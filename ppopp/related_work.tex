\section{Related work}

% Explicit task level parallelism
A lot of languages and libraries have been created to allow easy
expression of task-level parallelism.  Cilk's \cite{blumofe96cilk}
model of fork/join parallelism matches our function-spawning model
most closely, although a Cilk {\tt sync} always joins with all tasks 
spawned in the current procedure activation rather than being able to 
pick a specific task to join. While OpenMP \cite{dagum98openmp} 
originally focused almost exclusively on loops, version 3.0 adds 
support for task parallelism.  Other notable examples
include Java's concurrency library \cite{lea00java} and X10
\cite{charles05x10}.

% Other limits surveys
Several studies have been made on instruction-level parallelism
\cite{wall91limits, postiff99limits, austin92dynamic, lam92limits,
mak09limits}, but fewer have tried to separate out task-level
parallelism from instruction-level parallelism.  Kreaseck et al.\
\cite{Kreaseck00limitsof} explored limits of speculative task-level
parallelism by pre-emptively [what do you mean? /KFF] 
executing function calls, similar to the
way we hoist spawns.  They have however imposed the restriction that
spawned function calls must be joined at their original call sites,
which is a restriction we have felt to be unnecessary, and thus have
not imposed in our analysis.  In our model function calls can be
joined as late as dependencies allow (but always before the parent
call returns).  Other studies \cite{warg01limits, oplinger99insearch}
have shown that data value prediction, especially in regard to return
values, is effective at increasing task-level parallelism.  This is
something we wish to consider further in the future.

% Race detection
There are parallels between Embla and on-the-fly data race detection
\cite{MellorCrummey91onthefly, ha02space, savage97eraser}.  Both use
instrumentation to infer properties of the program through dynamic
analysis.  The main difference is that while race detection seeks to
find unsafe parallelism (i.e.\ bugs) in an explicitly multi-threaded
program, Embla seeks to find potential safe parallelism in a
sequentially written program.

% Automatic parallelisation
There has been much research into automatic parallelisation
\cite{kennedy02optimizing, Blume94polaris, ottoni05automatic,
ottoni07global}, most of which using only static analysis.  However,
the lack of precision when statically analysing dependencies remains a
barrier, something which dynamic analysis tools such as Embla can deal
with.  We believe that both static and dynamic analyses are necessary
for automatic parallelisation, as even though run-time information is
much more precise about dependencies, it does incur overheads, some of
which may be eliminated by static means.

% Thread-level speculation
One way of achieving better automatic parallelisation is to use
thread-level speculation (TLS) \cite{Rundberg01anall-software,
gregory05stampede, welc05safe}.  Indeed, some of the speculative
task-level parallelism uncovered in this study can be exploited in
this way.  One important factor affecting the performance of TLS is
the selection of good candidate threads for speculation, as frequent
rollbacks would offset any gains in parallelism \cite{johnson04mincut,
liu06posh}.  We believe that Embla can be used to identify good
candidates, as it allows us to look at the frequencies at which
potential dependencies materialise.
