\section{Implementation}

\subsection{Computing Dependencies}   \label{snca}

\begin{figure} \small
\hrulefill
\[
\begin{picture}(160,60)(70,15)
\put(120,65){\makebox(60,10)[c]{\it A:\ \tt f}}
\put(150,65){\line(-2,-1){50}}
% \put(150,65){\line( 0,-1){25}}
\put(150,65){\line( 2,-1){50}}
% \put(95,45){\makebox(20,10)[r]{\it 14}}
% \put(150,45){\makebox(20,10)[l]{\it 15}}
% \put(185,45){\makebox(20,10)[l]{\it 16}}
% \put(70,50){\makebox(20,10)[r]{\ldots}}
% \put(210,50){\makebox(20,10)[l]{\ldots}}
\put(170,30){\makebox(60,10)[c]{\it C:\ \tt g2}}
% \put(120,30){\makebox(60,10)[c]{\it C:\ \tt inc}}
\put(70,30){\makebox(60,10)[c]{\it B:\ \tt g1}}
\put(70,15){\makebox(60,10)[cb]{{\tt *a\ =}\ \ldots}}
\put(170,15){\makebox(60,10)[cb]{\ldots\ {\tt =\ *b}}}
\end{picture}
\]
\hrulefill
\caption{Part of the execution tree of the program in Figure~\ref{datadeps}
% edges are annotated with the line number of the corresponding call.
} 
\label{ffextree}
\end{figure}

Embla traces dependences between individual instructions in the binary
code during program execution. It then maps these dependencies to 
dependencies between pairs of source lines in the same function. The
instructions causing the dependence need not be part of the function 
within which the dependence is reported. For example, in 
Figure~\ref{datadeps} the instructions causing the dependence (which we
will from now on refer to as the {\em endpoints} of the dependence) 
are part of {\tt g1} and {\tt g2}, respectively, wheras the dependence 
will be reported as a dependence between the {\em calls to} {\tt g1} and 
{\tt g2} in {\tt f}.

Embla maintains an {\em execution tree} that at each moment captures 
the history of execution up to that moment. Every execution tree node 
corresponds to an
individual function call, and the path from the node to the root of the
tree corresponds to the call stack at the moment of the call. For
example, Figure~\ref{ffextree} depicts a fragment of the execution tree
for the program in Figure~\ref{datadeps}, capturing the calls of {\tt g1} 
and {\tt g2} from {\tt f}.

Embla computes the source-level dependences
from the instruction-level ones using the execution tree as follows. For
every instruction-level dependence, Embla identifies the function calls
where the endpoints occurred (nodes {\it B} and {\it C} for the example
above), and computes the {\em nearest common ancestor} node (NCA) of
those nodes in the execution tree. The NCA corresponds to a function
call with two instructions that are dependent because of
the instruction-level dependence (the calls to {\tt g1} and {\tt g2} in 
{\tt f} in the example).

\newcommand{\tracepile}{trace pile}


Embla uses two main data structures: The {\em \tracepile}, which represents
the execution tree, and the {\em memory table} which maps addresses to tree
nodes corresponding to the last write and subsequent reads of that
location. The \tracepile\ contains the part of the execution tree
corresponding to the part of the instruction trace that has been
seen so far. Each item in the trace pile corresponds to a memory reference 
or a procedure call or return; an execution tree node corresponds to
a subset of the items.

For each item {\tt n}, {\tt n.parent} is the parent node in the
execution tree (there is also other information associated with a
node, such as what source line corresponds to the node).  The NCA is
computed by following the {\tt parent} links, starting at the earlier
of the instructions corresponding to the dependence, until a node
corresponding to an activation record currently on the call stack is
found. This is the NCA since the later instruction in the pair (which
is also the current instruction), and hence all of its ancestors, are
on the stack.

Once a procedure call has returned, we will not distinguish between 
different events in the subtree corresponding to its (completed) 
execution. The NCA computation will simply skip them, following the 
{\tt parent} links until it finds a node on the stack (which will be
the call instruction at the root of this subtree). Thus we can 
periodically compact the \tracepile\ by replacing subtrees
corresponding to completed calls by their root nodes, saving vast amounts
of memory.
After compaction, the \tracepile\ contains the items associated with the
tree nodes corresponding to the stack, with call instruction items
representing entire subtrees.

Embla uses the Valgrind instrumentation infrastructure which emulates
the user mode instruction set architecture. An interesting alternative
would be to sample dependencies using hardware data breakpoints
(Accumem's Virtual Performance Expert uses this approach for cache
profiling, but since single dependencies are much more important than
single cache misses, it is not clear if sampling is suitable for
dependence analysis).

Our examples of profiling
output use C, but
the profiling is done at instruction level, and the result is
mapped to source level using debugging information.


