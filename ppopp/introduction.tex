\section{Introduction}

Parallel programming is no longer optional.  In order to enjoy continued
performance gains with future generation multicore processors,
application developers must parallelize all software, old and
new
% \cite{TEL95,ONHWC96,KAB03,VIAVAC05}
.  
% For scalable parallel
% performance, program execution must be divided into large numbers of
% independent tasks that can be scheduled on available cores and hardware
% threads by runtime systems.  
While automatic parallelization based on static analysis 
% \cite{KA02}
is sometimes feasible, currently most software requires manual
parallelization.
Since this is a difficult task, there is urgent need for efficient tool support. 
In particular, tools that assist the programmer in understanding the potential 
for parallelization in the code, finding promising parts of the code to parallelize, 
and in validating that the resulting parallel code is correct.

In this paper, we concentrate on the first issue; estimating the potential 
for parallelism in a sequential program. The objective is to construct a tool
which, when given a sequential program and an input, can estimate the 
amount of parallelism that would be available if the program was parallelized
using a language like OpenMP or Cilk. We validate our results against span 
measurements of explicitly parallel Cilk programs.

For explicitly parallel languages, 
such tools are based on the parallelization provided by the programmer, and 
estimates the performance of that particular parallel program. For programs written
in a sequential language, the tool must construct
an {\em estimated parallelization}. This requires two kinds of information:
\begin{itemize}
\item
Information about the program, in particular about the dependencies between 
different parts, since these determine which parallelization is legal (this 
information is also needed to find parallelization opportunities and validating 
correctness).
\item
Information about the parallel execution mechanisms and how these are going
to be integrated into the program, that is, what parallelizing transformations 
will be used.
\end{itemize}
We have implemented a parallelism measurement tool by extending the existing
tool Embla [cite!] (which captures dependence information by an instrumented 
execution of a program) with functionality for simulating the effects of various parallel 
programming models, all based on independent fork-join task parallelism
%\cite{Conway63}
, a framework used in many parallel programming environments.
%\cite{BJKLR96, Lea00, DM98, LF00}.

In traditional studies of parallelism limits, the focus is typically
on hardware support for some model of parallel execution, for instance 
instruction level parallelism in the seminal study by Wall \cite{wall91limits} 
or speculative module level parallelism \cite{warg01limits}. Parameters
such as the amount of hardware resources (buffer space, function
units, \ldots) are then varied. Our focus is rather on explicit
parallelization of the program source code. This leads to a somewhat
different set of parameters to vary and different constraints for
parallelization. For instance, in contrast to e.g. Wall, 
we use control dependence information
in our baseline model since that is typically available for source
level transformations. On the other hand, our baseline model is
static, in the sense that each code section is parallelized in one way
only; this parallelization must be correct in all executions of that
section. This is consistent with source level parallelization, and
special measures have to be taken to lift the restriction (for
instance function cloning, to allow code to be parallelized for some 
call sites and not for others). To fully implement this model, speculative 
execution (dependence speculation) has to be used. Section~\ref{smethod} 
discusses our parallelization models in depth.

We have used our prototype tool to investigate the potential for 
parallelization of three collections of programs; the SPEC CPU 2000 
integer programs, miBench and the example programs distributed with
the Cilk 5.4.6 distribution. While the two first collections contain
sequential programs, the last one is made up of explicitly parallel 
programs in the Cilk 5 language. These have the property that eliding
the parallel construct leaves correct sequential code, and we have run 
our tool on that code as validation of our approach. We can thus contrast 
programs from these different sources as well as show the behaviour of 
the different models, including the effects of control and dependence speculation.
Section~\ref{sresults} reports the results of these experiments.




\begin{figure}
\small
\hrulefill
\[
\begin{minipage}[t]{3cm}
\begin{alltt}
   p();
   q();
   r();
\end{alltt}
\end{minipage}
\begin{minipage}[t]{3cm}
\begin{alltt}
   spawn p();
   q();
   sync;
   r();
\end{alltt}
\end{minipage} 
\]
\hrulefill
\caption{Example of fork-join parallelism.}
\label{fforkjoin}
\end{figure}

Consider the program fragment in
Figure~\ref{fforkjoin} (left):
Suppose that the calls to {\tt p()} and {\tt q()} are independent,
but that the call to {\tt r()} depends on the earlier calls. Then
the call to {\tt p()} can be
executed in parallel with
the call to {\tt q()}, as shown to the right.
Here we assume the availability of a construct {\tt spawn} to start
the call in parallel and {\tt sync} to wait for all {\tt spawn}'d
activities to terminate (cf. Cilk
%\cite{BJKLR96}
).
% As long as {\tt p()} and {\tt q()} are
% independent, the parallel program will produce identical results to
% the sequential version.  Therefore it is sufficient to understand
% (debug, verify, \ldots) the sequential program; everything except
% performance carries over to the parallel version.


