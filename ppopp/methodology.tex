\section{Models of Parallelism} \label{smethod}

One of the main aims of our tool is to separate out task-level parallelism from instruction-level parallelism in a program.
Instruction-level parallelism is fine-grain, local and typically exploited in superscalar and VLIW machines.
It is however unsuitable for multi-core processors, as threading and communication overheads would dominate.
Thus we would like to only consider parallelism at \emph{task}-level, where a task is a set of instructions that can run independently of other concurrent tasks, and is large enough to justify the costs of spawning a new thread to execute it.

To mirror the task model of popular parallel programming environments such as Cilk~\cite{blumofe96cilk}, Java~\cite{lea00java}, TBB~\cite{reinders07intel} and TPL~\cite{leijen07parallel}, we define a task in Embla as a function call.
The main reason for this is that functions are mainly used by programmers as an abstraction over a sequence of operations, and as such tend to be relatively self-contained and independent and would usually consist of a substantial amount of work.
Also, the use of existing programming constructs such as functions as task boundaries means that the task-level parallelism that we discover can be extracted by both programmer and compiler easily.

In our model, each function call is spawned at its call site as a task.
The calling thread can then continue to execute statements that follow the function call without waiting for the call to return, until control reaches a statement that has a dependence on the call.
At this point the calling thread must \emph{synchronize} on the task, i.e.\ wait for the task to complete before going any further.
In addition, each function call must synchronize on all tasks it has spawned that are still running before returning.

We calculate inherent parallelism by constructing a \emph{dynamic dependence graph} for each function call, an example of which is shown in Figure~\ref{example-depgraph}.
A dynamic dependence graph is a directed acyclic graph $G=(V,E)$ where each node $v\in V$ corresponds to an \emph{instantiation}, or execution of a \emph{line} of the program\footnote{Ideally we would like to have a node corresponding to the each execution of a \emph{statement}, but the gcc debug information that we use contains only line not statement information.}, and each edge $e\in E$ corresponds to a dependence between two line instantiations.
Such a dependence means that they cannot be executed in parallel---the source of the dependence must be completed before the target can begin.
The \emph{cost} of a node $v$, $\mathit{cost}(v)$, is the minimum time required to execute the instantiation.
In our model this is basically the number of instructions executed on that line, with the additional requirement that if one of the instructions is a function call, then the length of the critical path of that call is included in the cost of the node.
The \emph{critical path} $\mathit{CP}$ is the path in $G$ with the largest total cost.
The amount of inherent parallelism for a function call is then the cost of serial execution divided by the length of the critical path, i.e. $\sum_{v\in V} \mathit{cost}(v)/\sum_{v\in \mathit{CP}} \mathit{cost}(v)$.
The inherent parallelism in the entire execution of a program is then the inherent parallelism for its \texttt{main} function.

\begin{figure}
  \begin{center}
  \small
  \input{example.depgraph}
  \end{center}
  \nocaptionrule \caption{Example dependence graph from Embla}
  \label{example-depgraph}
\end{figure}

We now describe in greater detail some parameters of our model, which can be varied to fit different parallel programming models and to examine the effects of different optimizations.

\subsection{Data dependences}
Data dependences arise when two lines of a program access the same memory location, and at least one of them writes to it.
They can be categorized into true (read-after-write), anti- (write-after-read) and output (write-after-write) dependences.
In our tool data dependences can be \emph{exact} or \emph{aggregated}.
With exact dependences, each instantiation of a line has its own set of dependences---different instantiations of the same line may therefore have different dependences, e.g.\ if they access different memory locations.
When data dependences are aggregated, however, there is just one set of dependences for each line, namely the union of the dependences for each instantiation.
This means that the dynamic dependence graph will be built with each instantiation of a line having the same set of dependences.
The difference between the two can be illustrated in the program in Figure~\ref{datadeps}, where there is a dependence between the calls to \texttt{g1} and \texttt{g2} in \texttt{f} only when \texttt{a} and \texttt{b} alias.
A model using aggregated data dependences would ascribe that dependence to all calls to \texttt{f}, resulting in no parallelism, while one using exact dependences would only place it in the second call, resulting in some parallelism in the first call.

\begin{figure}
  \begin{center}
  \small
  \begin{verbatim}
    void g1(int *a) {
      ...
      *a = ...; // Writes to *a
    }
  
    void g2(int *b) {
      ... = *b; // Reads from *b
      ...
    }

    int f(int *a, int *b) {
      int x;
      g1(a);
      g2(b);
    }

    int main(int argc, char *argv[]) {
      int x=0, y=1;
      f(&x, &y);
      f(&x, &x);
    }
  \end{verbatim}
  \end{center}
  \caption{Program illustrating the difference between aggregated and exact data dependences}
  \label{datadeps}
\end{figure}

Aggregated dependences are used when we would like to consider parallelism that can be achieved by inserting synchronization points in source code, necessitating the same synchronization point for all instantiations of a line.
Exact dependences, on the other hand, allow us to consider the parallelism that is possible by more dynamic techniques such as thread-level speculation (TLS)~\cite{Rundberg01anall-software,gregory05stampede, welc05safe}.
This difference can again be illustrated in the example above.
We cannot statically parallelize the calls to \texttt{g1} and \texttt{g2} as that would lead to a race condition in the second call of \texttt{f}.
With TLS however, we can attempt to run \texttt{g1} and \texttt{g2} in parallel, which results in potential parallelism in the first call to \texttt{f}, at the expense of a possible performance penalty of a rollback in the second call.

\subsection{Control dependences}
When whether a line is executed is not known until the execution of another (typically a branch), the former line is said to be control-dependent on the latter, and must not begin execution until the latter has completed.
In our tool, control dependences are calculated by constructing Control Dependence Graphs~\cite{ferrante87program}.
However, control dependences can also be turned off completely.
This in effect gives us the parallelism achievable with perfect (100\% accuracy) control speculation.
Previous studies on branch prediction~\cite{smith98study} suggest that even simple and compile-time prediction algorithms can result in high accuracies for many programs, meaning that control speculation is probably viable.

\subsection{Granularity}
In order to restrict parallelism to be task-level only, our default model only spawns lines with function calls.
Lines with simple statements and not function calls must be executed in order as they are usually too fine-grain.
Naturally this also excludes other forms of task-level parallelism where tasks are not delineated by functions.
As a result, our tool also contains an option to relax this constraint and spawn all lines, whether they contain function calls or not.
While this allows us to see greater task-level parallelism, some of the parallelism discovered will be too fine-grain to be useful.
Nonetheless, based on the line-level parallelism discovered, the programmer can then decide to refactor certain lines into their own spawnable function, once they have determined that the parallelism is sufficiently coarse-grain.

\subsection{Loop iterations}
Loops are generally considered to be another source of task-level parallelism, and are the primary parallel programming construct in programming libraries and environments such as OpenMP~\cite{dagum98openmp} and TBB~\cite{reinders07intel}.
There is an option to parallelize loops in our tool, where an established algorithm \cite{aho86compilers, muchnick97advanced} is used to identify natural loops, iterations of which can be spawned as tasks in the same way as function calls.

\subsection{Spawn hoisting}

\begin{figure}
  \begin{center}
  \scriptsize
  \begin{SubFloat}{\label{spawn:orig}Original program extract}
    \begin{minipage}{0.7in}
      \begin{verbatim}
int x, y, r=0;

x = fib(n-1);
r = r + x;
y = fib(n-2);
r = r + y;
return r;


      \end{verbatim}
    \end{minipage}%
  \end{SubFloat}%
  \qquad
  \begin{SubFloat}{\label{spawn:without}Best possible parallelization \emph{without} spawn hoisting}
    \begin{minipage}{1.0in}
      \begin{verbatim}
int x, y, r=0;

x = spawn fib(n-1);
sync;
r = r + x;
y = spawn fib(n-2);
sync;
r = r + y;
return r;
      \end{verbatim}
    \end{minipage}%
  \end{SubFloat}%
  \qquad
  \begin{SubFloat}{\label{spawn:with}Best possible parallelization \emph{with} spawn hoisting}
    \begin{minipage}{1.0in}
      \begin{verbatim}
int x, y, r;

x = spawn fib(n-1);
y = spawn fib(n-2);
r=0;
sync;
r = r + x;
r = r + y;
return r;
      \end{verbatim}
    \end{minipage}%
  \end{SubFloat}%
  \end{center}
  \caption{An illustration of spawn hoisting on part of a function that calculates a Fibonacci number.}
  \label{spawn}
\end{figure}

In our default model, a task is only spawned when control reaches its original site.
It is possible, however, that some tasks can be started earlier, resulting in greater parallelism.
This is illustrated in Figure~\ref{spawn}, where the line immediately following each call to \texttt{fib} is dependent on the result of the call in the original program (Figure~\ref{spawn:orig}).
As we need to synchronize on a spawned task before we reach any line that depends on it, we would need to insert a \texttt{sync} immediately after each call (Figure~\ref{spawn:without}).
If we allow function spawns to be hoisted, however, more opportunities for parallelism exist, as shown in Figure~\ref{spawn:with}, where both calls to \texttt{fib} have been hoisted to the beginning of the program extract.
provides To examine the extra parallelism this optimization allows, our tool provides an option to allow each spawn to be hoisted to as early a point as dependences allow.

