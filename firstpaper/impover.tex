% -*- eval: (local-set-key "\M-q" 'undefined) -*-
%
% The line above will probably make Emacs ask if it is ok to evaluate the
% expression.  Answer y.

\section{Overview of the Implementation of Embla}



\newcommand{\backlink}[3]
{\begin{picture}(#1,#2)
\put(0,0){\line(1,0){#1}}
\put(#1,0){\line(0,1){#2}}
\put(#1,#2){\vector(-1,0){#3}}
\end{picture}}

\newcommand{\rwblock}
{\begin{picture}(100,40)(-40,0)
\put(0,25){\line(4,3){20}}
\put(0,15){\line(4,-3){20}}
\put(-40,25){\line(1,0){40}}
\put(-40,15){\line(1,0){40}}
\put(20,0){\framebox(40,40){\ }}
\put(20,30){\line(1,0){40}}
\put(20,30){\makebox(40,10){Last write}}
\put(20,20){\makebox(40,10){Last reads}}
\put(20,5){\makebox(40,10){\vdots}}
\end{picture}}
\begin{figure*}
\small
\hrulefill
\[
\begin{picture}(250,210)(-30,0)
% The memory table
\put(-30,30){\framebox(40,180)[t]{\ }}
\put(-30,10){\makebox(60,10){The memory table}}
% A block
\put(-30,145){\rwblock}
\put(70,177){\vector(3,-2){30}}
\put(-30,85){\rwblock}
\put(70,113){\vector(4,-1){30}}
\put(70,97){\vector(3,-2){30}}
\put(-30,25){\rwblock}
\put(70,50){\vector(2,-1){30}}
% Trace pile
\put(100,30){\framebox(40,180)[t]{\ }}
\put(100,190){\framebox(40,10){Open C}}
\put(100,170){\makebox(40,10){\vdots}}
\put(100,160){\line(1,0){40}}
\put(100,150){\makebox(40,10){Reg}}
\put(140,157){\backlink{8}{36}{6}}
\put(100,140){\framebox(40,10){Closed C}}
\put(140,147){\backlink{16}{46}{6}}
\put(100,120){\makebox(40,10){\vdots}}
\put(100,100){\framebox(40,10){Reg}}
\put(140,107){\backlink{8}{36}{6}}
\put(100,85){\makebox(40,10){\vdots}}
\put(100,70){\framebox(40,10){Reg}}
\put(140,77){\backlink{16}{66}{6}}
\put(100,50){\line(1,0){40}}
\put(100,55){\makebox(40,10){\vdots}}
\put(100,40){\makebox(40,10){Return}}
\put(140,47){\backlink{24}{96}{6}}
\put(100,30){\framebox(40,10){Reg}}
\put(140,37){\backlink{32}{156}{14}}
\end{picture}
%
% The next picture
%
\begin{picture}(200,210)(-30,0)
% The memory table
\put(-30,30){\framebox(40,180)[t]{\ }}
\put(-30,10){\makebox(60,10){The memory table}}
% A block
\put(-30,145){\rwblock}
\put(70,181){\vector(4,3){30}}
\put(-30,85){\rwblock}
\put(70,108){\vector(1,2){30}}
\put(70,102){\vector(1,2){30}}
\put(-30,25){\rwblock}
\put(70,52){\vector(1,3){30}}
% Trace pile
\put(100,30){\framebox(40,180)[t]{\ }}
\put(100,200){\makebox(40,10){Open C}}
\put(100,200){\line(1,0){40}}
\put(100,185){\makebox(40,10){\vdots}}
\put(100,170){\framebox(40,10){Reg}}
\put(140,177){\backlink{8}{26}{6}}
\put(100,160){\makebox(40,10){Closed C}}
\put(140,167){\backlink{16}{36}{6}}
\put(100,150){\framebox(40,10){Return}}
\put(140,157){\backlink{8}{6}{6}}
\put(100,140){\makebox(40,10){Reg}}
\put(140,147){\backlink{24}{56}{6}}
\put(100,140){\line(1,0){40}}
\end{picture}
\]
\hrulefill
\caption{The main Embla data structures}
\label{fnyembladata}
\end{figure*}


Embla is based on instrumented execution of binary code. Although
our examples of profiling output use a high level language (C),
the profiling itself is on instruction level, followed by 
mapping the information to source level using debugging information 
in the standard way.

Embla uses the Valgrind instrumentation infrastructure, so there
is no offline code rewriting; the Embla tool behaves like an emulator
of the hardware.

Figure~\ref{fembladata} shows the three fundamental data structures 
in Embla:
\begin{description}
\item[The memory table:]
Contains, for each block of memory, information about the most 
recent write and all reads since that write. 
Blocksize is a tradeoff between memory use and precision; the examples in 
this paper were run with single byte blocks. 
\item[The activation tree:]
Contains a record for each stack frame allocated during execution. 
At any given point during execution, 
the rightmost branch of the activation tree corresponds to the stack.
\item[The result table:]
A hash table mapping dependency signatures to counters of RAW, WAR 
and WAR dependencies. Dependency signatures identify the source and
target of the dependency as well as other information of interest.
\end{description}
The figure also shows how the data structures are used when the
profiled program does a memory read. From the memory table we find the
activation frame that was active when the last write was made and we
compute the {\em nearest common ancestor} (NCA) of that an the
current frame. The flow dependence we find will be attributed to the
function that the NCA corresponds to. 

If a dependency endpoint is part of the NCA (if the NCA is the current
frame or the frame pointed to from the memory table) it is labelled as
direct, otherwise it is indirect. If any of the activation frames on
the path from an endpoint to the NCA is on the black list, we have a
weak endpoint.


Some anti and output dependencies are due to reusing the same
locations for (logically) distinct data items. This happens for
instance for stack locations when arguments are passed on the stack
(as is common on the x86, for instance). Consider the following
example:
\begin{verbatim}
    foo(x);
    bar(y);
\end{verbatim}
The value of {\tt x} will be pushed on the stack. Then {\tt foo} will
be called and will read {\tt x}. When that call has returned, {\tt y}
will be pushed on the stack on the same address as {\tt x}, creating
an anti dependence with {\tt foo}'s read. This kind of dependence can 
easily be removed by a
compiler. In fact, if the call to {\tt foo} is made asynchronously, in
another thread, it will automatically be made on a different stack and
the dependence disappears. Register allocation is another source of
anti and output dependencies since distinct variables are packed into
the same register. In this case, the allocation can be changed to
remove dependenies, a transformation known as {\em renaming}.

We currently try to avoid these spurious dependencies in the following
way. The memory table contains time stamps (these are simply the
number of (emulated) instructions executed since the start). We also
keep track of the changing stack size during execution so that we can
see if a memory location has been first in, then out of, then in the
stack again. Thus when we see a memory location (now part of the
stack) with time stamp $t$, we want to know if the stack has been
small enough to exclude that location at some point $t'$ such that 
$t<t'$. 

To this end, we maintain a {\em min stack} consisting of pairs $(t,s)$
where $t$ is a time stamp and $s$ is a stack pointer value. Each pair
$(t,s)$ in the min stack indicates that, at time $t$, the stack
pointer was $s$ and at all times $t'$ the stack has been larger (since
the x86 stack grows from high to low addresses, a small stack
corresponds to a large stack pointer). In the min stack, both
time stamps and stack sizes grow towards the top (that is, the stack
pointers decrease).

When the
call stack grows, we push a new pair, consisting of the current time
and the new stack pointer value, on top of the min stack. When the
stack shrinks, we pop all pairs representing larger stacks off the min
stack.

Thus, when we record a dependence at address $a$ with time stamp $t$,
we see if we can find a pair $(s,t')$ in the min stack with $a<s$ and
$t<t'$ (a smaller stack at a later time).

We avoid spurious anti and output dependencies due to register
allocation by not computing dependencies on registers and compiling
the traced program without optimization. 
