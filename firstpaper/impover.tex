\section{Overview of the Implementation of Embla}

Embla is based on instrumented execution of binary code. Although
our examples of profiling output use a high level language (C),
the profiling itself is on instruction level, followed by 
mapping the information to source level using debugging information 
in the standard way.

Embla uses the Valgrind instrumentation infrastructure, so there
is no offline code rewriting; the Embla tool behaves like an emulator
of the hardware.

There are three fundamental data structures in Embla.
\begin{description}
\item[The memory table:]
Contains information about the most recent reads and writes to each
block of memory. Blocksize is a tradeoff between memory use and 
precision. 
\item[The activation tree:]
Contains a record for each stack frame allocated during execution. 
The rightmost branch of the activation tree corresponds to the stack.
\item[The result table:]
A hash table mapping dependency signatures to counters of RAW, WAR 
and WAR dependencies. Dependency signatures identify the source and
target of the dependency as well as other information of interest.
\end{description}

Some anti and output dependencies are due to reusing the same
locations for (logically) distinct data items. This happens for
instance for stack locations when arguments are passed on the stack
(as is common on the x86, for instance). Consider the following
example:
\begin{verbatim}
    foo(x);
    bar(y);
\end{verbatim}
The value of {\tt x} will be pushed on the stack. Then {\tt foo} will
be called and will read {\tt x}. When that call has returned, {\tt y}
will be pushed on the stack on the same address as {\tt x}, creating
an anti dependence with {\tt foo}'s read. This kind of dependence can 
easily be removed by a
compiler. In fact, if the call to {\tt foo} is made asynchronously, in
another thread, it will automatically be made on a different stack and
the dependence disappears. Register allocation is another source of
anti and output dependencies since distinct variables are packed into
the same register. In this case, the allocation can be changed to
remove dependenies, a transformation known as {\em renaming}. 
Other anti dependencies are due to updates
to the same data structure and cannot be eliminated by renaming.

