\section{Overview of the Implementation of Embla}



\begin{figure} \small
\hrulefill
\[
\begin{picture}(160,90)(70,10)
\put(120,90){\makebox(60,10)[c]{\it A:\ \tt main}}
\put(150,90){\line(-1,-1){50}}
\put(150,90){\line( 0,-1){50}}
\put(150,90){\line( 1,-1){50}}
\put(100,60){\makebox(20,10)[r]{\it 14}}
\put(150,60){\makebox(20,10)[l]{\it 15}}
\put(180,60){\makebox(20,10)[l]{\it 16}}
\put(80,50){\makebox(20,10)[r]{\ldots}}
\put(200,50){\makebox(20,10)[l]{\ldots}}
\put(170,30){\makebox(60,10)[c]{\it D:\ \tt inc}}
\put(120,30){\makebox(60,10)[c]{\it C:\ \tt inc}}
\put(70,30){\makebox(60,10)[c]{\it B:\ \tt inc}}
\put(70,10){\makebox(60,10)[c]{{\tt *q=}\ \ldots}}
\put(170,10){\makebox(60,10)[c]{\ldots\ {\tt *q}\ \ldots}}
\end{picture}
\]
\hrulefill
\caption{Part of the call tree of Example 1, edges are annotated 
with the line number of the corresponding call.} 
\label{ffextree}
\end{figure}


Embla is based on instrumented execution of binary code. Although
our examples of profiling output use a high level language (C),
the profiling itself is on instruction level, followed by 
mapping the information to source level using debugging information 
in the standard way.

Embla uses the Valgrind instrumentation infrastructure, so there
is no offline code rewriting; the Embla tool behaves like an emulator
of the hardware.

Tracing programs to find dependencies is by no means a new endavour. Most 
earlier efforts have however been aimed at studying different hardware 
enhancements. In that context, the desired result is the execution time 
under various hardware assumptions. Thus it was sufficient to keep track
of the execution of the dynamic instruction stream, including dependencies
between instructions; mapping the dependencies to source code was unnecessary.

It is possible to find the minimal execution time by keeping track of the 
earliest possible execution
time of each instruction in the dynamic instruction stream. For each instruction 
that writes to a memory location or a register, that location or register is 
associated with the execution time of the writing instruction. The earliest
execution time of an intruction that reads memory or registers is taken as the
maximum of the time stamps of the items read plus the latency of the instruction 
itself.


A tool like Embla, on the other hand, that is designed to help programmers 
rewrite source code, poses different demands. Here, we need to project 
the dependencies discovered in the dynamic instruction stream onto the 
source code in such a way as to inform the programmer about legal code 
transformations. Since we are dealing with transformations on the level of
procedure calls, we need to reflect the dependencies we find to that level 
as well. That is, we always want to know the dependencies between different 
lines in the same procedure.

Consider the program from Figure~\ref{ffirstex} where we have three consecutive
calls to {\tt inc} in lines 14--16. Figure~\ref{ffextree} gives the relevant 
part of the dynamic call tree. The flow dependence between line 14 and 16 is due 
to the write in {\tt inc} from the first call and the read from the second call.
Note that the same instructions in other calling contexts give rise to 
different dependencies, eg the flow dependence from line 15 (call to {\tt inc})
to line 17 (call to {\tt printf}).

We accomplish this mapping by associating each memory location with a dynamic
instruction execution event and we also keep track of how these events 
relate to the dynamic call tree. Thus, for a memory reference event $E$ we 
look up its data address to find the previous reference event $E'$ to that 
location. Then we find the {\em nearest common anscestor} (NCA) of the call
tree nodes of $E$ and $E'$ (for instance, the NCA of {\it B} and {\it D} is
{\it A} in Figure~\ref{ffextree}). The NCA can be calculated from the previous
reference event $E'$ itself by noting that all anscestors of the current
reference event $E$ correspond to frames on the current call stack. If we mark 
such nodes, we can search from $E'$ 
towards the root of the call tree; the first node we reach that is on the 
stack is the NCA.
The line number (in the function corresponding to the NCA) of the source or 
destination of the dependence is the line number of the last edge on the path 
from the relevant event to the NCA (14 and 16, respectively, in 
Figure~\ref{ffextree}).


\begin{figure}
\small
\hrulefill
\begin{verbatim}
   DependenceEdge( oldEvent, currEvent ) {
       oldLine = LineOf( oldEvent ) 
       ncaNode = NodeOf( oldEvent )
       while( ncaNode is not on stack )
           oldLine = LineOf( ncaNode )
           ncaNode = ParentOf( ncaNode )
       if( ncaNode != NodeOf( currEvent ) )
           currLine = LineOf( NextOnStack( ncaNode ) )
       else
           currLine = LineOf( currEvent )
       return ( oldLine, currLine )
    }
\end{verbatim}
\hrulefill
\caption{Finding constructing the dependence edge from the events corresponding
to a previous and a new reference}
\label{fdepedge}
\end{figure}    

\begin{figure}
\small
\hrulefill
\begin{verbatim}
   DependenceEdge( oldEvent, currEvent ) {
       oldLine = oldEvent.line
       ncaNode = olsEvent.node
       while( ncaNode is not on stack )
           oldLine = ncaNode.line
           ncaNode = ncaNode.parent
       if( ncaNode != currEvent.node )
           currLine = ncaNode.nextOnStack.line
       else
           currLine = currEvent.line
       return ( oldLine, currLine )
    }
\end{verbatim}
\hrulefill
\caption{Finding constructing the dependence edge from the events corresponding
to a previous and a new reference}
\end{figure}    

The algorithm for computing the dependence edge given a data address and the 
current reference event is given in Figure~\ref{fdepedge}. Here an event 
represents the execution of a memory reference and a node corresponds to a 
procedure invocation. For an event $E$, {\tt LineOf($E$)} is the source line
corresponding to $E$ and {\tt NodeOf($E$)} is the procedure invocation that $E$
is part of. For a node $N$, {\tt LineOf($N$)} is the source line associated 
with the procedure call corresponding to $N$, {\tt ParentOf($N$)} is the 
parent node in the call tree and, if $N$ is in the stack, {\tt NextOnStack($N$)} 
is the node on top of $N$ in the stack ({\tt ParentOf(NextOnStack($N$))} = $N$).

Path compression can be used to decrease the number of iterations of the loop.
Every node $N$ that is visited but is not on the stack can have its parent set 
to its closest ancestor on the stack. We conjecture that this reduces the 
complexity of the algorithm to essentially constant time. We can however do even 
better since the output of the edge calculation does only depend on the immediate
descendants of nodes on the stack. Thus every subtree $T$ of the call tree, 
including the events at its leaves, such
that the root of $T$ is not on the stack can be represented by the root of $T$
alone.


\newcommand{\backlink}[3]
{\begin{picture}(#1,#2)
\put(0,0){\line(1,0){#1}}
\put(#1,0){\line(0,1){#2}}
\put(#1,#2){\vector(-1,0){#3}}
\end{picture}}

\newcommand{\rwblock}
{\begin{picture}(100,40)(-40,0)
\put(0,25){\line(4,3){20}}
\put(0,15){\line(4,-3){20}}
\put(-40,25){\line(1,0){40}}
\put(-40,15){\line(1,0){40}}
\put(20,0){\framebox(40,40){\ }}
\put(20,30){\line(1,0){40}}
\put(20,30){\makebox(40,10){Last write}}
\put(20,20){\makebox(40,10){Last reads}}
\put(20,5){\makebox(40,10){\vdots}}
\end{picture}}
\begin{figure*}
\small
\hrulefill
\[
\begin{picture}(250,210)(-30,0)
% The memory table
\put(-30,30){\framebox(40,180)[t]{\ }}
\put(-30,10){\makebox(60,10){The memory table}}
% A block
\put(-30,145){\rwblock}
\put(70,177){\vector(3,-2){30}}
\put(-30,85){\rwblock}
\put(70,113){\vector(4,-1){30}}
\put(70,97){\vector(3,-2){30}}
\put(-30,25){\rwblock}
\put(70,50){\vector(2,-1){30}}
% Trace pile
\put(100,30){\framebox(40,180)[t]{\ }}
\put(100,190){\framebox(40,10){Open C}}
\put(100,170){\makebox(40,10){\vdots}}
\put(100,160){\line(1,0){40}}
\put(100,150){\makebox(40,10){Reg}}
\put(140,157){\backlink{8}{36}{6}}
\put(100,140){\framebox(40,10){Closed C}}
\put(140,147){\backlink{16}{46}{6}}
\put(100,120){\makebox(40,10){\vdots}}
\put(100,100){\framebox(40,10){Reg}}
\put(140,107){\backlink{8}{36}{6}}
\put(100,85){\makebox(40,10){\vdots}}
\put(100,70){\framebox(40,10){Reg}}
\put(140,77){\backlink{16}{66}{6}}
\put(100,50){\line(1,0){40}}
\put(100,55){\makebox(40,10){\vdots}}
\put(100,40){\makebox(40,10){Return}}
\put(140,47){\backlink{24}{96}{6}}
\put(100,30){\framebox(40,10){Reg}}
\put(140,37){\backlink{32}{156}{14}}
\end{picture}
%
% The next picture
%
\begin{picture}(200,210)(-30,0)
% The memory table
\put(-30,30){\framebox(40,180)[t]{\ }}
\put(-30,10){\makebox(60,10){The memory table}}
% A block
\put(-30,145){\rwblock}
\put(70,181){\vector(4,3){30}}
\put(-30,85){\rwblock}
\put(70,108){\vector(1,2){30}}
\put(70,102){\vector(1,2){30}}
\put(-30,25){\rwblock}
\put(70,52){\vector(1,3){30}}
% Trace pile
\put(100,30){\framebox(40,180)[t]{\ }}
\put(100,200){\makebox(40,10){Open C}}
\put(100,200){\line(1,0){40}}
\put(100,185){\makebox(40,10){\vdots}}
\put(100,170){\framebox(40,10){Reg}}
\put(140,177){\backlink{8}{26}{6}}
\put(100,160){\makebox(40,10){Closed C}}
\put(140,167){\backlink{16}{36}{6}}
\put(100,150){\framebox(40,10){Return}}
\put(140,157){\backlink{8}{6}{6}}
\put(100,140){\makebox(40,10){Reg}}
\put(140,147){\backlink{24}{56}{6}}
\put(100,140){\line(1,0){40}}
\end{picture}
\]
\hrulefill
\caption{Compacting the trace pile: before and after}
\label{fnyembladata}
\end{figure*}


Figure~\ref{fnyembladata} shows the three fundamental data structures 
in Embla:
\begin{description}
\item[The memory table:]
Contains, for each block of memory, information about the most 
recent write and all reads since that write. 
Blocksize is a tradeoff between memory use and precision; the examples in 
this paper were run with single byte blocks. 
\item[The trace pile:]
Contains trace records representing instruction execution events that 
Embla needs to refer to. Most trace records contain 
\begin{itemize}
\item
a pointer to a record containing information about the program 
address of the corresponding instruction, and
\item
a pointer to the previous trace record representing a procedure call.
\end{itemize}
% \item[The activation tree:]
% Contains a record for each stack frame allocated during execution. 
% At any given point during execution, 
% the rightmost branch of the activation tree corresponds to the stack.
\item[The result table:]
A hash table mapping dependency signatures to counters of RAW, WAR 
and WAR dependencies. Dependency signatures identify the source and
target of the dependency as well as other information of interest.
\end{description}
In order to find all dependencies between a pair of lines, we also have
to map the references occurring during the execution of procedures 
called at those lines to the lines themselves. In general there may be 
many activation records on top of each other, which raises the question 
of how many records to ``unwind''. Consider the program in 
figure~\ref{ffirstex}. Part of the call tree of the program is given 
in figure~\ref{ffextree}. The dependence between lines 14 and 16 is due to
the write by the {\tt inc} function when called from line 14 and the
read by {\tt inc} when called from line 16. 








The figure also shows how the data structures are used when the
profiled program does a memory read. From the memory table we find the
activation frame that was active when the last write was made and we
compute the {\em nearest common ancestor} (NCA) of that an the
current frame. The flow dependence we find will be attributed to the
function that the NCA corresponds to. 

If a dependency endpoint is part of the NCA (if the NCA is the current
frame or the frame pointed to from the memory table) it is labelled as
direct, otherwise it is indirect. If any of the activation frames on
the path from an endpoint to the NCA is on the black list, we have a
weak endpoint.


Some anti and output dependencies are due to reusing the same
locations for (logically) distinct data items. This happens for
instance for stack locations when arguments are passed on the stack
(as is common on the x86, for instance). Consider the following
example:
\begin{verbatim}
    foo(x);
    bar(y);
\end{verbatim}
The value of {\tt x} will be pushed on the stack. Then {\tt foo} will
be called and will read {\tt x}. When that call has returned, {\tt y}
will be pushed on the stack on the same address as {\tt x}, creating
an anti dependence with {\tt foo}'s read. This kind of dependence can 
easily be removed by a
compiler. In fact, if the call to {\tt foo} is made asynchronously, in
another thread, it will automatically be made on a different stack and
the dependence disappears. Register allocation is another source of
anti and output dependencies since distinct variables are packed into
the same register. In this case, the allocation can be changed to
remove dependenies, a transformation known as {\em renaming}.

We currently try to avoid these spurious dependencies in the following
way. The memory table contains time stamps (these are simply the
number of (emulated) instructions executed since the start). We also
keep track of the changing stack size during execution so that we can
see if a memory location has been first in, then out of, then in the
stack again. Thus when we see a memory location (now part of the
stack) with time stamp $t$, we want to know if the stack has been
small enough to exclude that location at some point $t'$ such that 
$t<t'$. 

To this end, we maintain a {\em min stack} consisting of pairs $(t,s)$
where $t$ is a time stamp and $s$ is a stack pointer value. Each pair
$(t,s)$ in the min stack indicates that, at time $t$, the stack
pointer was $s$ and at all times $t'$ the stack has been larger (since
the x86 stack grows from high to low addresses, a small stack
corresponds to a large stack pointer). In the min stack, both
time stamps and stack sizes grow towards the top (that is, the stack
pointers decrease).

When the
call stack grows, we push a new pair, consisting of the current time
and the new stack pointer value, on top of the min stack. When the
stack shrinks, we pop all pairs representing larger stacks off the min
stack.

Thus, when we record a dependence at address $a$ with time stamp $t$,
we see if we can find a pair $(s,t')$ in the min stack with $a<s$ and
$t<t'$ (a smaller stack at a later time).

We avoid spurious anti and output dependencies due to register
allocation by not computing dependencies on registers and compiling
the traced program without optimization. 
