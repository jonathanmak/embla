\begin{abstract}

With the widespread use of chip multithreaded (CMT) processors, there
is an urgent need for tools and methodologies supporting
parallelization of existing applications. Writing explicitly threaded
code is a challenging task, especially since such code is naturally
nondeterministic.

In this paper we present a novel tool supporting the process of
parallelizing programs by hand. The tool, Embla, allows the user to
profile the dependencies in a sequential program, thus finding
opportunities for parallelization. While most tools for this problem
rely on static analysis, Embla is a profiler that records dependencies
as they arise during program execution. It is thus exact in contrast
to static tool which by necessity make conservative
approximation. Also, since the tool deals with the machine code of the
program, it is completely language independent.

\begin{comment_env}
Suggestion for changes: 

With the proliferation of multicore processors, there is an urgent need for
tools and methodologies supporting parallelization of existing
applications.  In this paper we present a novel tool for aiding
programmers in parallelizing programs. The tool, Embla, is based on the
Valgrind framework, and allows the user to
discover the data dependencies in a sequential program, thereby exposing
opportunities for parallelization.   Embla performs a dynamic analysis,
and records dependencies as they
arise during program execution.  It reports an optimistic view of
parallelizable sequences, and ignores dependencies that do not arise during
execution.  In contrast to static analysis tools,
which by necessity make conservative approximation, Embla is able to find
more parallelism in sequential programs, and relies on the programmer to
transform the program in a correct manner. 

Moreover, since the tool instruments with the machine code of the program,
it is completely language independent. 
\end{comment_env}

\end{abstract}

