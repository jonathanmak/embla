\documentclass{acm_proc_article-sp}

\usepackage{epic}
\usepackage{color}
\usepackage{verbdef}
\usepackage{alltt}

\newcommand{\comment}[1]{\textit{[ #1 ]}}
\newenvironment{comment_env}
  {\begin{itshape}}
  {\end{itshape}}

\begin{document}

\title{Embla -- Data Dependence Profiling for Parallel Programming }
\author{Karl-Filip Fax\'en, Lars Albertsson, Konstantin Popov, Sverker Janson\\
       Swedish Institute of Computer Science\\
       Box 1263\\
       SE-164 29 Kista\\
       Sweden\\
       \{kff,lalle,kost,sverker\}@sics.se}
\date{}
\maketitle

\begin{abstract}

With the proliferation of multicore processors, there is an urgent need for
tools and methodologies supporting parallelization of existing
applications.  In this paper we present a novel tool for aiding
programmers in parallelizing programs. The tool, Embla, is based on the
Valgrind framework, and allows the user to
discover the data dependencies in a sequential program, thereby exposing
opportunities for parallelization.   Embla performs a dynamic analysis,
and records dependencies as they
arise during program execution.  It reports an optimistic view of
parallelizable sequences, and ignores dependencies that do not arise during
execution.  In contrast to static analysis tools,
which by necessity make conservative approximation, Embla is able to find
more parallelism in sequential programs, and relies on the programmer to
transform the program in a correct manner. 
Moreover, since the tool instruments the machine code of the program,
it is completely language independent. 

\end{abstract}


% -*- eval: (local-set-key "\M-q" 'undefined) -*-
%
% The line above will probably make Emacs ask if it is ok to evaluate the
% expression.  Answer y.

\section{Introduction}

As wire delays and thermal problems have made improvements in clock
frequency much more difficult to realize~\cite{KAB03}, and increased levels of
instruction level parallelism faces problems of design complexity and
control dependencies, improving microprocessor performance has come to
depend increasingly on thread level parallelism~\cite{Sutter05}. Multicore chips, also
known as CMPs (chip multiprocessors)~\cite{ONHWC96} pack several processor cores on a
single die, typically with per core L1 caches but sharing the L2
cache. Processor cores can also be multithreaded, executing
instructions from several threads simultaneously~\cite{TEL95}. Collectively 
these techniques can be referred to as {\em CMT} (chip
multithreading). Studies revealed a large amount of inherent parallelism
in typical applications, which is however difficult to utilize just by
superscalar/pipelined architectures~\cite{AS92}, thereby justifying CMT designs.

From the point of view of the software, a CMT processor looks just
like a traditional shared memory symmetric multiprocessor. In particular, a CMT
program consists of multiple threads with shared state. 
Unfortunately, such code is much
harder to write than single threaded code, both because there are many
more program states to consider and because the relative execution
speed of threads are, in general, nondeterministic.  Not only is it
difficult to write multithreaded programs; most existing code is
single threaded (with the exception of code for numeric computation
and server oriented code).

We think that a practical way of resolving this issue is through hand
parallelization of sequential code. This method would both apply to
the existing code base and to newly developed code. This paper
presents a tool, Embla, to assist in this process.


While the dependence information that Embla finds can be used to 
support different models of parallelism, 
Embla is initially aimed mainly at supporting procedure-level 
fork-join parallelism. To a first approximation, this means executing 
procedure calls asynchronously. Consider the program fragment below:
\begin{alltt}
   foo();
   bar();
   baz();
\end{alltt}
Suppose that {\tt foo()} and {\tt bar()} are {\em independent}, ie that 
executing {\tt bar()} before {\tt foo()} does not change the meaning of
the program\footnote{There are many ways to formally give meaning to a
program, although it has never been done for C.}. A sufficient condition 
for independence
in this case is that {\tt foo()} and {\tt bar()} do not do any I/O and
that neither call writes to a memory location that the other call 
accesses. In that case the call to {\tt foo()} can be executed
asynchronously, 
by a different thread, and in parallel with the call to {\tt bar()}.
Suppose further that {\tt foo()} and {\tt baz()} are not independent,
so {\tt baz()} can not be executed until the call {\tt foo()} is completed.
We can express this as 
\begin{alltt}
   spawn foo();
   bar();
   sync;
   baz();
\end{alltt}
where {\tt spawn} starts the call in parallel and {\tt sync} waits
for all {\tt spawn}'d acivities to terminate (here we have borrowed 
programming constructs from the Cilk programming language 
\cite{BJKLR96,frigo98implementation}).

This style of parallelism can be implemented efficiently (as is done
for instance in the Cilk language) and is easy to understand although
not as powerful as general thread parallelism with arbitrary
synchronization. In general, we can find more parallelism in a program 
by using the latter, but in many cases, fork-join paralleism suffices 
(especially for CMT processors which are not massively parallel).
In particular, as long as {\tt foo()} and {\tt bar()}
are independent, the behavior of the parallel program is identical to
the behavior of the sequential version. Therefore it is sufficient to
understand (debug, verify, \ldots) the sequential program; everything
except performance carries over to the parallel version.

The price for this approach is that one must find independent
procedure calls (program fragments, in general). Thus there must be
such calls in the code and the programmer must be able to make sure
that they are in fact independent. The first issue depends of course
on the algorithms used in the program.

The second issue is typically
dealt with using static analysis tools as is done in parallelizing
compilers. Such analysers are however very complex and must by
necessity be conservative (since they are static, they must use
approximations which are always safe). Good precision also comes at a
high computational cost, especially when analysing large programs.
Consequently, it has proved difficult to parallelize programs 
automatically in 
practice, and most production codes are written in an explicitly 
parallel way.

Embla takes a different approach by observing program execution
directly and recording the dependencies that occur. This means that
Embla in effect is a testing tool which should make it easy to 
integrate into existing software development processes.




\section{Using Embla}

\begin{figure} 
\small
\verbdef\jkffxb$#include <stdlib.h> $
\verbdef\jkffxc$#include <stdio.h> $
\verbdef\jkffxd$ $
\verbdef\jkffxe$static void inc(int *p) $
\verbdef\jkffxf${ $
\verbdef\jkffxg$   *p=*p+1; $
\verbdef\jkffxh$} $
\verbdef\jkffxi$ $
\verbdef\jkffxj$int main(int argc, char **argv) $
\verbdef\jkffxba${ $
\verbdef\jkffxbb$   int *q=NULL,n=0; $
\verbdef\jkffxbc$ $
\verbdef\jkffxbd$   q = (int*) malloc( sizeof(int) ); $
\verbdef\jkffxbe$   inc(q); $
\verbdef\jkffxbf$   inc(&n); $
\verbdef\jkffxbg$   inc(q); $
\verbdef\jkffxbh$   printf( "%d\n", *q+n ); $
\verbdef\jkffxbi$   q = (int*) malloc( sizeof(int) ); $
\verbdef\jkffxbj$   return q==NULL; $
\verbdef\jkffxca$} $
\hrulefill
\[
\begin{picture}(420,200)(-65,-200)

\put(-65,-10){\makebox(15,10)[r]{\it 1:}}%
\put(0,-10){\makebox(100,10)[l]{\jkffxb}}
\put(-65,-20){\makebox(15,10)[r]{\it 2:}}%
\put(0,-20){\makebox(100,10)[l]{\jkffxc}}
\put(-65,-30){\makebox(15,10)[r]{\it 3:}}%
\put(0,-30){\makebox(100,10)[l]{\jkffxd}}
\put(-65,-40){\makebox(15,10)[r]{\it 4:}}%
\put(0,-40){\makebox(100,10)[l]{\jkffxe}}
\put(-65,-50){\makebox(15,10)[r]{\it 5:}}%
\put(0,-50){\makebox(100,10)[l]{\jkffxf}}
{\color{black} \dottedline{3}(-10,-45)(0,-45)}
\put(-65,-60){\makebox(15,10)[r]{\it 6:}}%
\put(0,-60){\makebox(100,10)[l]{\jkffxg}}
\put(-65,-70){\makebox(15,10)[r]{\it 7:}}%
\put(0,-70){\makebox(100,10)[l]{\jkffxh}}
{\color{black} \dottedline{3}(-10,-65)(0,-65)}
\put(-65,-80){\makebox(15,10)[r]{\it 8:}}%
\put(0,-80){\makebox(100,10)[l]{\jkffxi}}
\put(-65,-90){\makebox(15,10)[r]{\it 9:}}%
\put(0,-90){\makebox(100,10)[l]{\jkffxj}}
\put(-65,-100){\makebox(15,10)[r]{\it 10:}}%
\put(0,-100){\makebox(100,10)[l]{\jkffxba}}
{\color{black} \dottedline{3}(-10,-95)(0,-95)}
\put(-65,-110){\makebox(15,10)[r]{\it 11:}}%
\put(0,-110){\makebox(100,10)[l]{\jkffxbb}}
{\color{black} \dottedline{3}(-10,-105)(0,-105)}
\put(-65,-120){\makebox(15,10)[r]{\it 12:}}%
\put(0,-120){\makebox(100,10)[l]{\jkffxbc}}
\put(-65,-130){\makebox(15,10)[r]{\it 13:}}%
\put(0,-130){\makebox(100,10)[l]{\jkffxbd}}
{\color{black} \dottedline{3}(-10,-125)(0,-125)}
\put(-65,-140){\makebox(15,10)[r]{\it 14:}}%
\put(0,-140){\makebox(100,10)[l]{\jkffxbe}}
{\color{black} \dottedline{3}(-10,-135)(0,-135)}
\put(-65,-150){\makebox(15,10)[r]{\it 15:}}%
\put(0,-150){\makebox(100,10)[l]{\jkffxbf}}
{\color{black} \dottedline{3}(-10,-145)(0,-145)}
\put(-65,-160){\makebox(15,10)[r]{\it 16:}}%
\put(0,-160){\makebox(100,10)[l]{\jkffxbg}}
{\color{black} \dottedline{3}(-10,-155)(0,-155)}
\put(-65,-170){\makebox(15,10)[r]{\it 17:}}%
\put(0,-170){\makebox(100,10)[l]{\jkffxbh}}
{\color{black} \dottedline{3}(-10,-165)(0,-165)}
\put(-65,-180){\makebox(15,10)[r]{\it 18:}}%
\put(0,-180){\makebox(100,10)[l]{\jkffxbi}}
{\color{black} \dottedline{3}(-10,-175)(0,-175)}
\put(-65,-190){\makebox(15,10)[r]{\it 19:}}%
\put(0,-190){\makebox(100,10)[l]{\jkffxbj}}
{\color{black} \dottedline{3}(-10,-185)(0,-185)}
\put(-65,-200){\makebox(15,10)[r]{\it 20:}}%
\put(0,-200){\makebox(100,10)[l]{\jkffxca}}
{\color{black} \dottedline{3}(-10,-195)(0,-195)}

\put(-40,-165){\color{black}\circle*{2}}
\put(-10,-45){\color{black}\circle*{2}}
\put(-30,-175){\color{black}\circle*{2}}
\put(-30,-185){\color{black}\circle*{2}}
\put(-10,-65){\color{black}\circle*{2}}
\put(-10,-95){\color{black}\circle*{2}}
\put(-40,-105){\color{black}\circle*{2}}
\put(-10,-195){\color{black}\circle*{2}}
\put(-30,-125){\color{black}\circle*{2}}
\put(-20,-135){\color{black}\circle*{2}}
\put(-40,-145){\color{black}\circle*{2}}
\put(-20,-155){\color{black}\circle*{2}}
{\color{red}\dashline[200]{3}(-10.0,-48.0)(-10.0,-62.0)}%
{\color{red}\dashline[200]{3}(-10.0,-62.0)(-9.0,-59.0)}%
{\color{red}\dashline[200]{3}(-10.0,-62.0)(-11.0,-59.0)}%
{\color{red}\dashline[200]{3}(-10.0,-98.0)(-10.0,-192.0)}%
{\color{red}\dashline[200]{3}(-10.0,-192.0)(-9.0,-189.0)}%
{\color{red}\dashline[200]{3}(-10.0,-192.0)(-11.0,-189.0)}%
{\color{blue}\dashline[200]{3}(-38.7,-107.7)(-31.3,-122.3)}%
{\color{blue}\dashline[200]{3}(-31.3,-122.3)(-31.8,-119.2)}%
{\color{blue}\dashline[200]{3}(-31.3,-122.3)(-33.6,-120.1)}%
{\color{red}\dashline[200]{3}(-40.0,-108.0)(-40.0,-142.0)}%
{\color{red}\dashline[200]{3}(-40.0,-142.0)(-39.0,-139.0)}%
{\color{red}\dashline[200]{3}(-40.0,-142.0)(-41.0,-139.0)}%
{\color{red}\dashline[200]{3}(-27.9,-127.1)(-22.1,-132.9)}%
{\color{red}\dashline[200]{3}(-22.1,-132.9)(-23.5,-130.1)}%
{\color{red}\dashline[200]{3}(-22.1,-132.9)(-24.9,-131.5)}%
{\color{red}\dashline[200]{3}(-29.1,-127.8)(-20.9,-152.2)}%
{\color{red}\dashline[200]{3}(-20.9,-152.2)(-20.9,-149.0)}%
{\color{red}\dashline[200]{3}(-20.9,-152.2)(-22.8,-149.6)}%
{\color{red}\dashline[200]{3}(-30.7,-127.9)(-39.3,-162.1)}%
{\color{red}\dashline[200]{3}(-39.3,-162.1)(-37.6,-159.4)}%
{\color{red}\dashline[200]{3}(-39.3,-162.1)(-39.5,-158.9)}%
{\color{cyan}\dashline[200]{3}(-30.0,-128.0)(-30.0,-172.0)}%
{\color{cyan}\dashline[200]{3}(-30.0,-172.0)(-29.0,-169.0)}%
{\color{cyan}\dashline[200]{3}(-30.0,-172.0)(-31.0,-169.0)}%
{\color{red}\dashline[200]{3}(-20.0,-138.0)(-20.0,-152.0)}%
{\color{red}\dashline[200]{3}(-20.0,-152.0)(-19.0,-149.0)}%
{\color{red}\dashline[200]{3}(-20.0,-152.0)(-21.0,-149.0)}%
{\color{green}\dashline[200]{3}(-20.7,-137.9)(-29.3,-172.1)}%
{\color{green}\dashline[200]{3}(-29.3,-172.1)(-27.6,-169.4)}%
{\color{green}\dashline[200]{3}(-29.3,-172.1)(-29.5,-168.9)}%
{\color{red}\dashline[200]{3}(-40.0,-148.0)(-40.0,-162.0)}%
{\color{red}\dashline[200]{3}(-40.0,-162.0)(-39.0,-159.0)}%
{\color{red}\dashline[200]{3}(-40.0,-162.0)(-41.0,-159.0)}%
{\color{red}\dashline[200]{3}(-22.7,-156.3)(-37.3,-163.7)}%
{\color{red}\dashline[200]{3}(-37.3,-163.7)(-34.2,-163.2)}%
{\color{red}\dashline[200]{3}(-37.3,-163.7)(-35.1,-161.4)}%
{\color{green}\dashline[200]{3}(-21.3,-157.7)(-28.7,-172.3)}%
{\color{green}\dashline[200]{3}(-28.7,-172.3)(-26.4,-170.1)}%
{\color{green}\dashline[200]{3}(-28.7,-172.3)(-28.2,-169.2)}%
{\color{green}\dashline[200]{3}(-37.9,-167.1)(-32.1,-172.9)}%
{\color{green}\dashline[200]{3}(-32.1,-172.9)(-33.5,-170.1)}%
{\color{green}\dashline[200]{3}(-32.1,-172.9)(-34.9,-171.5)}%
{\color{red}\dashline[200]{3}(-30.0,-178.0)(-30.0,-182.0)}%
{\color{red}\dashline[200]{3}(-30.0,-182.0)(-29.0,-179.0)}%
{\color{red}\dashline[200]{3}(-30.0,-182.0)(-31.0,-179.0)}%
\end{picture}
\]
\hrulefill

\caption{Example program with dependency graph} \label{ffirstex}
\end{figure}

To get a feeling for what dependency profiling is and what Embla can do, 
let us turn to the admittedly contrieved program in Figure~\ref{ffirstex}. 

\subsection{A first look: the dependency graph}

In Figure~\ref{ffirstex} we see, from 
left to right, line numbers, colored data dependency arrows and source 
lines. 

The presence of a data dependency arrow between a pair of lines indicates
that Embla has found one or more data dependencies between them
(Embla does not find control dependencies). A data dependence is a pair
of references, not both reads, to overlapping memory
locations with no interveaning write. We will refer to these
references as the {\em endpoints} of the dependence.
For instance, in the figure, 
there is a (red) arrow from line 13 to line 14 corresponding to
the assignment to {\tt q} (the {\em early} endpoint) followed by its use 
as an argument in {\tt inc(q)} (the {\em late} endpoint).

Depending on whether the two endpoints of a dependence
are reads or writes, data dependencies are typically divided into 
three classes:
\begin{description}
\item[Flow:]
A true data dependency where the location is first written then
read. Also known as {\em read after write} (RAW). Shown in 
{\bf \color{red} red} in the figure.
\item[Anti:]
A dependence caused by reuse of a location that is first read and then
written. Also known as {\em write after read} (WAR). Shown in 
{\bf \color{green} green} in the figure.
\item[Output:]
Similar to an anti dependence, but the second reference is also a
write. Also known as {\em write after write} (WAW). Shown in 
{\bf \color{blue} blue} in the figure.
\end{description}
If there are multiple dependencies of different class between the 
same pair of lines, the color selection is prioritized in the
order the dependencies are presented above. That is, if there is
some flow dependence, the arrow is red. Otherwise if there is 
some antidependence, the arrow is green, otherwise blue. We will 
come to the light blue arrows shortly.


We show dependencies in different colors since there are program
transformations that in some cases can be used to eliminate 
dependencies. Since we are discussing an approach to tool-supported
hand parallelisation, program transformations to remove dependencies
and increase parallelism are potentially useful. Hence Embla is 
designed to support them by tipping the user off as to which
dependencies might be removable.

Flow dependencies are unavoidable since they represent the data flow 
in the program, but anti and output dependencies can sometimes be 
removed by {\em renaming}. Consider the (green) arrows from line 
14, 16 and 17 to line 18 in the figure. These correspond to three
reads of {\tt q} followed by an assignment to {\tt q}. Here we 
are dealing with two different values which happen to be stored 
in the same variable. We could easily invent a new variable, e.g. 
{\tt r}, and replace {\tt q} by {\tt r} on line 18 and 19. This 
transformation would remove the dependencies from line 14, 16 and
17 to line 18. 

Similarly, an output dependence like the one from line 11 to line 13
(blue) can sometimes, and indeed in this case, be eliminated by 
removing the first write (the initialization of {\tt q} on line 11).
Renaming is of course also applicable to output dependencies.

For each of the dependency arrows in the figure that 
we have dicussed up to now, the endpoints have been part of the 
code for {\tt main}
itself. Embla also tracks references made in function calls. For
instance, there is a (red) flow dependence from line 14 to line 16
representing the write in the first invocation of {\tt inc} to the 
{\tt malloc}'d area pointed to by {\tt q} and the subsequent read 
of the same location by a later invocation of {\tt inc}. 
Note that these dependencies 
are reported as pertaining to {\tt main} rather than {\tt inc},
although the endpoints are part of the latter function. 
But the importance of the dependence is that, in {\tt main}, the calls
on line 14 and 16 can not be made in parallel.

Dependencies between different invocations of {\tt malloc} form 
an interesting special case.
Each call to {\tt malloc} manipulates (updates) 
administrative data structures like the free list. Embla will
report dependencies arising from these updates, 
effectively serializing all calls to 
{\tt malloc}, and thus in addition all calls to functions calling
{\tt malloc}. In reality, these calls need not be serialized since
program semantics typically does not depend on the exact addresses 
that data are allocated. It suffices that a thread safe 
implementation of {\tt malloc} is used.

Embla maintains a black list of functions that behave in this way.
In the examples in this paper, the black list consists of the 
{\tt malloc} family, but of course other functions can be included 
as appropriate. 

We give all dependencies where both endpoints are in blacklisted functions 
a {\bf \color{cyan}light blue} color, independent of whether they are 
flow, anti or output. This is because they will disappear when the
program is run in parallel.

An important thing to remember when interpreting the output of Embla 
is that although it takes the form of annotated source code, the 
profiling is performed on the executable machine code. Thus one must
keep the mappning of source code to machine code in mind. A case in 
point is the (red) flow dependence between the opening brace of a 
function and its closing brace (for instance from line 5 to line 7).
This dependence corresponds to pushing the frame pointer onto the 
stack as part of the function prologue and popping it in the 
epilogue.

\begin{figure}
\small
\hrulefill
\begin{alltt}
#include <stdlib.h>
#include <stdio.h>

static void inc(int *p)
\verb+{+
   *p=*p+1;
\verb+}+

int main(int argc, char **argv)
\verb+{+
   int *q=NULL,n=0;

   {\color{red}spawn} inc(&n);
   q = (int*) malloc( sizeof(int) );
   inc(q);
   inc(q);
   {\color{red}synch;}
   printf( "%d\verb+\+n", *q+n );
   q = (int*) malloc( sizeof(int) );
   return q==NULL;
\verb+}+
\end{alltt}
\hrulefill
\caption{The first example parallelized.}
\label{fparfirstex}
\end{figure}

What, then, does the graph in Figure~\ref{ffirstex} tell us about
the available parallelism in the program? It tells us that 
the call to {\tt inc} on line 15 is independent on lines 13, 14
and 16. Consequently, it can be parallelized, and we show the
result in Figure~\ref{fparfirstex} where we have again used the 
Cilk primitives {\tt spawn} and {\tt synch}.

\begin{figure} 
\small
\verbdef\jkffxb$#include <stdlib.h> $
\verbdef\jkffxc$#include <stdio.h> $
\verbdef\jkffxd$ $
\verbdef\jkffxe$static void inc(int *p) $
\verbdef\jkffxf${ $
\verbdef\jkffxg$   *p=*p+1; $
\verbdef\jkffxh$} $
\verbdef\jkffxi$ $
\verbdef\jkffxj$int main(int argc, char **argv) $
\verbdef\jkffxba${ $
\verbdef\jkffxbb$   int *q=NULL,n=0; $
\verbdef\jkffxbc$ $
\verbdef\jkffxbd$   q = (int*) malloc( sizeof(int) ); $
\verbdef\jkffxbe$   inc(q); $
\verbdef\jkffxbf$   inc(&n); $
\verbdef\jkffxbg$   inc(q); $
\verbdef\jkffxbh$   printf( "%d\n", *q+n ); $
\verbdef\jkffxbi$   q = (int*) malloc( sizeof(int) ); $
\verbdef\jkffxbj$   return q==NULL; $
\verbdef\jkffxca$} $
\hrulefill
\[
\begin{picture}(420,200)(-86,-200)

\put(-86,0){\makebox(15,10)[r]{\it 0:}}%
\put(0,0){\makebox(100,10)[l]{}}
\put(-86,-10){\makebox(15,10)[r]{\it 1:}}%
\put(0,-10){\makebox(100,10)[l]{\jkffxb}}
\put(-86,-20){\makebox(15,10)[r]{\it 2:}}%
\put(0,-20){\makebox(100,10)[l]{\jkffxc}}
\put(-86,-30){\makebox(15,10)[r]{\it 3:}}%
\put(0,-30){\makebox(100,10)[l]{\jkffxd}}
\put(-86,-40){\makebox(15,10)[r]{\it 4:}}%
\put(0,-40){\makebox(100,10)[l]{\jkffxe}}
\put(-86,-50){\makebox(15,10)[r]{\it 5:}}%
\put(0,-50){\makebox(100,10)[l]{\jkffxf}}
{\color{black} \dottedline{3}(-63,-50)(-5,-50)}\put(-86,-60){\makebox(15,10)[r]{\it 6:}}%
\put(0,-60){\makebox(100,10)[l]{\jkffxg}}
{\color{black} \dottedline{3}(-63,-60)(-5,-60)}\put(-86,-70){\makebox(15,10)[r]{\it 7:}}%
\put(0,-70){\makebox(100,10)[l]{\jkffxh}}
\put(-86,-80){\makebox(15,10)[r]{\it 8:}}%
\put(0,-80){\makebox(100,10)[l]{\jkffxi}}
\put(-86,-90){\makebox(15,10)[r]{\it 9:}}%
\put(0,-90){\makebox(100,10)[l]{\jkffxj}}
\put(-86,-100){\makebox(15,10)[r]{\it 10:}}%
\put(0,-100){\makebox(100,10)[l]{\jkffxba}}
{\color{black} \dottedline{3}(-63,-100)(-5,-100)}\put(-86,-110){\makebox(15,10)[r]{\it 11:}}%
\put(0,-110){\makebox(100,10)[l]{\jkffxbb}}
{\color{black} \dottedline{3}(-63,-110)(-5,-110)}\put(-86,-120){\makebox(15,10)[r]{\it 12:}}%
\put(0,-120){\makebox(100,10)[l]{\jkffxbc}}
{\color{black} \dottedline{3}(-63,-120)(-5,-120)}\put(-86,-130){\makebox(15,10)[r]{\it 13:}}%
\put(0,-130){\makebox(100,10)[l]{\jkffxbd}}
{\color{black} \dottedline{3}(-63,-130)(-5,-130)}\put(-86,-140){\makebox(15,10)[r]{\it 14:}}%
\put(0,-140){\makebox(100,10)[l]{\jkffxbe}}
{\color{black} \dottedline{3}(-63,-140)(-5,-140)}\put(-86,-150){\makebox(15,10)[r]{\it 15:}}%
\put(0,-150){\makebox(100,10)[l]{\jkffxbf}}
{\color{black} \dottedline{3}(-63,-150)(-5,-150)}\put(-86,-160){\makebox(15,10)[r]{\it 16:}}%
\put(0,-160){\makebox(100,10)[l]{\jkffxbg}}
{\color{black} \dottedline{3}(-63,-160)(-5,-160)}\put(-86,-170){\makebox(15,10)[r]{\it 17:}}%
\put(0,-170){\makebox(100,10)[l]{\jkffxbh}}
{\color{black} \dottedline{3}(-63,-170)(-5,-170)}\put(-86,-180){\makebox(15,10)[r]{\it 18:}}%
\put(0,-180){\makebox(100,10)[l]{\jkffxbi}}
{\color{black} \dottedline{3}(-63,-180)(-5,-180)}\put(-86,-190){\makebox(15,10)[r]{\it 19:}}%
\put(0,-190){\makebox(100,10)[l]{\jkffxbj}}
{\color{black} \dottedline{3}(-63,-190)(-5,-190)}\put(-86,-200){\makebox(15,10)[r]{\it 20:}}%
\put(0,-200){\makebox(100,10)[l]{\jkffxca}}

\color{red}
\put(-6,-95){\circle*{2}}
\put(-6,-97){\vector(0,-1){96}}
\put(-8,-97){\linethickness{0.7pt}\line(1,0){0}}
\put(-6,-195){\circle*{2}}

\color{blue}
\put(-12,-105){\circle*{2}}
\put(-12,-107){\vector(0,-1){16}}
\put(-14,-107){\linethickness{0.7pt}\line(1,0){0}}
\put(-12,-125){\circle*{2}}

\color{red}
\put(-18,-105){\circle*{2}}
\put(-18,-107){\vector(0,-1){36}}
\put(-20,-107){\linethickness{0.7pt}\line(1,0){0}}
\put(-18,-145){\circle*{2}\hspace{-2\unitlength}\circle{4}}

\color{red}
\put(-24,-122){\circle*{2}}
\put(-26,-124){\linethickness{0.7pt}\line(1,0){0}}
\put(-24,-127){\circle{4}}

\color{red}
\put(-30,-125){\circle*{2}}
\put(-30,-127){\vector(0,-1){6}}
\put(-32,-127){\linethickness{0.7pt}\line(1,0){0}}
\put(-30,-135){\circle*{2}}

\color{red}
\put(-36,-125){\circle*{2}}
\put(-36,-127){\vector(0,-1){26}}
\put(-38,-127){\linethickness{0.7pt}\line(1,0){0}}
\put(-36,-155){\circle*{2}}

\color{red}
\put(-42,-125){\circle*{2}}
\put(-42,-127){\vector(0,-1){36}}
\put(-44,-127){\linethickness{0.7pt}\line(1,0){0}}
\put(-42,-165){\circle*{2}}

\color{red}
\put(-48,-125){\circle{4}}
\put(-48,-127){\vector(0,-1){46}}
\put(-50,-127){\linethickness{0.7pt}\line(1,0){4}}
\put(-48,-175){\circle{4}}

\color{red}
\put(-12,-132){\circle*{2}}
\put(-14,-134){\linethickness{0.7pt}\line(1,0){0}}
\put(-12,-137){\circle*{2}\hspace{-2\unitlength}\circle{4}}

\color{green}
\put(-24,-135){\circle*{2}}
\put(-24,-137){\vector(0,-1){36}}
\put(-26,-137){\linethickness{0.7pt}\line(1,0){0}}
\put(-24,-175){\circle*{2}}

\color{red}
\put(-54,-135){\circle*{2}\hspace{-2\unitlength}\circle{4}}
\put(-54,-137){\vector(0,-1){16}}
\put(-56,-137){\linethickness{0.7pt}\line(1,0){4}}
\put(-54,-155){\circle*{2}\hspace{-2\unitlength}\circle{4}}

\color{red}
\put(-12,-142){\circle*{2}}
\put(-14,-144){\linethickness{0.7pt}\line(1,0){0}}
\put(-12,-147){\circle*{2}\hspace{-2\unitlength}\circle{4}}

\color{red}
\put(-30,-145){\circle*{2}\hspace{-2\unitlength}\circle{4}}
\put(-30,-147){\vector(0,-1){16}}
\put(-32,-147){\linethickness{0.7pt}\line(1,0){0}}
\put(-30,-165){\circle*{2}}

\color{red}
\put(-12,-152){\circle*{2}}
\put(-14,-154){\linethickness{0.7pt}\line(1,0){0}}
\put(-12,-157){\circle*{2}\hspace{-2\unitlength}\circle{4}}

\color{green}
\put(-18,-155){\circle*{2}}
\put(-18,-157){\vector(0,-1){16}}
\put(-20,-157){\linethickness{0.7pt}\line(1,0){0}}
\put(-18,-175){\circle*{2}}

\color{red}
\put(-60,-155){\circle*{2}\hspace{-2\unitlength}\circle{4}}
\put(-60,-157){\vector(0,-1){6}}
\put(-62,-157){\linethickness{0.7pt}\line(1,0){4}}
\put(-60,-165){\circle*{2}}

\color{red}
\put(-12,-162){\circle*{2}}
\put(-14,-164){\linethickness{0.7pt}\line(1,0){0}}
\put(-12,-167){\circle*{2}\hspace{-2\unitlength}\circle{4}}

\color{green}
\put(-36,-165){\circle*{2}}
\put(-36,-167){\vector(0,-1){6}}
\put(-38,-167){\linethickness{0.7pt}\line(1,0){0}}
\put(-36,-175){\circle*{2}}

\color{red}
\put(-12,-172){\circle*{2}}
\put(-14,-174){\linethickness{0.7pt}\line(1,0){0}}
\put(-12,-177){\circle{4}}

\color{red}
\put(-30,-175){\circle*{2}}
\put(-30,-177){\vector(0,-1){6}}
\put(-32,-177){\linethickness{0.7pt}\line(1,0){0}}
\put(-30,-185){\circle*{2}}

\color{red}
\put(-6,-45){\circle*{2}}
\put(-6,-47){\vector(0,-1){16}}
\put(-8,-47){\linethickness{0.7pt}\line(1,0){0}}
\put(-6,-65){\circle*{2}}

\color{green}
\put(-12,-52){\circle*{2}}
\put(-14,-54){\linethickness{0.7pt}\line(1,0){4}}
\put(-12,-57){\circle*{2}}

\color{green}
\put(-18,-52){\circle*{2}}
\put(-20,-54){\linethickness{0.7pt}\line(1,0){0}}
\put(-18,-57){\circle*{2}}

\end{picture}
\]
\hrulefill

\caption{First example program with dependency list.}
\label{ffirstexlist}
\end{figure}

\subsection{Digging deeper: the dependency list}

The dependency graph we have discussed above is an abstracted form of
the information provided by Embla. If the region of interest lacks
dependencies we are done; in order to motivate the transformation
shown in Figure~\ref{fparfirstex}, we only need the dependency
graph. But if Embla discovers more dependencies than we had hoped, we
might wish to see if we can transform the program to eliminate them.
For that task, it is useful to have a more detailed look.

Hence we now turn to Figure~\ref{ffirstexlist} where we see a
graphical representation of the individual dependencies found by
Embla. There are three major dimensions in which this plot gives more
details:

\paragraph*{All dependencies are shown} 

Rather than giving a single arrow if there is more than one kind of
dependence, every kind is shown individually. If there is both a flow
and an anti dependence between a pair of lines, there are two arrows
(one red and one green). There is however no example of this in the
figure.

Dependency lists also show {\em self dependencies}, where the source
and target of the dependence is on the same line. in that case, no
arrow is shown, only the two endpoint symbols. 

\paragraph*{Characterization of dependence endpoints}

% Each dependence corresponds to a pair of memory referencing 
% instruction executions, referred to as the {\em dependence 
% endpoints}. 
The dependence endpoints are not always part of the function
containing the dependence but may be part of a function called
(indirectly) from the line indicated by the dependency arrow. For instance, the
endpoints of the flow dependence from line 14 to line 16 in the
example are part of the function {\tt inc}. This observation gives
rise to the following three kinds of endpoints:
\begin{description}
\item[$\cdot\ $] 
A small dot represents a {\em direct dependency endpoint}, an 
instruction that is
generated from the source code at the indicated line. 
% Thus the 
% instruction is part of the function or procedure containing the 
% line.
\item[$\circ\hskip-0.35em\cdot\ $]
A dot in a circle represents an {\em indirect dependency endpoint},
an instruction that is part of a function (procedure) that was 
(transitively) invoked by a function (procedure) call at the indicated 
line.
\item[$\circ\ $]
An unfilled circle represents a {\em weak dependecy endpoint},
which is like an indirect endpoint but the function containing 
the instruction is on a blacklist of functions which, 
like {\tt malloc}, would not give rise to dependencies in the 
parallelized program.
\end{description}
A dependency can have endpoints of different kind. For instance,
function calls give rise to flow dependencies between the write of the
argument and its subsequent read in the called function (seen in the
second column from the right in Figure~\ref{ffirstexlist} for the
calls on lines 14 through 18.

\paragraph*{Data location}

Another dimension that the dependency list shows is where in memory
the location associated with the dependency is situated. Currently, we
distinguish references to the stack from other references but a more
fine-grained characterization is certainly possible.
\begin{description}
\item[$\downarrow\ $]
An unadorned arrow represents a
 {\em stack dependence} where the location has been a part of 
the stack from the first reference to the second (inclusive).
\item[$\bar{\downarrow}\ $]
An arrow with a bar across represents a
{\em heap dependence} where the location has never been part of the stack.
\end{description}
Dependencies where the data location has been part of the stack, then
not part of the stack, then part of the stack again are filtered away
and not shown, as discussed in the implementation section.

\begin{figure} 
\small
\verbdef\jkffxb$#include <stdlib.h> $
\verbdef\jkffxc$#include <stdio.h> $
\verbdef\jkffxd$ $
\verbdef\jkffxe$static int a[]  $
\verbdef\jkffxf$  = {17, 3, 84, 89, 4, 5, 23, 43,  $
\verbdef\jkffxg$     21, 7, 2, 1, 55, 63, 21}; $
\verbdef\jkffxh$static int n = 15; $
\verbdef\jkffxi$ $
\verbdef\jkffxj$static int *part(int *a, int n) $
\verbdef\jkffxba${ $
\verbdef\jkffxbb$   int i = a[0]; $
\verbdef\jkffxbc$   int k = a[n-1]; $
\verbdef\jkffxbd$   int *lp = a; $
\verbdef\jkffxbe$   int *hp = a+n-1; $
\verbdef\jkffxbf$ $
\verbdef\jkffxbg$   while( lp<hp ) { $
\verbdef\jkffxbh$      if( k<i ) { $
\verbdef\jkffxbi$         *lp = k; $
\verbdef\jkffxbj$          lp++; $
\verbdef\jkffxca$          k = *lp; $
\verbdef\jkffxcb$      } else { $
\verbdef\jkffxcc$          *hp = k; $
\verbdef\jkffxcd$           hp--; $
\verbdef\jkffxce$           k = *hp; $
\verbdef\jkffxcf$      } $
\verbdef\jkffxcg$   } $
\verbdef\jkffxch$   *lp = i; $
\verbdef\jkffxci$   return lp; $
\verbdef\jkffxcj$} $
\verbdef\jkffxda$ $
\verbdef\jkffxdb$static void qs(int *a, int n) $
\verbdef\jkffxdc${ $
\verbdef\jkffxdd$   if( n>1 ) { $
\verbdef\jkffxde$      int *lp = part( a, n ); $
\verbdef\jkffxdf$      int m = lp-a; $
\verbdef\jkffxdg$      qs( a, m ); $
\verbdef\jkffxdh$      qs( lp+1, n-m-1 ); $
\verbdef\jkffxdi$   } $
\verbdef\jkffxdj$} $
\verbdef\jkffxea$ $
\verbdef\jkffxeb$  $
\verbdef\jkffxec$ $
\verbdef\jkffxed$int main(int argc, char **argv) $
\verbdef\jkffxee${ $
\verbdef\jkffxef$   int i; $
\verbdef\jkffxeg$    $
\verbdef\jkffxeh$   qs( a, n ); $
\verbdef\jkffxei$   for( i=0; i<n; i++ ) { $
\verbdef\jkffxej$      printf( "%d ", a[i] ); $
\verbdef\jkffxfa$   } $
\verbdef\jkffxfb$   printf( "\n" ); $
\verbdef\jkffxfc$ $
\verbdef\jkffxfd$} $
\hrulefill
\[
\begin{picture}(420,530)(-80,-530)

\put(0,-10){\makebox(100,10)[l]{\jkffxb}}
\put(0,-20){\makebox(100,10)[l]{\jkffxc}}
\put(0,-30){\makebox(100,10)[l]{\jkffxd}}
\put(0,-40){\makebox(100,10)[l]{\jkffxe}}
\put(0,-50){\makebox(100,10)[l]{\jkffxf}}
\put(0,-60){\makebox(100,10)[l]{\jkffxg}}
\put(0,-70){\makebox(100,10)[l]{\jkffxh}}
\put(0,-80){\makebox(100,10)[l]{\jkffxi}}
{\color{black} \dottedline{3}(-10,-75)(-0,-75)}
\put(0,-90){\makebox(100,10)[l]{\jkffxj}}
{\color{black} \dottedline{3}(-10,-85)(-0,-85)}
\put(0,-100){\makebox(100,10)[l]{\jkffxba}}
{\color{black} \dottedline{3}(-10,-95)(-0,-95)}
\put(0,-110){\makebox(100,10)[l]{\jkffxbb}}
{\color{black} \dottedline{3}(-10,-105)(-0,-105)}
\put(0,-120){\makebox(100,10)[l]{\jkffxbc}}
{\color{black} \dottedline{3}(-10,-115)(-0,-115)}
\put(0,-130){\makebox(100,10)[l]{\jkffxbd}}
\put(0,-140){\makebox(100,10)[l]{\jkffxbe}}
{\color{black} \dottedline{3}(-10,-135)(-0,-135)}
\put(0,-150){\makebox(100,10)[l]{\jkffxbf}}
{\color{black} \dottedline{3}(-10,-145)(-0,-145)}
\put(0,-160){\makebox(100,10)[l]{\jkffxbg}}
{\color{black} \dottedline{3}(-10,-155)(-0,-155)}
\put(0,-170){\makebox(100,10)[l]{\jkffxbh}}
{\color{black} \dottedline{3}(-10,-165)(-0,-165)}
\put(0,-180){\makebox(100,10)[l]{\jkffxbi}}
{\color{black} \dottedline{3}(-10,-175)(-0,-175)}
\put(0,-190){\makebox(100,10)[l]{\jkffxbj}}
\put(0,-200){\makebox(100,10)[l]{\jkffxca}}
{\color{black} \dottedline{3}(-10,-195)(-0,-195)}
\put(0,-210){\makebox(100,10)[l]{\jkffxcb}}
{\color{black} \dottedline{3}(-10,-205)(-0,-205)}
\put(0,-220){\makebox(100,10)[l]{\jkffxcc}}
{\color{black} \dottedline{3}(-10,-215)(-0,-215)}
\put(0,-230){\makebox(100,10)[l]{\jkffxcd}}
\put(0,-240){\makebox(100,10)[l]{\jkffxce}}
\put(0,-250){\makebox(100,10)[l]{\jkffxcf}}
{\color{black} \dottedline{3}(-10,-245)(-0,-245)}
\put(0,-260){\makebox(100,10)[l]{\jkffxcg}}
{\color{black} \dottedline{3}(-10,-255)(-0,-255)}
\put(0,-270){\makebox(100,10)[l]{\jkffxch}}
{\color{black} \dottedline{3}(-10,-265)(-0,-265)}
\put(0,-280){\makebox(100,10)[l]{\jkffxci}}
\put(0,-290){\makebox(100,10)[l]{\jkffxcj}}
\put(0,-300){\makebox(100,10)[l]{\jkffxda}}
{\color{black} \dottedline{3}(-10,-295)(-0,-295)}
\put(0,-310){\makebox(100,10)[l]{\jkffxdb}}
\put(0,-320){\makebox(100,10)[l]{\jkffxdc}}
{\color{black} \dottedline{3}(-10,-315)(-0,-315)}
\put(0,-330){\makebox(100,10)[l]{\jkffxdd}}
{\color{black} \dottedline{3}(-10,-325)(-0,-325)}
\put(0,-340){\makebox(100,10)[l]{\jkffxde}}
{\color{black} \dottedline{3}(-10,-335)(-0,-335)}
\put(0,-350){\makebox(100,10)[l]{\jkffxdf}}
{\color{black} \dottedline{3}(-10,-345)(-0,-345)}
\put(0,-360){\makebox(100,10)[l]{\jkffxdg}}
\put(0,-370){\makebox(100,10)[l]{\jkffxdh}}
{\color{black} \dottedline{3}(-10,-365)(-0,-365)}
\put(0,-380){\makebox(100,10)[l]{\jkffxdi}}
\put(0,-390){\makebox(100,10)[l]{\jkffxdj}}
\put(0,-400){\makebox(100,10)[l]{\jkffxea}}
\put(0,-410){\makebox(100,10)[l]{\jkffxeb}}
\put(0,-420){\makebox(100,10)[l]{\jkffxec}}
{\color{black} \dottedline{3}(-10,-415)(-0,-415)}
\put(0,-430){\makebox(100,10)[l]{\jkffxed}}
\put(0,-440){\makebox(100,10)[l]{\jkffxee}}
\put(0,-450){\makebox(100,10)[l]{\jkffxef}}
{\color{black} \dottedline{3}(-10,-445)(-0,-445)}
\put(0,-460){\makebox(100,10)[l]{\jkffxeg}}
{\color{black} \dottedline{3}(-10,-455)(-0,-455)}
\put(0,-470){\makebox(100,10)[l]{\jkffxeh}}
{\color{black} \dottedline{3}(-10,-465)(-0,-465)}
\put(0,-480){\makebox(100,10)[l]{\jkffxei}}
\put(0,-490){\makebox(100,10)[l]{\jkffxej}}
{\color{black} \dottedline{3}(-10,-485)(-0,-485)}
\put(0,-500){\makebox(100,10)[l]{\jkffxfa}}
\put(0,-510){\makebox(100,10)[l]{\jkffxfb}}
{\color{black} \dottedline{3}(-10,-505)(-0,-505)}
\put(0,-520){\makebox(100,10)[l]{\jkffxfc}}
\put(0,-530){\makebox(100,10)[l]{\jkffxfd}}

\put(-30,-255){\color{black}\circle*{2}}
\put(-20,-165){\color{black}\circle*{2}}
\put(-30,-345){\color{black}\circle*{2}}
\put(-30,-445){\color{black}\circle*{2}}
\put(-80,-175){\color{black}\circle*{2}}
\put(-10,-265){\color{black}\circle*{2}}
\put(-20,-455){\color{black}\circle*{2}}
\put(-10,-365){\color{black}\circle*{2}}
\put(-30,-465){\color{black}\circle*{2}}
\put(-10,-75){\color{black}\circle*{2}}
\put(-30,-485){\color{black}\circle*{2}}
\put(-70,-85){\color{black}\circle*{2}}
\put(-60,-95){\color{black}\circle*{2}}
\put(-40,-105){\color{black}\circle*{2}}
\put(-60,-195){\color{black}\circle*{2}}
\put(-50,-205){\color{black}\circle*{2}}
\put(-50,-115){\color{black}\circle*{2}}
\put(-10,-295){\color{black}\circle*{2}}
\put(-60,-215){\color{black}\circle*{2}}
\put(-40,-135){\color{black}\circle*{2}}
\put(-30,-315){\color{black}\circle*{2}}
\put(-10,-505){\color{black}\circle*{2}}
\put(-10,-415){\color{black}\circle*{2}}
\put(-80,-145){\color{black}\circle*{2}}
\put(-20,-325){\color{black}\circle*{2}}
\put(-70,-245){\color{black}\circle*{2}}
\put(-20,-155){\color{black}\circle*{2}}
\put(-20,-335){\color{black}\circle*{2}}
{\color{red}\dashline[200]{3}(-10.0,-77.0)(-10.0,-263.0)}%
{\color{red}\dashline[200]{3}(-10.0,-263.0)(-9.0,-261.0)}%
{\color{red}\dashline[200]{3}(-10.0,-263.0)(-11.0,-261.0)}%
{\color{red}\dashline[200]{3}(-70.3,-87.0)(-79.7,-143.0)}%
{\color{red}\dashline[200]{3}(-79.7,-143.0)(-78.4,-141.2)}%
{\color{red}\dashline[200]{3}(-79.7,-143.0)(-80.3,-140.9)}%
{\color{green}\dashline[200]{3}(-68.8,-86.6)(-21.2,-153.4)}%
{\color{green}\dashline[200]{3}(-21.2,-153.4)(-21.5,-151.2)}%
{\color{green}\dashline[200]{3}(-21.2,-153.4)(-23.1,-152.3)}%
{\color{red}\dashline[200]{3}(-70.0,-87.0)(-70.0,-243.0)}%
{\color{red}\dashline[200]{3}(-70.0,-243.0)(-69.0,-241.0)}%
{\color{red}\dashline[200]{3}(-70.0,-243.0)(-71.0,-241.0)}%
{\color{red}\dashline[200]{3}(-60.7,-96.9)(-79.3,-143.1)}%
{\color{red}\dashline[200]{3}(-79.3,-143.1)(-77.6,-141.7)}%
{\color{red}\dashline[200]{3}(-79.3,-143.1)(-79.4,-140.9)}%
{\color{red}\dashline[200]{3}(-58.9,-96.7)(-21.1,-153.3)}%
{\color{red}\dashline[200]{3}(-21.1,-153.3)(-21.4,-151.1)}%
{\color{red}\dashline[200]{3}(-21.1,-153.3)(-23.1,-152.2)}%
{\color{red}\dashline[200]{3}(-60.0,-97.0)(-60.0,-193.0)}%
{\color{red}\dashline[200]{3}(-60.0,-193.0)(-59.0,-191.0)}%
{\color{red}\dashline[200]{3}(-60.0,-193.0)(-61.0,-191.0)}%
{\color{green}\dashline[200]{3}(-60.1,-97.0)(-69.9,-243.0)}%
{\color{green}\dashline[200]{3}(-69.9,-243.0)(-68.7,-241.1)}%
{\color{green}\dashline[200]{3}(-69.9,-243.0)(-70.7,-240.9)}%
{\color{red}\dashline[200]{3}(-40.0,-107.0)(-40.0,-133.0)}%
{\color{red}\dashline[200]{3}(-40.0,-133.0)(-39.0,-131.0)}%
{\color{red}\dashline[200]{3}(-40.0,-133.0)(-41.0,-131.0)}%
{\color{red}\dashline[200]{3}(-39.3,-106.9)(-20.7,-153.1)}%
{\color{red}\dashline[200]{3}(-20.7,-153.1)(-20.6,-150.9)}%
{\color{red}\dashline[200]{3}(-20.7,-153.1)(-22.4,-151.7)}%
{\color{red}\dashline[200]{3}(-39.4,-106.9)(-20.6,-163.1)}%
{\color{red}\dashline[200]{3}(-20.6,-163.1)(-20.3,-160.9)}%
{\color{red}\dashline[200]{3}(-20.6,-163.1)(-22.2,-161.5)}%
{\color{red}\dashline[200]{3}(-40.4,-107.0)(-69.6,-243.0)}%
{\color{red}\dashline[200]{3}(-69.6,-243.0)(-68.2,-241.3)}%
{\color{red}\dashline[200]{3}(-69.6,-243.0)(-70.1,-240.9)}%
{\color{red}\dashline[200]{3}(-39.9,-107.0)(-30.1,-253.0)}%
{\color{red}\dashline[200]{3}(-30.1,-253.0)(-29.3,-250.9)}%
{\color{red}\dashline[200]{3}(-30.1,-253.0)(-31.3,-251.1)}%
{\color{red}\dashline[200]{3}(-49.1,-116.8)(-40.9,-133.2)}%
{\color{red}\dashline[200]{3}(-40.9,-133.2)(-40.9,-131.0)}%
{\color{red}\dashline[200]{3}(-40.9,-133.2)(-42.7,-131.9)}%
{\color{red}\dashline[200]{3}(-50.2,-117.0)(-59.8,-193.0)}%
{\color{red}\dashline[200]{3}(-59.8,-193.0)(-58.5,-191.2)}%
{\color{red}\dashline[200]{3}(-59.8,-193.0)(-60.5,-190.9)}%
{\color{red}\dashline[200]{3}(-50.0,-117.0)(-50.0,-203.0)}%
{\color{red}\dashline[200]{3}(-50.0,-203.0)(-49.0,-201.0)}%
{\color{red}\dashline[200]{3}(-50.0,-203.0)(-51.0,-201.0)}%
{\color{green}\dashline[200]{3}(-38.1,-136.1)(-20.3,-162.8)}%
{\color{green}\dashline[200]{3}(-20.3,-162.8)(-20.6,-160.6)}%
{\color{green}\dashline[200]{3}(-20.3,-162.8)(-22.2,-161.7)}%
{\color{green}\dashline[200]{3}(-39.3,-137.1)(-48.7,-203.2)}%
{\color{green}\dashline[200]{3}(-48.7,-203.2)(-47.5,-201.3)}%
{\color{green}\dashline[200]{3}(-48.7,-203.2)(-49.4,-201.0)}%
{\color{green}\dashline[200]{3}(-79.0,-147.0)(-79.0,-173.0)}%
{\color{green}\dashline[200]{3}(-79.0,-173.0)(-78.0,-171.0)}%
{\color{green}\dashline[200]{3}(-79.0,-173.0)(-80.0,-171.0)}%
{\color{green}\dashline[200]{3}(-78.5,-146.6)(-59.6,-212.8)}%
{\color{green}\dashline[200]{3}(-59.6,-212.8)(-59.2,-210.6)}%
{\color{green}\dashline[200]{3}(-59.6,-212.8)(-61.1,-211.2)}%
{\color{green}\dashline[200]{3}(-19.0,-157.0)(-19.0,-163.0)}%
{\color{green}\dashline[200]{3}(-19.0,-163.0)(-18.0,-161.0)}%
{\color{green}\dashline[200]{3}(-19.0,-163.0)(-20.0,-161.0)}%
{\color{green}\dashline[200]{3}(-21.6,-156.6)(-77.8,-175.3)}%
{\color{green}\dashline[200]{3}(-77.8,-175.3)(-75.6,-175.6)}%
{\color{green}\dashline[200]{3}(-77.8,-175.3)(-76.2,-173.7)}%
{\color{red}\dashline[200]{3}(-21.9,-163.9)(-39.7,-137.2)}%
{\color{red}\dashline[200]{3}(-39.7,-137.2)(-39.4,-139.4)}%
{\color{red}\dashline[200]{3}(-39.7,-137.2)(-37.8,-138.3)}%
{\color{red}\dashline[200]{3}(-21.0,-163.0)(-21.0,-157.0)}%
{\color{red}\dashline[200]{3}(-21.0,-157.0)(-22.0,-159.0)}%
{\color{red}\dashline[200]{3}(-21.0,-157.0)(-20.0,-159.0)}%
{\color{red}\dashline[200]{3}(-21.8,-166.3)(-77.9,-175.7)}%
{\color{red}\dashline[200]{3}(-77.9,-175.7)(-75.7,-176.3)}%
{\color{red}\dashline[200]{3}(-77.9,-175.7)(-76.1,-174.3)}%
{\color{red}\dashline[200]{3}(-21.1,-166.7)(-68.9,-243.3)}%
{\color{red}\dashline[200]{3}(-68.9,-243.3)(-67.0,-242.1)}%
{\color{red}\dashline[200]{3}(-68.9,-243.3)(-68.7,-241.1)}%
{\color{red}\dashline[200]{3}(-20.2,-167.0)(-29.8,-253.0)}%
{\color{red}\dashline[200]{3}(-29.8,-253.0)(-28.6,-251.1)}%
{\color{red}\dashline[200]{3}(-29.8,-253.0)(-30.6,-250.9)}%
{\color{red}\dashline[200]{3}(-81.0,-173.0)(-81.0,-147.0)}%
{\color{red}\dashline[200]{3}(-81.0,-147.0)(-82.0,-149.0)}%
{\color{red}\dashline[200]{3}(-81.0,-147.0)(-80.0,-149.0)}%
{\color{red}\dashline[200]{3}(-78.4,-173.4)(-22.2,-154.7)}%
{\color{red}\dashline[200]{3}(-22.2,-154.7)(-24.4,-154.4)}%
{\color{red}\dashline[200]{3}(-22.2,-154.7)(-23.8,-156.3)}%
{\color{green}\dashline[200]{3}(-78.2,-173.7)(-22.1,-164.3)}%
{\color{green}\dashline[200]{3}(-22.1,-164.3)(-24.3,-163.7)}%
{\color{green}\dashline[200]{3}(-22.1,-164.3)(-23.9,-165.7)}%
{\color{red}\dashline[200]{3}(-78.6,-176.4)(-61.4,-193.6)}%
{\color{red}\dashline[200]{3}(-61.4,-193.6)(-62.1,-191.5)}%
{\color{red}\dashline[200]{3}(-61.4,-193.6)(-63.5,-192.9)}%
{\color{green}\dashline[200]{3}(-79.7,-177.0)(-70.3,-243.0)}%
{\color{green}\dashline[200]{3}(-70.3,-243.0)(-69.6,-240.9)}%
{\color{green}\dashline[200]{3}(-70.3,-243.0)(-71.6,-241.2)}%
{\color{green}\dashline[200]{3}(-57.9,-195.7)(-50.7,-202.9)}%
{\color{green}\dashline[200]{3}(-50.7,-202.9)(-51.4,-200.8)}%
{\color{green}\dashline[200]{3}(-50.7,-202.9)(-52.8,-202.2)}%
{\color{green}\dashline[200]{3}(-59.0,-197.0)(-59.0,-213.0)}%
{\color{green}\dashline[200]{3}(-59.0,-213.0)(-58.0,-211.0)}%
{\color{green}\dashline[200]{3}(-59.0,-213.0)(-60.0,-211.0)}%
{\color{red}\dashline[200]{3}(-50.7,-202.9)(-41.3,-136.8)}%
{\color{red}\dashline[200]{3}(-41.3,-136.8)(-42.5,-138.7)}%
{\color{red}\dashline[200]{3}(-41.3,-136.8)(-40.6,-139.0)}%
{\color{red}\dashline[200]{3}(-52.1,-204.3)(-59.3,-197.1)}%
{\color{red}\dashline[200]{3}(-59.3,-197.1)(-58.6,-199.2)}%
{\color{red}\dashline[200]{3}(-59.3,-197.1)(-57.2,-197.8)}%
{\color{red}\dashline[200]{3}(-50.7,-207.1)(-57.9,-214.3)}%
{\color{red}\dashline[200]{3}(-57.9,-214.3)(-55.8,-213.6)}%
{\color{red}\dashline[200]{3}(-57.9,-214.3)(-57.2,-212.2)}%
{\color{red}\dashline[200]{3}(-61.5,-213.4)(-80.4,-147.2)}%
{\color{red}\dashline[200]{3}(-80.4,-147.2)(-80.8,-149.4)}%
{\color{red}\dashline[200]{3}(-80.4,-147.2)(-78.9,-148.8)}%
{\color{red}\dashline[200]{3}(-58.9,-213.3)(-21.1,-156.7)}%
{\color{red}\dashline[200]{3}(-21.1,-156.7)(-23.1,-157.8)}%
{\color{red}\dashline[200]{3}(-21.1,-156.7)(-21.4,-158.9)}%
{\color{red}\dashline[200]{3}(-61.0,-213.0)(-61.0,-197.0)}%
{\color{red}\dashline[200]{3}(-61.0,-197.0)(-62.0,-199.0)}%
{\color{red}\dashline[200]{3}(-61.0,-197.0)(-60.0,-199.0)}%
{\color{green}\dashline[200]{3}(-59.3,-212.9)(-52.1,-205.7)}%
{\color{green}\dashline[200]{3}(-52.1,-205.7)(-54.2,-206.4)}%
{\color{green}\dashline[200]{3}(-52.1,-205.7)(-52.8,-207.8)}%
{\color{green}\dashline[200]{3}(-60.6,-216.9)(-69.4,-243.1)}%
{\color{green}\dashline[200]{3}(-69.4,-243.1)(-67.8,-241.5)}%
{\color{green}\dashline[200]{3}(-69.4,-243.1)(-69.7,-240.9)}%
{\color{red}\dashline[200]{3}(-10.0,-297.0)(-10.0,-363.0)}%
{\color{red}\dashline[200]{3}(-10.0,-363.0)(-9.0,-361.0)}%
{\color{red}\dashline[200]{3}(-10.0,-363.0)(-11.0,-361.0)}%
{\color{red}\dashline[200]{3}(-28.6,-316.4)(-21.4,-323.6)}%
{\color{red}\dashline[200]{3}(-21.4,-323.6)(-22.1,-321.5)}%
{\color{red}\dashline[200]{3}(-21.4,-323.6)(-23.5,-322.9)}%
{\color{red}\dashline[200]{3}(-29.1,-316.8)(-20.9,-333.2)}%
{\color{red}\dashline[200]{3}(-20.9,-333.2)(-20.9,-331.0)}%
{\color{red}\dashline[200]{3}(-20.9,-333.2)(-22.7,-331.9)}%
{\color{red}\dashline[200]{3}(-30.0,-317.0)(-30.0,-343.0)}%
{\color{red}\dashline[200]{3}(-30.0,-343.0)(-29.0,-341.0)}%
{\color{red}\dashline[200]{3}(-30.0,-343.0)(-31.0,-341.0)}%
{\color{red}\dashline[200]{3}(-20.0,-327.0)(-20.0,-333.0)}%
{\color{red}\dashline[200]{3}(-20.0,-333.0)(-19.0,-331.0)}%
{\color{red}\dashline[200]{3}(-20.0,-333.0)(-21.0,-331.0)}%
{\color{red}\dashline[200]{3}(-20.9,-326.8)(-29.1,-343.2)}%
{\color{red}\dashline[200]{3}(-29.1,-343.2)(-27.3,-341.9)}%
{\color{red}\dashline[200]{3}(-29.1,-343.2)(-29.1,-341.0)}%
{\color{red}\dashline[200]{3}(-10.0,-417.0)(-10.0,-503.0)}%
{\color{red}\dashline[200]{3}(-10.0,-503.0)(-9.0,-501.0)}%
{\color{red}\dashline[200]{3}(-10.0,-503.0)(-11.0,-501.0)}%
{\color{red}\dashline[200]{3}(-30.0,-447.0)(-30.0,-463.0)}%
{\color{red}\dashline[200]{3}(-30.0,-463.0)(-29.0,-461.0)}%
{\color{red}\dashline[200]{3}(-30.0,-463.0)(-31.0,-461.0)}%
{\color{red}\dashline[200]{3}(-20.7,-457.1)(-27.9,-464.3)}%
{\color{red}\dashline[200]{3}(-27.9,-464.3)(-25.8,-463.6)}%
{\color{red}\dashline[200]{3}(-27.9,-464.3)(-27.2,-462.2)}%
{\color{green}\dashline[200]{3}(-29.3,-462.9)(-22.1,-455.7)}%
{\color{green}\dashline[200]{3}(-22.1,-455.7)(-24.2,-456.4)}%
{\color{green}\dashline[200]{3}(-22.1,-455.7)(-22.8,-457.8)}%
{\color{red}\dashline[200]{3}(-30.0,-467.0)(-30.0,-483.0)}%
{\color{red}\dashline[200]{3}(-30.0,-483.0)(-29.0,-481.0)}%
{\color{red}\dashline[200]{3}(-30.0,-483.0)(-31.0,-481.0)}%
\end{picture}
\]
\hrulefill

\caption{The dependency graph of quicksort}
\label{fquickg}
\end{figure}

\begin{figure} 
\small
\verbdef\jkffxb$#include <stdlib.h> $
\verbdef\jkffxc$#include <stdio.h> $
\verbdef\jkffxd$ $
\verbdef\jkffxe$static int a[]  $
\verbdef\jkffxf$  = {17, 3, 84, 89, 4, 5, 23, 43,  $
\verbdef\jkffxg$     21, 7, 2, 1, 55, 63, 21}; $
\verbdef\jkffxh$static int n = 15; $
\verbdef\jkffxi$ $
\verbdef\jkffxj$static int *part(int *a, int n) $
\verbdef\jkffxba${ $
\verbdef\jkffxbb$   int i = a[0]; $
\verbdef\jkffxbc$   int k = a[n-1]; $
\verbdef\jkffxbd$   int *lp = a; $
\verbdef\jkffxbe$   int *hp = a+n-1; $
\verbdef\jkffxbf$ $
\verbdef\jkffxbg$   while( lp<hp ) { $
\verbdef\jkffxbh$      if( k<i ) { $
\verbdef\jkffxbi$         *lp = k; $
\verbdef\jkffxbj$          lp++; $
\verbdef\jkffxca$          k = *lp; $
\verbdef\jkffxcb$      } else { $
\verbdef\jkffxcc$          *hp = k; $
\verbdef\jkffxcd$           hp--; $
\verbdef\jkffxce$           k = *hp; $
\verbdef\jkffxcf$      } $
\verbdef\jkffxcg$   } $
\verbdef\jkffxch$   *lp = i; $
\verbdef\jkffxci$   return lp; $
\verbdef\jkffxcj$} $
\verbdef\jkffxda$ $
\verbdef\jkffxdb$static void qs(int *a, int n) $
\verbdef\jkffxdc${ $
\verbdef\jkffxdd$   if( n>1 ) { $
\verbdef\jkffxde$      int *lp = part( a, n ); $
\verbdef\jkffxdf$      int m = lp-a; $
\verbdef\jkffxdg$      qs( a, m ); $
\verbdef\jkffxdh$      qs( lp+1, n-m-1 ); $
\verbdef\jkffxdi$   } $
\verbdef\jkffxdj$} $
\verbdef\jkffxea$ $
\verbdef\jkffxeb$  $
\verbdef\jkffxec$ $
\verbdef\jkffxed$int main(int argc, char **argv) $
\verbdef\jkffxee${ $
\verbdef\jkffxef$   int i; $
\verbdef\jkffxeg$    $
\verbdef\jkffxeh$   qs( a, n ); $
\verbdef\jkffxei$   for( i=0; i<n; i++ ) { $
\verbdef\jkffxej$      printf( "%d ", a[i] ); $
\verbdef\jkffxfa$   } $
\verbdef\jkffxfb$   printf( "\n" ); $
\verbdef\jkffxfc$ $
\verbdef\jkffxfd$} $
\hrulefill
\[
\begin{picture}(420,530)(-192,-530)

\put(0,0){\makebox(100,10)[l]{}}
\put(0,-10){\makebox(100,10)[l]{\jkffxb}}
\put(0,-20){\makebox(100,10)[l]{\jkffxc}}
\put(0,-30){\makebox(100,10)[l]{\jkffxd}}
\put(0,-40){\makebox(100,10)[l]{\jkffxe}}
\put(0,-50){\makebox(100,10)[l]{\jkffxf}}
\put(0,-60){\makebox(100,10)[l]{\jkffxg}}
\put(0,-70){\makebox(100,10)[l]{\jkffxh}}
\put(0,-80){\makebox(100,10)[l]{\jkffxi}}
{\color{black} \dottedline{3}(-189,-80)(-5,-80)}\put(0,-90){\makebox(100,10)[l]{\jkffxj}}
{\color{black} \dottedline{3}(-189,-90)(-5,-90)}\put(0,-100){\makebox(100,10)[l]{\jkffxba}}
{\color{black} \dottedline{3}(-189,-100)(-5,-100)}\put(0,-110){\makebox(100,10)[l]{\jkffxbb}}
{\color{black} \dottedline{3}(-189,-110)(-5,-110)}\put(0,-120){\makebox(100,10)[l]{\jkffxbc}}
{\color{black} \dottedline{3}(-189,-120)(-5,-120)}\put(0,-130){\makebox(100,10)[l]{\jkffxbd}}
{\color{black} \dottedline{3}(-189,-130)(-5,-130)}\put(0,-140){\makebox(100,10)[l]{\jkffxbe}}
{\color{black} \dottedline{3}(-189,-140)(-5,-140)}\put(0,-150){\makebox(100,10)[l]{\jkffxbf}}
{\color{black} \dottedline{3}(-189,-150)(-5,-150)}\put(0,-160){\makebox(100,10)[l]{\jkffxbg}}
{\color{black} \dottedline{3}(-189,-160)(-5,-160)}\put(0,-170){\makebox(100,10)[l]{\jkffxbh}}
{\color{black} \dottedline{3}(-189,-170)(-5,-170)}\put(0,-180){\makebox(100,10)[l]{\jkffxbi}}
{\color{black} \dottedline{3}(-189,-180)(-5,-180)}\put(0,-190){\makebox(100,10)[l]{\jkffxbj}}
{\color{black} \dottedline{3}(-189,-190)(-5,-190)}\put(0,-200){\makebox(100,10)[l]{\jkffxca}}
{\color{black} \dottedline{3}(-189,-200)(-5,-200)}\put(0,-210){\makebox(100,10)[l]{\jkffxcb}}
{\color{black} \dottedline{3}(-189,-210)(-5,-210)}\put(0,-220){\makebox(100,10)[l]{\jkffxcc}}
{\color{black} \dottedline{3}(-189,-220)(-5,-220)}\put(0,-230){\makebox(100,10)[l]{\jkffxcd}}
{\color{black} \dottedline{3}(-189,-230)(-5,-230)}\put(0,-240){\makebox(100,10)[l]{\jkffxce}}
{\color{black} \dottedline{3}(-189,-240)(-5,-240)}\put(0,-250){\makebox(100,10)[l]{\jkffxcf}}
{\color{black} \dottedline{3}(-189,-250)(-5,-250)}\put(0,-260){\makebox(100,10)[l]{\jkffxcg}}
{\color{black} \dottedline{3}(-189,-260)(-5,-260)}\put(0,-270){\makebox(100,10)[l]{\jkffxch}}
\put(0,-280){\makebox(100,10)[l]{\jkffxci}}
\put(0,-290){\makebox(100,10)[l]{\jkffxcj}}
\put(0,-300){\makebox(100,10)[l]{\jkffxda}}
{\color{black} \dottedline{3}(-189,-300)(-5,-300)}\put(0,-310){\makebox(100,10)[l]{\jkffxdb}}
{\color{black} \dottedline{3}(-189,-310)(-5,-310)}\put(0,-320){\makebox(100,10)[l]{\jkffxdc}}
{\color{black} \dottedline{3}(-189,-320)(-5,-320)}\put(0,-330){\makebox(100,10)[l]{\jkffxdd}}
{\color{black} \dottedline{3}(-189,-330)(-5,-330)}\put(0,-340){\makebox(100,10)[l]{\jkffxde}}
{\color{black} \dottedline{3}(-189,-340)(-5,-340)}\put(0,-350){\makebox(100,10)[l]{\jkffxdf}}
{\color{black} \dottedline{3}(-189,-350)(-5,-350)}\put(0,-360){\makebox(100,10)[l]{\jkffxdg}}
{\color{black} \dottedline{3}(-189,-360)(-5,-360)}\put(0,-370){\makebox(100,10)[l]{\jkffxdh}}
\put(0,-380){\makebox(100,10)[l]{\jkffxdi}}
\put(0,-390){\makebox(100,10)[l]{\jkffxdj}}
\put(0,-400){\makebox(100,10)[l]{\jkffxea}}
\put(0,-410){\makebox(100,10)[l]{\jkffxeb}}
\put(0,-420){\makebox(100,10)[l]{\jkffxec}}
{\color{black} \dottedline{3}(-189,-420)(-5,-420)}\put(0,-430){\makebox(100,10)[l]{\jkffxed}}
{\color{black} \dottedline{3}(-189,-430)(-5,-430)}\put(0,-440){\makebox(100,10)[l]{\jkffxee}}
{\color{black} \dottedline{3}(-189,-440)(-5,-440)}\put(0,-450){\makebox(100,10)[l]{\jkffxef}}
{\color{black} \dottedline{3}(-189,-450)(-5,-450)}\put(0,-460){\makebox(100,10)[l]{\jkffxeg}}
{\color{black} \dottedline{3}(-189,-460)(-5,-460)}\put(0,-470){\makebox(100,10)[l]{\jkffxeh}}
{\color{black} \dottedline{3}(-189,-470)(-5,-470)}\put(0,-480){\makebox(100,10)[l]{\jkffxei}}
{\color{black} \dottedline{3}(-189,-480)(-5,-480)}\put(0,-490){\makebox(100,10)[l]{\jkffxej}}
{\color{black} \dottedline{3}(-189,-490)(-5,-490)}\put(0,-500){\makebox(100,10)[l]{\jkffxfa}}
{\color{black} \dottedline{3}(-189,-500)(-5,-500)}\put(0,-510){\makebox(100,10)[l]{\jkffxfb}}
\put(0,-520){\makebox(100,10)[l]{\jkffxfc}}
\put(0,-530){\makebox(100,10)[l]{\jkffxfd}}

\color{red}
\put(-6,-295){\circle*{2}}
\put(-6,-297){\vector(0,-1){66}}
\put(-8,-297){\linethickness{0.7pt}\line(1,0){0}}
\put(-6,-365){\circle*{2}}

\color{red}
\put(-12,-312){\circle*{2}}
\put(-14,-314){\linethickness{0.7pt}\line(1,0){0}}
\put(-12,-317){\circle*{2}\hspace{-2\unitlength}\circle{4}}

\color{red}
\put(-18,-315){\circle*{2}}
\put(-18,-317){\vector(0,-1){6}}
\put(-20,-317){\linethickness{0.7pt}\line(1,0){0}}
\put(-18,-325){\circle*{2}}

\color{red}
\put(-24,-315){\circle*{2}}
\put(-24,-317){\vector(0,-1){26}}
\put(-26,-317){\linethickness{0.7pt}\line(1,0){0}}
\put(-24,-345){\circle*{2}}

\color{red}
\put(-30,-315){\circle*{2}\hspace{-2\unitlength}\circle{4}}
\put(-30,-317){\vector(0,-1){16}}
\put(-32,-317){\linethickness{0.7pt}\line(1,0){4}}
\put(-30,-335){\circle*{2}\hspace{-2\unitlength}\circle{4}}

\color{red}
\put(-36,-315){\circle*{2}\hspace{-2\unitlength}\circle{4}}
\put(-36,-317){\vector(0,-1){26}}
\put(-38,-317){\linethickness{0.7pt}\line(1,0){4}}
\put(-36,-345){\circle*{2}\hspace{-2\unitlength}\circle{4}}

\color{red}
\put(-12,-325){\circle*{2}}
\put(-12,-327){\vector(0,-1){6}}
\put(-14,-327){\linethickness{0.7pt}\line(1,0){0}}
\put(-12,-335){\circle*{2}}

\color{red}
\put(-42,-325){\circle*{2}}
\put(-42,-327){\vector(0,-1){16}}
\put(-44,-327){\linethickness{0.7pt}\line(1,0){0}}
\put(-42,-345){\circle*{2}}

\color{red}
\put(-18,-332){\circle*{2}}
\put(-20,-334){\linethickness{0.7pt}\line(1,0){0}}
\put(-18,-337){\circle*{2}\hspace{-2\unitlength}\circle{4}}

\color{red}
\put(-12,-342){\circle*{2}}
\put(-14,-344){\linethickness{0.7pt}\line(1,0){0}}
\put(-12,-347){\circle*{2}\hspace{-2\unitlength}\circle{4}}

\color{red}
\put(-6,-75){\circle*{2}}
\put(-6,-77){\vector(0,-1){186}}
\put(-8,-77){\linethickness{0.7pt}\line(1,0){0}}
\put(-6,-265){\circle*{2}}

\color{red}
\put(-12,-85){\circle*{2}}
\put(-12,-87){\vector(0,-1){56}}
\put(-14,-87){\linethickness{0.7pt}\line(1,0){0}}
\put(-12,-145){\circle*{2}}

\color{green}
\put(-18,-85){\circle*{2}}
\put(-18,-87){\vector(0,-1){66}}
\put(-20,-87){\linethickness{0.7pt}\line(1,0){4}}
\put(-18,-155){\circle*{2}}

\color{green}
\put(-24,-85){\circle*{2}}
\put(-24,-87){\vector(0,-1){156}}
\put(-26,-87){\linethickness{0.7pt}\line(1,0){4}}
\put(-24,-245){\circle*{2}}

\color{red}
\put(-30,-85){\circle*{2}}
\put(-30,-87){\vector(0,-1){156}}
\put(-32,-87){\linethickness{0.7pt}\line(1,0){0}}
\put(-30,-245){\circle*{2}}

\color{red}
\put(-36,-95){\circle*{2}}
\put(-36,-97){\vector(0,-1){46}}
\put(-38,-97){\linethickness{0.7pt}\line(1,0){0}}
\put(-36,-145){\circle*{2}}

\color{red}
\put(-42,-95){\circle*{2}}
\put(-42,-97){\vector(0,-1){56}}
\put(-44,-97){\linethickness{0.7pt}\line(1,0){0}}
\put(-42,-155){\circle*{2}}

\color{green}
\put(-48,-95){\circle*{2}}
\put(-48,-97){\vector(0,-1){96}}
\put(-50,-97){\linethickness{0.7pt}\line(1,0){4}}
\put(-48,-195){\circle*{2}}

\color{red}
\put(-54,-95){\circle*{2}}
\put(-54,-97){\vector(0,-1){96}}
\put(-56,-97){\linethickness{0.7pt}\line(1,0){0}}
\put(-54,-195){\circle*{2}}

\color{green}
\put(-60,-95){\circle*{2}}
\put(-60,-97){\vector(0,-1){146}}
\put(-62,-97){\linethickness{0.7pt}\line(1,0){4}}
\put(-60,-245){\circle*{2}}

\color{red}
\put(-66,-105){\circle*{2}}
\put(-66,-107){\vector(0,-1){26}}
\put(-68,-107){\linethickness{0.7pt}\line(1,0){0}}
\put(-66,-135){\circle*{2}}

\color{red}
\put(-72,-105){\circle*{2}}
\put(-72,-107){\vector(0,-1){46}}
\put(-74,-107){\linethickness{0.7pt}\line(1,0){0}}
\put(-72,-155){\circle*{2}}

\color{red}
\put(-78,-105){\circle*{2}}
\put(-78,-107){\vector(0,-1){56}}
\put(-80,-107){\linethickness{0.7pt}\line(1,0){0}}
\put(-78,-165){\circle*{2}}

\color{red}
\put(-84,-105){\circle*{2}}
\put(-84,-107){\vector(0,-1){136}}
\put(-86,-107){\linethickness{0.7pt}\line(1,0){0}}
\put(-84,-245){\circle*{2}}

\color{red}
\put(-90,-105){\circle*{2}}
\put(-90,-107){\vector(0,-1){146}}
\put(-92,-107){\linethickness{0.7pt}\line(1,0){0}}
\put(-90,-255){\circle*{2}}

\color{red}
\put(-96,-115){\circle*{2}}
\put(-96,-117){\vector(0,-1){16}}
\put(-98,-117){\linethickness{0.7pt}\line(1,0){0}}
\put(-96,-135){\circle*{2}}

\color{red}
\put(-102,-115){\circle*{2}}
\put(-102,-117){\vector(0,-1){76}}
\put(-104,-117){\linethickness{0.7pt}\line(1,0){0}}
\put(-102,-195){\circle*{2}}

\color{red}
\put(-108,-115){\circle*{2}}
\put(-108,-117){\vector(0,-1){86}}
\put(-110,-117){\linethickness{0.7pt}\line(1,0){0}}
\put(-108,-205){\circle*{2}}

\color{green}
\put(-114,-135){\circle*{2}}
\put(-114,-137){\vector(0,-1){26}}
\put(-116,-137){\linethickness{0.7pt}\line(1,0){0}}
\put(-114,-165){\circle*{2}}

\color{green}
\put(-120,-135){\circle*{2}}
\put(-120,-137){\vector(0,-1){66}}
\put(-122,-137){\linethickness{0.7pt}\line(1,0){0}}
\put(-120,-205){\circle*{2}}

\color{green}
\put(-66,-145){\circle*{2}}
\put(-66,-147){\vector(0,-1){26}}
\put(-68,-147){\linethickness{0.7pt}\line(1,0){0}}
\put(-66,-175){\circle*{2}}

\color{green}
\put(-96,-145){\circle*{2}}
\put(-96,-147){\vector(0,-1){66}}
\put(-98,-147){\linethickness{0.7pt}\line(1,0){0}}
\put(-96,-215){\circle*{2}}

\color{green}
\put(-12,-155){\circle*{2}}
\put(-12,-157){\vector(0,-1){6}}
\put(-14,-157){\linethickness{0.7pt}\line(1,0){0}}
\put(-12,-165){\circle*{2}}

\color{green}
\put(-36,-155){\circle*{2}}
\put(-36,-157){\vector(0,-1){16}}
\put(-38,-157){\linethickness{0.7pt}\line(1,0){0}}
\put(-36,-175){\circle*{2}}

\color{red}
\put(-126,-165){\circle*{2}}
\put(-126,-163){\vector(0,1){26}}
\put(-128,-137){\linethickness{0.7pt}\line(1,0){0}}
\put(-126,-135){\circle*{2}}

\color{red}
\put(-132,-165){\circle*{2}}
\put(-132,-163){\vector(0,1){6}}
\put(-134,-157){\linethickness{0.7pt}\line(1,0){0}}
\put(-132,-155){\circle*{2}}

\color{red}
\put(-18,-162){\circle*{2}}
\put(-20,-164){\linethickness{0.7pt}\line(1,0){0}}
\put(-18,-167){\circle*{2}}

\color{green}
\put(-42,-162){\circle*{2}}
\put(-44,-164){\linethickness{0.7pt}\line(1,0){0}}
\put(-42,-167){\circle*{2}}

\color{red}
\put(-72,-165){\circle*{2}}
\put(-72,-167){\vector(0,-1){6}}
\put(-74,-167){\linethickness{0.7pt}\line(1,0){0}}
\put(-72,-175){\circle*{2}}

\color{red}
\put(-138,-165){\circle*{2}}
\put(-138,-167){\vector(0,-1){76}}
\put(-140,-167){\linethickness{0.7pt}\line(1,0){0}}
\put(-138,-245){\circle*{2}}

\color{red}
\put(-144,-165){\circle*{2}}
\put(-144,-167){\vector(0,-1){86}}
\put(-146,-167){\linethickness{0.7pt}\line(1,0){0}}
\put(-144,-255){\circle*{2}}

\color{red}
\put(-150,-175){\circle*{2}}
\put(-150,-173){\vector(0,1){26}}
\put(-152,-147){\linethickness{0.7pt}\line(1,0){0}}
\put(-150,-145){\circle*{2}}

\color{green}
\put(-156,-175){\circle*{2}}
\put(-156,-173){\vector(0,1){16}}
\put(-158,-157){\linethickness{0.7pt}\line(1,0){4}}
\put(-156,-155){\circle*{2}}

\color{red}
\put(-162,-175){\circle*{2}}
\put(-162,-173){\vector(0,1){16}}
\put(-164,-157){\linethickness{0.7pt}\line(1,0){0}}
\put(-162,-155){\circle*{2}}

\color{green}
\put(-168,-175){\circle*{2}}
\put(-168,-173){\vector(0,1){6}}
\put(-170,-167){\linethickness{0.7pt}\line(1,0){0}}
\put(-168,-165){\circle*{2}}

\color{red}
\put(-12,-175){\circle*{2}}
\put(-12,-177){\vector(0,-1){16}}
\put(-14,-177){\linethickness{0.7pt}\line(1,0){0}}
\put(-12,-195){\circle*{2}}

\color{green}
\put(-18,-175){\circle*{2}}
\put(-18,-177){\vector(0,-1){66}}
\put(-20,-177){\linethickness{0.7pt}\line(1,0){4}}
\put(-18,-245){\circle*{2}}

\color{green}
\put(-36,-195){\circle*{2}}
\put(-36,-197){\vector(0,-1){6}}
\put(-38,-197){\linethickness{0.7pt}\line(1,0){0}}
\put(-36,-205){\circle*{2}}

\color{green}
\put(-42,-195){\circle*{2}}
\put(-42,-197){\vector(0,-1){16}}
\put(-44,-197){\linethickness{0.7pt}\line(1,0){0}}
\put(-42,-215){\circle*{2}}

\color{red}
\put(-174,-205){\circle*{2}}
\put(-174,-203){\vector(0,1){66}}
\put(-176,-137){\linethickness{0.7pt}\line(1,0){0}}
\put(-174,-135){\circle*{2}}

\color{red}
\put(-66,-205){\circle*{2}}
\put(-66,-203){\vector(0,1){6}}
\put(-68,-197){\linethickness{0.7pt}\line(1,0){0}}
\put(-66,-195){\circle*{2}}

\color{red}
\put(-12,-202){\circle*{2}}
\put(-14,-204){\linethickness{0.7pt}\line(1,0){0}}
\put(-12,-207){\circle*{2}}

\color{green}
\put(-48,-202){\circle*{2}}
\put(-50,-204){\linethickness{0.7pt}\line(1,0){0}}
\put(-48,-207){\circle*{2}}

\color{red}
\put(-54,-205){\circle*{2}}
\put(-54,-207){\vector(0,-1){6}}
\put(-56,-207){\linethickness{0.7pt}\line(1,0){0}}
\put(-54,-215){\circle*{2}}

\color{red}
\put(-180,-215){\circle*{2}}
\put(-180,-213){\vector(0,1){66}}
\put(-182,-147){\linethickness{0.7pt}\line(1,0){0}}
\put(-180,-145){\circle*{2}}

\color{red}
\put(-186,-215){\circle*{2}}
\put(-186,-213){\vector(0,1){56}}
\put(-188,-157){\linethickness{0.7pt}\line(1,0){0}}
\put(-186,-155){\circle*{2}}

\color{green}
\put(-72,-215){\circle*{2}}
\put(-72,-213){\vector(0,1){16}}
\put(-74,-197){\linethickness{0.7pt}\line(1,0){4}}
\put(-72,-195){\circle*{2}}

\color{red}
\put(-78,-215){\circle*{2}}
\put(-78,-213){\vector(0,1){16}}
\put(-80,-197){\linethickness{0.7pt}\line(1,0){0}}
\put(-78,-195){\circle*{2}}

\color{green}
\put(-102,-215){\circle*{2}}
\put(-102,-213){\vector(0,1){6}}
\put(-104,-207){\linethickness{0.7pt}\line(1,0){0}}
\put(-102,-205){\circle*{2}}

\color{green}
\put(-12,-215){\circle*{2}}
\put(-12,-217){\vector(0,-1){26}}
\put(-14,-217){\linethickness{0.7pt}\line(1,0){4}}
\put(-12,-245){\circle*{2}}

\color{red}
\put(-6,-415){\circle*{2}}
\put(-6,-417){\vector(0,-1){86}}
\put(-8,-417){\linethickness{0.7pt}\line(1,0){0}}
\put(-6,-505){\circle*{2}}

\color{red}
\put(-12,-442){\circle*{2}}
\put(-14,-444){\linethickness{0.7pt}\line(1,0){0}}
\put(-12,-447){\circle*{2}\hspace{-2\unitlength}\circle{4}}

\color{red}
\put(-18,-445){\circle*{2}\hspace{-2\unitlength}\circle{4}}
\put(-18,-447){\vector(0,-1){16}}
\put(-20,-447){\linethickness{0.7pt}\line(1,0){4}}
\put(-18,-465){\circle*{2}}

\color{red}
\put(-12,-452){\circle*{2}}
\put(-14,-454){\linethickness{0.7pt}\line(1,0){0}}
\put(-12,-457){\circle*{2}}

\color{green}
\put(-24,-452){\circle*{2}}
\put(-26,-454){\linethickness{0.7pt}\line(1,0){0}}
\put(-24,-457){\circle*{2}}

\color{red}
\put(-30,-455){\circle*{2}}
\put(-30,-457){\vector(0,-1){6}}
\put(-32,-457){\linethickness{0.7pt}\line(1,0){0}}
\put(-30,-465){\circle*{2}}

\color{green}
\put(-36,-465){\circle*{2}}
\put(-36,-463){\vector(0,1){6}}
\put(-38,-457){\linethickness{0.7pt}\line(1,0){0}}
\put(-36,-455){\circle*{2}}

\color{red}
\put(-12,-462){\circle*{2}}
\put(-14,-464){\linethickness{0.7pt}\line(1,0){0}}
\put(-12,-467){\circle*{2}\hspace{-2\unitlength}\circle{4}}

\color{red}
\put(-24,-462){\circle*{2}\hspace{-2\unitlength}\circle{4}}
\put(-26,-464){\linethickness{0.7pt}\line(1,0){4}}
\put(-24,-467){\circle*{2}\hspace{-2\unitlength}\circle{4}}

\color{red}
\put(-42,-465){\circle*{2}\hspace{-2\unitlength}\circle{4}}
\put(-42,-467){\vector(0,-1){16}}
\put(-44,-467){\linethickness{0.7pt}\line(1,0){4}}
\put(-42,-485){\circle*{2}\hspace{-2\unitlength}\circle{4}}

\color{green}
\put(-48,-465){\circle*{2}\hspace{-2\unitlength}\circle{4}}
\put(-48,-467){\vector(0,-1){16}}
\put(-50,-467){\linethickness{0.7pt}\line(1,0){4}}
\put(-48,-485){\circle*{2}\hspace{-2\unitlength}\circle{4}}

\color{blue}
\put(-54,-465){\circle*{2}\hspace{-2\unitlength}\circle{4}}
\put(-54,-467){\vector(0,-1){16}}
\put(-56,-467){\linethickness{0.7pt}\line(1,0){4}}
\put(-54,-485){\circle*{2}\hspace{-2\unitlength}\circle{4}}

\color{red}
\put(-12,-482){\circle*{2}}
\put(-14,-484){\linethickness{0.7pt}\line(1,0){0}}
\put(-12,-487){\circle*{2}\hspace{-2\unitlength}\circle{4}}

\end{picture}
\]
\hrulefill

\caption{The dependency list of quicksort}
\label{fquickl}
\end{figure}

\subsection{A look at quicksort}

Figures~\ref{fquickg} and \ref{fquickl} give another example;
quicksort (we have suppressed all dependencies except those in the
function {\tt qs} itself which implements the recursion).
This is an interesting program since it uses a recursive divide and
conquer algorithm which gives hope for a lot of parallelism. However,
the implementation is sequential and uses in-place update of a single
array. The question is whether Embla can find the parallelism that
should be buried there. 

To find out, let us turn to the definition of {\tt qs} in
figure~\ref{fquickg} where we find the two recursive calls at lines 36
and 37. What we do not find is a data dependecy between them; hence
Embla has found that the functions are independent. We know of no 
static analysis that could find this independence, especially as it 
is expressed using pointer arithmetic rather than indexing operations,
as is representative of legacy C code.

\section{Dependence Attribution}



\begin{figure} \small
\hrulefill
\[
\begin{picture}(160,90)(70,10)
\put(120,90){\makebox(60,10)[c]{\it A:\ \tt main}}
\put(150,90){\line(-1,-1){50}}
\put(150,90){\line( 0,-1){50}}
\put(150,90){\line( 1,-1){50}}
\put(100,60){\makebox(20,10)[r]{\it 14}}
\put(150,60){\makebox(20,10)[l]{\it 15}}
\put(180,60){\makebox(20,10)[l]{\it 16}}
\put(80,50){\makebox(20,10)[r]{\ldots}}
\put(200,50){\makebox(20,10)[l]{\ldots}}
\put(170,30){\makebox(60,10)[c]{\it D:\ \tt inc}}
\put(120,30){\makebox(60,10)[c]{\it C:\ \tt inc}}
\put(70,30){\makebox(60,10)[c]{\it B:\ \tt inc}}
\put(70,10){\makebox(60,10)[c]{{\tt *q=}\ \ldots}}
\put(170,10){\makebox(60,10)[c]{\ldots\ {\tt *q}\ \ldots}}
\end{picture}
\]
\hrulefill
\caption{Part of the call tree of Example 1, edges are annotated 
with the line number of the corresponding call.} 
\label{ffextree}
\end{figure}

Tracing programs to find dependencies is by no means a new endavour. Most 
earlier efforts have however been aimed at studying different hardware 
enhancements. In that context, the desired result is the execution time 
under various hardware assumptions. Thus it was sufficient to keep track
of the execution of the dynamic instruction stream, including dependencies
between instructions; mapping the dependencies to source code was unnecessary.

It is possible to find the minimal execution time by keeping track of the 
earliest possible execution
time of each instruction in the dynamic instruction stream. For each instruction 
that writes to a memory location or a register, that location or register is 
associated with the execution time of the writing instruction. The earliest
execution time of an intruction that reads memory or registers is taken as the
maximum of the time stamps of the items read plus the latency of the instruction 
itself.


A tool like Embla, on the other hand, that is designed to help programmers 
rewrite source code, poses different demands. Here, we need to project 
the dependencies discovered in the dynamic instruction stream onto the 
source code in such a way as to inform the programmer about legal code 
transformations. Since we are dealing with transformations on the level of
procedure calls, we need to reflect the dependencies we find to that level 
as well. That is, we always want to know the dependencies between different 
lines in the same procedure.

\subsection{From dependency endpoints to dependency edges: the call tree}

Consider the program from Figure~\ref{ffirstex} where we have three consecutive
calls to {\tt inc} in lines 14--16. Figure~\ref{ffextree} gives the relevant 
part of the dynamic call tree. The flow dependence between line 14 and 16 is due 
to the write in {\tt inc} from the first call and the read from the second call.
Note that the same instructions in other calling contexts give rise to 
different dependencies, eg the flow dependence from line 15 (call to {\tt inc})
to line 17 (call to {\tt printf}).

We accomplish this mapping by associating each memory location with a dynamic
instruction execution event and we also keep track of how these events 
relate to the dynamic call tree. Thus, for a memory reference event $E$ we 
look up its data address to find the previous reference event $E'$ to that 
location. Then we find the {\em nearest common anscestor} (NCA) of the call
tree nodes of $E$ and $E'$ (for instance, the NCA of {\it B} and {\it D} is
{\it A} in Figure~\ref{ffextree}). The NCA can be calculated from the previous
reference event $E'$ itself by noting that all anscestors of the current
reference event $E$ correspond to frames on the current call stack. If we mark 
such nodes, we can search from $E'$ 
towards the root of the call tree; the first node we reach that is on the 
stack is the NCA.
The line number (in the function corresponding to the NCA) of the source or 
destination of the dependence is the line number of the last edge on the path 
from the relevant event to the NCA (14 and 16, respectively, in 
Figure~\ref{ffextree}).


\begin{figure}
\small
\hrulefill
\begin{verbatim}
   DependenceEdge( oldEvent, currEvent ) {
       oldLine = oldEvent.line;
       ncaNode = oldEvent.node;
       while( ncaNode is not on stack ) {
           oldLine = ncaNode.line;
           ncaNode = ncaNode.parent;
       }
       if( ncaNode != currEvent.node )
           currLine = ncaNode.nextOnStack.line;
       else
           currLine = currEvent.line;
       return ( oldLine, currLine );
    }
\end{verbatim}
\hrulefill
\caption{Constructing the dependence edge from the events corresponding
to a previous and a new reference}
\label{fdepedge}
\end{figure}    

The algorithm for computing the dependence edge given a data address and 
the current reference event is given in Figure~\ref{fdepedge}. Here an 
event represents the execution of a memory reference and a node corresponds 
to a procedure invocation. An event {\tt e} has two attributes: {\tt e.line} 
is the source line corresponding to {\tt e} and {\tt e.node} is the procedure 
invocation that {\tt e} is part of. Similarly, if {\tt n} is a node, then 
{\tt n.line} is the source line associated with the procedure call 
corresponding to {\tt n}, {\tt n.parent} is the parent node in the call tree 
and, if {\tt n} is in the stack, {\tt n.nextOnStack} is the node on top of 
{\tt n} in the stack ({\tt n.nextOnStack.parent} = {\tt n}).

Path compression can be used to decrease the number of iterations of the 
{\tt while} loop.
Every node {\tt n} that is visited but is not on the stack can have its parent set 
to its closest ancestor on the stack. We conjecture that this reduces the 
complexity of the algorithm to essentially constant time. We can however do even 
better since the output of the edge calculation does only depend on the immediate
descendants of nodes on the stack. Thus every subtree $T$ of the call tree, 
including the events at its leaves, such
that the root of $T$ is not on the stack can be represented by the root of $T$
alone.

\subsection{The state of a memory location: the memory table}

When computing the dependencies created by a memory reference, we need to 
map the data address of the reference to information about the previous
reference to that location. What information we need depends on whether 
the new refenece is a read or a write:
\begin{description}
\item[READ]
We need the latest write to construct a flow dependence (RAW).
\item[WRITE]
We need the latest write to construct an output dependence (WAW) as well as
all reads since that write to construct anti dependencies (WAR).
\end{description}
To see that we need all reads since the last write, consider the following
example:
\begin{verbatim}
  x = 1;
  y = x+1;
  z = x+2;
  x = 3;
\end{verbatim}
If only the last read is kept, the dependence between the second and fourth
lines will be lost, and since subsequent reads do not yield dependencies, 
there is no transitive dependence between these lines.
Hence we keep track, for each memory location, of the the event corresponding 
to the last write and a list of events corresponding to subsequent reads.

An important aspect of compaction is that when a subtree of the dynamic
call tree is compacted to be represented by its root node, the memory table
needs to be updated to avoid dangling pointers to eliminated events. All
pointers are then forwarded to point to the representative nodes. In this
process, pointers to previously distict events now may point at the same 
tree nodes. 

\newcommand{\backlink}[3]
{\begin{picture}(#1,#2)
\put(0,0){\line(1,0){#1}}
\put(#1,0){\line(0,1){#2}}
\put(#1,#2){\vector(-1,0){#3}}
\end{picture}}

\newcommand{\rwblock}
{\begin{picture}(100,40)(-40,0)
\put(0,25){\line(4,3){20}}
\put(0,15){\line(4,-3){20}}
\put(-40,25){\line(1,0){40}}
\put(-40,15){\line(1,0){40}}
\put(20,0){\framebox(40,40){\ }}
\put(20,30){\line(1,0){40}}
\put(20,30){\makebox(40,10){Last write}}
\put(20,20){\makebox(40,10){Last reads}}
\put(20,5){\makebox(40,10){\vdots}}
\end{picture}}
\begin{figure*}
\small
\hrulefill
\[
\begin{picture}(250,210)(-30,0)
% The memory table
\put(-30,30){\framebox(40,180)[t]{\ }}
\put(-30,10){\makebox(60,10){The memory table}}
% A block
\put(-30,145){\rwblock}
\put(70,177){\vector(3,-2){30}}
\put(-30,85){\rwblock}
\put(70,113){\vector(4,-1){30}}
\put(70,97){\vector(3,-2){30}}
\put(-30,25){\rwblock}
\put(70,50){\vector(2,-1){30}}
% Trace pile
\put(100,30){\framebox(40,180)[t]{\ }}
\put(100,190){\framebox(40,10){Open C}}
\put(100,170){\makebox(40,10){\vdots}}
\put(100,160){\line(1,0){40}}
\put(100,150){\makebox(40,10){Reg}}
\put(140,157){\backlink{8}{36}{6}}
\put(100,140){\framebox(40,10){Closed C}}
\put(140,147){\backlink{16}{46}{6}}
\put(100,120){\makebox(40,10){\vdots}}
\put(100,100){\framebox(40,10){Reg}}
\put(140,107){\backlink{8}{36}{6}}
\put(100,85){\makebox(40,10){\vdots}}
\put(100,70){\framebox(40,10){Reg}}
\put(140,77){\backlink{16}{66}{6}}
\put(100,50){\line(1,0){40}}
\put(100,55){\makebox(40,10){\vdots}}
\put(100,40){\makebox(40,10){Return}}
\put(140,47){\backlink{24}{96}{6}}
\put(100,30){\framebox(40,10){Reg}}
\put(140,37){\backlink{32}{156}{14}}
\end{picture}
%
% The next picture
%
\begin{picture}(200,210)(-30,0)
% The memory table
\put(-30,30){\framebox(40,180)[t]{\ }}
\put(-30,10){\makebox(60,10){The memory table}}
% A block
\put(-30,145){\rwblock}
\put(70,181){\vector(4,3){30}}
\put(-30,85){\rwblock}
\put(70,105){\vector(1,2){30}}
\put(-30,25){\rwblock}
\put(70,52){\vector(1,3){30}}
% Trace pile
\put(100,30){\framebox(40,180)[t]{\ }}
\put(100,200){\makebox(40,10){Open C}}
\put(100,200){\line(1,0){40}}
\put(100,185){\makebox(40,10){\vdots}}
\put(100,170){\framebox(40,10){Reg}}
\put(140,177){\backlink{8}{26}{6}}
\put(100,160){\makebox(40,10){Closed C}}
\put(140,167){\backlink{16}{36}{6}}
\put(100,150){\framebox(40,10){Return}}
\put(140,157){\backlink{8}{6}{6}}
\put(100,140){\makebox(40,10){Reg}}
\put(140,147){\backlink{24}{56}{6}}
\put(100,140){\line(1,0){40}}
\end{picture}
\]
\hrulefill
\caption{Compacting the trace pile: before and after}
\label{fnyembladata}
\end{figure*}


Figure~\ref{fnyembladata} shows the three fundamental data structures 
in Embla:
\begin{description}
\item[The memory table:]
Contains, for each block of memory, information about the most 
recent write and all reads since that write. 
Blocksize is a tradeoff between memory use and precision; the examples in 
this paper were run with single byte blocks. 
\item[The trace pile:]
Contains trace records representing instruction execution events that 
Embla needs to refer to. Most trace records contain 
\begin{itemize}
\item
a pointer to a record containing information about the program 
address of the corresponding instruction, and
\item
a pointer to the previous trace record representing a procedure call.
\end{itemize}
% \item[The activation tree:]
% Contains a record for each stack frame allocated during execution. 
% At any given point during execution, 
% the rightmost branch of the activation tree corresponds to the stack.
\item[The result table:]
A hash table mapping dependency signatures to counters of RAW, WAR 
and WAR dependencies. Dependency signatures identify the source and
target of the dependency as well as other information of interest.
\end{description}


The figure also shows how the data structures are used when the
profiled program does a memory read. From the memory table we find the
activation frame that was active when the last write was made and we
compute the {\em nearest common ancestor} (NCA) of that an the
current frame. The flow dependence we find will be attributed to the
function that the NCA corresponds to. 

If a dependency endpoint is part of the NCA (if the NCA is the current
frame or the frame pointed to from the memory table) it is labelled as
direct, otherwise it is indirect. If any of the activation frames on
the path from an endpoint to the NCA is on the black list, we have a
weak endpoint.


Some anti and output dependencies are due to reusing the same
locations for (logically) distinct data items. This happens for
instance for stack locations when arguments are passed on the stack
(as is common on the x86, for instance). Consider the following
example:
\begin{verbatim}
    foo(x);
    bar(y);
\end{verbatim}
The value of {\tt x} will be pushed on the stack. Then {\tt foo} will
be called and will read {\tt x}. When that call has returned, {\tt y}
will be pushed on the stack on the same address as {\tt x}, creating
an anti dependence with {\tt foo}'s read. This kind of dependence can 
easily be removed by a
compiler. In fact, if the call to {\tt foo} is made asynchronously, in
another thread, it will automatically be made on a different stack and
the dependence disappears. Register allocation is another source of
anti and output dependencies since distinct variables are packed into
the same register. In this case, the allocation can be changed to
remove dependenies, a transformation known as {\em renaming}.

We currently try to avoid these spurious dependencies in the following
way. The memory table contains time stamps (these are simply the
number of (emulated) instructions executed since the start). We also
keep track of the changing stack size during execution so that we can
see if a memory location has been first in, then out of, then in the
stack again. Thus when we see a memory location (now part of the
stack) with time stamp $t$, we want to know if the stack has been
small enough to exclude that location at some point $t'$ such that 
$t<t'$. 

To this end, we maintain a {\em min stack} consisting of pairs $(t,s)$
where $t$ is a time stamp and $s$ is a stack pointer value. Each pair
$(t,s)$ in the min stack indicates that, at time $t$, the stack
pointer was $s$ and at all times $t'$ the stack has been larger (since
the x86 stack grows from high to low addresses, a small stack
corresponds to a large stack pointer). In the min stack, both
time stamps and stack sizes grow towards the top (that is, the stack
pointers decrease).

When the
call stack grows, we push a new pair, consisting of the current time
and the new stack pointer value, on top of the min stack. When the
stack shrinks, we pop all pairs representing larger stacks off the min
stack.

Thus, when we record a dependence at address $a$ with time stamp $t$,
we see if we can find a pair $(s,t')$ in the min stack with $a<s$ and
$t<t'$ (a smaller stack at a later time).

We avoid spurious anti and output dependencies due to register
allocation by not computing dependencies on registers and compiling
the traced program without optimization. 

\section{Overview of the Implementation of Embla}


\begin{figure*}
\small
\hrulefill
\[
\begin{picture}(500,210)(-30,0)
% The memory table
\put(-30,30){\framebox(40,180)[t]{\ }}
\put(-30,100){\line(1,0){40}}
\put(-30,110){\line(1,0){40}}
\put(-30,10){\makebox(60,10){The memory table}}
% A block
\put(10,110){\line(4,3){20}}
\put(10,100){\line(4,-3){20}}
\put(30,85){\framebox(40,40){\ }}
\put(30,115){\line(1,0){40}}
\put(30,115){\makebox(40,10){Last write}}
\put(30,105){\makebox(40,10){Last reads}}
\put(30,90){\makebox(40,10){\vdots}}
% An event
\put(70,125){\line(2,1){20}}
\put(70,115){\line(2,-1){20}}
\put(90,105){\framebox(40,30){\ }}
\put(90,125){\line(1,0){40}}
\put(90,115){\line(1,0){40}}
\put(90,105){\makebox(40,10){Time}}
\put(90,115){\makebox(40,10){Instr}}
\put(90,125){\makebox(40,10){Context}}
\put(130,130){\vector(1,-2){40}}
% The activation tree
% Level 0 (root)
\put(270,180){\framebox(30,30)[tl]{\it 1}}
% Level 1 (1 node)
\put(235,130){\framebox(30,30)[tl]{\it 2}} \put(250,160){\vector(3,2){30}}
\put(270,150){\makebox(50,10)[l]{\it NCA(4,7)}}
% Level 2 (2 nodes)
\put(190,80){\framebox(30,30)[tl]{\it 3}} \put(205,110){\vector(2,1){40}}
\put(280,80){\framebox(30,30)[tl]{\it 6}} \put(295,110){\vector(-2,1){40}}
% Level 3 (3 nodes)
\put(170,30){\framebox(30,30)[tl]{\it 4}} \put(185,60){\vector(3,4){15}}
\put(210,30){\framebox(30,30)[tl]{\it 5}} \put(225,60){\vector(-3,4){15}}
\put(260,30){\framebox(30,30)[tl]{\it 7}} \put(275,60){\vector(3,4){15}}
\put(295,50){\makebox(50,10)[l]{\it Current frame}}
% caption
\put(170,10){\makebox(80,10){The activation tree}}
% The result table
\put(360,30){\framebox(80,180){\ }}
\put(360,10){\makebox(80,10){The result table}}

\end{picture}
\]
\hrulefill
\caption{The main Embla data structures}
\label{fembladata}
\end{figure*}

Embla is based on instrumented execution of binary code. Although
our examples of profiling output use a high level language (C),
the profiling itself is on instruction level, followed by 
mapping the information to source level using debugging information 
in the standard way.

Embla uses the Valgrind instrumentation infrastructure, so there
is no offline code rewriting; the Embla tool behaves like an emulator
of the hardware.

Figure~\ref{fembladata} shows the three fundamental data structures 
in Embla:
\begin{description}
\item[The memory table:]
Contains, for each block of memory, information about the most 
recent write and all reads since that write. 
Blocksize is a tradeoff between memory use and precision; the examples in 
this paper were run with single byte blocks. 
\item[The activation tree:]
Contains a record for each stack frame allocated during execution. 
At any given point during execution, 
the rightmost branch of the activation tree corresponds to the stack.
\item[The result table:]
A hash table mapping dependency signatures to counters of RAW, WAR 
and WAR dependencies. Dependency signatures identify the source and
target of the dependency as well as other information of interest.
\end{description}
The figure also shows how the data structures are used when the
profiled program does a memory read. From the memory table we find the
activation frame that was active when the last write was made and we
compute the {\em nearest common ancestor} (NCA) of that an the
current frame. The flow dependence we find will be attributed to the
function that the NCA corresponds to. 

If a dependency endpoint is part of the NCA (if the NCA is the current
frame or the frame pointed to from the memory table) it is labelled as
direct, otherwise it is indirect. If any of the activation frames on
the path from an endpoint to the NCA is on the black list, we have a
weak endpoint.


Some anti and output dependencies are due to reusing the same
locations for (logically) distinct data items. This happens for
instance for stack locations when arguments are passed on the stack
(as is common on the x86, for instance). Consider the following
example:
\begin{verbatim}
    foo(x);
    bar(y);
\end{verbatim}
The value of {\tt x} will be pushed on the stack. Then {\tt foo} will
be called and will read {\tt x}. When that call has returned, {\tt y}
will be pushed on the stack on the same address as {\tt x}, creating
an anti dependence with {\tt foo}'s read. This kind of dependence can 
easily be removed by a
compiler. In fact, if the call to {\tt foo} is made asynchronously, in
another thread, it will automatically be made on a different stack and
the dependence disappears. Register allocation is another source of
anti and output dependencies since distinct variables are packed into
the same register. In this case, the allocation can be changed to
remove dependenies, a transformation known as {\em renaming}.

We currently try to avoid these spurious dependencies in the following
way. The memory table contains time stamps (these are simply the
number of (emulated) instructions executed since the start). We also
keep track of the changing stack size during execution so that we can
see if a memory location has been first in, then out of, then in the
stack again. Thus when we see a memory location (now part of the
stack) with time stamp $t$, we want to know if the stack has been
small enough to exclude that location at some point $t'$ such that 
$t'>t$. 

To this end, we maintain a {\em min stack} consisting of pairs $(t,s)$
where $t$ is a time stamp and $s$ is a stack pointer value. Each pair
$(t,s)$ in the min stack indicates that, at time $t$, the stack
pointer was $s$ and at all times $t'$ the stack has been larger (since
the x86 stack grows from high to low addresses, a small stack
corresponds to a large stack pointer). In the min stack, both
time stamps and stack sizes grow towards the top (that is, the stack
pointers decrease).

When the
call stack grows, we push a new pair, consisting of the current time
and the new stack pointer value, on top of the min stack. When the
stack shrinks, we pop all pairs representing larger stacks off the min
stack.

Thus, when we record a dependence at address $a$ with time stamp $t$,
we see if we can find a pair $(s,t')$ in the min stack with $a<s$ and
$t<t'$ (a smaller stack a a later time).

We avoid spurious anti and output dependencies due to register
allocation by not computing dependencies on registers and compiling
the traced program without optimization. 

%\section{The Tool}

Embla helps the user find dependencies in a program. Technically, a
dependece is two references, not both reads, to overlapping memory
locations with no interveaning write. What makes Embla special is the
way dependencies are reported. Since the idea is to support manual,
function call level parallelization, we are not really interested in
knowing that a memory read on line 11754 in function {\tt foo}
sometimes reads a value last written in line 3411 in function {\tt
bar}. Rather, the user is concerned with a completely different
function {\tt baz} and would like to know whether the call to {\tt
foobar} on line 2355 can be made in parallel with the call to {\tt
barfoo} on line 2356.



\subsection{Types of dependencies}

There are several different types of dependences between two
references. These types can be characterized according to several
dimensions. The first dimension concerns the read and write nature of
the references:
\begin{description}
\item[Flow:]
A true data dependency wher the location is first written then
read. Also known as {\em read after write} (RAW).
\item[Anti:]
A dependence caused by reuse of a location that is first read and then
written. Also known as {\em write after read} (WAR).
\item[Output:]
Similar to an anti dependence, but the second reference is also a
write. Also known as {\em write after write} (WAW).
\end{description}
Some anti and output dependencies are due to reusing the same
locations for (logically) distinct data items. This happens for
instance for stack locations when arguments are passed on the stack
(as is common on the x86, for instance). Consider the following
example:
\begin{verbatim}
    foo(x);
    bar(y);
\end{verbatim}
The value of {\tt x} will be pushed on the stack. Then {\tt foo} will
be called and will read {\tt x}. When that call has returned, {\tt y}
will be pushed on the stack on the same address as {\tt x}, creating
an anti dependence with {\tt foo}'s read. This kind of dependence can 
easily be removed by a
compiler. In fact, if the call to {\tt foo} is made asynchronously, in
another thread, it will automatically be made on a different stack and
the dependence disappears. Register allocation is another source of
anti and output dependencies since distinct variables are packed into
the same register. In this case, the allocation can be changed to
remove dependenies, a transformation known as {\em renaming}. 
Other anti dependencies are due to updates
to the same data structure and cannot be eliminated by renaming.

Another dimension that Embla recognizes is where in memory the
location associated with the dependency is situated. In particular, we
want to catch the cases where anti and output dependencies are created
by reuse of memory for new stack frames. This happens when the stack 
shrinks and
then grows again; the new stack frames use the same memory locations,
but only for memory management reasons. With this in mind we have the
following categories:
\begin{description}
\item[s(tack):]
The location has been a part of the stack from the first reference to
the second (inclusive).
\item[f(alse):]
The location was part of the stack when the first reference was made,
then the stack shrunk enough that the location was not part of the
stack and then it became part of the stack again.
\item[o(ther):]
The location has never been part of the stack.
\end{description}

We also categorize dependencies according to properties of the two 
memory references making up the source and target of the dependence. 
This is both in order to give more precise information and to flag
additional dependencies as spurious. 

A typical case in point is 
{\tt malloc}. Each call to {\tt malloc} manipulates (updates) 
administrative data structures like the free list. Embla will
report these as dependencies, effectively serializing all calls to 
{\tt malloc}, and thus in addition all calls to functions calling
{\tt malloc}. In reality, these calls need not be serialized since
program semantics typically does not depend on the exact addresses 
that data are allocated. It suffices that a thread safe 
implementation of {\tt malloc} is used.

Dependency endpoints correspond to instruction execution events and
are reported as line numbers with one of the following codes:
\begin{description}
\item[$\epsilon$:] 
No code at all represents an instruction that is
generated from the source code at the indicated line. Thus the 
instruction is part of the function or procedure containing the 
line.
\item[c:]
The instruction was executed by a function (procedure) that was 
(transitively) invoked by a function (procedure) call at the indicated 
line.
\item[h:]
Like {\bf c}, but the function containing the instruction was 
{\em hidden}, that is, it was on a blacklist of functions which, 
like {\tt malloc}, would not give rise to dependencies in the 
parallelized program.
\end{description}


\begin{figure} 
\small
\verbdef\jkffxb$#include <stdlib.h> $
\verbdef\jkffxc$#include <stdio.h> $
\verbdef\jkffxd$ $
\verbdef\jkffxe$struct {int a; int b;} has_ab; $
\verbdef\jkffxf$ $
\verbdef\jkffxg$int main(int argc, char **argv) $
\verbdef\jkffxh${ $
\verbdef\jkffxi$   has_ab.a = 1; $
\verbdef\jkffxj$   has_ab.b = 1; $
\verbdef\jkffxba$   printf("%d ", has_ab.a); $
\verbdef\jkffxbb$   has_ab.a = 2; $
\verbdef\jkffxbc$   has_ab.b = 2; $
\verbdef\jkffxbd$   printf("%d %d\n", has_ab.a, has_ab.b); $
\verbdef\jkffxbe$ $
\verbdef\jkffxbf$} $
\hrulefill
\[
\begin{picture}(420,150)(-54,-150)

\put(0,0){\makebox(100,10)[l]{}}
\put(0,-10){\makebox(100,10)[l]{\jkffxb}}
\put(0,-20){\makebox(100,10)[l]{\jkffxc}}
\put(0,-30){\makebox(100,10)[l]{\jkffxd}}
\put(0,-40){\makebox(100,10)[l]{\jkffxe}}
\put(0,-50){\makebox(100,10)[l]{\jkffxf}}
\put(0,-60){\makebox(100,10)[l]{\jkffxg}}
\put(0,-70){\makebox(100,10)[l]{\jkffxh}}
{\color{black} \dottedline{3}(-51,-70)(-5,-70)}\put(0,-80){\makebox(100,10)[l]{\jkffxi}}
{\color{black} \dottedline{3}(-51,-80)(-5,-80)}\put(0,-90){\makebox(100,10)[l]{\jkffxj}}
{\color{black} \dottedline{3}(-51,-90)(-5,-90)}\put(0,-100){\makebox(100,10)[l]{\jkffxba}}
{\color{black} \dottedline{3}(-51,-100)(-5,-100)}\put(0,-110){\makebox(100,10)[l]{\jkffxbb}}
{\color{black} \dottedline{3}(-51,-110)(-5,-110)}\put(0,-120){\makebox(100,10)[l]{\jkffxbc}}
{\color{black} \dottedline{3}(-51,-120)(-5,-120)}\put(0,-130){\makebox(100,10)[l]{\jkffxbd}}
{\color{black} \dottedline{3}(-51,-130)(-5,-130)}\put(0,-140){\makebox(100,10)[l]{\jkffxbe}}
{\color{black} \dottedline{3}(-51,-140)(-5,-140)}\put(0,-150){\makebox(100,10)[l]{\jkffxbf}}

\color{red}
\put(-6,-65){\circle*{2}}
\put(-6,-67){\vector(0,-1){76}}
\put(-8,-67){\linethickness{0.7pt}\line(1,0){0}}
\put(-6,-145){\circle*{2}}

\color{red}
\put(-12,-75){\circle*{2}}
\put(-12,-77){\vector(0,-1){16}}
\put(-14,-77){\linethickness{0.7pt}\line(1,0){4}}
\put(-12,-95){\circle*{2}}

\color{blue}
\put(-18,-85){\circle*{2}}
\put(-18,-87){\vector(0,-1){26}}
\put(-20,-87){\linethickness{0.7pt}\line(1,0){4}}
\put(-18,-115){\circle*{2}}

\color{red}
\put(-24,-92){\circle*{2}}
\put(-26,-94){\linethickness{0.7pt}\line(1,0){0}}
\put(-24,-97){\circle*{2}\hspace{-2\unitlength}\circle{4}}

\color{green}
\put(-30,-95){\circle*{2}}
\put(-30,-97){\vector(0,-1){6}}
\put(-32,-97){\linethickness{0.7pt}\line(1,0){4}}
\put(-30,-105){\circle*{2}}

\color{red}
\put(-36,-95){\circle*{2}\hspace{-2\unitlength}\circle{4}}
\put(-36,-97){\vector(0,-1){26}}
\put(-38,-97){\linethickness{0.7pt}\line(1,0){4}}
\put(-36,-125){\circle*{2}\hspace{-2\unitlength}\circle{4}}

\color{green}
\put(-42,-95){\circle*{2}\hspace{-2\unitlength}\circle{4}}
\put(-42,-97){\vector(0,-1){26}}
\put(-44,-97){\linethickness{0.7pt}\line(1,0){4}}
\put(-42,-125){\circle*{2}\hspace{-2\unitlength}\circle{4}}

\color{blue}
\put(-48,-95){\circle*{2}\hspace{-2\unitlength}\circle{4}}
\put(-48,-97){\vector(0,-1){26}}
\put(-50,-97){\linethickness{0.7pt}\line(1,0){4}}
\put(-48,-125){\circle*{2}\hspace{-2\unitlength}\circle{4}}

\color{red}
\put(-12,-105){\circle*{2}}
\put(-12,-107){\vector(0,-1){16}}
\put(-14,-107){\linethickness{0.7pt}\line(1,0){4}}
\put(-12,-125){\circle*{2}}

\color{red}
\put(-24,-115){\circle*{2}}
\put(-24,-117){\vector(0,-1){6}}
\put(-26,-117){\linethickness{0.7pt}\line(1,0){4}}
\put(-24,-125){\circle*{2}}

\color{red}
\put(-18,-122){\circle*{2}}
\put(-20,-124){\linethickness{0.7pt}\line(1,0){0}}
\put(-18,-127){\circle*{2}\hspace{-2\unitlength}\circle{4}}

\end{picture}
\]
\hrulefill

\end{figure}

\begin{figure} 
\small
\verbdef\jkffxb$#include <stdio.h> $
\verbdef\jkffxc$#include <stdlib.h> $
\verbdef\jkffxd$ $
\verbdef\jkffxe$#define N 5 $
\verbdef\jkffxf$ $
\verbdef\jkffxg$typedef struct _ilist { $
\verbdef\jkffxh$   int val; $
\verbdef\jkffxi$   struct _ilist *next; $
\verbdef\jkffxj$} ilist; $
\verbdef\jkffxba$ $
\verbdef\jkffxbb$static ilist* mklist(int n) $
\verbdef\jkffxbc${ $
\verbdef\jkffxbd$  int i; $
\verbdef\jkffxbe$  ilist *p = NULL; $
\verbdef\jkffxbf$ $
\verbdef\jkffxbg$  for( i=0; i<n; i++ ) { $
\verbdef\jkffxbh$    ilist *t = (ilist *) malloc( sizeof(ilist) ); $
\verbdef\jkffxbi$    t->val = i; $
\verbdef\jkffxbj$    t->next = p; $
\verbdef\jkffxca$    p = t; $
\verbdef\jkffxcb$  } $
\verbdef\jkffxcc$  return p; $
\verbdef\jkffxcd$} $
\verbdef\jkffxce$ $
\verbdef\jkffxcf$static int sumlist(ilist* p) $
\verbdef\jkffxcg${ $
\verbdef\jkffxch$   ilist *q; $
\verbdef\jkffxci$   int  sum = 0; $
\verbdef\jkffxcj$ $
\verbdef\jkffxda$   for( q=p; q!=NULL; q=q->next ) { $
\verbdef\jkffxdb$      sum += q->val; $
\verbdef\jkffxdc$   } $
\verbdef\jkffxdd$   return sum; $
\verbdef\jkffxde$} $
\verbdef\jkffxdf$ $
\verbdef\jkffxdg$ $
\verbdef\jkffxdh$int main(int argc, char **argv)  $
\verbdef\jkffxdi${ $
\verbdef\jkffxdj$  int m,n; $
\verbdef\jkffxea$  ilist *p,*q; $
\verbdef\jkffxeb$ $
\verbdef\jkffxec$  p = mklist( N ); $
\verbdef\jkffxed$  q = mklist( N ); $
\verbdef\jkffxee$  m = sumlist( p ); $
\verbdef\jkffxef$  n = sumlist( q ); $
\verbdef\jkffxeg$ $
\verbdef\jkffxeh$  printf("%d\n", m+n); $
\verbdef\jkffxei$ $
\verbdef\jkffxej$  return 0; $
\verbdef\jkffxfa$ $
\verbdef\jkffxfb$} $
\hrulefill
\[
\begin{picture}(420,510)(-78,-510)

\put(0,0){\makebox(100,10)[l]{}}
\put(0,-10){\makebox(100,10)[l]{\jkffxb}}
\put(0,-20){\makebox(100,10)[l]{\jkffxc}}
\put(0,-30){\makebox(100,10)[l]{\jkffxd}}
\put(0,-40){\makebox(100,10)[l]{\jkffxe}}
\put(0,-50){\makebox(100,10)[l]{\jkffxf}}
\put(0,-60){\makebox(100,10)[l]{\jkffxg}}
\put(0,-70){\makebox(100,10)[l]{\jkffxh}}
\put(0,-80){\makebox(100,10)[l]{\jkffxi}}
\put(0,-90){\makebox(100,10)[l]{\jkffxj}}
\put(0,-100){\makebox(100,10)[l]{\jkffxba}}
\put(0,-110){\makebox(100,10)[l]{\jkffxbb}}
\put(0,-120){\makebox(100,10)[l]{\jkffxbc}}
{\color{black} \dottedline{3}(-75,-120)(-5,-120)}\put(0,-130){\makebox(100,10)[l]{\jkffxbd}}
{\color{black} \dottedline{3}(-75,-130)(-5,-130)}\put(0,-140){\makebox(100,10)[l]{\jkffxbe}}
{\color{black} \dottedline{3}(-75,-140)(-5,-140)}\put(0,-150){\makebox(100,10)[l]{\jkffxbf}}
{\color{black} \dottedline{3}(-75,-150)(-5,-150)}\put(0,-160){\makebox(100,10)[l]{\jkffxbg}}
{\color{black} \dottedline{3}(-75,-160)(-5,-160)}\put(0,-170){\makebox(100,10)[l]{\jkffxbh}}
{\color{black} \dottedline{3}(-75,-170)(-5,-170)}\put(0,-180){\makebox(100,10)[l]{\jkffxbi}}
{\color{black} \dottedline{3}(-75,-180)(-5,-180)}\put(0,-190){\makebox(100,10)[l]{\jkffxbj}}
{\color{black} \dottedline{3}(-75,-190)(-5,-190)}\put(0,-200){\makebox(100,10)[l]{\jkffxca}}
{\color{black} \dottedline{3}(-75,-200)(-5,-200)}\put(0,-210){\makebox(100,10)[l]{\jkffxcb}}
{\color{black} \dottedline{3}(-75,-210)(-5,-210)}\put(0,-220){\makebox(100,10)[l]{\jkffxcc}}
{\color{black} \dottedline{3}(-75,-220)(-5,-220)}\put(0,-230){\makebox(100,10)[l]{\jkffxcd}}
\put(0,-240){\makebox(100,10)[l]{\jkffxce}}
\put(0,-250){\makebox(100,10)[l]{\jkffxcf}}
\put(0,-260){\makebox(100,10)[l]{\jkffxcg}}
{\color{black} \dottedline{3}(-75,-260)(-5,-260)}\put(0,-270){\makebox(100,10)[l]{\jkffxch}}
{\color{black} \dottedline{3}(-75,-270)(-5,-270)}\put(0,-280){\makebox(100,10)[l]{\jkffxci}}
{\color{black} \dottedline{3}(-75,-280)(-5,-280)}\put(0,-290){\makebox(100,10)[l]{\jkffxcj}}
{\color{black} \dottedline{3}(-75,-290)(-5,-290)}\put(0,-300){\makebox(100,10)[l]{\jkffxda}}
{\color{black} \dottedline{3}(-75,-300)(-5,-300)}\put(0,-310){\makebox(100,10)[l]{\jkffxdb}}
{\color{black} \dottedline{3}(-75,-310)(-5,-310)}\put(0,-320){\makebox(100,10)[l]{\jkffxdc}}
{\color{black} \dottedline{3}(-75,-320)(-5,-320)}\put(0,-330){\makebox(100,10)[l]{\jkffxdd}}
{\color{black} \dottedline{3}(-75,-330)(-5,-330)}\put(0,-340){\makebox(100,10)[l]{\jkffxde}}
\put(0,-350){\makebox(100,10)[l]{\jkffxdf}}
\put(0,-360){\makebox(100,10)[l]{\jkffxdg}}
\put(0,-370){\makebox(100,10)[l]{\jkffxdh}}
\put(0,-380){\makebox(100,10)[l]{\jkffxdi}}
{\color{black} \dottedline{3}(-75,-380)(-5,-380)}\put(0,-390){\makebox(100,10)[l]{\jkffxdj}}
{\color{black} \dottedline{3}(-75,-390)(-5,-390)}\put(0,-400){\makebox(100,10)[l]{\jkffxea}}
{\color{black} \dottedline{3}(-75,-400)(-5,-400)}\put(0,-410){\makebox(100,10)[l]{\jkffxeb}}
{\color{black} \dottedline{3}(-75,-410)(-5,-410)}\put(0,-420){\makebox(100,10)[l]{\jkffxec}}
{\color{black} \dottedline{3}(-75,-420)(-5,-420)}\put(0,-430){\makebox(100,10)[l]{\jkffxed}}
{\color{black} \dottedline{3}(-75,-430)(-5,-430)}\put(0,-440){\makebox(100,10)[l]{\jkffxee}}
{\color{black} \dottedline{3}(-75,-440)(-5,-440)}\put(0,-450){\makebox(100,10)[l]{\jkffxef}}
{\color{black} \dottedline{3}(-75,-450)(-5,-450)}\put(0,-460){\makebox(100,10)[l]{\jkffxeg}}
{\color{black} \dottedline{3}(-75,-460)(-5,-460)}\put(0,-470){\makebox(100,10)[l]{\jkffxeh}}
{\color{black} \dottedline{3}(-75,-470)(-5,-470)}\put(0,-480){\makebox(100,10)[l]{\jkffxei}}
{\color{black} \dottedline{3}(-75,-480)(-5,-480)}\put(0,-490){\makebox(100,10)[l]{\jkffxej}}
{\color{black} \dottedline{3}(-75,-490)(-5,-490)}\put(0,-500){\makebox(100,10)[l]{\jkffxfa}}
{\color{black} \dottedline{3}(-75,-500)(-5,-500)}\put(0,-510){\makebox(100,10)[l]{\jkffxfb}}

\color{red}
\put(-6,-255){\circle*{2}}
\put(-6,-257){\vector(0,-1){76}}
\put(-8,-257){\linethickness{0.7pt}\line(1,0){0}}
\put(-6,-335){\circle*{2}}

\color{red}
\put(-12,-275){\circle*{2}}
\put(-12,-277){\vector(0,-1){26}}
\put(-14,-277){\linethickness{0.7pt}\line(1,0){0}}
\put(-12,-305){\circle*{2}}

\color{red}
\put(-18,-292){\circle*{2}}
\put(-20,-294){\linethickness{0.7pt}\line(1,0){0}}
\put(-18,-297){\circle*{2}}

\color{green}
\put(-24,-292){\circle*{2}}
\put(-26,-294){\linethickness{0.7pt}\line(1,0){0}}
\put(-24,-297){\circle*{2}}

\color{red}
\put(-30,-295){\circle*{2}}
\put(-30,-297){\vector(0,-1){6}}
\put(-32,-297){\linethickness{0.7pt}\line(1,0){0}}
\put(-30,-305){\circle*{2}}

\color{green}
\put(-36,-305){\circle*{2}}
\put(-36,-303){\vector(0,1){6}}
\put(-38,-297){\linethickness{0.7pt}\line(1,0){0}}
\put(-36,-295){\circle*{2}}

\color{red}
\put(-18,-302){\circle*{2}}
\put(-20,-304){\linethickness{0.7pt}\line(1,0){0}}
\put(-18,-307){\circle*{2}}

\color{green}
\put(-24,-302){\circle*{2}}
\put(-26,-304){\linethickness{0.7pt}\line(1,0){0}}
\put(-24,-307){\circle*{2}}

\color{red}
\put(-42,-305){\circle*{2}}
\put(-42,-307){\vector(0,-1){16}}
\put(-44,-307){\linethickness{0.7pt}\line(1,0){0}}
\put(-42,-325){\circle*{2}}

\color{red}
\put(-6,-115){\circle*{2}}
\put(-6,-117){\vector(0,-1){106}}
\put(-8,-117){\linethickness{0.7pt}\line(1,0){0}}
\put(-6,-225){\circle*{2}}

\color{red}
\put(-12,-135){\circle*{2}}
\put(-12,-137){\vector(0,-1){46}}
\put(-14,-137){\linethickness{0.7pt}\line(1,0){0}}
\put(-12,-185){\circle*{2}}

\color{red}
\put(-18,-152){\circle*{2}}
\put(-20,-154){\linethickness{0.7pt}\line(1,0){0}}
\put(-18,-157){\circle*{2}}

\color{green}
\put(-24,-152){\circle*{2}}
\put(-26,-154){\linethickness{0.7pt}\line(1,0){0}}
\put(-24,-157){\circle*{2}}

\color{red}
\put(-30,-155){\circle*{2}}
\put(-30,-157){\vector(0,-1){16}}
\put(-32,-157){\linethickness{0.7pt}\line(1,0){0}}
\put(-30,-175){\circle*{2}}

\color{red}
\put(-18,-162){\circle*{2}}
\put(-20,-164){\linethickness{0.7pt}\line(1,0){0}}
\put(-18,-167){\circle{4}}

\color{red}
\put(-24,-165){\circle*{2}}
\put(-24,-167){\vector(0,-1){6}}
\put(-26,-167){\linethickness{0.7pt}\line(1,0){0}}
\put(-24,-175){\circle*{2}}

\color{red}
\put(-36,-165){\circle*{2}}
\put(-36,-167){\vector(0,-1){16}}
\put(-38,-167){\linethickness{0.7pt}\line(1,0){0}}
\put(-36,-185){\circle*{2}}

\color{red}
\put(-42,-165){\circle*{2}}
\put(-42,-167){\vector(0,-1){26}}
\put(-44,-167){\linethickness{0.7pt}\line(1,0){0}}
\put(-42,-195){\circle*{2}}

\color{red}
\put(-48,-162){\circle{4}}
\put(-50,-164){\linethickness{0.7pt}\line(1,0){4}}
\put(-48,-167){\circle{4}}

\color{green}
\put(-54,-175){\circle*{2}}
\put(-54,-173){\vector(0,1){16}}
\put(-56,-157){\linethickness{0.7pt}\line(1,0){0}}
\put(-54,-155){\circle*{2}}

\color{green}
\put(-60,-175){\circle*{2}}
\put(-60,-173){\vector(0,1){6}}
\put(-62,-167){\linethickness{0.7pt}\line(1,0){0}}
\put(-60,-165){\circle*{2}}

\color{green}
\put(-66,-185){\circle*{2}}
\put(-66,-183){\vector(0,1){16}}
\put(-68,-167){\linethickness{0.7pt}\line(1,0){0}}
\put(-66,-165){\circle*{2}}

\color{green}
\put(-18,-185){\circle*{2}}
\put(-18,-187){\vector(0,-1){6}}
\put(-20,-187){\linethickness{0.7pt}\line(1,0){0}}
\put(-18,-195){\circle*{2}}

\color{green}
\put(-72,-195){\circle*{2}}
\put(-72,-193){\vector(0,1){26}}
\put(-74,-167){\linethickness{0.7pt}\line(1,0){0}}
\put(-72,-165){\circle*{2}}

\color{red}
\put(-24,-195){\circle*{2}}
\put(-24,-193){\vector(0,1){6}}
\put(-26,-187){\linethickness{0.7pt}\line(1,0){0}}
\put(-24,-185){\circle*{2}}

\color{red}
\put(-12,-195){\circle*{2}}
\put(-12,-197){\vector(0,-1){16}}
\put(-14,-197){\linethickness{0.7pt}\line(1,0){0}}
\put(-12,-215){\circle*{2}}

\color{red}
\put(-6,-375){\circle*{2}}
\put(-6,-377){\vector(0,-1){126}}
\put(-8,-377){\linethickness{0.7pt}\line(1,0){0}}
\put(-6,-505){\circle*{2}}

\color{red}
\put(-12,-412){\circle*{2}}
\put(-14,-414){\linethickness{0.7pt}\line(1,0){0}}
\put(-12,-417){\circle*{2}\hspace{-2\unitlength}\circle{4}}

\color{red}
\put(-18,-415){\circle*{2}}
\put(-18,-417){\vector(0,-1){16}}
\put(-20,-417){\linethickness{0.7pt}\line(1,0){0}}
\put(-18,-435){\circle*{2}}

\color{red}
\put(-24,-415){\circle*{2}\hspace{-2\unitlength}\circle{4}}
\put(-24,-417){\vector(0,-1){16}}
\put(-26,-417){\linethickness{0.7pt}\line(1,0){4}}
\put(-24,-435){\circle*{2}\hspace{-2\unitlength}\circle{4}}

\color{red}
\put(-30,-415){\circle{4}}
\put(-30,-417){\vector(0,-1){6}}
\put(-32,-417){\linethickness{0.7pt}\line(1,0){4}}
\put(-30,-425){\circle{4}}

\color{red}
\put(-12,-422){\circle*{2}}
\put(-14,-424){\linethickness{0.7pt}\line(1,0){0}}
\put(-12,-427){\circle*{2}\hspace{-2\unitlength}\circle{4}}

\color{red}
\put(-36,-425){\circle*{2}}
\put(-36,-427){\vector(0,-1){16}}
\put(-38,-427){\linethickness{0.7pt}\line(1,0){0}}
\put(-36,-445){\circle*{2}}

\color{red}
\put(-42,-425){\circle*{2}\hspace{-2\unitlength}\circle{4}}
\put(-42,-427){\vector(0,-1){16}}
\put(-44,-427){\linethickness{0.7pt}\line(1,0){4}}
\put(-42,-445){\circle*{2}\hspace{-2\unitlength}\circle{4}}

\color{red}
\put(-12,-432){\circle*{2}}
\put(-14,-434){\linethickness{0.7pt}\line(1,0){0}}
\put(-12,-437){\circle*{2}\hspace{-2\unitlength}\circle{4}}

\color{red}
\put(-30,-435){\circle*{2}}
\put(-30,-437){\vector(0,-1){26}}
\put(-32,-437){\linethickness{0.7pt}\line(1,0){0}}
\put(-30,-465){\circle*{2}}

\color{red}
\put(-12,-442){\circle*{2}}
\put(-14,-444){\linethickness{0.7pt}\line(1,0){0}}
\put(-12,-447){\circle*{2}\hspace{-2\unitlength}\circle{4}}

\color{red}
\put(-18,-445){\circle*{2}}
\put(-18,-447){\vector(0,-1){16}}
\put(-20,-447){\linethickness{0.7pt}\line(1,0){0}}
\put(-18,-465){\circle*{2}}

\color{red}
\put(-12,-462){\circle*{2}}
\put(-14,-464){\linethickness{0.7pt}\line(1,0){0}}
\put(-12,-467){\circle*{2}\hspace{-2\unitlength}\circle{4}}

\end{picture}
\]
\hrulefill

\end{figure}

\begin{figure} 
\small
\verbdef\jkffxb$#include <stdlib.h> $
\verbdef\jkffxc$#include <stdio.h> $
\verbdef\jkffxd$ $
\verbdef\jkffxe$struct {int a; int b;} has_ab; $
\verbdef\jkffxf$ $
\verbdef\jkffxg$int main(int argc, char **argv) $
\verbdef\jkffxh${ $
\verbdef\jkffxi$   has_ab.a = 1; $
\verbdef\jkffxj$   has_ab.b = 1; $
\verbdef\jkffxba$   printf("%d ", has_ab.a); $
\verbdef\jkffxbb$   has_ab.a = 2; $
\verbdef\jkffxbc$   has_ab.b = 2; $
\verbdef\jkffxbd$   printf("%d %d\n", has_ab.a, has_ab.b); $
\verbdef\jkffxbe$ $
\verbdef\jkffxbf$} $
\hrulefill
\[
\begin{picture}(420,150)(-40,-150)

\put(0,-10){\makebox(100,10)[l]{\jkffxb}}
\put(0,-20){\makebox(100,10)[l]{\jkffxc}}
\put(0,-30){\makebox(100,10)[l]{\jkffxd}}
\put(0,-40){\makebox(100,10)[l]{\jkffxe}}
\put(0,-50){\makebox(100,10)[l]{\jkffxf}}
\put(0,-60){\makebox(100,10)[l]{\jkffxg}}
\put(0,-70){\makebox(100,10)[l]{\jkffxh}}
{\color{black} \dottedline{3}(-10,-65)(-0,-65)}
\put(0,-80){\makebox(100,10)[l]{\jkffxi}}
{\color{black} \dottedline{3}(-10,-75)(-0,-75)}
\put(0,-90){\makebox(100,10)[l]{\jkffxj}}
{\color{black} \dottedline{3}(-10,-85)(-0,-85)}
\put(0,-100){\makebox(100,10)[l]{\jkffxba}}
{\color{black} \dottedline{3}(-10,-95)(-0,-95)}
\put(0,-110){\makebox(100,10)[l]{\jkffxbb}}
{\color{black} \dottedline{3}(-10,-105)(-0,-105)}
\put(0,-120){\makebox(100,10)[l]{\jkffxbc}}
{\color{black} \dottedline{3}(-10,-115)(-0,-115)}
\put(0,-130){\makebox(100,10)[l]{\jkffxbd}}
{\color{black} \dottedline{3}(-10,-125)(-0,-125)}
\put(0,-140){\makebox(100,10)[l]{\jkffxbe}}
\put(0,-150){\makebox(100,10)[l]{\jkffxbf}}
{\color{black} \dottedline{3}(-10,-145)(-0,-145)}

\put(-10,-65){\color{black}\circle*{2}}
\put(-40,-75){\color{black}\circle*{2}}
\put(-20,-85){\color{black}\circle*{2}}
\put(-40,-95){\color{black}\circle*{2}}
\put(-30,-105){\color{black}\circle*{2}}
\put(-30,-115){\color{black}\circle*{2}}
\put(-40,-125){\color{black}\circle*{2}}
\put(-10,-145){\color{black}\circle*{2}}
\put(-10.0,-67.0){\color{red}\vector(0,-1){76.0}}
{\color{red}\dashline[100]{2}(-10.0,-67.0)(-10.0,-143.0)}
\put(-40.0,-77.0){\color{red}\vector(0,-1){16.0}}
{\color{red}\dashline[100]{2}(-40.0,-77.0)(-40.0,-93.0)}
\put(-20.6,-86.9){\color{blue}\vector(-1,-3){8.7}}
{\color{blue}\dashline[100]{2}(-20.6,-86.9)(-29.4,-113.1)}
\put(-38.6,-96.4){\color{green}\vector(1,-1){7.2}}
{\color{green}\dashline[100]{2}(-38.6,-96.4)(-31.4,-103.6)}
\put(-40.0,-97.0){\color{red}\vector(0,-1){26.0}}
{\color{red}\dashline[100]{2}(-40.0,-97.0)(-40.0,-123.0)}
\put(-30.9,-106.8){\color{red}\vector(-1,-2){8.2}}
{\color{red}\dashline[100]{2}(-30.9,-106.8)(-39.1,-123.2)}
\put(-31.4,-116.4){\color{red}\vector(-1,-1){7.2}}
{\color{red}\dashline[100]{2}(-31.4,-116.4)(-38.6,-123.6)}
\end{picture}
\]
\hrulefill

\end{figure}

\begin{figure} 
\small
\verbdef\jkffxb$#include <stdio.h> $
\verbdef\jkffxc$#include <stdlib.h> $
\verbdef\jkffxd$ $
\verbdef\jkffxe$#define N 5 $
\verbdef\jkffxf$ $
\verbdef\jkffxg$typedef struct _ilist { $
\verbdef\jkffxh$   int val; $
\verbdef\jkffxi$   struct _ilist *next; $
\verbdef\jkffxj$} ilist; $
\verbdef\jkffxba$ $
\verbdef\jkffxbb$static ilist* mklist(int n) $
\verbdef\jkffxbc${ $
\verbdef\jkffxbd$  int i; $
\verbdef\jkffxbe$  ilist *p = NULL; $
\verbdef\jkffxbf$ $
\verbdef\jkffxbg$  for( i=0; i<n; i++ ) { $
\verbdef\jkffxbh$    ilist *t = (ilist *) malloc( sizeof(ilist) ); $
\verbdef\jkffxbi$    t->val = i; $
\verbdef\jkffxbj$    t->next = p; $
\verbdef\jkffxca$    p = t; $
\verbdef\jkffxcb$  } $
\verbdef\jkffxcc$  return p; $
\verbdef\jkffxcd$} $
\verbdef\jkffxce$ $
\verbdef\jkffxcf$static int sumlist(ilist* p) $
\verbdef\jkffxcg${ $
\verbdef\jkffxch$   ilist *q; $
\verbdef\jkffxci$   int  sum = 0; $
\verbdef\jkffxcj$ $
\verbdef\jkffxda$   for( q=p; q!=NULL; q=q->next ) { $
\verbdef\jkffxdb$      sum += q->val; $
\verbdef\jkffxdc$   } $
\verbdef\jkffxdd$   return sum; $
\verbdef\jkffxde$} $
\verbdef\jkffxdf$ $
\verbdef\jkffxdg$ $
\verbdef\jkffxdh$int main(int argc, char **argv)  $
\verbdef\jkffxdi${ $
\verbdef\jkffxdj$  int m,n; $
\verbdef\jkffxea$  ilist *p,*q; $
\verbdef\jkffxeb$ $
\verbdef\jkffxec$  p = mklist( N ); $
\verbdef\jkffxed$  q = mklist( N ); $
\verbdef\jkffxee$  m = sumlist( p ); $
\verbdef\jkffxef$  n = sumlist( q ); $
\verbdef\jkffxeg$ $
\verbdef\jkffxeh$  printf("%d\n", m+n); $
\verbdef\jkffxei$ $
\verbdef\jkffxej$  return 0; $
\verbdef\jkffxfa$ $
\verbdef\jkffxfb$} $
\hrulefill
\[
\begin{picture}(420,510)(-40,-510)

\put(0,-10){\makebox(100,10)[l]{\jkffxb}}
\put(0,-20){\makebox(100,10)[l]{\jkffxc}}
\put(0,-30){\makebox(100,10)[l]{\jkffxd}}
\put(0,-40){\makebox(100,10)[l]{\jkffxe}}
\put(0,-50){\makebox(100,10)[l]{\jkffxf}}
\put(0,-60){\makebox(100,10)[l]{\jkffxg}}
\put(0,-70){\makebox(100,10)[l]{\jkffxh}}
\put(0,-80){\makebox(100,10)[l]{\jkffxi}}
\put(0,-90){\makebox(100,10)[l]{\jkffxj}}
\put(0,-100){\makebox(100,10)[l]{\jkffxba}}
\put(0,-110){\makebox(100,10)[l]{\jkffxbb}}
\put(0,-120){\makebox(100,10)[l]{\jkffxbc}}
{\color{black} \dottedline{3}(-10,-115)(-0,-115)}
\put(0,-130){\makebox(100,10)[l]{\jkffxbd}}
\put(0,-140){\makebox(100,10)[l]{\jkffxbe}}
{\color{black} \dottedline{3}(-10,-135)(-0,-135)}
\put(0,-150){\makebox(100,10)[l]{\jkffxbf}}
\put(0,-160){\makebox(100,10)[l]{\jkffxbg}}
{\color{black} \dottedline{3}(-10,-155)(-0,-155)}
\put(0,-170){\makebox(100,10)[l]{\jkffxbh}}
{\color{black} \dottedline{3}(-10,-165)(-0,-165)}
\put(0,-180){\makebox(100,10)[l]{\jkffxbi}}
{\color{black} \dottedline{3}(-10,-175)(-0,-175)}
\put(0,-190){\makebox(100,10)[l]{\jkffxbj}}
{\color{black} \dottedline{3}(-10,-185)(-0,-185)}
\put(0,-200){\makebox(100,10)[l]{\jkffxca}}
{\color{black} \dottedline{3}(-10,-195)(-0,-195)}
\put(0,-210){\makebox(100,10)[l]{\jkffxcb}}
\put(0,-220){\makebox(100,10)[l]{\jkffxcc}}
{\color{black} \dottedline{3}(-10,-215)(-0,-215)}
\put(0,-230){\makebox(100,10)[l]{\jkffxcd}}
{\color{black} \dottedline{3}(-10,-225)(-0,-225)}
\put(0,-240){\makebox(100,10)[l]{\jkffxce}}
\put(0,-250){\makebox(100,10)[l]{\jkffxcf}}
\put(0,-260){\makebox(100,10)[l]{\jkffxcg}}
{\color{black} \dottedline{3}(-10,-255)(-0,-255)}
\put(0,-270){\makebox(100,10)[l]{\jkffxch}}
\put(0,-280){\makebox(100,10)[l]{\jkffxci}}
{\color{black} \dottedline{3}(-10,-275)(-0,-275)}
\put(0,-290){\makebox(100,10)[l]{\jkffxcj}}
\put(0,-300){\makebox(100,10)[l]{\jkffxda}}
{\color{black} \dottedline{3}(-10,-295)(-0,-295)}
\put(0,-310){\makebox(100,10)[l]{\jkffxdb}}
{\color{black} \dottedline{3}(-10,-305)(-0,-305)}
\put(0,-320){\makebox(100,10)[l]{\jkffxdc}}
\put(0,-330){\makebox(100,10)[l]{\jkffxdd}}
{\color{black} \dottedline{3}(-10,-325)(-0,-325)}
\put(0,-340){\makebox(100,10)[l]{\jkffxde}}
{\color{black} \dottedline{3}(-10,-335)(-0,-335)}
\put(0,-350){\makebox(100,10)[l]{\jkffxdf}}
\put(0,-360){\makebox(100,10)[l]{\jkffxdg}}
\put(0,-370){\makebox(100,10)[l]{\jkffxdh}}
\put(0,-380){\makebox(100,10)[l]{\jkffxdi}}
{\color{black} \dottedline{3}(-10,-375)(-0,-375)}
\put(0,-390){\makebox(100,10)[l]{\jkffxdj}}
\put(0,-400){\makebox(100,10)[l]{\jkffxea}}
\put(0,-410){\makebox(100,10)[l]{\jkffxeb}}
\put(0,-420){\makebox(100,10)[l]{\jkffxec}}
{\color{black} \dottedline{3}(-10,-415)(-0,-415)}
\put(0,-430){\makebox(100,10)[l]{\jkffxed}}
{\color{black} \dottedline{3}(-10,-425)(-0,-425)}
\put(0,-440){\makebox(100,10)[l]{\jkffxee}}
{\color{black} \dottedline{3}(-10,-435)(-0,-435)}
\put(0,-450){\makebox(100,10)[l]{\jkffxef}}
{\color{black} \dottedline{3}(-10,-445)(-0,-445)}
\put(0,-460){\makebox(100,10)[l]{\jkffxeg}}
\put(0,-470){\makebox(100,10)[l]{\jkffxeh}}
{\color{black} \dottedline{3}(-10,-465)(-0,-465)}
\put(0,-480){\makebox(100,10)[l]{\jkffxei}}
\put(0,-490){\makebox(100,10)[l]{\jkffxej}}
\put(0,-500){\makebox(100,10)[l]{\jkffxfa}}
\put(0,-510){\makebox(100,10)[l]{\jkffxfb}}
{\color{black} \dottedline{3}(-10,-505)(-0,-505)}

\put(-20,-435){\color{black}\circle*{2}}
\put(-40,-165){\color{black}\circle*{2}}
\put(-10,-255){\color{black}\circle*{2}}
\put(-30,-445){\color{black}\circle*{2}}
\put(-20,-175){\color{black}\circle*{2}}
\put(-30,-185){\color{black}\circle*{2}}
\put(-30,-275){\color{black}\circle*{2}}
\put(-20,-465){\color{black}\circle*{2}}
\put(-10,-375){\color{black}\circle*{2}}
\put(-40,-195){\color{black}\circle*{2}}
\put(-10,-115){\color{black}\circle*{2}}
\put(-20,-295){\color{black}\circle*{2}}
\put(-40,-215){\color{black}\circle*{2}}
\put(-30,-305){\color{black}\circle*{2}}
\put(-30,-135){\color{black}\circle*{2}}
\put(-10,-225){\color{black}\circle*{2}}
\put(-30,-415){\color{black}\circle*{2}}
\put(-10,-505){\color{black}\circle*{2}}
\put(-30,-325){\color{black}\circle*{2}}
\put(-30,-425){\color{black}\circle*{2}}
\put(-20,-155){\color{black}\circle*{2}}
\put(-10,-335){\color{black}\circle*{2}}
\put(-10,-117){\color{red}\vector(0,-1){106}}
\put(-30,-137){\color{red}\vector(0,-1){46}}
\put(-19,-157){\color{red}\vector(0,-1){16}}
\put(-37,-166){\color{red}\vector(2,-1){16}}
\put(-38,-167){\color{red}\vector(1,-2){8}}
\put(-39,-167){\color{red}\vector(0,-1){26}}
\put(-21,-173){\color{green}\vector(0,1){16}}
\put(-22,-173){\color{green}\vector(-2,1){16}}
\put(-31,-182){\color{green}\vector(-1,2){8}}
\put(-30,-185){\color{green}\vector(-1,-1){7}}
\put(-41,-193){\color{green}\vector(0,1){26}}
\put(-39,-194){\color{red}\vector(1,1){7}}
\put(-40,-197){\color{red}\vector(0,-1){16}}
\put(-10,-257){\color{red}\vector(0,-1){76}}
\put(-30,-277){\color{red}\vector(0,-1){26}}
\put(-20,-295){\color{red}\vector(-1,-1){7}}
\put(-29,-304){\color{green}\vector(1,1){7}}
\put(-30,-307){\color{red}\vector(0,-1){16}}
\put(-10,-377){\color{red}\vector(0,-1){126}}
\put(-30,-417){\color{cyan}\vector(0,-1){6}}
\put(-29,-416){\color{red}\vector(1,-2){8}}
\put(-30,-427){\color{red}\vector(0,-1){16}}
\put(-20,-437){\color{red}\vector(0,-1){26}}
\put(-29,-446){\color{red}\vector(1,-2){8}}
\end{picture}
\]
\hrulefill

\end{figure}

\begin{figure} 
\small
\verbdef\jkffxb$#include <stdlib.h> $
\verbdef\jkffxc$#include <stdio.h> $
\verbdef\jkffxd$ $
\verbdef\jkffxe$static int nfib(int n) $
\verbdef\jkffxf${ $
\verbdef\jkffxg$   int result; $
\verbdef\jkffxh$   if( n < 2 ) { $
\verbdef\jkffxi$     result = 1; $
\verbdef\jkffxj$   } else { $
\verbdef\jkffxba$     int a = nfib( n-1 ); $
\verbdef\jkffxbb$     int b = nfib( n-2 ); $
\verbdef\jkffxbc$     result = a+b; $
\verbdef\jkffxbd$   } $
\verbdef\jkffxbe$   return result; $
\verbdef\jkffxbf$} $
\verbdef\jkffxbg$      $
\verbdef\jkffxbh$  $
\verbdef\jkffxbi$ $
\verbdef\jkffxbj$int main(int argc, char **argv) $
\verbdef\jkffxca${ $
\verbdef\jkffxcb$   int m = nfib( 8 ); $
\verbdef\jkffxcc$    $
\verbdef\jkffxcd$   printf( "%d\n", m ); $
\verbdef\jkffxce$} $
\hrulefill
\[
\begin{picture}(420,240)(-36,-240)

\put(0,0){\makebox(100,10)[l]{}}
\put(0,-10){\makebox(100,10)[l]{\jkffxb}}
\put(0,-20){\makebox(100,10)[l]{\jkffxc}}
\put(0,-30){\makebox(100,10)[l]{\jkffxd}}
\put(0,-40){\makebox(100,10)[l]{\jkffxe}}
\put(0,-50){\makebox(100,10)[l]{\jkffxf}}
{\color{black} \dottedline{3}(-33,-50)(-5,-50)}\put(0,-60){\makebox(100,10)[l]{\jkffxg}}
{\color{black} \dottedline{3}(-33,-60)(-5,-60)}\put(0,-70){\makebox(100,10)[l]{\jkffxh}}
{\color{black} \dottedline{3}(-33,-70)(-5,-70)}\put(0,-80){\makebox(100,10)[l]{\jkffxi}}
{\color{black} \dottedline{3}(-33,-80)(-5,-80)}\put(0,-90){\makebox(100,10)[l]{\jkffxj}}
{\color{black} \dottedline{3}(-33,-90)(-5,-90)}\put(0,-100){\makebox(100,10)[l]{\jkffxba}}
{\color{black} \dottedline{3}(-33,-100)(-5,-100)}\put(0,-110){\makebox(100,10)[l]{\jkffxbb}}
{\color{black} \dottedline{3}(-33,-110)(-5,-110)}\put(0,-120){\makebox(100,10)[l]{\jkffxbc}}
{\color{black} \dottedline{3}(-33,-120)(-5,-120)}\put(0,-130){\makebox(100,10)[l]{\jkffxbd}}
{\color{black} \dottedline{3}(-33,-130)(-5,-130)}\put(0,-140){\makebox(100,10)[l]{\jkffxbe}}
{\color{black} \dottedline{3}(-33,-140)(-5,-140)}\put(0,-150){\makebox(100,10)[l]{\jkffxbf}}
\put(0,-160){\makebox(100,10)[l]{\jkffxbg}}
\put(0,-170){\makebox(100,10)[l]{\jkffxbh}}
\put(0,-180){\makebox(100,10)[l]{\jkffxbi}}
\put(0,-190){\makebox(100,10)[l]{\jkffxbj}}
\put(0,-200){\makebox(100,10)[l]{\jkffxca}}
{\color{black} \dottedline{3}(-33,-200)(-5,-200)}\put(0,-210){\makebox(100,10)[l]{\jkffxcb}}
{\color{black} \dottedline{3}(-33,-210)(-5,-210)}\put(0,-220){\makebox(100,10)[l]{\jkffxcc}}
{\color{black} \dottedline{3}(-33,-220)(-5,-220)}\put(0,-230){\makebox(100,10)[l]{\jkffxcd}}
{\color{black} \dottedline{3}(-33,-230)(-5,-230)}\put(0,-240){\makebox(100,10)[l]{\jkffxce}}

\color{red}
\put(-6,-45){\circle*{2}}
\put(-6,-47){\vector(0,-1){96}}
\put(-8,-47){\linethickness{0.7pt}\line(1,0){0}}
\put(-6,-145){\circle*{2}}

\color{red}
\put(-12,-75){\circle*{2}}
\put(-12,-77){\vector(0,-1){56}}
\put(-14,-77){\linethickness{0.7pt}\line(1,0){0}}
\put(-12,-135){\circle*{2}}

\color{red}
\put(-18,-92){\circle*{2}}
\put(-20,-94){\linethickness{0.7pt}\line(1,0){0}}
\put(-18,-97){\circle*{2}\hspace{-2\unitlength}\circle{4}}

\color{red}
\put(-24,-95){\circle*{2}}
\put(-24,-97){\vector(0,-1){16}}
\put(-26,-97){\linethickness{0.7pt}\line(1,0){0}}
\put(-24,-115){\circle*{2}}

\color{red}
\put(-18,-102){\circle*{2}}
\put(-20,-104){\linethickness{0.7pt}\line(1,0){0}}
\put(-18,-107){\circle*{2}\hspace{-2\unitlength}\circle{4}}

\color{red}
\put(-30,-105){\circle*{2}}
\put(-30,-107){\vector(0,-1){6}}
\put(-32,-107){\linethickness{0.7pt}\line(1,0){0}}
\put(-30,-115){\circle*{2}}

\color{red}
\put(-18,-115){\circle*{2}}
\put(-18,-117){\vector(0,-1){16}}
\put(-20,-117){\linethickness{0.7pt}\line(1,0){0}}
\put(-18,-135){\circle*{2}}

\color{red}
\put(-6,-195){\circle*{2}}
\put(-6,-197){\vector(0,-1){36}}
\put(-8,-197){\linethickness{0.7pt}\line(1,0){0}}
\put(-6,-235){\circle*{2}}

\color{red}
\put(-12,-202){\circle*{2}}
\put(-14,-204){\linethickness{0.7pt}\line(1,0){0}}
\put(-12,-207){\circle*{2}\hspace{-2\unitlength}\circle{4}}

\color{red}
\put(-18,-205){\circle*{2}}
\put(-18,-207){\vector(0,-1){16}}
\put(-20,-207){\linethickness{0.7pt}\line(1,0){0}}
\put(-18,-225){\circle*{2}}

\color{red}
\put(-12,-222){\circle*{2}}
\put(-14,-224){\linethickness{0.7pt}\line(1,0){0}}
\put(-12,-227){\circle*{2}\hspace{-2\unitlength}\circle{4}}

\end{picture}
\]
\hrulefill

\end{figure}

\begin{figure} 
\small
\verbdef\jkffxb$#include <stdlib.h> $
\verbdef\jkffxc$#include <stdio.h> $
\verbdef\jkffxd$ $
\verbdef\jkffxe$static int nfib(int n) $
\verbdef\jkffxf${ $
\verbdef\jkffxg$   int result; $
\verbdef\jkffxh$   if( n < 2 ) { $
\verbdef\jkffxi$     result = 1; $
\verbdef\jkffxj$   } else { $
\verbdef\jkffxba$     int a = nfib( n-1 ); $
\verbdef\jkffxbb$     int b = nfib( n-2 ); $
\verbdef\jkffxbc$     result = a+b; $
\verbdef\jkffxbd$   } $
\verbdef\jkffxbe$   return result; $
\verbdef\jkffxbf$} $
\verbdef\jkffxbg$      $
\verbdef\jkffxbh$  $
\verbdef\jkffxbi$ $
\verbdef\jkffxbj$int main(int argc, char **argv) $
\verbdef\jkffxca${ $
\verbdef\jkffxcb$   int m = nfib( 8 ); $
\verbdef\jkffxcc$    $
\verbdef\jkffxcd$   printf( "%d\n", m ); $
\verbdef\jkffxce$} $
\hrulefill
\[
\begin{picture}(420,240)(-40,-240)

\put(0,-10){\makebox(100,10)[l]{\jkffxb}}
\put(0,-20){\makebox(100,10)[l]{\jkffxc}}
\put(0,-30){\makebox(100,10)[l]{\jkffxd}}
\put(0,-40){\makebox(100,10)[l]{\jkffxe}}
\put(0,-50){\makebox(100,10)[l]{\jkffxf}}
{\color{black} \dottedline{3}(-10,-45)(-0,-45)}
\put(0,-60){\makebox(100,10)[l]{\jkffxg}}
\put(0,-70){\makebox(100,10)[l]{\jkffxh}}
\put(0,-80){\makebox(100,10)[l]{\jkffxi}}
{\color{black} \dottedline{3}(-10,-75)(-0,-75)}
\put(0,-90){\makebox(100,10)[l]{\jkffxj}}
\put(0,-100){\makebox(100,10)[l]{\jkffxba}}
{\color{black} \dottedline{3}(-10,-95)(-0,-95)}
\put(0,-110){\makebox(100,10)[l]{\jkffxbb}}
{\color{black} \dottedline{3}(-10,-105)(-0,-105)}
\put(0,-120){\makebox(100,10)[l]{\jkffxbc}}
{\color{black} \dottedline{3}(-10,-115)(-0,-115)}
\put(0,-130){\makebox(100,10)[l]{\jkffxbd}}
\put(0,-140){\makebox(100,10)[l]{\jkffxbe}}
{\color{black} \dottedline{3}(-10,-135)(-0,-135)}
\put(0,-150){\makebox(100,10)[l]{\jkffxbf}}
{\color{black} \dottedline{3}(-10,-145)(-0,-145)}
\put(0,-160){\makebox(100,10)[l]{\jkffxbg}}
\put(0,-170){\makebox(100,10)[l]{\jkffxbh}}
\put(0,-180){\makebox(100,10)[l]{\jkffxbi}}
\put(0,-190){\makebox(100,10)[l]{\jkffxbj}}
\put(0,-200){\makebox(100,10)[l]{\jkffxca}}
{\color{black} \dottedline{3}(-10,-195)(-0,-195)}
\put(0,-210){\makebox(100,10)[l]{\jkffxcb}}
{\color{black} \dottedline{3}(-10,-205)(-0,-205)}
\put(0,-220){\makebox(100,10)[l]{\jkffxcc}}
\put(0,-230){\makebox(100,10)[l]{\jkffxcd}}
{\color{black} \dottedline{3}(-10,-225)(-0,-225)}
\put(0,-240){\makebox(100,10)[l]{\jkffxce}}
{\color{black} \dottedline{3}(-10,-235)(-0,-235)}

\put(-10,-45){\color{black}\circle*{2}}
\put(-30,-75){\color{black}\circle*{2}}
\put(-20,-95){\color{black}\circle*{2}}
\put(-10,-195){\color{black}\circle*{2}}
\put(-40,-105){\color{black}\circle*{2}}
\put(-20,-205){\color{black}\circle*{2}}
\put(-20,-115){\color{black}\circle*{2}}
\put(-20,-225){\color{black}\circle*{2}}
\put(-30,-135){\color{black}\circle*{2}}
\put(-10,-235){\color{black}\circle*{2}}
\put(-10,-145){\color{black}\circle*{2}}
\put(-10,-47){\color{red}\vector(0,-1){96}}
\put(-30,-77){\color{red}\vector(0,-1){56}}
\put(-20,-97){\color{red}\vector(0,-1){16}}
\put(-38,-105){\color{red}\vector(2,-1){16}}
\put(-20,-116){\color{red}\vector(-1,-2){8}}
\put(-10,-197){\color{red}\vector(0,-1){36}}
\put(-20,-207){\color{red}\vector(0,-1){16}}
\end{picture}
\]
\hrulefill

\end{figure}



\section{Discussion}

In this section we would like to discuss our approach to parallel
programming and contrast it with two alternatives: Automatic 
(compiler) parallelisation and explicit thread parallelism.

While the information provided by Embla can be used for any hand 
parallelization strategy, it appears to be especially suitable to
combine with a fork-join framework. In this case, the programmer 
simply annotates function calls that should be made asynchronously 
and in addition determines the join points. For such programs we
know that in the absence of dependencies petween parallel parts,
the programs are free of both deadlocks and race conditions and are 
furthermore deterministic (if the sequential version of the program 
is deterministic).

Of course, since Embla is a testing tool, it may be the case that not
all relevant imputs have been used, allowing a dependency to slip
though the net. This will manifest itself as a bug in the parallel
program, possibly in a nondeterministic manner. However, the root
cause of the bug is the missed dependency, which is deterministic and
which can be found by running Embla on the sequential program 
with the offending input. Thus,
while Embla does not give proof of absence of dependencies (as a
static checker would), it reduces the problem of debugging a
multithreaded program to the problem of debugging a sequential
program. Here all of the standard testing machinery (regression 
testing, test coverage tools, \ldots) can be used.


% -*- eval: (local-set-key "\M-q" 'undefined) -*-
%
% The line above will probably make Emacs ask if it is ok to evaluate the
% expression.  Answer y.

\section{Related Work}

\section{Future Work}

\comment{Measure time.  Combine with profiling.}

\comment{Combine data from multiple runs.}

\comment{Independent paths crossing procedure boundaries.}

\comment{Top list with parallelisable paths.}


\bibliographystyle{plain}
\bibliography{embib}

\end{document}
