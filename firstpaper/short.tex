\documentclass{acm_proc_article-sp}

\usepackage{epic}
\usepackage{color}
\usepackage{verbdef}
\usepackage{alltt}

\newcommand{\comment}[1]{\textit{[ #1 ]}}
\newenvironment{comment_env}
  {\begin{itshape}}
  {\end{itshape}}

\begin{document}

\title{Embla -- Data Dependence Profiling for Parallel Programming }
\subtitle{Extended Abstract}
\author{Karl-Filip Fax\'en, Lars Albertsson, Konstantin Popov, Sverker Janson\\
       Swedish Institute of Computer Science\\
       Box 1263\\
       SE-164 29 Kista\\
       Sweden\\
       \{kff,lalle,kost,sverker\}@sics.se}
\date{}
\maketitle

\begin{abstract}

With the proliferation of multicore processors, there is an urgent need for
tools and methodologies supporting parallelization of existing
applications.  In this paper we present a novel tool for aiding
programmers in parallelizing programs. The tool, Embla, is based on the
Valgrind framework, and allows the user to
discover the data dependencies in a sequential program, thereby exposing
opportunities for parallelization.  Embla performs a dynamic analysis,
and records dependencies as they
arise during program execution.  It reports an optimistic view of
parallelizable sequences, and ignores dependencies that do not arise during
execution.  In contrast to static analysis tools,
which by necessity make conservative approximation, Embla is able to find
more parallelism in sequential programs, and relies on the programmer to
transform the program in a correct manner. 
Moreover, since the tool instruments the machine code of the program,
it is largely language independent. 

\end{abstract}

% Intro start ---------------------

\section{Introduction}

Parallel programming is no longer optional.  To enjoy continued
performance gains with future generation multicore processors,
application developers must parallelize all software, old and
new~\cite{TEL95,ONHWC96,KAB03,VIAVAC05}.  For scalable parallel
performance, program execution must be divided into large numbers of
independent tasks that can be scheduled on available cores and hardware
threads by runtime systems.  For some classes of programs, static
analysis and automatic parallelization is feasible~\cite{KA02}, but with
the current state-of-the-art, most software requires manual
parallelization.  Our work aims to help developers find the potential
for parallelism in programs, in particular by providing efficient tool
support.  In this paper, we present a data dependence profiling approach
to the parallelization problem, an efficient algorithm to project data
dependencies onto relevant parts of the program code, and its
implementation, the tool Embla.

\begin{figure}
\small
\hrulefill
\[
\begin{minipage}[t]{3cm}
\begin{alltt}
   p();
   q();
   r();
\end{alltt}
\end{minipage}
\begin{minipage}[t]{3cm}
\begin{alltt}
   spawn p();
   q();
   sync;
   r();
\end{alltt}
\end{minipage} 
\]
\hrulefill
\caption{Example of Fork/join parallelism.}
\label{fforkjoin}
\end{figure}

We will focus on parallelization by introducing procedure-level
fork-join parallelism. The fork-join framework was first introduced
in~\cite{Conway63} and used in many parallel programming environments,
including Cilk~\cite{BJKLR96}, the Java fork/join
framework~\cite{Lea00}, OpenMP~\cite{DM98}, and the Filaments
package~\cite{LF00}. Consider the program fragment in
Figure~\ref{fforkjoin} (left):
Suppose that the calls to {\tt p()} and {\tt q()} are independent,
but that the call to {\tt r()} depends on the earlier calls. Then
the call to {\tt p()} can be
executed in parallel with
the call to {\tt q()}, as shown to the right.
Here we assume the availability of a construct {\tt spawn} to start
the call in parallel and {\tt sync} to wait for all {\tt spawn}'d
activities to terminate (cf.~\cite{BJKLR96}).
% As long as {\tt p()} and {\tt q()} are
% independent, the parallel program will produce identical results to
% the sequential version.  Therefore it is sufficient to understand
% (debug, verify, \ldots) the sequential program; everything except
% performance carries over to the parallel version.

In order to use this approach, one must find independent
procedure calls.  The availability of independent
calls in the program depends on the algorithms used and can be further
limited by sequential programming artifacts, such as re-use of
variables and sequential book-keeping in an otherwise parallelizable
algorithm.  Data dependence information can potentially help
identify and remove such obstacles to parallel execution, but this
will not be further discussed here.

Parallelizing compilers mostly target loop parallelization based on
static data dependence analysis methods~\cite{KA02}.  Such analysers are
by necessity conservative, and use
approximations that are always safe.  
% Good precision comes at a high
% computational cost, especially when analysing large programs.
Analysing more general code, e.g., with pointers, remains a major
challenge~TODOREF\cite{}.  Consequently, it has proved difficult to
parallelize programs automatically, and most production codes are either
written in an explicitly parallel way or rely on speculative, run-time
parallelization techniques~\cite{PO03,CL03}.

In contrast, Embla
observes the actual data dependencies that occur during program
execution, project them onto relevant program parts, and interpret the
lack of a runtime data dependency as an indication that the program
parts involved are likely to be independent.
Developers will be responsible for selecting
program inputs that generate representative program executions with
good coverage.  In the future, Embla will measure the potential speed
improvement for parallelizing independent program parts and produce a
report with suggested program transforms that will yield the maximum
benefit.  The reporting mechanism is future work, however.  This paper
presents the mechanism for collecting runtime data dependence information.

Should a dependency remain undetected, it might manifest itself as a 
difference in behaviour between the parallel and sequential versions of the
program for some input. Rerunning the sequential program under
Embla with the offending input will, however, yield the missing dependency.
Thus, debugging the parallel program is reduced to debugging the sequential 
program, in contrast to the case of ad hoc construction of multithreaded
code.

In addition to data dependencies, and Embla could easily be extended to
deal with I/O dependencies that limit parallelism them. We will, however,
leave these extensions for future work.

% End intro --------------------

% Begin using -----------------

\section{Using Embla}

\begin{figure} 
\small
\verbdef\jkffxb$#include <stdlib.h> $
\verbdef\jkffxc$#include <stdio.h> $
\verbdef\jkffxd$ $
\verbdef\jkffxe$static void inc(int *p) $
\verbdef\jkffxf${ $
\verbdef\jkffxg$   *p=*p+1; $
\verbdef\jkffxh$} $
\verbdef\jkffxi$ $
\verbdef\jkffxj$int main(int argc, char **argv) $
\verbdef\jkffxba${ $
\verbdef\jkffxbb$   int *q=NULL,n=0; $
\verbdef\jkffxbc$ $
\verbdef\jkffxbd$   q = (int*) malloc( sizeof(int) ); $
\verbdef\jkffxbe$   inc(q); $
\verbdef\jkffxbf$   inc(&n); $
\verbdef\jkffxbg$   inc(q); $
\verbdef\jkffxbh$   printf( "%d\n", *q+n ); $
\verbdef\jkffxbi$   q = (int*) malloc( sizeof(int) ); $
\verbdef\jkffxbj$   return q==NULL; $
\verbdef\jkffxca$} $
\hrulefill
\[
\begin{picture}(420,200)(-65,-200)

\put(-65,-10){\makebox(15,10)[r]{\it 1:}}%
\put(0,-10){\makebox(100,10)[l]{\jkffxb}}
\put(-65,-20){\makebox(15,10)[r]{\it 2:}}%
\put(0,-20){\makebox(100,10)[l]{\jkffxc}}
\put(-65,-30){\makebox(15,10)[r]{\it 3:}}%
\put(0,-30){\makebox(100,10)[l]{\jkffxd}}
\put(-65,-40){\makebox(15,10)[r]{\it 4:}}%
\put(0,-40){\makebox(100,10)[l]{\jkffxe}}
\put(-65,-50){\makebox(15,10)[r]{\it 5:}}%
\put(0,-50){\makebox(100,10)[l]{\jkffxf}}
{\color{black} \dottedline{3}(-10,-45)(0,-45)}
\put(-65,-60){\makebox(15,10)[r]{\it 6:}}%
\put(0,-60){\makebox(100,10)[l]{\jkffxg}}
\put(-65,-70){\makebox(15,10)[r]{\it 7:}}%
\put(0,-70){\makebox(100,10)[l]{\jkffxh}}
{\color{black} \dottedline{3}(-10,-65)(0,-65)}
\put(-65,-80){\makebox(15,10)[r]{\it 8:}}%
\put(0,-80){\makebox(100,10)[l]{\jkffxi}}
\put(-65,-90){\makebox(15,10)[r]{\it 9:}}%
\put(0,-90){\makebox(100,10)[l]{\jkffxj}}
\put(-65,-100){\makebox(15,10)[r]{\it 10:}}%
\put(0,-100){\makebox(100,10)[l]{\jkffxba}}
{\color{black} \dottedline{3}(-10,-95)(0,-95)}
\put(-65,-110){\makebox(15,10)[r]{\it 11:}}%
\put(0,-110){\makebox(100,10)[l]{\jkffxbb}}
{\color{black} \dottedline{3}(-10,-105)(0,-105)}
\put(-65,-120){\makebox(15,10)[r]{\it 12:}}%
\put(0,-120){\makebox(100,10)[l]{\jkffxbc}}
\put(-65,-130){\makebox(15,10)[r]{\it 13:}}%
\put(0,-130){\makebox(100,10)[l]{\jkffxbd}}
{\color{black} \dottedline{3}(-10,-125)(0,-125)}
\put(-65,-140){\makebox(15,10)[r]{\it 14:}}%
\put(0,-140){\makebox(100,10)[l]{\jkffxbe}}
{\color{black} \dottedline{3}(-10,-135)(0,-135)}
\put(-65,-150){\makebox(15,10)[r]{\it 15:}}%
\put(0,-150){\makebox(100,10)[l]{\jkffxbf}}
{\color{black} \dottedline{3}(-10,-145)(0,-145)}
\put(-65,-160){\makebox(15,10)[r]{\it 16:}}%
\put(0,-160){\makebox(100,10)[l]{\jkffxbg}}
{\color{black} \dottedline{3}(-10,-155)(0,-155)}
\put(-65,-170){\makebox(15,10)[r]{\it 17:}}%
\put(0,-170){\makebox(100,10)[l]{\jkffxbh}}
{\color{black} \dottedline{3}(-10,-165)(0,-165)}
\put(-65,-180){\makebox(15,10)[r]{\it 18:}}%
\put(0,-180){\makebox(100,10)[l]{\jkffxbi}}
{\color{black} \dottedline{3}(-10,-175)(0,-175)}
\put(-65,-190){\makebox(15,10)[r]{\it 19:}}%
\put(0,-190){\makebox(100,10)[l]{\jkffxbj}}
{\color{black} \dottedline{3}(-10,-185)(0,-185)}
\put(-65,-200){\makebox(15,10)[r]{\it 20:}}%
\put(0,-200){\makebox(100,10)[l]{\jkffxca}}
{\color{black} \dottedline{3}(-10,-195)(0,-195)}

\put(-40,-165){\color{black}\circle*{2}}
\put(-10,-45){\color{black}\circle*{2}}
\put(-30,-175){\color{black}\circle*{2}}
\put(-30,-185){\color{black}\circle*{2}}
\put(-10,-65){\color{black}\circle*{2}}
\put(-10,-95){\color{black}\circle*{2}}
\put(-40,-105){\color{black}\circle*{2}}
\put(-10,-195){\color{black}\circle*{2}}
\put(-30,-125){\color{black}\circle*{2}}
\put(-20,-135){\color{black}\circle*{2}}
\put(-40,-145){\color{black}\circle*{2}}
\put(-20,-155){\color{black}\circle*{2}}
{\color{red}\dashline[200]{3}(-10.0,-48.0)(-10.0,-62.0)}%
{\color{red}\dashline[200]{3}(-10.0,-62.0)(-9.0,-59.0)}%
{\color{red}\dashline[200]{3}(-10.0,-62.0)(-11.0,-59.0)}%
{\color{red}\dashline[200]{3}(-10.0,-98.0)(-10.0,-192.0)}%
{\color{red}\dashline[200]{3}(-10.0,-192.0)(-9.0,-189.0)}%
{\color{red}\dashline[200]{3}(-10.0,-192.0)(-11.0,-189.0)}%
{\color{blue}\dashline[200]{3}(-38.7,-107.7)(-31.3,-122.3)}%
{\color{blue}\dashline[200]{3}(-31.3,-122.3)(-31.8,-119.2)}%
{\color{blue}\dashline[200]{3}(-31.3,-122.3)(-33.6,-120.1)}%
{\color{red}\dashline[200]{3}(-40.0,-108.0)(-40.0,-142.0)}%
{\color{red}\dashline[200]{3}(-40.0,-142.0)(-39.0,-139.0)}%
{\color{red}\dashline[200]{3}(-40.0,-142.0)(-41.0,-139.0)}%
{\color{red}\dashline[200]{3}(-27.9,-127.1)(-22.1,-132.9)}%
{\color{red}\dashline[200]{3}(-22.1,-132.9)(-23.5,-130.1)}%
{\color{red}\dashline[200]{3}(-22.1,-132.9)(-24.9,-131.5)}%
{\color{red}\dashline[200]{3}(-29.1,-127.8)(-20.9,-152.2)}%
{\color{red}\dashline[200]{3}(-20.9,-152.2)(-20.9,-149.0)}%
{\color{red}\dashline[200]{3}(-20.9,-152.2)(-22.8,-149.6)}%
{\color{red}\dashline[200]{3}(-30.7,-127.9)(-39.3,-162.1)}%
{\color{red}\dashline[200]{3}(-39.3,-162.1)(-37.6,-159.4)}%
{\color{red}\dashline[200]{3}(-39.3,-162.1)(-39.5,-158.9)}%
{\color{cyan}\dashline[200]{3}(-30.0,-128.0)(-30.0,-172.0)}%
{\color{cyan}\dashline[200]{3}(-30.0,-172.0)(-29.0,-169.0)}%
{\color{cyan}\dashline[200]{3}(-30.0,-172.0)(-31.0,-169.0)}%
{\color{red}\dashline[200]{3}(-20.0,-138.0)(-20.0,-152.0)}%
{\color{red}\dashline[200]{3}(-20.0,-152.0)(-19.0,-149.0)}%
{\color{red}\dashline[200]{3}(-20.0,-152.0)(-21.0,-149.0)}%
{\color{green}\dashline[200]{3}(-20.7,-137.9)(-29.3,-172.1)}%
{\color{green}\dashline[200]{3}(-29.3,-172.1)(-27.6,-169.4)}%
{\color{green}\dashline[200]{3}(-29.3,-172.1)(-29.5,-168.9)}%
{\color{red}\dashline[200]{3}(-40.0,-148.0)(-40.0,-162.0)}%
{\color{red}\dashline[200]{3}(-40.0,-162.0)(-39.0,-159.0)}%
{\color{red}\dashline[200]{3}(-40.0,-162.0)(-41.0,-159.0)}%
{\color{red}\dashline[200]{3}(-22.7,-156.3)(-37.3,-163.7)}%
{\color{red}\dashline[200]{3}(-37.3,-163.7)(-34.2,-163.2)}%
{\color{red}\dashline[200]{3}(-37.3,-163.7)(-35.1,-161.4)}%
{\color{green}\dashline[200]{3}(-21.3,-157.7)(-28.7,-172.3)}%
{\color{green}\dashline[200]{3}(-28.7,-172.3)(-26.4,-170.1)}%
{\color{green}\dashline[200]{3}(-28.7,-172.3)(-28.2,-169.2)}%
{\color{green}\dashline[200]{3}(-37.9,-167.1)(-32.1,-172.9)}%
{\color{green}\dashline[200]{3}(-32.1,-172.9)(-33.5,-170.1)}%
{\color{green}\dashline[200]{3}(-32.1,-172.9)(-34.9,-171.5)}%
{\color{red}\dashline[200]{3}(-30.0,-178.0)(-30.0,-182.0)}%
{\color{red}\dashline[200]{3}(-30.0,-182.0)(-29.0,-179.0)}%
{\color{red}\dashline[200]{3}(-30.0,-182.0)(-31.0,-179.0)}%
\end{picture}
\]
\hrulefill

\caption{Example program with dependency graph} \label{ffirstex}
\end{figure}

To get a feeling for what dependency profiling is and what Embla can do, 
let us turn to the admittedly contrived program in Figure~\ref{ffirstex}
were we see, from left to right, line numbers, data dependency 
arrows and source lines. 

A data dependence is a pair
of references, not both reads, to overlapping memory
locations with no intervening write. We will refer to these
references as the {\em endpoints} of the dependence.
For instance, in the figure, 
there is an arrow from line 13 to line 14 corresponding to
the assignment to {\tt q} (the {\em early} endpoint) followed by its use 
as an argument in {\tt inc(q)} (the {\em late} endpoint). Embla
internally distinguishes between flow (RAW), anti (WAR) and output (WAW) 
dependencies, but in the figures in this paper, we do not make this
distinction. Embla can be instructed to show dependency types, which can be
useful for figuring out the reasons for individual dependencies.

For each of the dependency arrows in the figure that 
we have discussed up to now, the endpoints have been part of the 
code for {\tt main}
itself. Embla also tracks references made in function calls. For
instance, there is a flow dependence from line 14 to line 16
representing the write in the first invocation of {\tt inc} to the 
{\tt malloc}'d area pointed to by {\tt q} and the subsequent read 
of the same location by a later invocation of {\tt inc}. 
Note that these dependencies 
are reported as pertaining to {\tt main} rather than {\tt inc},
although the endpoints are part of the latter function. 
But the importance of the dependence is that, in {\tt main}, the calls
on line 14 and 16 can not be made in parallel.

The dependency given with a dotted arrow 
(from line 13 to line 18) is due to manipulation of administrative 
data structures by {\tt malloc}. If taken at face value such dependencies will
serialize all calls to {\tt malloc}, but fortunately, the exact order
of memory allocations is not important. If the 
parallelized version of the program uses a thread safe 
implementation of {\tt malloc} these dependencies are irrelevant and
can be ignored. Embla maintains a suppression list, with functions that behave 
in this way.



% End using ------------------

% Begin algorithm -------------------

\section{The Dependence Attribution Algorithm}

\begin{figure} \small
\hrulefill
\[
\begin{picture}(160,60)(70,15)
\put(120,65){\makebox(60,10)[c]{\it A:\ \tt main}}
\put(150,65){\line(-2,-1){50}}
\put(150,65){\line( 0,-1){25}}
\put(150,65){\line( 2,-1){50}}
\put(95,45){\makebox(20,10)[r]{\it 14}}
\put(150,45){\makebox(20,10)[l]{\it 15}}
\put(185,45){\makebox(20,10)[l]{\it 16}}
\put(70,50){\makebox(20,10)[r]{\ldots}}
\put(210,50){\makebox(20,10)[l]{\ldots}}
\put(170,30){\makebox(60,10)[c]{\it D:\ \tt inc}}
\put(120,30){\makebox(60,10)[c]{\it C:\ \tt inc}}
\put(70,30){\makebox(60,10)[c]{\it B:\ \tt inc}}
\put(70,15){\makebox(60,10)[cb]{{\tt *q=}\ \ldots}}
\put(170,15){\makebox(60,10)[cb]{\ldots\ {\tt *q}\ \ldots}}
\end{picture}
\]
\hrulefill
\caption{Part of the call tree of Example 1, edges are annotated 
with the line number of the corresponding call.} 
\label{ffextree}
\end{figure}

% To be useful to the programmer, the instruction level dependencies detected 
% by Embla must be mapped to source level. As discussed above, the dependency
% that Embla reports is not always in the same function as the dependency endpoints
% Embla is designed to help programmers 
% rewrite source code, so we need to project 
% the dependencies discovered in the dynamic instruction stream onto the 
% source code in such a way as to inform the programmer about legal code 
% transformations. Since we are dealing with transformations on the level of
% procedure calls, we need to reflect the dependencies we find to that level 
% as well. That is, we always want to know the dependencies between different 
% lines in the same procedure.
We are interested in dependencies between program statements, and in 
this section we discuss how to compute these dependencies efficiently from
the instruction level dependencies directly observed by Embla.

\subsection{The execution tree}

Consider the {\em dynamic instruction stream} $S$ where each element 
corresponds to the execution
of an instruction. We can view this stream as a tree where the leaves are 
events that do not correspond to call or return instructions, and each 
non-leaf node is a call
event followed by a sequence of nodes and a return event. We call
this tree the {\em execution tree} of $S$. 
Each node corresponds to some execution of a procedure body.
Figure~\ref{ffextree} shows part of
the execution tree of Figure~\ref{ffirstex} where we have omitted the 
leaves. The nodes marked {\it B}, {\it C}, and {\it D} correspond to 
the three consecutive calls to {\tt inc} at lines 14--16 of {\tt main}.

The transformations we target will execute {\em siblings}, subtrees with
the same parent, in parallel.  Hence, we 
are interested in dependencies between siblings.  Sibling dependencies
arise from the 
dependencies in the instruction stream; if $M$ and $N$ are siblings and
there is an instruction level dependency from an instruction in $M$ 
to an instruction in $N$ we have a (tree) dependency between $M$ and $N$
and we call $M$ the {\em source} and $N$ the {\em target} of the dependency.

In fact, each instruction level dependency yields exactly one dependency
between siblings in the execution tree since there is only one node
in the execution tree  
where the two events fall in two distinct children (siblings of each other)
that could potentially be rearranged. We call that node the {\em nearest
common ancestor} (NCA) of the endpoints of the instruction level dependency.
Note that either or both of the children could be a leaf, in which case the
corresponding dependency endpoint would be direct.
For example, in figure~\ref{ffextree}
the dependency between the write in {\it B} and the read in {\it D} yields
only the dependence between the nodes {\it B} and {\it D} and the NCA of
{\it B} and {\it D} is {\it A}. 

The dependence is then reported as a dependence between the source lines
associated with the source and target nodes, respectively; in the example
between lines 14 and 16 in {\tt main}.

\newcommand{\tracepile}{trace pile}

\subsection{Computing dependencies}

Embla uses two main data structures: The \tracepile, which implements 
the execution tree, and the memory table which maps addresses to tree
nodes corresponding to the last write and subsequent reads of that
location. For each reference, we look up the data address in the memory 
table. If the reference is a read, we use the latest write to generate
a flow dependency (RAW). If it is a write, we use the latest write to 
construct an output dependence (WAW) as well as all reads since that 
write to construct anti dependencies (WAR).

If there are several reads with no intervening write, a subsequent write
(anti) depends on all of them. Since the reads do not depend on each other,
we need to keep track of all of them in the memory table to generate the
anti dependency edges explicitly. Since the write depends on them all and
all subsequent references depend on the write, the read list can then be
deallocated.

The \tracepile\ contains the nodes of the execution tree in the same order
as in the instruction stream. For each node {\tt n}, {\tt n.line} is the 
source line associated with the procedure call 
corresponding to {\tt n}, {\tt n.parent} is the parent node in the execution tree 
and, if {\tt n} on the path between the root node of the tree and the most recent 
event (leaf node), {\tt n.next} is the last child of {\tt n} (the node one step 
closer to the most recent one), so {\tt n.next.parent} = {\tt n}. This
path corresponds to the call stack; {\tt n.next} is the stack frame on
top of {\tt n}.

\begin{figure}
\small
\hrulefill
\begin{verbatim}
   DependenceEdge( oldEvent, currEvent ) {
       oldLine = oldEvent.line;
       ncaNode = oldEvent.node;
       while( ncaNode is not on stack ) {
           oldLine = ncaNode.line;
           ncaNode = ncaNode.parent;
       }
       if( ncaNode != currEvent.node )
           currLine = ncaNode.next.line;
       else
           currLine = currEvent.line;
       return ( oldLine, currLine );
    }
\end{verbatim}
\hrulefill
\caption{Constructing the dependence edge from the events corresponding
to a previous and a new reference}
\label{fdepedge}
\end{figure}    

For each instruction level dependency found using the memory table,
a source level dependency is computed using the algorithm in 
Figure~\ref{fdepedge}. Here we
have made use of the fact that the NCA must be part of the path from the
late instruction level endpoint to the root node in the execution tree. 
Thus
we can search from the early endpoint towards the root; the first node
on the stack is the NCA.

Path compression can be used to decrease the number of iterations of the 
{\tt while} loop.
Every node {\tt n} that is visited but is not on the stack can have its parent set
to its closest ancestor on the stack. We conjecture that this reduces the 
complexity of the algorithm to essentially constant time. 
% We can however do even better since the output of the tree dependence 
% calculation does only depend on the immediate
% descendants of nodes on the stack. Thus every subtree $T$ of the call tree, 
% including the events at its leaves, such
% that the root of $T$ is not on the stack can be represented by the root of $T$
% alone.


We can do better than path compression by {\em compacting} the \tracepile. 
Once a procedure call has returned, we will not distinguish between 
different events during its (completed) execution. We will only use the
(identical) {\tt parent} fields of the nodes to look for the NCA. We 
thus periodically compact the \tracepile\ by updating the pointers in 
the memory table by following the {\tt parent} fields and reclaim the no longer
referenced tree nodes.




An important aspect of compaction is that when a subtree of the dynamic
call tree is compacted to be represented by its root node, the memory table
needs to be updated to avoid dangling pointers to eliminated events. All
pointers are then forwarded to point to the representative nodes. In this
process, pointers to previously distinct events now may point at the same 
tree nodes. In this case it is unnecessary to represent more than one 
copy of each pointer, thus compacting the read lists. This optimization is 
crucial in practice.



% End algorithm -----------------------

\section{Implementation and Preliminary Experiments}

Embla is based on instrumented execution of binary code. Although
our examples of profiling output use a high level language (C),
the profiling itself is on instruction level, followed by 
mapping the information to source level using debugging information 
in the standard way.

Embla uses the Valgrind instrumentation infrastructure, so there
is no offline code rewriting; the Embla tool behaves like an emulator
of the hardware.

\bibliographystyle{plain}
\bibliography{embib}

\end{document}
