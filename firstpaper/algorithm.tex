\section{Dependence Attribution}



\begin{figure} \small
\hrulefill
\[
\begin{picture}(160,90)(70,10)
\put(120,90){\makebox(60,10)[c]{\it A:\ \tt main}}
\put(150,90){\line(-1,-1){50}}
\put(150,90){\line( 0,-1){50}}
\put(150,90){\line( 1,-1){50}}
\put(100,60){\makebox(20,10)[r]{\it 14}}
\put(150,60){\makebox(20,10)[l]{\it 15}}
\put(180,60){\makebox(20,10)[l]{\it 16}}
\put(80,50){\makebox(20,10)[r]{\ldots}}
\put(200,50){\makebox(20,10)[l]{\ldots}}
\put(170,30){\makebox(60,10)[c]{\it D:\ \tt inc}}
\put(120,30){\makebox(60,10)[c]{\it C:\ \tt inc}}
\put(70,30){\makebox(60,10)[c]{\it B:\ \tt inc}}
\put(70,10){\makebox(60,10)[c]{{\tt *q=}\ \ldots}}
\put(170,10){\makebox(60,10)[c]{\ldots\ {\tt *q}\ \ldots}}
\end{picture}
\]
\hrulefill
\caption{Part of the call tree of Example 1, edges are annotated 
with the line number of the corresponding call.} 
\label{ffextree}
\end{figure}

Tracing programs to find dependencies is by no means a new endavour. Most 
earlier efforts have however been aimed at studying different hardware 
enhancements. In that context, the desired result is the execution time 
under various hardware assumptions. Thus it was sufficient to keep track
of the execution of the dynamic instruction stream, including dependencies
between instructions; mapping the dependencies to source code was unnecessary.

It is possible to find the minimal execution time by keeping track of the 
earliest possible execution
time of each instruction in the dynamic instruction stream. For each instruction 
that writes to a memory location or a register, that location or register is 
associated with the execution time of the writing instruction. The earliest
execution time of an intruction that reads memory or registers is taken as the
maximum of the time stamps of the items read plus the latency of the instruction 
itself.


A tool like Embla, on the other hand, that is designed to help programmers 
rewrite source code, poses different demands. Here, we need to project 
the dependencies discovered in the dynamic instruction stream onto the 
source code in such a way as to inform the programmer about legal code 
transformations. Since we are dealing with transformations on the level of
procedure calls, we need to reflect the dependencies we find to that level 
as well. That is, we always want to know the dependencies between different 
lines in the same procedure.

\subsection{From dependency endpoints to dependency edges: the call tree}

Consider the program from Figure~\ref{ffirstex} where we have three consecutive
calls to {\tt inc} in lines 14--16. Figure~\ref{ffextree} gives the relevant 
part of the dynamic call tree. The flow dependence between line 14 and 16 is due 
to the write in {\tt inc} from the first call and the read from the second call.
Note that the same instructions in other calling contexts give rise to 
different dependencies, eg the flow dependence from line 15 (call to {\tt inc})
to line 17 (call to {\tt printf}).

We accomplish this mapping by associating each memory location with a dynamic
instruction execution event and we also keep track of how these events 
relate to the dynamic call tree. Thus, for a memory reference event $E$ we 
look up its data address to find the previous reference event $E'$ to that 
location. Then we find the {\em nearest common anscestor} (NCA) of the call
tree nodes of $E$ and $E'$ (for instance, the NCA of {\it B} and {\it D} is
{\it A} in Figure~\ref{ffextree}). The NCA can be calculated from the previous
reference event $E'$ itself by noting that all anscestors of the current
reference event $E$ correspond to frames on the current call stack. If we mark 
such nodes, we can search from $E'$ 
towards the root of the call tree; the first node we reach that is on the 
stack is the NCA.
The line number (in the function corresponding to the NCA) of the source or 
destination of the dependence is the line number of the last edge on the path 
from the relevant event to the NCA (14 and 16, respectively, in 
Figure~\ref{ffextree}).


\begin{figure}
\small
\hrulefill
\begin{verbatim}
   DependenceEdge( oldEvent, currEvent ) {
       oldLine = oldEvent.line;
       ncaNode = oldEvent.node;
       while( ncaNode is not on stack ) {
           oldLine = ncaNode.line;
           ncaNode = ncaNode.parent;
       }
       if( ncaNode != currEvent.node )
           currLine = ncaNode.nextOnStack.line;
       else
           currLine = currEvent.line;
       return ( oldLine, currLine );
    }
\end{verbatim}
\hrulefill
\caption{Constructing the dependence edge from the events corresponding
to a previous and a new reference}
\label{fdepedge}
\end{figure}    

The algorithm for computing the dependence edge given a data address and 
the current reference event is given in Figure~\ref{fdepedge}. Here an 
event represents the execution of a memory reference and a node corresponds 
to a procedure invocation. An event {\tt e} has two attributes: {\tt e.line} 
is the source line corresponding to {\tt e} and {\tt e.node} is the procedure 
invocation that {\tt e} is part of. Similarly, if {\tt n} is a node, then 
{\tt n.line} is the source line associated with the procedure call 
corresponding to {\tt n}, {\tt n.parent} is the parent node in the call tree 
and, if {\tt n} is in the stack, {\tt n.nextOnStack} is the node on top of 
{\tt n} in the stack ({\tt n.nextOnStack.parent} = {\tt n}).

Path compression \cite{GF64} can be used to decrease the number of iterations 
of the {\tt while} loop.
Every node {\tt n} that is visited but is not on the stack can have its parent set 
to its closest ancestor on the stack. We conjecture that this reduces the 
complexity of the algorithm to essentially constant time. We can however do even 
better since the output of the edge calculation does only depend on the immediate
descendants of nodes on the stack. Thus every subtree $T$ of the call tree, 
including the events at its leaves, such
that the root of $T$ is not on the stack can be represented by the root of $T$
alone.

\subsection{The state of a memory location: the memory table}

When computing the dependencies created by a memory reference, we need to 
map the data address of the reference to information about the previous
reference to that location. What information we need depends on whether 
the new refenece is a read or a write:
\begin{description}
\item[READ]
We need the latest write to construct a flow dependence (RAW).
\item[WRITE]
We need the latest write to construct an output dependence (WAW) as well as
all reads since that write to construct anti dependencies (WAR).
\end{description}
To see that we need all reads since the last write, consider the following
example:
\begin{verbatim}
  x = 1;
  y = x+1;
  z = x+2;
  x = 3;
\end{verbatim}
If only the last read is kept, the dependence between the second and fourth
lines will be lost, and since subsequent reads do not yield dependencies, 
there is no transitive dependence between these lines.
Hence we keep track, for each memory location, of the the event corresponding 
to the last write and a list of events corresponding to subsequent reads.

An important aspect of compaction is that when a subtree of the dynamic
call tree is compacted to be represented by its root node, the memory table
needs to be updated to avoid dangling pointers to eliminated events. All
pointers are then forwarded to point to the representative nodes. In this
process, pointers to previously distinct events now may point at the same 
tree nodes. In this case it is unnecessary to represent more than one 
copy of each pointer, thus compacting the read lists. This optimization is 
crucial in practice.

\newcommand{\backlink}[3]
{\begin{picture}(#1,#2)
\put(0,0){\line(1,0){#1}}
\put(#1,0){\line(0,1){#2}}
\put(#1,#2){\vector(-1,0){#3}}
\end{picture}}

\newcommand{\rwblock}
{\begin{picture}(100,40)(-40,0)
\put(0,25){\line(4,3){20}}
\put(0,15){\line(4,-3){20}}
\put(-40,25){\line(1,0){40}}
\put(-40,15){\line(1,0){40}}
\put(20,0){\framebox(40,40){\ }}
\put(20,30){\line(1,0){40}}
\put(20,30){\makebox(40,10){Last write}}
\put(20,20){\makebox(40,10){Last reads}}
\put(20,5){\makebox(40,10){\vdots}}
\end{picture}}
\begin{figure*}
\small
\hrulefill
\[
\begin{picture}(250,210)(-30,0)
% The memory table
\put(-30,30){\framebox(40,180)[t]{\ }}
\put(-30,10){\makebox(60,10){The memory table}}
% A block
\put(-30,145){\rwblock}
\put(70,177){\vector(3,-2){30}}
\put(-30,85){\rwblock}
\put(70,113){\vector(4,-1){30}}
\put(70,97){\vector(3,-2){30}}
\put(-30,25){\rwblock}
\put(70,50){\vector(2,-1){30}}
% Trace pile
\put(100,30){\framebox(40,180)[t]{\ }}
\put(100,190){\framebox(40,10){Open C}}
\put(100,170){\makebox(40,10){\vdots}}
\put(100,160){\line(1,0){40}}
\put(100,150){\makebox(40,10){Reg}}
\put(140,157){\backlink{8}{36}{6}}
\put(100,140){\framebox(40,10){Closed C}}
\put(140,147){\backlink{16}{46}{6}}
\put(100,120){\makebox(40,10){\vdots}}
\put(100,100){\framebox(40,10){Reg}}
\put(140,107){\backlink{8}{36}{6}}
\put(100,85){\makebox(40,10){\vdots}}
\put(100,70){\framebox(40,10){Reg}}
\put(140,77){\backlink{16}{66}{6}}
\put(100,50){\line(1,0){40}}
\put(100,55){\makebox(40,10){\vdots}}
\put(100,40){\makebox(40,10){Return}}
\put(140,47){\backlink{24}{96}{6}}
\put(100,30){\framebox(40,10){Reg}}
\put(140,37){\backlink{32}{156}{14}}
\end{picture}
%
% The next picture
%
\begin{picture}(200,210)(-30,0)
% The memory table
\put(-30,30){\framebox(40,180)[t]{\ }}
\put(-30,10){\makebox(60,10){The memory table}}
% A block
\put(-30,145){\rwblock}
\put(70,181){\vector(4,3){30}}
\put(-30,85){\rwblock}
\put(70,105){\vector(1,2){30}}
\put(-30,25){\rwblock}
\put(70,52){\vector(1,3){30}}
% Trace pile
\put(100,30){\framebox(40,180)[t]{\ }}
\put(100,200){\makebox(40,10){Open C}}
\put(100,200){\line(1,0){40}}
\put(100,185){\makebox(40,10){\vdots}}
\put(100,170){\framebox(40,10){Reg}}
\put(140,177){\backlink{8}{26}{6}}
\put(100,160){\makebox(40,10){Closed C}}
\put(140,167){\backlink{16}{36}{6}}
\put(100,150){\framebox(40,10){Return}}
\put(140,157){\backlink{8}{6}{6}}
\put(100,140){\makebox(40,10){Reg}}
\put(140,147){\backlink{24}{56}{6}}
\put(100,140){\line(1,0){40}}
\end{picture}
\]
\hrulefill
\caption{Compacting the trace pile: before and after}
\label{fnyembladata}
\end{figure*}


Figure~\ref{fnyembladata} shows the data structures we have discussed:
\begin{description}
\item[The memory table:]
Contains, for each block of memory, pointers to events or nodes representing
the most recent write and the reads since that write. 
Blocksize is a tradeoff between memory use and precision; the examples in 
this paper were run with single byte blocks. 
\item[The trace pile:]
Contains trace records representing events and call tree nodes; thus the 
trace pile implements the call tree. Most trace records contain 
\begin{itemize}
\item
a pointer to a record containing information about the program 
address of the corresponding instruction (corresponds to the {\tt line} 
attribute in the previous discussion), and
\item
a pointer to the previous trace record representing a procedure call (the 
{\tt node} or {\tt parent} attributes of events and nodes, respectively).
\end{itemize}
\end{description}

If a dependency endpoint is part of the NCA (if the NCA is the current
frame or the frame pointed to from the memory table) it is labelled as
direct, otherwise it is indirect. If any of the activation frames on
the path from an endpoint to the NCA is on the black list, we have a
weak endpoint.

\subsection{Eliminating spurious dependencies}

Some anti and output dependencies are due to reusing the same
locations for (logically) distinct data items. This happens for
instance for stack locations when arguments are passed on the stack
(as is common on the x86, for instance). Consider the following
example:
\begin{verbatim}
    foo(x);
    bar(y);
\end{verbatim}
The value of {\tt x} will be pushed on the stack. Then {\tt foo} will
be called and will read {\tt x}. When that call has returned, {\tt y}
will be pushed on the stack on the same address as {\tt x}, creating
an anti dependence with {\tt foo}'s read. This kind of dependence can 
easily be removed by a
compiler. In fact, if the call to {\tt foo} is made asynchronously, in
another thread, it will automatically be made on a different stack and
the dependence disappears. Register allocation is another source of
anti and output dependencies since distinct variables are packed into
the same register. In this case, the allocation can be changed to
remove dependencies, a transformation known as {\em renaming}.

The class of spurious dependencies that are caused by reuse of memory 
by the stack can be eliminated by keeping track of the minimal stack
size since each previous event (the trace pile keeps the events in their
original order). Thus we want to know if, since the last reference (of a 
particular kind) to a location currently part of the stack, the stack 
has been small enough that the location was out of the stack. In that
case that reference and the current reference do not give rise to any 
dependencies.

To this end, we maintain a {\em min stack} consisting of pairs $(t,s)$
where $t$ is a time stamp (an address in the trace pile) and $s$ is a 
stack pointer value. Each pair
$(t,s)$ in the min stack indicates that, at time $t$, the stack
pointer was $s$ and at all times $t'>t$ the stack has been larger.
Since the stack grows towards low addresses in some architectures and 
towards high addresses in other, we use the notation $a \in s$ to say 
that an address $a$ is part of the stack represented by stack pointer $s$.
For and stack growing towards lower addresses, $a\in s$ means $a \geq s$
wheras in the opposite case it means $a<s$.

When the
call stack grows, we push a new pair, consisting of the current time
and the new stack pointer value, on top of the min stack. When the
call stack shrinks, we pop all pairs representing larger stacks off the min
stack.

Thus, when we record a dependence at address $a$ with a previous 
reference at time stamp $t$,
we see if we can find a pair $(s,t')$ in the min stack with $a \not\in s$ and
$t<t'$ (a smaller stack at a later time).

We avoid spurious anti and output dependencies due to register
allocation by not computing dependencies on registers and compiling
the traced program without optimization. 
