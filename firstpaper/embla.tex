\section{The Tool}

Embla helps the user find dependencies in a program. Technically, a
dependece is two references, not both reads, to overlapping memory
locations with no interveaning write. What makes Embla special is the
way dependencies are reported. Since the idea is to support manual,
function call level parallelization, we are not really interested in
knowing that a memory read on line 11754 in function {\tt foo}
sometimes reads a value last written in line 3411 in function {\tt
bar}. Rather, the user is concerned with a completely different
function {\tt baz} and would like to know whether the call to {\tt
foobar} on line 2355 can be made in parallel with the call to {\tt
barfoo} on line 2356.



\subsection{Types of dependencies}

Another dimension that Embla recognizes is where in memory the
location associated with the dependency is situated. In particular, we
want to catch the cases where anti and output dependencies are created
by reuse of memory for new stack frames. This happens when the stack 
shrinks and
then grows again; the new stack frames use the same memory locations,
but only for memory management reasons. With this in mind we have the
following categories:
\begin{description}
\item[s(tack):]
The location has been a part of the stack from the first reference to
the second (inclusive).
\item[f(alse):]
The location was part of the stack when the first reference was made,
then the stack shrunk enough that the location was not part of the
stack and then it became part of the stack again.
\item[o(ther):]
The location has never been part of the stack.
\end{description}

We also categorize dependencies according to properties of the two 
memory references making up the source and target of the dependence. 
This is both in order to give more precise information and to flag
additional dependencies as spurious. 

Dependency endpoints correspond to instruction execution events and
are reported as line numbers with one of the following codes:
\begin{description}
\item[$\epsilon$:] 
No code at all represents an instruction that is
generated from the source code at the indicated line. Thus the 
instruction is part of the function or procedure containing the 
line.
\item[c:]
The instruction was executed by a function (procedure) that was 
(transitively) invoked by a function (procedure) call at the indicated 
line.
\item[h:]
Like {\bf c}, but the function containing the instruction was 
{\em hidden}, that is, it was on a blacklist of functions which, 
like {\tt malloc}, would not give rise to dependencies in the 
parallelized program.
\end{description}


\begin{figure} 
\small
\verbdef\jkffxb$#include <stdlib.h> $
\verbdef\jkffxc$#include <stdio.h> $
\verbdef\jkffxd$ $
\verbdef\jkffxe$struct {int a; int b;} has_ab; $
\verbdef\jkffxf$ $
\verbdef\jkffxg$int main(int argc, char **argv) $
\verbdef\jkffxh${ $
\verbdef\jkffxi$   has_ab.a = 1; $
\verbdef\jkffxj$   has_ab.b = 1; $
\verbdef\jkffxba$   printf("%d ", has_ab.a); $
\verbdef\jkffxbb$   has_ab.a = 2; $
\verbdef\jkffxbc$   has_ab.b = 2; $
\verbdef\jkffxbd$   printf("%d %d\n", has_ab.a, has_ab.b); $
\verbdef\jkffxbe$ $
\verbdef\jkffxbf$} $
\hrulefill
\[
\begin{picture}(420,150)(-60,-150)

\put(0,0){\makebox(100,10)[l]{}}
\put(0,-10){\makebox(100,10)[l]{\jkffxb}}
\put(0,-20){\makebox(100,10)[l]{\jkffxc}}
\put(0,-30){\makebox(100,10)[l]{\jkffxd}}
\put(0,-40){\makebox(100,10)[l]{\jkffxe}}
\put(0,-50){\makebox(100,10)[l]{\jkffxf}}
\put(0,-60){\makebox(100,10)[l]{\jkffxg}}
\put(0,-70){\makebox(100,10)[l]{\jkffxh}}
{\color{black} \dottedline{3}(-57,-70)(-5,-70)}\put(0,-80){\makebox(100,10)[l]{\jkffxi}}
{\color{black} \dottedline{3}(-57,-80)(-5,-80)}\put(0,-90){\makebox(100,10)[l]{\jkffxj}}
{\color{black} \dottedline{3}(-57,-90)(-5,-90)}\put(0,-100){\makebox(100,10)[l]{\jkffxba}}
{\color{black} \dottedline{3}(-57,-100)(-5,-100)}\put(0,-110){\makebox(100,10)[l]{\jkffxbb}}
{\color{black} \dottedline{3}(-57,-110)(-5,-110)}\put(0,-120){\makebox(100,10)[l]{\jkffxbc}}
{\color{black} \dottedline{3}(-57,-120)(-5,-120)}\put(0,-130){\makebox(100,10)[l]{\jkffxbd}}
{\color{black} \dottedline{3}(-57,-130)(-5,-130)}\put(0,-140){\makebox(100,10)[l]{\jkffxbe}}
{\color{black} \dottedline{3}(-57,-140)(-5,-140)}\put(0,-150){\makebox(100,10)[l]{\jkffxbf}}

\color{red}
\put(-6,-92){\circle*{2}}
\put(-8,-94){\linethickness{0.7pt}\line(1,0){0}}
\put(-6,-97){\circle*{2}\hspace{-2\unitlength}\circle{4}}

\color{red}
\put(-12,-75){\circle*{2}}
\put(-12,-77){\vector(0,-1){16}}
\put(-14,-77){\linethickness{0.7pt}\line(1,0){4}}
\put(-12,-95){\circle*{2}}

\color{green}
\put(-18,-95){\circle*{2}}
\put(-18,-97){\vector(0,-1){6}}
\put(-20,-97){\linethickness{0.7pt}\line(1,0){4}}
\put(-18,-105){\circle*{2}}

\color{blue}
\put(-24,-85){\circle*{2}}
\put(-24,-87){\vector(0,-1){26}}
\put(-26,-87){\linethickness{0.7pt}\line(1,0){4}}
\put(-24,-115){\circle*{2}}

\color{red}
\put(-30,-95){\circle*{2}\hspace{-2\unitlength}\circle{4}}
\put(-30,-97){\vector(0,-1){26}}
\put(-32,-97){\linethickness{0.7pt}\line(1,0){4}}
\put(-30,-125){\circle*{2}\hspace{-2\unitlength}\circle{4}}

\color{green}
\put(-36,-95){\circle*{2}\hspace{-2\unitlength}\circle{4}}
\put(-36,-97){\vector(0,-1){26}}
\put(-38,-97){\linethickness{0.7pt}\line(1,0){4}}
\put(-36,-125){\circle*{2}\hspace{-2\unitlength}\circle{4}}

\color{blue}
\put(-42,-95){\circle*{2}\hspace{-2\unitlength}\circle{4}}
\put(-42,-97){\vector(0,-1){26}}
\put(-44,-97){\linethickness{0.7pt}\line(1,0){4}}
\put(-42,-125){\circle*{2}\hspace{-2\unitlength}\circle{4}}

\color{red}
\put(-6,-122){\circle*{2}}
\put(-8,-124){\linethickness{0.7pt}\line(1,0){0}}
\put(-6,-127){\circle*{2}\hspace{-2\unitlength}\circle{4}}

\color{green}
\put(-48,-95){\circle*{2}\hspace{-2\unitlength}\circle{4}}
\put(-48,-97){\vector(0,-1){26}}
\put(-50,-97){\linethickness{0.7pt}\line(1,0){0}}
\put(-48,-125){\circle*{2}}

\color{red}
\put(-12,-105){\circle*{2}}
\put(-12,-107){\vector(0,-1){16}}
\put(-14,-107){\linethickness{0.7pt}\line(1,0){4}}
\put(-12,-125){\circle*{2}}

\color{red}
\put(-18,-115){\circle*{2}}
\put(-18,-117){\vector(0,-1){6}}
\put(-20,-117){\linethickness{0.7pt}\line(1,0){4}}
\put(-18,-125){\circle*{2}}

\color{red}
\put(-54,-65){\circle*{2}}
\put(-54,-67){\vector(0,-1){76}}
\put(-56,-67){\linethickness{0.7pt}\line(1,0){0}}
\put(-54,-145){\circle*{2}}

\end{picture}
\]
\hrulefill

\end{figure}

\begin{figure} 
\small
\verbdef\jkffxb$#include <stdio.h> $
\verbdef\jkffxc$#include <stdlib.h> $
\verbdef\jkffxd$ $
\verbdef\jkffxe$#define N 5 $
\verbdef\jkffxf$ $
\verbdef\jkffxg$typedef struct _ilist { $
\verbdef\jkffxh$   int val; $
\verbdef\jkffxi$   struct _ilist *next; $
\verbdef\jkffxj$} ilist; $
\verbdef\jkffxba$ $
\verbdef\jkffxbb$static ilist* mklist(int n) $
\verbdef\jkffxbc${ $
\verbdef\jkffxbd$  int i; $
\verbdef\jkffxbe$  ilist *p = NULL; $
\verbdef\jkffxbf$ $
\verbdef\jkffxbg$  for( i=0; i<n; i++ ) { $
\verbdef\jkffxbh$    ilist *t = (ilist *) malloc( sizeof(ilist) ); $
\verbdef\jkffxbi$    t->val = i; $
\verbdef\jkffxbj$    t->next = p; $
\verbdef\jkffxca$    p = t; $
\verbdef\jkffxcb$  } $
\verbdef\jkffxcc$  return p; $
\verbdef\jkffxcd$} $
\verbdef\jkffxce$ $
\verbdef\jkffxcf$static int sumlist(ilist* p) $
\verbdef\jkffxcg${ $
\verbdef\jkffxch$   ilist *q; $
\verbdef\jkffxci$   int  sum = 0; $
\verbdef\jkffxcj$ $
\verbdef\jkffxda$   for( q=p; q!=NULL; q=q->next ) { $
\verbdef\jkffxdb$      sum += q->val; $
\verbdef\jkffxdc$   } $
\verbdef\jkffxdd$   return sum; $
\verbdef\jkffxde$} $
\verbdef\jkffxdf$ $
\verbdef\jkffxdg$ $
\verbdef\jkffxdh$int main(int argc, char **argv)  $
\verbdef\jkffxdi${ $
\verbdef\jkffxdj$  int m,n; $
\verbdef\jkffxea$  ilist *p,*q; $
\verbdef\jkffxeb$ $
\verbdef\jkffxec$  p = mklist( N ); $
\verbdef\jkffxed$  q = mklist( N ); $
\verbdef\jkffxee$  m = sumlist( p ); $
\verbdef\jkffxef$  n = sumlist( q ); $
\verbdef\jkffxeg$ $
\verbdef\jkffxeh$  printf("%d\n", m+n); $
\verbdef\jkffxei$ $
\verbdef\jkffxej$  return 0; $
\verbdef\jkffxfa$ $
\verbdef\jkffxfb$} $
\hrulefill
\[
\begin{picture}(420,510)(-78,-510)

\put(0,0){\makebox(100,10)[l]{}}
\put(0,-10){\makebox(100,10)[l]{\jkffxb}}
\put(0,-20){\makebox(100,10)[l]{\jkffxc}}
\put(0,-30){\makebox(100,10)[l]{\jkffxd}}
\put(0,-40){\makebox(100,10)[l]{\jkffxe}}
\put(0,-50){\makebox(100,10)[l]{\jkffxf}}
\put(0,-60){\makebox(100,10)[l]{\jkffxg}}
\put(0,-70){\makebox(100,10)[l]{\jkffxh}}
\put(0,-80){\makebox(100,10)[l]{\jkffxi}}
\put(0,-90){\makebox(100,10)[l]{\jkffxj}}
\put(0,-100){\makebox(100,10)[l]{\jkffxba}}
\put(0,-110){\makebox(100,10)[l]{\jkffxbb}}
\put(0,-120){\makebox(100,10)[l]{\jkffxbc}}
{\color{black} \dottedline{3}(-75,-120)(-5,-120)}\put(0,-130){\makebox(100,10)[l]{\jkffxbd}}
{\color{black} \dottedline{3}(-75,-130)(-5,-130)}\put(0,-140){\makebox(100,10)[l]{\jkffxbe}}
{\color{black} \dottedline{3}(-75,-140)(-5,-140)}\put(0,-150){\makebox(100,10)[l]{\jkffxbf}}
{\color{black} \dottedline{3}(-75,-150)(-5,-150)}\put(0,-160){\makebox(100,10)[l]{\jkffxbg}}
{\color{black} \dottedline{3}(-75,-160)(-5,-160)}\put(0,-170){\makebox(100,10)[l]{\jkffxbh}}
{\color{black} \dottedline{3}(-75,-170)(-5,-170)}\put(0,-180){\makebox(100,10)[l]{\jkffxbi}}
{\color{black} \dottedline{3}(-75,-180)(-5,-180)}\put(0,-190){\makebox(100,10)[l]{\jkffxbj}}
{\color{black} \dottedline{3}(-75,-190)(-5,-190)}\put(0,-200){\makebox(100,10)[l]{\jkffxca}}
{\color{black} \dottedline{3}(-75,-200)(-5,-200)}\put(0,-210){\makebox(100,10)[l]{\jkffxcb}}
{\color{black} \dottedline{3}(-75,-210)(-5,-210)}\put(0,-220){\makebox(100,10)[l]{\jkffxcc}}
{\color{black} \dottedline{3}(-75,-220)(-5,-220)}\put(0,-230){\makebox(100,10)[l]{\jkffxcd}}
\put(0,-240){\makebox(100,10)[l]{\jkffxce}}
\put(0,-250){\makebox(100,10)[l]{\jkffxcf}}
\put(0,-260){\makebox(100,10)[l]{\jkffxcg}}
{\color{black} \dottedline{3}(-75,-260)(-5,-260)}\put(0,-270){\makebox(100,10)[l]{\jkffxch}}
{\color{black} \dottedline{3}(-75,-270)(-5,-270)}\put(0,-280){\makebox(100,10)[l]{\jkffxci}}
{\color{black} \dottedline{3}(-75,-280)(-5,-280)}\put(0,-290){\makebox(100,10)[l]{\jkffxcj}}
{\color{black} \dottedline{3}(-75,-290)(-5,-290)}\put(0,-300){\makebox(100,10)[l]{\jkffxda}}
{\color{black} \dottedline{3}(-75,-300)(-5,-300)}\put(0,-310){\makebox(100,10)[l]{\jkffxdb}}
{\color{black} \dottedline{3}(-75,-310)(-5,-310)}\put(0,-320){\makebox(100,10)[l]{\jkffxdc}}
{\color{black} \dottedline{3}(-75,-320)(-5,-320)}\put(0,-330){\makebox(100,10)[l]{\jkffxdd}}
{\color{black} \dottedline{3}(-75,-330)(-5,-330)}\put(0,-340){\makebox(100,10)[l]{\jkffxde}}
\put(0,-350){\makebox(100,10)[l]{\jkffxdf}}
\put(0,-360){\makebox(100,10)[l]{\jkffxdg}}
\put(0,-370){\makebox(100,10)[l]{\jkffxdh}}
\put(0,-380){\makebox(100,10)[l]{\jkffxdi}}
{\color{black} \dottedline{3}(-75,-380)(-5,-380)}\put(0,-390){\makebox(100,10)[l]{\jkffxdj}}
{\color{black} \dottedline{3}(-75,-390)(-5,-390)}\put(0,-400){\makebox(100,10)[l]{\jkffxea}}
{\color{black} \dottedline{3}(-75,-400)(-5,-400)}\put(0,-410){\makebox(100,10)[l]{\jkffxeb}}
{\color{black} \dottedline{3}(-75,-410)(-5,-410)}\put(0,-420){\makebox(100,10)[l]{\jkffxec}}
{\color{black} \dottedline{3}(-75,-420)(-5,-420)}\put(0,-430){\makebox(100,10)[l]{\jkffxed}}
{\color{black} \dottedline{3}(-75,-430)(-5,-430)}\put(0,-440){\makebox(100,10)[l]{\jkffxee}}
{\color{black} \dottedline{3}(-75,-440)(-5,-440)}\put(0,-450){\makebox(100,10)[l]{\jkffxef}}
{\color{black} \dottedline{3}(-75,-450)(-5,-450)}\put(0,-460){\makebox(100,10)[l]{\jkffxeg}}
{\color{black} \dottedline{3}(-75,-460)(-5,-460)}\put(0,-470){\makebox(100,10)[l]{\jkffxeh}}
{\color{black} \dottedline{3}(-75,-470)(-5,-470)}\put(0,-480){\makebox(100,10)[l]{\jkffxei}}
{\color{black} \dottedline{3}(-75,-480)(-5,-480)}\put(0,-490){\makebox(100,10)[l]{\jkffxej}}
{\color{black} \dottedline{3}(-75,-490)(-5,-490)}\put(0,-500){\makebox(100,10)[l]{\jkffxfa}}
{\color{black} \dottedline{3}(-75,-500)(-5,-500)}\put(0,-510){\makebox(100,10)[l]{\jkffxfb}}

\color{red}
\put(-6,-255){\circle*{2}}
\put(-6,-257){\vector(0,-1){76}}
\put(-8,-257){\linethickness{0.7pt}\line(1,0){0}}
\put(-6,-335){\circle*{2}}

\color{red}
\put(-12,-275){\circle*{2}}
\put(-12,-277){\vector(0,-1){26}}
\put(-14,-277){\linethickness{0.7pt}\line(1,0){0}}
\put(-12,-305){\circle*{2}}

\color{red}
\put(-18,-292){\circle*{2}}
\put(-20,-294){\linethickness{0.7pt}\line(1,0){0}}
\put(-18,-297){\circle*{2}}

\color{green}
\put(-24,-292){\circle*{2}}
\put(-26,-294){\linethickness{0.7pt}\line(1,0){0}}
\put(-24,-297){\circle*{2}}

\color{red}
\put(-30,-295){\circle*{2}}
\put(-30,-297){\vector(0,-1){6}}
\put(-32,-297){\linethickness{0.7pt}\line(1,0){0}}
\put(-30,-305){\circle*{2}}

\color{green}
\put(-36,-305){\circle*{2}}
\put(-36,-303){\vector(0,1){6}}
\put(-38,-297){\linethickness{0.7pt}\line(1,0){0}}
\put(-36,-295){\circle*{2}}

\color{red}
\put(-18,-302){\circle*{2}}
\put(-20,-304){\linethickness{0.7pt}\line(1,0){0}}
\put(-18,-307){\circle*{2}}

\color{green}
\put(-24,-302){\circle*{2}}
\put(-26,-304){\linethickness{0.7pt}\line(1,0){0}}
\put(-24,-307){\circle*{2}}

\color{red}
\put(-42,-305){\circle*{2}}
\put(-42,-307){\vector(0,-1){16}}
\put(-44,-307){\linethickness{0.7pt}\line(1,0){0}}
\put(-42,-325){\circle*{2}}

\color{red}
\put(-6,-115){\circle*{2}}
\put(-6,-117){\vector(0,-1){106}}
\put(-8,-117){\linethickness{0.7pt}\line(1,0){0}}
\put(-6,-225){\circle*{2}}

\color{red}
\put(-12,-135){\circle*{2}}
\put(-12,-137){\vector(0,-1){46}}
\put(-14,-137){\linethickness{0.7pt}\line(1,0){0}}
\put(-12,-185){\circle*{2}}

\color{red}
\put(-18,-152){\circle*{2}}
\put(-20,-154){\linethickness{0.7pt}\line(1,0){0}}
\put(-18,-157){\circle*{2}}

\color{green}
\put(-24,-152){\circle*{2}}
\put(-26,-154){\linethickness{0.7pt}\line(1,0){0}}
\put(-24,-157){\circle*{2}}

\color{red}
\put(-30,-155){\circle*{2}}
\put(-30,-157){\vector(0,-1){16}}
\put(-32,-157){\linethickness{0.7pt}\line(1,0){0}}
\put(-30,-175){\circle*{2}}

\color{red}
\put(-18,-162){\circle*{2}}
\put(-20,-164){\linethickness{0.7pt}\line(1,0){0}}
\put(-18,-167){\circle{4}}

\color{red}
\put(-24,-165){\circle*{2}}
\put(-24,-167){\vector(0,-1){6}}
\put(-26,-167){\linethickness{0.7pt}\line(1,0){0}}
\put(-24,-175){\circle*{2}}

\color{red}
\put(-36,-165){\circle*{2}}
\put(-36,-167){\vector(0,-1){16}}
\put(-38,-167){\linethickness{0.7pt}\line(1,0){0}}
\put(-36,-185){\circle*{2}}

\color{red}
\put(-42,-165){\circle*{2}}
\put(-42,-167){\vector(0,-1){26}}
\put(-44,-167){\linethickness{0.7pt}\line(1,0){0}}
\put(-42,-195){\circle*{2}}

\color{red}
\put(-48,-162){\circle{4}}
\put(-50,-164){\linethickness{0.7pt}\line(1,0){4}}
\put(-48,-167){\circle{4}}

\color{green}
\put(-54,-175){\circle*{2}}
\put(-54,-173){\vector(0,1){16}}
\put(-56,-157){\linethickness{0.7pt}\line(1,0){0}}
\put(-54,-155){\circle*{2}}

\color{green}
\put(-60,-175){\circle*{2}}
\put(-60,-173){\vector(0,1){6}}
\put(-62,-167){\linethickness{0.7pt}\line(1,0){0}}
\put(-60,-165){\circle*{2}}

\color{green}
\put(-66,-185){\circle*{2}}
\put(-66,-183){\vector(0,1){16}}
\put(-68,-167){\linethickness{0.7pt}\line(1,0){0}}
\put(-66,-165){\circle*{2}}

\color{green}
\put(-18,-185){\circle*{2}}
\put(-18,-187){\vector(0,-1){6}}
\put(-20,-187){\linethickness{0.7pt}\line(1,0){0}}
\put(-18,-195){\circle*{2}}

\color{green}
\put(-72,-195){\circle*{2}}
\put(-72,-193){\vector(0,1){26}}
\put(-74,-167){\linethickness{0.7pt}\line(1,0){0}}
\put(-72,-165){\circle*{2}}

\color{red}
\put(-24,-195){\circle*{2}}
\put(-24,-193){\vector(0,1){6}}
\put(-26,-187){\linethickness{0.7pt}\line(1,0){0}}
\put(-24,-185){\circle*{2}}

\color{red}
\put(-12,-195){\circle*{2}}
\put(-12,-197){\vector(0,-1){16}}
\put(-14,-197){\linethickness{0.7pt}\line(1,0){0}}
\put(-12,-215){\circle*{2}}

\color{red}
\put(-6,-375){\circle*{2}}
\put(-6,-377){\vector(0,-1){126}}
\put(-8,-377){\linethickness{0.7pt}\line(1,0){0}}
\put(-6,-505){\circle*{2}}

\color{red}
\put(-12,-412){\circle*{2}}
\put(-14,-414){\linethickness{0.7pt}\line(1,0){0}}
\put(-12,-417){\circle*{2}\hspace{-2\unitlength}\circle{4}}

\color{red}
\put(-18,-415){\circle*{2}}
\put(-18,-417){\vector(0,-1){16}}
\put(-20,-417){\linethickness{0.7pt}\line(1,0){0}}
\put(-18,-435){\circle*{2}}

\color{red}
\put(-24,-415){\circle*{2}\hspace{-2\unitlength}\circle{4}}
\put(-24,-417){\vector(0,-1){16}}
\put(-26,-417){\linethickness{0.7pt}\line(1,0){4}}
\put(-24,-435){\circle*{2}\hspace{-2\unitlength}\circle{4}}

\color{red}
\put(-30,-415){\circle{4}}
\put(-30,-417){\vector(0,-1){6}}
\put(-32,-417){\linethickness{0.7pt}\line(1,0){4}}
\put(-30,-425){\circle{4}}

\color{red}
\put(-12,-422){\circle*{2}}
\put(-14,-424){\linethickness{0.7pt}\line(1,0){0}}
\put(-12,-427){\circle*{2}\hspace{-2\unitlength}\circle{4}}

\color{red}
\put(-36,-425){\circle*{2}}
\put(-36,-427){\vector(0,-1){16}}
\put(-38,-427){\linethickness{0.7pt}\line(1,0){0}}
\put(-36,-445){\circle*{2}}

\color{red}
\put(-42,-425){\circle*{2}\hspace{-2\unitlength}\circle{4}}
\put(-42,-427){\vector(0,-1){16}}
\put(-44,-427){\linethickness{0.7pt}\line(1,0){4}}
\put(-42,-445){\circle*{2}\hspace{-2\unitlength}\circle{4}}

\color{red}
\put(-12,-432){\circle*{2}}
\put(-14,-434){\linethickness{0.7pt}\line(1,0){0}}
\put(-12,-437){\circle*{2}\hspace{-2\unitlength}\circle{4}}

\color{red}
\put(-30,-435){\circle*{2}}
\put(-30,-437){\vector(0,-1){26}}
\put(-32,-437){\linethickness{0.7pt}\line(1,0){0}}
\put(-30,-465){\circle*{2}}

\color{red}
\put(-12,-442){\circle*{2}}
\put(-14,-444){\linethickness{0.7pt}\line(1,0){0}}
\put(-12,-447){\circle*{2}\hspace{-2\unitlength}\circle{4}}

\color{red}
\put(-18,-445){\circle*{2}}
\put(-18,-447){\vector(0,-1){16}}
\put(-20,-447){\linethickness{0.7pt}\line(1,0){0}}
\put(-18,-465){\circle*{2}}

\color{red}
\put(-12,-462){\circle*{2}}
\put(-14,-464){\linethickness{0.7pt}\line(1,0){0}}
\put(-12,-467){\circle*{2}\hspace{-2\unitlength}\circle{4}}

\end{picture}
\]
\hrulefill

\end{figure}

\begin{figure} 
\small
\verbdef\jkffxb$#include <stdlib.h> $
\verbdef\jkffxc$#include <stdio.h> $
\verbdef\jkffxd$ $
\verbdef\jkffxe$struct {int a; int b;} has_ab; $
\verbdef\jkffxf$ $
\verbdef\jkffxg$int main(int argc, char **argv) $
\verbdef\jkffxh${ $
\verbdef\jkffxi$   has_ab.a = 1; $
\verbdef\jkffxj$   has_ab.b = 1; $
\verbdef\jkffxba$   printf("%d ", has_ab.a); $
\verbdef\jkffxbb$   has_ab.a = 2; $
\verbdef\jkffxbc$   has_ab.b = 2; $
\verbdef\jkffxbd$   printf("%d %d\n", has_ab.a, has_ab.b); $
\verbdef\jkffxbe$ $
\verbdef\jkffxbf$} $
\hrulefill
\[
\begin{picture}(420,150)(-30,-150)

\put(0,-10){\makebox(100,10)[l]{\jkffxb}}
\put(0,-20){\makebox(100,10)[l]{\jkffxc}}
\put(0,-30){\makebox(100,10)[l]{\jkffxd}}
\put(0,-40){\makebox(100,10)[l]{\jkffxe}}
\put(0,-50){\makebox(100,10)[l]{\jkffxf}}
\put(0,-60){\makebox(100,10)[l]{\jkffxg}}
\put(0,-70){\makebox(100,10)[l]{\jkffxh}}
{\color{black} \dottedline{3}(-10,-65)(-0,-65)}
\put(0,-80){\makebox(100,10)[l]{\jkffxi}}
{\color{black} \dottedline{3}(-10,-75)(-0,-75)}
\put(0,-90){\makebox(100,10)[l]{\jkffxj}}
{\color{black} \dottedline{3}(-10,-85)(-0,-85)}
\put(0,-100){\makebox(100,10)[l]{\jkffxba}}
{\color{black} \dottedline{3}(-10,-95)(-0,-95)}
\put(0,-110){\makebox(100,10)[l]{\jkffxbb}}
{\color{black} \dottedline{3}(-10,-105)(-0,-105)}
\put(0,-120){\makebox(100,10)[l]{\jkffxbc}}
{\color{black} \dottedline{3}(-10,-115)(-0,-115)}
\put(0,-130){\makebox(100,10)[l]{\jkffxbd}}
{\color{black} \dottedline{3}(-10,-125)(-0,-125)}
\put(0,-140){\makebox(100,10)[l]{\jkffxbe}}
\put(0,-150){\makebox(100,10)[l]{\jkffxbf}}
{\color{black} \dottedline{3}(-10,-145)(-0,-145)}

\put(-10,-65){\color{black}\circle*{2}}
\put(-30,-75){\color{black}\circle*{2}}
\put(-20,-85){\color{black}\circle*{2}}
\put(-30,-95){\color{black}\circle*{2}}
\put(-30,-105){\color{black}\circle*{2}}
\put(-20,-115){\color{black}\circle*{2}}
\put(-20,-125){\color{black}\circle*{2}}
\put(-10,-145){\color{black}\circle*{2}}
\put(-10,-67){\color{red}\vector(0,-1){76}}
\put(-30,-77){\color{red}\vector(0,-1){16}}
\put(-20,-87){\color{blue}\vector(0,-1){26}}
\put(-30,-97){\color{green}\vector(0,-1){6}}
\put(-29,-96){\color{red}\vector(1,-3){8}}
\put(-29,-106){\color{red}\vector(1,-2){8}}
\put(-20,-117){\color{red}\vector(0,-1){6}}
\end{picture}
\]
\hrulefill

\end{figure}

\begin{figure} 
\small
\verbdef\jkffxb$#include <stdio.h> $
\verbdef\jkffxc$#include <stdlib.h> $
\verbdef\jkffxd$ $
\verbdef\jkffxe$#define N 5 $
\verbdef\jkffxf$ $
\verbdef\jkffxg$typedef struct _ilist { $
\verbdef\jkffxh$   int val; $
\verbdef\jkffxi$   struct _ilist *next; $
\verbdef\jkffxj$} ilist; $
\verbdef\jkffxba$ $
\verbdef\jkffxbb$static ilist* mklist(int n) $
\verbdef\jkffxbc${ $
\verbdef\jkffxbd$  int i; $
\verbdef\jkffxbe$  ilist *p = NULL; $
\verbdef\jkffxbf$ $
\verbdef\jkffxbg$  for( i=0; i<n; i++ ) { $
\verbdef\jkffxbh$    ilist *t = (ilist *) malloc( sizeof(ilist) ); $
\verbdef\jkffxbi$    t->val = i; $
\verbdef\jkffxbj$    t->next = p; $
\verbdef\jkffxca$    p = t; $
\verbdef\jkffxcb$  } $
\verbdef\jkffxcc$  return p; $
\verbdef\jkffxcd$} $
\verbdef\jkffxce$ $
\verbdef\jkffxcf$static int sumlist(ilist* p) $
\verbdef\jkffxcg${ $
\verbdef\jkffxch$   ilist *q; $
\verbdef\jkffxci$   int  sum = 0; $
\verbdef\jkffxcj$ $
\verbdef\jkffxda$   for( q=p; q!=NULL; q=q->next ) { $
\verbdef\jkffxdb$      sum += q->val; $
\verbdef\jkffxdc$   } $
\verbdef\jkffxdd$   return sum; $
\verbdef\jkffxde$} $
\verbdef\jkffxdf$ $
\verbdef\jkffxdg$ $
\verbdef\jkffxdh$int main(int argc, char **argv)  $
\verbdef\jkffxdi${ $
\verbdef\jkffxdj$  int m,n; $
\verbdef\jkffxea$  ilist *p,*q; $
\verbdef\jkffxeb$ $
\verbdef\jkffxec$  p = mklist( N ); $
\verbdef\jkffxed$  q = mklist( N ); $
\verbdef\jkffxee$  m = sumlist( p ); $
\verbdef\jkffxef$  n = sumlist( q ); $
\verbdef\jkffxeg$ $
\verbdef\jkffxeh$  printf("%d\n", m+n); $
\verbdef\jkffxei$ $
\verbdef\jkffxej$  return 0; $
\verbdef\jkffxfa$ $
\verbdef\jkffxfb$} $
\hrulefill
\[
\begin{picture}(420,510)(-40,-510)

\put(0,-10){\makebox(100,10)[l]{\jkffxb}}
\put(0,-20){\makebox(100,10)[l]{\jkffxc}}
\put(0,-30){\makebox(100,10)[l]{\jkffxd}}
\put(0,-40){\makebox(100,10)[l]{\jkffxe}}
\put(0,-50){\makebox(100,10)[l]{\jkffxf}}
\put(0,-60){\makebox(100,10)[l]{\jkffxg}}
\put(0,-70){\makebox(100,10)[l]{\jkffxh}}
\put(0,-80){\makebox(100,10)[l]{\jkffxi}}
\put(0,-90){\makebox(100,10)[l]{\jkffxj}}
\put(0,-100){\makebox(100,10)[l]{\jkffxba}}
\put(0,-110){\makebox(100,10)[l]{\jkffxbb}}
\put(0,-120){\makebox(100,10)[l]{\jkffxbc}}
{\color{black} \dottedline{3}(-10,-115)(-0,-115)}
\put(0,-130){\makebox(100,10)[l]{\jkffxbd}}
\put(0,-140){\makebox(100,10)[l]{\jkffxbe}}
{\color{black} \dottedline{3}(-10,-135)(-0,-135)}
\put(0,-150){\makebox(100,10)[l]{\jkffxbf}}
\put(0,-160){\makebox(100,10)[l]{\jkffxbg}}
{\color{black} \dottedline{3}(-10,-155)(-0,-155)}
\put(0,-170){\makebox(100,10)[l]{\jkffxbh}}
{\color{black} \dottedline{3}(-10,-165)(-0,-165)}
\put(0,-180){\makebox(100,10)[l]{\jkffxbi}}
{\color{black} \dottedline{3}(-10,-175)(-0,-175)}
\put(0,-190){\makebox(100,10)[l]{\jkffxbj}}
{\color{black} \dottedline{3}(-10,-185)(-0,-185)}
\put(0,-200){\makebox(100,10)[l]{\jkffxca}}
{\color{black} \dottedline{3}(-10,-195)(-0,-195)}
\put(0,-210){\makebox(100,10)[l]{\jkffxcb}}
\put(0,-220){\makebox(100,10)[l]{\jkffxcc}}
{\color{black} \dottedline{3}(-10,-215)(-0,-215)}
\put(0,-230){\makebox(100,10)[l]{\jkffxcd}}
{\color{black} \dottedline{3}(-10,-225)(-0,-225)}
\put(0,-240){\makebox(100,10)[l]{\jkffxce}}
\put(0,-250){\makebox(100,10)[l]{\jkffxcf}}
\put(0,-260){\makebox(100,10)[l]{\jkffxcg}}
{\color{black} \dottedline{3}(-10,-255)(-0,-255)}
\put(0,-270){\makebox(100,10)[l]{\jkffxch}}
\put(0,-280){\makebox(100,10)[l]{\jkffxci}}
{\color{black} \dottedline{3}(-10,-275)(-0,-275)}
\put(0,-290){\makebox(100,10)[l]{\jkffxcj}}
\put(0,-300){\makebox(100,10)[l]{\jkffxda}}
{\color{black} \dottedline{3}(-10,-295)(-0,-295)}
\put(0,-310){\makebox(100,10)[l]{\jkffxdb}}
{\color{black} \dottedline{3}(-10,-305)(-0,-305)}
\put(0,-320){\makebox(100,10)[l]{\jkffxdc}}
\put(0,-330){\makebox(100,10)[l]{\jkffxdd}}
{\color{black} \dottedline{3}(-10,-325)(-0,-325)}
\put(0,-340){\makebox(100,10)[l]{\jkffxde}}
{\color{black} \dottedline{3}(-10,-335)(-0,-335)}
\put(0,-350){\makebox(100,10)[l]{\jkffxdf}}
\put(0,-360){\makebox(100,10)[l]{\jkffxdg}}
\put(0,-370){\makebox(100,10)[l]{\jkffxdh}}
\put(0,-380){\makebox(100,10)[l]{\jkffxdi}}
{\color{black} \dottedline{3}(-10,-375)(-0,-375)}
\put(0,-390){\makebox(100,10)[l]{\jkffxdj}}
\put(0,-400){\makebox(100,10)[l]{\jkffxea}}
\put(0,-410){\makebox(100,10)[l]{\jkffxeb}}
\put(0,-420){\makebox(100,10)[l]{\jkffxec}}
{\color{black} \dottedline{3}(-10,-415)(-0,-415)}
\put(0,-430){\makebox(100,10)[l]{\jkffxed}}
{\color{black} \dottedline{3}(-10,-425)(-0,-425)}
\put(0,-440){\makebox(100,10)[l]{\jkffxee}}
{\color{black} \dottedline{3}(-10,-435)(-0,-435)}
\put(0,-450){\makebox(100,10)[l]{\jkffxef}}
{\color{black} \dottedline{3}(-10,-445)(-0,-445)}
\put(0,-460){\makebox(100,10)[l]{\jkffxeg}}
\put(0,-470){\makebox(100,10)[l]{\jkffxeh}}
{\color{black} \dottedline{3}(-10,-465)(-0,-465)}
\put(0,-480){\makebox(100,10)[l]{\jkffxei}}
\put(0,-490){\makebox(100,10)[l]{\jkffxej}}
\put(0,-500){\makebox(100,10)[l]{\jkffxfa}}
\put(0,-510){\makebox(100,10)[l]{\jkffxfb}}
{\color{black} \dottedline{3}(-10,-505)(-0,-505)}

\put(-30,-435){\color{black}\circle*{2}}
\put(-30,-165){\color{black}\circle*{2}}
\put(-10,-255){\color{black}\circle*{2}}
\put(-20,-445){\color{black}\circle*{2}}
\put(-20,-175){\color{black}\circle*{2}}
\put(-40,-185){\color{black}\circle*{2}}
\put(-30,-275){\color{black}\circle*{2}}
\put(-30,-465){\color{black}\circle*{2}}
\put(-10,-375){\color{black}\circle*{2}}
\put(-30,-195){\color{black}\circle*{2}}
\put(-10,-115){\color{black}\circle*{2}}
\put(-20,-295){\color{black}\circle*{2}}
\put(-30,-215){\color{black}\circle*{2}}
\put(-30,-305){\color{black}\circle*{2}}
\put(-40,-135){\color{black}\circle*{2}}
\put(-10,-225){\color{black}\circle*{2}}
\put(-30,-415){\color{black}\circle*{2}}
\put(-10,-505){\color{black}\circle*{2}}
\put(-30,-325){\color{black}\circle*{2}}
\put(-20,-425){\color{black}\circle*{2}}
\put(-20,-155){\color{black}\circle*{2}}
\put(-10,-335){\color{black}\circle*{2}}
\put(-10.0,-117.0){\color{red}\vector(0,-1){106.0}}
{\color{red}\dashline[100]{2}(-10.0,-117.0)(-10.0,-223.0)}
\put(-40.0,-137.0){\color{red}\vector(0,-1){46.0}}
{\color{red}\dashline[100]{2}(-40.0,-137.0)(-40.0,-183.0)}
\put(-19.0,-157.0){\color{red}\vector(0,-1){16.0}}
{\color{red}\dashline[100]{2}(-19.0,-157.0)(-19.0,-173.0)}
\put(-27.9,-165.7){\color{red}\vector(1,-1){7.2}}
{\color{red}\dashline[100]{2}(-27.9,-165.7)(-20.7,-172.9)}
\put(-30.0,-167.2){\color{red}\vector(-1,-2){8.2}}
{\color{red}\dashline[100]{2}(-30.0,-167.2)(-38.2,-183.7)}
\put(-29.0,-167.0){\color{red}\vector(0,-1){26.0}}
{\color{red}\dashline[100]{2}(-29.0,-167.0)(-29.0,-193.0)}
\put(-21.0,-173.0){\color{green}\vector(0,1){16.0}}
{\color{green}\dashline[100]{2}(-21.0,-173.0)(-21.0,-157.0)}
\put(-22.1,-174.3){\color{green}\vector(-1,1){7.2}}
{\color{green}\dashline[100]{2}(-22.1,-174.3)(-29.3,-167.1)}
\put(-40.0,-182.8){\color{green}\vector(1,2){8.2}}
{\color{green}\dashline[100]{2}(-40.0,-182.8)(-31.8,-166.3)}
\put(-37.9,-185.7){\color{green}\vector(1,-1){7.2}}
{\color{green}\dashline[100]{2}(-37.9,-185.7)(-30.7,-192.9)}
\put(-31.0,-193.0){\color{green}\vector(0,1){26.0}}
{\color{green}\dashline[100]{2}(-31.0,-193.0)(-31.0,-167.0)}
\put(-32.1,-194.3){\color{red}\vector(-1,1){7.2}}
{\color{red}\dashline[100]{2}(-32.1,-194.3)(-39.3,-187.1)}
\put(-30.0,-197.0){\color{red}\vector(0,-1){16.0}}
{\color{red}\dashline[100]{2}(-30.0,-197.0)(-30.0,-213.0)}
\put(-10.0,-257.0){\color{red}\vector(0,-1){76.0}}
{\color{red}\dashline[100]{2}(-10.0,-257.0)(-10.0,-333.0)}
\put(-30.0,-277.0){\color{red}\vector(0,-1){26.0}}
{\color{red}\dashline[100]{2}(-30.0,-277.0)(-30.0,-303.0)}
\put(-20.7,-297.1){\color{red}\vector(-1,-1){7.2}}
{\color{red}\dashline[100]{2}(-20.7,-297.1)(-27.9,-304.3)}
\put(-29.3,-302.9){\color{green}\vector(1,1){7.2}}
{\color{green}\dashline[100]{2}(-29.3,-302.9)(-22.1,-295.7)}
\put(-30.0,-307.0){\color{red}\vector(0,-1){16.0}}
{\color{red}\dashline[100]{2}(-30.0,-307.0)(-30.0,-323.0)}
\put(-10.0,-377.0){\color{red}\vector(0,-1){126.0}}
{\color{red}\dashline[100]{2}(-10.0,-377.0)(-10.0,-503.0)}
\put(-28.6,-416.4){\color{cyan}\vector(1,-1){7.2}}
{\color{cyan}\dashline[100]{2}(-28.6,-416.4)(-21.4,-423.6)}
\put(-30.0,-417.0){\color{red}\vector(0,-1){16.0}}
{\color{red}\dashline[100]{2}(-30.0,-417.0)(-30.0,-433.0)}
\put(-20.0,-427.0){\color{red}\vector(0,-1){16.0}}
{\color{red}\dashline[100]{2}(-20.0,-427.0)(-20.0,-443.0)}
\put(-30.0,-437.0){\color{red}\vector(0,-1){26.0}}
{\color{red}\dashline[100]{2}(-30.0,-437.0)(-30.0,-463.0)}
\put(-20.9,-446.8){\color{red}\vector(-1,-2){8.2}}
{\color{red}\dashline[100]{2}(-20.9,-446.8)(-29.1,-463.2)}
\end{picture}
\]
\hrulefill

\end{figure}

\begin{figure} 
\small
\verbdef\jkffxb$#include <stdlib.h> $
\verbdef\jkffxc$#include <stdio.h> $
\verbdef\jkffxd$ $
\verbdef\jkffxe$static int nfib(int n) $
\verbdef\jkffxf${ $
\verbdef\jkffxg$   int result; $
\verbdef\jkffxh$   if( n < 2 ) { $
\verbdef\jkffxi$     result = 1; $
\verbdef\jkffxj$   } else { $
\verbdef\jkffxba$     int a = nfib( n-1 ); $
\verbdef\jkffxbb$     int b = nfib( n-2 ); $
\verbdef\jkffxbc$     result = a+b; $
\verbdef\jkffxbd$   } $
\verbdef\jkffxbe$   return result; $
\verbdef\jkffxbf$} $
\verbdef\jkffxbg$      $
\verbdef\jkffxbh$  $
\verbdef\jkffxbi$ $
\verbdef\jkffxbj$int main(int argc, char **argv) $
\verbdef\jkffxca${ $
\verbdef\jkffxcb$   int m = nfib( 8 ); $
\verbdef\jkffxcc$    $
\verbdef\jkffxcd$   printf( "%d\n", m ); $
\verbdef\jkffxce$} $
\hrulefill
\[
\begin{picture}(420,240)(-36,-240)

\put(0,0){\makebox(100,10)[l]{}}
\put(0,-10){\makebox(100,10)[l]{\jkffxb}}
\put(0,-20){\makebox(100,10)[l]{\jkffxc}}
\put(0,-30){\makebox(100,10)[l]{\jkffxd}}
\put(0,-40){\makebox(100,10)[l]{\jkffxe}}
\put(0,-50){\makebox(100,10)[l]{\jkffxf}}
{\color{black} \dottedline{3}(-33,-50)(-5,-50)}\put(0,-60){\makebox(100,10)[l]{\jkffxg}}
{\color{black} \dottedline{3}(-33,-60)(-5,-60)}\put(0,-70){\makebox(100,10)[l]{\jkffxh}}
{\color{black} \dottedline{3}(-33,-70)(-5,-70)}\put(0,-80){\makebox(100,10)[l]{\jkffxi}}
{\color{black} \dottedline{3}(-33,-80)(-5,-80)}\put(0,-90){\makebox(100,10)[l]{\jkffxj}}
{\color{black} \dottedline{3}(-33,-90)(-5,-90)}\put(0,-100){\makebox(100,10)[l]{\jkffxba}}
{\color{black} \dottedline{3}(-33,-100)(-5,-100)}\put(0,-110){\makebox(100,10)[l]{\jkffxbb}}
{\color{black} \dottedline{3}(-33,-110)(-5,-110)}\put(0,-120){\makebox(100,10)[l]{\jkffxbc}}
{\color{black} \dottedline{3}(-33,-120)(-5,-120)}\put(0,-130){\makebox(100,10)[l]{\jkffxbd}}
{\color{black} \dottedline{3}(-33,-130)(-5,-130)}\put(0,-140){\makebox(100,10)[l]{\jkffxbe}}
{\color{black} \dottedline{3}(-33,-140)(-5,-140)}\put(0,-150){\makebox(100,10)[l]{\jkffxbf}}
\put(0,-160){\makebox(100,10)[l]{\jkffxbg}}
\put(0,-170){\makebox(100,10)[l]{\jkffxbh}}
\put(0,-180){\makebox(100,10)[l]{\jkffxbi}}
\put(0,-190){\makebox(100,10)[l]{\jkffxbj}}
\put(0,-200){\makebox(100,10)[l]{\jkffxca}}
{\color{black} \dottedline{3}(-33,-200)(-5,-200)}\put(0,-210){\makebox(100,10)[l]{\jkffxcb}}
{\color{black} \dottedline{3}(-33,-210)(-5,-210)}\put(0,-220){\makebox(100,10)[l]{\jkffxcc}}
{\color{black} \dottedline{3}(-33,-220)(-5,-220)}\put(0,-230){\makebox(100,10)[l]{\jkffxcd}}
{\color{black} \dottedline{3}(-33,-230)(-5,-230)}\put(0,-240){\makebox(100,10)[l]{\jkffxce}}

\color{red}
\put(-6,-45){\circle*{2}}
\put(-6,-47){\vector(0,-1){96}}
\put(-8,-47){\linethickness{0.7pt}\line(1,0){0}}
\put(-6,-145){\circle*{2}}

\color{red}
\put(-12,-75){\circle*{2}}
\put(-12,-77){\vector(0,-1){56}}
\put(-14,-77){\linethickness{0.7pt}\line(1,0){0}}
\put(-12,-135){\circle*{2}}

\color{red}
\put(-18,-92){\circle*{2}}
\put(-20,-94){\linethickness{0.7pt}\line(1,0){0}}
\put(-18,-97){\circle*{2}\hspace{-2\unitlength}\circle{4}}

\color{red}
\put(-24,-95){\circle*{2}}
\put(-24,-97){\vector(0,-1){16}}
\put(-26,-97){\linethickness{0.7pt}\line(1,0){0}}
\put(-24,-115){\circle*{2}}

\color{red}
\put(-18,-102){\circle*{2}}
\put(-20,-104){\linethickness{0.7pt}\line(1,0){0}}
\put(-18,-107){\circle*{2}\hspace{-2\unitlength}\circle{4}}

\color{red}
\put(-30,-105){\circle*{2}}
\put(-30,-107){\vector(0,-1){6}}
\put(-32,-107){\linethickness{0.7pt}\line(1,0){0}}
\put(-30,-115){\circle*{2}}

\color{red}
\put(-18,-115){\circle*{2}}
\put(-18,-117){\vector(0,-1){16}}
\put(-20,-117){\linethickness{0.7pt}\line(1,0){0}}
\put(-18,-135){\circle*{2}}

\color{red}
\put(-6,-195){\circle*{2}}
\put(-6,-197){\vector(0,-1){36}}
\put(-8,-197){\linethickness{0.7pt}\line(1,0){0}}
\put(-6,-235){\circle*{2}}

\color{red}
\put(-12,-202){\circle*{2}}
\put(-14,-204){\linethickness{0.7pt}\line(1,0){0}}
\put(-12,-207){\circle*{2}\hspace{-2\unitlength}\circle{4}}

\color{red}
\put(-18,-205){\circle*{2}}
\put(-18,-207){\vector(0,-1){16}}
\put(-20,-207){\linethickness{0.7pt}\line(1,0){0}}
\put(-18,-225){\circle*{2}}

\color{red}
\put(-12,-222){\circle*{2}}
\put(-14,-224){\linethickness{0.7pt}\line(1,0){0}}
\put(-12,-227){\circle*{2}\hspace{-2\unitlength}\circle{4}}

\end{picture}
\]
\hrulefill

\end{figure}

\begin{figure} 
\small
\verbdef\jkffxb$#include <stdlib.h> $
\verbdef\jkffxc$#include <stdio.h> $
\verbdef\jkffxd$ $
\verbdef\jkffxe$static int nfib(int n) $
\verbdef\jkffxf${ $
\verbdef\jkffxg$   int result; $
\verbdef\jkffxh$   if( n < 2 ) { $
\verbdef\jkffxi$     result = 1; $
\verbdef\jkffxj$   } else { $
\verbdef\jkffxba$     int a = nfib( n-1 ); $
\verbdef\jkffxbb$     int b = nfib( n-2 ); $
\verbdef\jkffxbc$     result = a+b; $
\verbdef\jkffxbd$   } $
\verbdef\jkffxbe$   return result; $
\verbdef\jkffxbf$} $
\verbdef\jkffxbg$      $
\verbdef\jkffxbh$  $
\verbdef\jkffxbi$ $
\verbdef\jkffxbj$int main(int argc, char **argv) $
\verbdef\jkffxca${ $
\verbdef\jkffxcb$   int m = nfib( 8 ); $
\verbdef\jkffxcc$    $
\verbdef\jkffxcd$   printf( "%d\n", m ); $
\verbdef\jkffxce$} $
\hrulefill
\[
\begin{picture}(420,240)(-40,-240)

\put(0,-10){\makebox(100,10)[l]{\jkffxb}}
\put(0,-20){\makebox(100,10)[l]{\jkffxc}}
\put(0,-30){\makebox(100,10)[l]{\jkffxd}}
\put(0,-40){\makebox(100,10)[l]{\jkffxe}}
\put(0,-50){\makebox(100,10)[l]{\jkffxf}}
{\color{black} \dottedline{3}(-10,-45)(-0,-45)}
\put(0,-60){\makebox(100,10)[l]{\jkffxg}}
\put(0,-70){\makebox(100,10)[l]{\jkffxh}}
\put(0,-80){\makebox(100,10)[l]{\jkffxi}}
{\color{black} \dottedline{3}(-10,-75)(-0,-75)}
\put(0,-90){\makebox(100,10)[l]{\jkffxj}}
\put(0,-100){\makebox(100,10)[l]{\jkffxba}}
{\color{black} \dottedline{3}(-10,-95)(-0,-95)}
\put(0,-110){\makebox(100,10)[l]{\jkffxbb}}
{\color{black} \dottedline{3}(-10,-105)(-0,-105)}
\put(0,-120){\makebox(100,10)[l]{\jkffxbc}}
{\color{black} \dottedline{3}(-10,-115)(-0,-115)}
\put(0,-130){\makebox(100,10)[l]{\jkffxbd}}
\put(0,-140){\makebox(100,10)[l]{\jkffxbe}}
{\color{black} \dottedline{3}(-10,-135)(-0,-135)}
\put(0,-150){\makebox(100,10)[l]{\jkffxbf}}
{\color{black} \dottedline{3}(-10,-145)(-0,-145)}
\put(0,-160){\makebox(100,10)[l]{\jkffxbg}}
\put(0,-170){\makebox(100,10)[l]{\jkffxbh}}
\put(0,-180){\makebox(100,10)[l]{\jkffxbi}}
\put(0,-190){\makebox(100,10)[l]{\jkffxbj}}
\put(0,-200){\makebox(100,10)[l]{\jkffxca}}
{\color{black} \dottedline{3}(-10,-195)(-0,-195)}
\put(0,-210){\makebox(100,10)[l]{\jkffxcb}}
{\color{black} \dottedline{3}(-10,-205)(-0,-205)}
\put(0,-220){\makebox(100,10)[l]{\jkffxcc}}
\put(0,-230){\makebox(100,10)[l]{\jkffxcd}}
{\color{black} \dottedline{3}(-10,-225)(-0,-225)}
\put(0,-240){\makebox(100,10)[l]{\jkffxce}}
{\color{black} \dottedline{3}(-10,-235)(-0,-235)}

\put(-10,-45){\color{black}\circle*{2}}
\put(-40,-75){\color{black}\circle*{2}}
\put(-30,-95){\color{black}\circle*{2}}
\put(-10,-195){\color{black}\circle*{2}}
\put(-20,-105){\color{black}\circle*{2}}
\put(-20,-205){\color{black}\circle*{2}}
\put(-30,-115){\color{black}\circle*{2}}
\put(-20,-225){\color{black}\circle*{2}}
\put(-40,-135){\color{black}\circle*{2}}
\put(-10,-235){\color{black}\circle*{2}}
\put(-10,-145){\color{black}\circle*{2}}
\put(-10.0,-47.0){\color{red}\vector(0,-1){96.0}}
{\color{red}\dashline[100]{2}(-10.0,-47.0)(-10.0,-143.0)}
\put(-40.0,-77.0){\color{red}\vector(0,-1){56.0}}
{\color{red}\dashline[100]{2}(-40.0,-77.0)(-40.0,-133.0)}
\put(-30.0,-97.0){\color{red}\vector(0,-1){16.0}}
{\color{red}\dashline[100]{2}(-30.0,-97.0)(-30.0,-113.0)}
\put(-21.4,-106.4){\color{red}\vector(-1,-1){7.2}}
{\color{red}\dashline[100]{2}(-21.4,-106.4)(-28.6,-113.6)}
\put(-30.9,-116.8){\color{red}\vector(-1,-2){8.2}}
{\color{red}\dashline[100]{2}(-30.9,-116.8)(-39.1,-133.2)}
\put(-10.0,-197.0){\color{red}\vector(0,-1){36.0}}
{\color{red}\dashline[100]{2}(-10.0,-197.0)(-10.0,-233.0)}
\put(-20.0,-207.0){\color{red}\vector(0,-1){16.0}}
{\color{red}\dashline[100]{2}(-20.0,-207.0)(-20.0,-223.0)}
\end{picture}
\]
\hrulefill

\end{figure}


\begin{figure*} 
\small
\verbdef\jkffxb$#include <stdlib.h> $
\verbdef\jkffxc$#include <stdio.h> $
\verbdef\jkffxd$ $
\verbdef\jkffxe$static int a[] =  $
\verbdef\jkffxf$   {17, 3, 84, 89, 4, 5, 23, 43,  $
\verbdef\jkffxg$    21, 7, 2, 1, 55, 63, 21}; $
\verbdef\jkffxh$static int n = 15; $
\verbdef\jkffxi$ $
\verbdef\jkffxj$static int *part(int *a, int n) $
\verbdef\jkffxba${ $
\verbdef\jkffxbb$   int i = a[0]; $
\verbdef\jkffxbc$   int k = a[n-1]; $
\verbdef\jkffxbd$   int *lp = a; $
\verbdef\jkffxbe$   int *hp = a+n-1; $
\verbdef\jkffxbf$ $
\verbdef\jkffxbg$   while( lp<hp ) { $
\verbdef\jkffxbh$      if( k<i ) { $
\verbdef\jkffxbi$         *lp = k; $
\verbdef\jkffxbj$          lp++; $
\verbdef\jkffxca$          k = *lp; $
\verbdef\jkffxcb$      } else { $
\verbdef\jkffxcc$          *hp = k; $
\verbdef\jkffxcd$           hp--; $
\verbdef\jkffxce$           k = *hp; $
\verbdef\jkffxcf$      } $
\verbdef\jkffxcg$   } $
\verbdef\jkffxch$   *lp = i; $
\verbdef\jkffxci$   return lp; $
\verbdef\jkffxcj$} $
\verbdef\jkffxda$ $
\verbdef\jkffxdb$static void qs(int *a, int n) $
\verbdef\jkffxdc${ $
\verbdef\jkffxdd$   if( n>1 ) { $
\verbdef\jkffxde$      int *lp = part( a, n ); $
\verbdef\jkffxdf$      int m = lp-a; $
\verbdef\jkffxdg$      qs( a, m ); $
\verbdef\jkffxdh$      qs( lp+1, n-m-1 ); $
\verbdef\jkffxdi$   } $
\verbdef\jkffxdj$} $
\verbdef\jkffxea$ $
\verbdef\jkffxeb$  $
\verbdef\jkffxec$ $
\verbdef\jkffxed$int main(int argc, char **argv) $
\verbdef\jkffxee${ $
\verbdef\jkffxef$   int i; $
\verbdef\jkffxeg$    $
\verbdef\jkffxeh$   qs( a, n ); $
\verbdef\jkffxei$   for( i=0; i<n; i++ ) { $
\verbdef\jkffxej$      printf( "%d ", a[i] ); $
\verbdef\jkffxfa$   } $
\verbdef\jkffxfb$   printf( "\n" ); $
\verbdef\jkffxfc$ $
\verbdef\jkffxfd$} $
\hrulefill
\[
\begin{picture}(420,530)(-68,-530)

\put(-68,0){\makebox(15,10)[r]{\it 0:}}%
\put(0,0){\makebox(100,10)[l]{}}
\put(-68,-10){\makebox(15,10)[r]{\it 1:}}%
\put(0,-10){\makebox(100,10)[l]{\jkffxb}}
\put(-68,-20){\makebox(15,10)[r]{\it 2:}}%
\put(0,-20){\makebox(100,10)[l]{\jkffxc}}
\put(-68,-30){\makebox(15,10)[r]{\it 3:}}%
\put(0,-30){\makebox(100,10)[l]{\jkffxd}}
\put(-68,-40){\makebox(15,10)[r]{\it 4:}}%
\put(0,-40){\makebox(100,10)[l]{\jkffxe}}
\put(-68,-50){\makebox(15,10)[r]{\it 5:}}%
\put(0,-50){\makebox(100,10)[l]{\jkffxf}}
\put(-68,-60){\makebox(15,10)[r]{\it 6:}}%
\put(0,-60){\makebox(100,10)[l]{\jkffxg}}
\put(-68,-70){\makebox(15,10)[r]{\it 7:}}%
\put(0,-70){\makebox(100,10)[l]{\jkffxh}}
\put(-68,-80){\makebox(15,10)[r]{\it 8:}}%
\put(0,-80){\makebox(100,10)[l]{\jkffxi}}
\put(-68,-90){\makebox(15,10)[r]{\it 9:}}%
\put(0,-90){\makebox(100,10)[l]{\jkffxj}}
\put(-68,-100){\makebox(15,10)[r]{\it 10:}}%
\put(0,-100){\makebox(100,10)[l]{\jkffxba}}
\put(-68,-110){\makebox(15,10)[r]{\it 11:}}%
\put(0,-110){\makebox(100,10)[l]{\jkffxbb}}
\put(-68,-120){\makebox(15,10)[r]{\it 12:}}%
\put(0,-120){\makebox(100,10)[l]{\jkffxbc}}
\put(-68,-130){\makebox(15,10)[r]{\it 13:}}%
\put(0,-130){\makebox(100,10)[l]{\jkffxbd}}
\put(-68,-140){\makebox(15,10)[r]{\it 14:}}%
\put(0,-140){\makebox(100,10)[l]{\jkffxbe}}
\put(-68,-150){\makebox(15,10)[r]{\it 15:}}%
\put(0,-150){\makebox(100,10)[l]{\jkffxbf}}
\put(-68,-160){\makebox(15,10)[r]{\it 16:}}%
\put(0,-160){\makebox(100,10)[l]{\jkffxbg}}
\put(-68,-170){\makebox(15,10)[r]{\it 17:}}%
\put(0,-170){\makebox(100,10)[l]{\jkffxbh}}
\put(-68,-180){\makebox(15,10)[r]{\it 18:}}%
\put(0,-180){\makebox(100,10)[l]{\jkffxbi}}
\put(-68,-190){\makebox(15,10)[r]{\it 19:}}%
\put(0,-190){\makebox(100,10)[l]{\jkffxbj}}
\put(-68,-200){\makebox(15,10)[r]{\it 20:}}%
\put(0,-200){\makebox(100,10)[l]{\jkffxca}}
\put(-68,-210){\makebox(15,10)[r]{\it 21:}}%
\put(0,-210){\makebox(100,10)[l]{\jkffxcb}}
\put(-68,-220){\makebox(15,10)[r]{\it 22:}}%
\put(0,-220){\makebox(100,10)[l]{\jkffxcc}}
\put(-68,-230){\makebox(15,10)[r]{\it 23:}}%
\put(0,-230){\makebox(100,10)[l]{\jkffxcd}}
\put(-68,-240){\makebox(15,10)[r]{\it 24:}}%
\put(0,-240){\makebox(100,10)[l]{\jkffxce}}
\put(-68,-250){\makebox(15,10)[r]{\it 25:}}%
\put(0,-250){\makebox(100,10)[l]{\jkffxcf}}
\put(-68,-260){\makebox(15,10)[r]{\it 26:}}%
\put(0,-260){\makebox(100,10)[l]{\jkffxcg}}
\put(-68,-270){\makebox(15,10)[r]{\it 27:}}%
\put(0,-270){\makebox(100,10)[l]{\jkffxch}}
\put(-68,-280){\makebox(15,10)[r]{\it 28:}}%
\put(0,-280){\makebox(100,10)[l]{\jkffxci}}
\put(-68,-290){\makebox(15,10)[r]{\it 29:}}%
\put(0,-290){\makebox(100,10)[l]{\jkffxcj}}
\put(-68,-300){\makebox(15,10)[r]{\it 30:}}%
\put(0,-300){\makebox(100,10)[l]{\jkffxda}}
\put(-68,-310){\makebox(15,10)[r]{\it 31:}}%
\put(0,-310){\makebox(100,10)[l]{\jkffxdb}}
\put(-68,-320){\makebox(15,10)[r]{\it 32:}}%
\put(0,-320){\makebox(100,10)[l]{\jkffxdc}}
{\color{black} \dottedline{3}(-45,-320)(-5,-320)}\put(-68,-330){\makebox(15,10)[r]{\it 33:}}%
\put(0,-330){\makebox(100,10)[l]{\jkffxdd}}
{\color{black} \dottedline{3}(-45,-330)(-5,-330)}\put(-68,-340){\makebox(15,10)[r]{\it 34:}}%
\put(0,-340){\makebox(100,10)[l]{\jkffxde}}
{\color{black} \dottedline{3}(-45,-340)(-5,-340)}\put(-68,-350){\makebox(15,10)[r]{\it 35:}}%
\put(0,-350){\makebox(100,10)[l]{\jkffxdf}}
{\color{black} \dottedline{3}(-45,-350)(-5,-350)}\put(-68,-360){\makebox(15,10)[r]{\it 36:}}%
\put(0,-360){\makebox(100,10)[l]{\jkffxdg}}
{\color{black} \dottedline{3}(-45,-360)(-5,-360)}\put(-68,-370){\makebox(15,10)[r]{\it 37:}}%
\put(0,-370){\makebox(100,10)[l]{\jkffxdh}}
{\color{black} \dottedline{3}(-45,-370)(-5,-370)}\put(-68,-380){\makebox(15,10)[r]{\it 38:}}%
\put(0,-380){\makebox(100,10)[l]{\jkffxdi}}
{\color{black} \dottedline{3}(-45,-380)(-5,-380)}\put(-68,-390){\makebox(15,10)[r]{\it 39:}}%
\put(0,-390){\makebox(100,10)[l]{\jkffxdj}}
\put(-68,-400){\makebox(15,10)[r]{\it 40:}}%
\put(0,-400){\makebox(100,10)[l]{\jkffxea}}
\put(-68,-410){\makebox(15,10)[r]{\it 41:}}%
\put(0,-410){\makebox(100,10)[l]{\jkffxeb}}
\put(-68,-420){\makebox(15,10)[r]{\it 42:}}%
\put(0,-420){\makebox(100,10)[l]{\jkffxec}}
\put(-68,-430){\makebox(15,10)[r]{\it 43:}}%
\put(0,-430){\makebox(100,10)[l]{\jkffxed}}
\put(-68,-440){\makebox(15,10)[r]{\it 44:}}%
\put(0,-440){\makebox(100,10)[l]{\jkffxee}}
\put(-68,-450){\makebox(15,10)[r]{\it 45:}}%
\put(0,-450){\makebox(100,10)[l]{\jkffxef}}
\put(-68,-460){\makebox(15,10)[r]{\it 46:}}%
\put(0,-460){\makebox(100,10)[l]{\jkffxeg}}
\put(-68,-470){\makebox(15,10)[r]{\it 47:}}%
\put(0,-470){\makebox(100,10)[l]{\jkffxeh}}
\put(-68,-480){\makebox(15,10)[r]{\it 48:}}%
\put(0,-480){\makebox(100,10)[l]{\jkffxei}}
\put(-68,-490){\makebox(15,10)[r]{\it 49:}}%
\put(0,-490){\makebox(100,10)[l]{\jkffxej}}
\put(-68,-500){\makebox(15,10)[r]{\it 50:}}%
\put(0,-500){\makebox(100,10)[l]{\jkffxfa}}
\put(-68,-510){\makebox(15,10)[r]{\it 51:}}%
\put(0,-510){\makebox(100,10)[l]{\jkffxfb}}
\put(-68,-520){\makebox(15,10)[r]{\it 52:}}%
\put(0,-520){\makebox(100,10)[l]{\jkffxfc}}
\put(-68,-530){\makebox(15,10)[r]{\it 53:}}%
\put(0,-530){\makebox(100,10)[l]{\jkffxfd}}

\color{red}
\put(-6,-315){\circle*{2}}
\put(-6,-317){\vector(0,-1){66}}
\put(-8,-317){\linethickness{0.7pt}\line(1,0){0}}
\put(-6,-385){\circle*{2}}

\color{red}
\put(-12,-332){\circle*{2}}
\put(-14,-334){\linethickness{0.7pt}\line(1,0){0}}
\put(-12,-337){\circle*{2}\hspace{-2\unitlength}\circle{4}}

\color{red}
\put(-18,-335){\circle*{2}}
\put(-18,-337){\vector(0,-1){6}}
\put(-20,-337){\linethickness{0.7pt}\line(1,0){0}}
\put(-18,-345){\circle*{2}}

\color{red}
\put(-24,-335){\circle*{2}}
\put(-24,-337){\vector(0,-1){26}}
\put(-26,-337){\linethickness{0.7pt}\line(1,0){0}}
\put(-24,-365){\circle*{2}}

\color{red}
\put(-30,-335){\circle*{2}\hspace{-2\unitlength}\circle{4}}
\put(-30,-337){\vector(0,-1){16}}
\put(-32,-337){\linethickness{0.7pt}\line(1,0){4}}
\put(-30,-355){\circle*{2}\hspace{-2\unitlength}\circle{4}}

\color{red}
\put(-36,-335){\circle*{2}\hspace{-2\unitlength}\circle{4}}
\put(-36,-337){\vector(0,-1){26}}
\put(-38,-337){\linethickness{0.7pt}\line(1,0){4}}
\put(-36,-365){\circle*{2}\hspace{-2\unitlength}\circle{4}}

\color{red}
\put(-12,-345){\circle*{2}}
\put(-12,-347){\vector(0,-1){6}}
\put(-14,-347){\linethickness{0.7pt}\line(1,0){0}}
\put(-12,-355){\circle*{2}}

\color{red}
\put(-42,-345){\circle*{2}}
\put(-42,-347){\vector(0,-1){16}}
\put(-44,-347){\linethickness{0.7pt}\line(1,0){0}}
\put(-42,-365){\circle*{2}}

\color{red}
\put(-18,-352){\circle*{2}}
\put(-20,-354){\linethickness{0.7pt}\line(1,0){0}}
\put(-18,-357){\circle*{2}\hspace{-2\unitlength}\circle{4}}

\color{red}
\put(-12,-362){\circle*{2}}
\put(-14,-364){\linethickness{0.7pt}\line(1,0){0}}
\put(-12,-367){\circle*{2}\hspace{-2\unitlength}\circle{4}}

\end{picture}
\]
\hrulefill

\end{figure*}

\begin{figure} 
\small
\verbdef\jkffxb$#include <stdlib.h> $
\verbdef\jkffxc$#include <stdio.h> $
\verbdef\jkffxd$ $
\verbdef\jkffxe$static int a[] =  $
\verbdef\jkffxf$   {17, 3, 84, 89, 4, 5, 23, 43,  $
\verbdef\jkffxg$    21, 7, 2, 1, 55, 63, 21}; $
\verbdef\jkffxh$static int n = 15; $
\verbdef\jkffxi$ $
\verbdef\jkffxj$static int *part(int *a, int n) $
\verbdef\jkffxba${ $
\verbdef\jkffxbb$   int i = a[0]; $
\verbdef\jkffxbc$   int k = a[n-1]; $
\verbdef\jkffxbd$   int *lp = a; $
\verbdef\jkffxbe$   int *hp = a+n-1; $
\verbdef\jkffxbf$ $
\verbdef\jkffxbg$   while( lp<hp ) { $
\verbdef\jkffxbh$      if( k<i ) { $
\verbdef\jkffxbi$         *lp = k; $
\verbdef\jkffxbj$          lp++; $
\verbdef\jkffxca$          k = *lp; $
\verbdef\jkffxcb$      } else { $
\verbdef\jkffxcc$          *hp = k; $
\verbdef\jkffxcd$           hp--; $
\verbdef\jkffxce$           k = *hp; $
\verbdef\jkffxcf$      } $
\verbdef\jkffxcg$   } $
\verbdef\jkffxch$   *lp = i; $
\verbdef\jkffxci$   return lp; $
\verbdef\jkffxcj$} $
\verbdef\jkffxda$ $
\verbdef\jkffxdb$static void qs(int *a, int n) $
\verbdef\jkffxdc${ $
\verbdef\jkffxdd$   if( n>1 ) { $
\verbdef\jkffxde$      int *lp = part( a, n ); $
\verbdef\jkffxdf$      int m = lp-a; $
\verbdef\jkffxdg$      qs( a, m ); $
\verbdef\jkffxdh$      qs( lp+1, n-m-1 ); $
\verbdef\jkffxdi$   } $
\verbdef\jkffxdj$} $
\verbdef\jkffxea$ $
\verbdef\jkffxeb$  $
\verbdef\jkffxec$ $
\verbdef\jkffxed$int main(int argc, char **argv) $
\verbdef\jkffxee${ $
\verbdef\jkffxef$   int i; $
\verbdef\jkffxeg$    $
\verbdef\jkffxeh$   qs( a, n ); $
\verbdef\jkffxei$   for( i=0; i<n; i++ ) { $
\verbdef\jkffxej$      printf( "%d ", a[i] ); $
\verbdef\jkffxfa$   } $
\verbdef\jkffxfb$   printf( "\n" ); $
\verbdef\jkffxfc$ $
\verbdef\jkffxfd$} $
\hrulefill
\[
\begin{picture}(420,530)(-100,-530)

\put(-100,-10){\makebox(15,10)[r]{\it 1:}}%
\put(0,-10){\makebox(100,10)[l]{\jkffxb}}
\put(-100,-20){\makebox(15,10)[r]{\it 2:}}%
\put(0,-20){\makebox(100,10)[l]{\jkffxc}}
\put(-100,-30){\makebox(15,10)[r]{\it 3:}}%
\put(0,-30){\makebox(100,10)[l]{\jkffxd}}
\put(-100,-40){\makebox(15,10)[r]{\it 4:}}%
\put(0,-40){\makebox(100,10)[l]{\jkffxe}}
\put(-100,-50){\makebox(15,10)[r]{\it 5:}}%
\put(0,-50){\makebox(100,10)[l]{\jkffxf}}
\put(-100,-60){\makebox(15,10)[r]{\it 6:}}%
\put(0,-60){\makebox(100,10)[l]{\jkffxg}}
\put(-100,-70){\makebox(15,10)[r]{\it 7:}}%
\put(0,-70){\makebox(100,10)[l]{\jkffxh}}
\put(-100,-80){\makebox(15,10)[r]{\it 8:}}%
\put(0,-80){\makebox(100,10)[l]{\jkffxi}}
\put(-100,-90){\makebox(15,10)[r]{\it 9:}}%
\put(0,-90){\makebox(100,10)[l]{\jkffxj}}
\put(-100,-100){\makebox(15,10)[r]{\it 10:}}%
\put(0,-100){\makebox(100,10)[l]{\jkffxba}}
{\color{black} \dottedline{3}(-10,-95)(0,-95)}
\put(-100,-110){\makebox(15,10)[r]{\it 11:}}%
\put(0,-110){\makebox(100,10)[l]{\jkffxbb}}
{\color{black} \dottedline{3}(-10,-105)(0,-105)}
\put(-100,-120){\makebox(15,10)[r]{\it 12:}}%
\put(0,-120){\makebox(100,10)[l]{\jkffxbc}}
{\color{black} \dottedline{3}(-10,-115)(0,-115)}
\put(-100,-130){\makebox(15,10)[r]{\it 13:}}%
\put(0,-130){\makebox(100,10)[l]{\jkffxbd}}
{\color{black} \dottedline{3}(-10,-125)(0,-125)}
\put(-100,-140){\makebox(15,10)[r]{\it 14:}}%
\put(0,-140){\makebox(100,10)[l]{\jkffxbe}}
{\color{black} \dottedline{3}(-10,-135)(0,-135)}
\put(-100,-150){\makebox(15,10)[r]{\it 15:}}%
\put(0,-150){\makebox(100,10)[l]{\jkffxbf}}
\put(-100,-160){\makebox(15,10)[r]{\it 16:}}%
\put(0,-160){\makebox(100,10)[l]{\jkffxbg}}
{\color{black} \dottedline{3}(-10,-155)(0,-155)}
\put(-100,-170){\makebox(15,10)[r]{\it 17:}}%
\put(0,-170){\makebox(100,10)[l]{\jkffxbh}}
{\color{black} \dottedline{3}(-10,-165)(0,-165)}
\put(-100,-180){\makebox(15,10)[r]{\it 18:}}%
\put(0,-180){\makebox(100,10)[l]{\jkffxbi}}
{\color{black} \dottedline{3}(-10,-175)(0,-175)}
\put(-100,-190){\makebox(15,10)[r]{\it 19:}}%
\put(0,-190){\makebox(100,10)[l]{\jkffxbj}}
{\color{black} \dottedline{3}(-10,-185)(0,-185)}
\put(-100,-200){\makebox(15,10)[r]{\it 20:}}%
\put(0,-200){\makebox(100,10)[l]{\jkffxca}}
{\color{black} \dottedline{3}(-10,-195)(0,-195)}
\put(-100,-210){\makebox(15,10)[r]{\it 21:}}%
\put(0,-210){\makebox(100,10)[l]{\jkffxcb}}
\put(-100,-220){\makebox(15,10)[r]{\it 22:}}%
\put(0,-220){\makebox(100,10)[l]{\jkffxcc}}
{\color{black} \dottedline{3}(-10,-215)(0,-215)}
\put(-100,-230){\makebox(15,10)[r]{\it 23:}}%
\put(0,-230){\makebox(100,10)[l]{\jkffxcd}}
{\color{black} \dottedline{3}(-10,-225)(0,-225)}
\put(-100,-240){\makebox(15,10)[r]{\it 24:}}%
\put(0,-240){\makebox(100,10)[l]{\jkffxce}}
{\color{black} \dottedline{3}(-10,-235)(0,-235)}
\put(-100,-250){\makebox(15,10)[r]{\it 25:}}%
\put(0,-250){\makebox(100,10)[l]{\jkffxcf}}
\put(-100,-260){\makebox(15,10)[r]{\it 26:}}%
\put(0,-260){\makebox(100,10)[l]{\jkffxcg}}
\put(-100,-270){\makebox(15,10)[r]{\it 27:}}%
\put(0,-270){\makebox(100,10)[l]{\jkffxch}}
{\color{black} \dottedline{3}(-10,-265)(0,-265)}
\put(-100,-280){\makebox(15,10)[r]{\it 28:}}%
\put(0,-280){\makebox(100,10)[l]{\jkffxci}}
{\color{black} \dottedline{3}(-10,-275)(0,-275)}
\put(-100,-290){\makebox(15,10)[r]{\it 29:}}%
\put(0,-290){\makebox(100,10)[l]{\jkffxcj}}
{\color{black} \dottedline{3}(-10,-285)(0,-285)}
\put(-100,-300){\makebox(15,10)[r]{\it 30:}}%
\put(0,-300){\makebox(100,10)[l]{\jkffxda}}
\put(-100,-310){\makebox(15,10)[r]{\it 31:}}%
\put(0,-310){\makebox(100,10)[l]{\jkffxdb}}
\put(-100,-320){\makebox(15,10)[r]{\it 32:}}%
\put(0,-320){\makebox(100,10)[l]{\jkffxdc}}
{\color{black} \dottedline{3}(-10,-315)(0,-315)}
\put(-100,-330){\makebox(15,10)[r]{\it 33:}}%
\put(0,-330){\makebox(100,10)[l]{\jkffxdd}}
\put(-100,-340){\makebox(15,10)[r]{\it 34:}}%
\put(0,-340){\makebox(100,10)[l]{\jkffxde}}
{\color{black} \dottedline{3}(-10,-335)(0,-335)}
\put(-100,-350){\makebox(15,10)[r]{\it 35:}}%
\put(0,-350){\makebox(100,10)[l]{\jkffxdf}}
{\color{black} \dottedline{3}(-10,-345)(0,-345)}
\put(-100,-360){\makebox(15,10)[r]{\it 36:}}%
\put(0,-360){\makebox(100,10)[l]{\jkffxdg}}
{\color{black} \dottedline{3}(-10,-355)(0,-355)}
\put(-100,-370){\makebox(15,10)[r]{\it 37:}}%
\put(0,-370){\makebox(100,10)[l]{\jkffxdh}}
{\color{black} \dottedline{3}(-10,-365)(0,-365)}
\put(-100,-380){\makebox(15,10)[r]{\it 38:}}%
\put(0,-380){\makebox(100,10)[l]{\jkffxdi}}
\put(-100,-390){\makebox(15,10)[r]{\it 39:}}%
\put(0,-390){\makebox(100,10)[l]{\jkffxdj}}
{\color{black} \dottedline{3}(-10,-385)(0,-385)}
\put(-100,-400){\makebox(15,10)[r]{\it 40:}}%
\put(0,-400){\makebox(100,10)[l]{\jkffxea}}
\put(-100,-410){\makebox(15,10)[r]{\it 41:}}%
\put(0,-410){\makebox(100,10)[l]{\jkffxeb}}
\put(-100,-420){\makebox(15,10)[r]{\it 42:}}%
\put(0,-420){\makebox(100,10)[l]{\jkffxec}}
\put(-100,-430){\makebox(15,10)[r]{\it 43:}}%
\put(0,-430){\makebox(100,10)[l]{\jkffxed}}
\put(-100,-440){\makebox(15,10)[r]{\it 44:}}%
\put(0,-440){\makebox(100,10)[l]{\jkffxee}}
{\color{black} \dottedline{3}(-10,-435)(0,-435)}
\put(-100,-450){\makebox(15,10)[r]{\it 45:}}%
\put(0,-450){\makebox(100,10)[l]{\jkffxef}}
\put(-100,-460){\makebox(15,10)[r]{\it 46:}}%
\put(0,-460){\makebox(100,10)[l]{\jkffxeg}}
\put(-100,-470){\makebox(15,10)[r]{\it 47:}}%
\put(0,-470){\makebox(100,10)[l]{\jkffxeh}}
{\color{black} \dottedline{3}(-10,-465)(0,-465)}
\put(-100,-480){\makebox(15,10)[r]{\it 48:}}%
\put(0,-480){\makebox(100,10)[l]{\jkffxei}}
{\color{black} \dottedline{3}(-10,-475)(0,-475)}
\put(-100,-490){\makebox(15,10)[r]{\it 49:}}%
\put(0,-490){\makebox(100,10)[l]{\jkffxej}}
{\color{black} \dottedline{3}(-10,-485)(0,-485)}
\put(-100,-500){\makebox(15,10)[r]{\it 50:}}%
\put(0,-500){\makebox(100,10)[l]{\jkffxfa}}
\put(-100,-510){\makebox(15,10)[r]{\it 51:}}%
\put(0,-510){\makebox(100,10)[l]{\jkffxfb}}
{\color{black} \dottedline{3}(-10,-505)(0,-505)}
\put(-100,-520){\makebox(15,10)[r]{\it 52:}}%
\put(0,-520){\makebox(100,10)[l]{\jkffxfc}}
\put(-100,-530){\makebox(15,10)[r]{\it 53:}}%
\put(0,-530){\makebox(100,10)[l]{\jkffxfd}}
{\color{black} \dottedline{3}(-10,-525)(0,-525)}

\put(-10,-525){\color{black}\circle*{2}}
\put(-10,-435){\color{black}\circle*{2}}
\put(-80,-165){\color{black}\circle*{2}}
\put(-20,-345){\color{black}\circle*{2}}
\put(-70,-265){\color{black}\circle*{2}}
\put(-20,-175){\color{black}\circle*{2}}
\put(-20,-355){\color{black}\circle*{2}}
\put(-30,-275){\color{black}\circle*{2}}
\put(-20,-185){\color{black}\circle*{2}}
\put(-30,-365){\color{black}\circle*{2}}
\put(-30,-465){\color{black}\circle*{2}}
\put(-10,-285){\color{black}\circle*{2}}
\put(-20,-475){\color{black}\circle*{2}}
\put(-10,-385){\color{black}\circle*{2}}
\put(-30,-485){\color{black}\circle*{2}}
\put(-10,-95){\color{black}\circle*{2}}
\put(-80,-195){\color{black}\circle*{2}}
\put(-70,-105){\color{black}\circle*{2}}
\put(-60,-115){\color{black}\circle*{2}}
\put(-40,-125){\color{black}\circle*{2}}
\put(-60,-215){\color{black}\circle*{2}}
\put(-50,-225){\color{black}\circle*{2}}
\put(-50,-135){\color{black}\circle*{2}}
\put(-10,-315){\color{black}\circle*{2}}
\put(-30,-505){\color{black}\circle*{2}}
\put(-60,-235){\color{black}\circle*{2}}
\put(-40,-155){\color{black}\circle*{2}}
\put(-30,-335){\color{black}\circle*{2}}
{\color{red}\dashline[200]{3}(-10.0,-97.0)(-10.0,-283.0)}%
{\color{red}\dashline[200]{3}(-10.0,-283.0)(-9.0,-281.0)}%
{\color{red}\dashline[200]{3}(-10.0,-283.0)(-11.0,-281.0)}%
{\color{red}\dashline[200]{3}(-70.3,-107.0)(-79.7,-163.0)}%
{\color{red}\dashline[200]{3}(-79.7,-163.0)(-78.4,-161.2)}%
{\color{red}\dashline[200]{3}(-79.7,-163.0)(-80.3,-160.9)}%
{\color{green}\dashline[200]{3}(-68.8,-106.6)(-21.2,-173.4)}%
{\color{green}\dashline[200]{3}(-21.2,-173.4)(-21.5,-171.2)}%
{\color{green}\dashline[200]{3}(-21.2,-173.4)(-23.1,-172.3)}%
{\color{red}\dashline[200]{3}(-70.0,-107.0)(-70.0,-263.0)}%
{\color{red}\dashline[200]{3}(-70.0,-263.0)(-69.0,-261.0)}%
{\color{red}\dashline[200]{3}(-70.0,-263.0)(-71.0,-261.0)}%
{\color{red}\dashline[200]{3}(-60.7,-116.9)(-79.3,-163.1)}%
{\color{red}\dashline[200]{3}(-79.3,-163.1)(-77.6,-161.7)}%
{\color{red}\dashline[200]{3}(-79.3,-163.1)(-79.4,-160.9)}%
{\color{red}\dashline[200]{3}(-58.9,-116.7)(-21.1,-173.3)}%
{\color{red}\dashline[200]{3}(-21.1,-173.3)(-21.4,-171.1)}%
{\color{red}\dashline[200]{3}(-21.1,-173.3)(-23.1,-172.2)}%
{\color{red}\dashline[200]{3}(-60.0,-117.0)(-60.0,-213.0)}%
{\color{red}\dashline[200]{3}(-60.0,-213.0)(-59.0,-211.0)}%
{\color{red}\dashline[200]{3}(-60.0,-213.0)(-61.0,-211.0)}%
{\color{green}\dashline[200]{3}(-60.1,-117.0)(-69.9,-263.0)}%
{\color{green}\dashline[200]{3}(-69.9,-263.0)(-68.7,-261.1)}%
{\color{green}\dashline[200]{3}(-69.9,-263.0)(-70.7,-260.9)}%
{\color{red}\dashline[200]{3}(-40.0,-127.0)(-40.0,-153.0)}%
{\color{red}\dashline[200]{3}(-40.0,-153.0)(-39.0,-151.0)}%
{\color{red}\dashline[200]{3}(-40.0,-153.0)(-41.0,-151.0)}%
{\color{red}\dashline[200]{3}(-39.3,-126.9)(-20.7,-173.1)}%
{\color{red}\dashline[200]{3}(-20.7,-173.1)(-20.6,-170.9)}%
{\color{red}\dashline[200]{3}(-20.7,-173.1)(-22.4,-171.7)}%
{\color{red}\dashline[200]{3}(-39.4,-126.9)(-20.6,-183.1)}%
{\color{red}\dashline[200]{3}(-20.6,-183.1)(-20.3,-180.9)}%
{\color{red}\dashline[200]{3}(-20.6,-183.1)(-22.2,-181.5)}%
{\color{red}\dashline[200]{3}(-40.4,-127.0)(-69.6,-263.0)}%
{\color{red}\dashline[200]{3}(-69.6,-263.0)(-68.2,-261.3)}%
{\color{red}\dashline[200]{3}(-69.6,-263.0)(-70.1,-260.9)}%
{\color{red}\dashline[200]{3}(-39.9,-127.0)(-30.1,-273.0)}%
{\color{red}\dashline[200]{3}(-30.1,-273.0)(-29.3,-270.9)}%
{\color{red}\dashline[200]{3}(-30.1,-273.0)(-31.3,-271.1)}%
{\color{red}\dashline[200]{3}(-49.1,-136.8)(-40.9,-153.2)}%
{\color{red}\dashline[200]{3}(-40.9,-153.2)(-40.9,-151.0)}%
{\color{red}\dashline[200]{3}(-40.9,-153.2)(-42.7,-151.9)}%
{\color{red}\dashline[200]{3}(-50.2,-137.0)(-59.8,-213.0)}%
{\color{red}\dashline[200]{3}(-59.8,-213.0)(-58.5,-211.2)}%
{\color{red}\dashline[200]{3}(-59.8,-213.0)(-60.5,-210.9)}%
{\color{red}\dashline[200]{3}(-50.0,-137.0)(-50.0,-223.0)}%
{\color{red}\dashline[200]{3}(-50.0,-223.0)(-49.0,-221.0)}%
{\color{red}\dashline[200]{3}(-50.0,-223.0)(-51.0,-221.0)}%
{\color{green}\dashline[200]{3}(-38.1,-156.1)(-20.3,-182.8)}%
{\color{green}\dashline[200]{3}(-20.3,-182.8)(-20.6,-180.6)}%
{\color{green}\dashline[200]{3}(-20.3,-182.8)(-22.2,-181.7)}%
{\color{green}\dashline[200]{3}(-39.3,-157.1)(-48.7,-223.2)}%
{\color{green}\dashline[200]{3}(-48.7,-223.2)(-47.5,-221.3)}%
{\color{green}\dashline[200]{3}(-48.7,-223.2)(-49.4,-221.0)}%
{\color{green}\dashline[200]{3}(-79.0,-167.0)(-79.0,-193.0)}%
{\color{green}\dashline[200]{3}(-79.0,-193.0)(-78.0,-191.0)}%
{\color{green}\dashline[200]{3}(-79.0,-193.0)(-80.0,-191.0)}%
{\color{green}\dashline[200]{3}(-78.5,-166.6)(-59.6,-232.8)}%
{\color{green}\dashline[200]{3}(-59.6,-232.8)(-59.2,-230.6)}%
{\color{green}\dashline[200]{3}(-59.6,-232.8)(-61.1,-231.2)}%
{\color{green}\dashline[200]{3}(-19.0,-177.0)(-19.0,-183.0)}%
{\color{green}\dashline[200]{3}(-19.0,-183.0)(-18.0,-181.0)}%
{\color{green}\dashline[200]{3}(-19.0,-183.0)(-20.0,-181.0)}%
{\color{green}\dashline[200]{3}(-21.6,-176.6)(-77.8,-195.3)}%
{\color{green}\dashline[200]{3}(-77.8,-195.3)(-75.6,-195.6)}%
{\color{green}\dashline[200]{3}(-77.8,-195.3)(-76.2,-193.7)}%
{\color{red}\dashline[200]{3}(-21.9,-183.9)(-39.7,-157.2)}%
{\color{red}\dashline[200]{3}(-39.7,-157.2)(-39.4,-159.4)}%
{\color{red}\dashline[200]{3}(-39.7,-157.2)(-37.8,-158.3)}%
{\color{red}\dashline[200]{3}(-21.0,-183.0)(-21.0,-177.0)}%
{\color{red}\dashline[200]{3}(-21.0,-177.0)(-22.0,-179.0)}%
{\color{red}\dashline[200]{3}(-21.0,-177.0)(-20.0,-179.0)}%
{\color{red}\dashline[200]{3}(-21.8,-186.3)(-77.9,-195.7)}%
{\color{red}\dashline[200]{3}(-77.9,-195.7)(-75.7,-196.3)}%
{\color{red}\dashline[200]{3}(-77.9,-195.7)(-76.1,-194.3)}%
{\color{red}\dashline[200]{3}(-21.1,-186.7)(-68.9,-263.3)}%
{\color{red}\dashline[200]{3}(-68.9,-263.3)(-67.0,-262.1)}%
{\color{red}\dashline[200]{3}(-68.9,-263.3)(-68.7,-261.1)}%
{\color{red}\dashline[200]{3}(-20.2,-187.0)(-29.8,-273.0)}%
{\color{red}\dashline[200]{3}(-29.8,-273.0)(-28.6,-271.1)}%
{\color{red}\dashline[200]{3}(-29.8,-273.0)(-30.6,-270.9)}%
{\color{red}\dashline[200]{3}(-81.0,-193.0)(-81.0,-167.0)}%
{\color{red}\dashline[200]{3}(-81.0,-167.0)(-82.0,-169.0)}%
{\color{red}\dashline[200]{3}(-81.0,-167.0)(-80.0,-169.0)}%
{\color{red}\dashline[200]{3}(-78.4,-193.4)(-22.2,-174.7)}%
{\color{red}\dashline[200]{3}(-22.2,-174.7)(-24.4,-174.4)}%
{\color{red}\dashline[200]{3}(-22.2,-174.7)(-23.8,-176.3)}%
{\color{green}\dashline[200]{3}(-78.2,-193.7)(-22.1,-184.3)}%
{\color{green}\dashline[200]{3}(-22.1,-184.3)(-24.3,-183.7)}%
{\color{green}\dashline[200]{3}(-22.1,-184.3)(-23.9,-185.7)}%
{\color{red}\dashline[200]{3}(-78.6,-196.4)(-61.4,-213.6)}%
{\color{red}\dashline[200]{3}(-61.4,-213.6)(-62.1,-211.5)}%
{\color{red}\dashline[200]{3}(-61.4,-213.6)(-63.5,-212.9)}%
{\color{green}\dashline[200]{3}(-79.7,-197.0)(-70.3,-263.0)}%
{\color{green}\dashline[200]{3}(-70.3,-263.0)(-69.6,-260.9)}%
{\color{green}\dashline[200]{3}(-70.3,-263.0)(-71.6,-261.2)}%
{\color{green}\dashline[200]{3}(-57.9,-215.7)(-50.7,-222.9)}%
{\color{green}\dashline[200]{3}(-50.7,-222.9)(-51.4,-220.8)}%
{\color{green}\dashline[200]{3}(-50.7,-222.9)(-52.8,-222.2)}%
{\color{green}\dashline[200]{3}(-59.0,-217.0)(-59.0,-233.0)}%
{\color{green}\dashline[200]{3}(-59.0,-233.0)(-58.0,-231.0)}%
{\color{green}\dashline[200]{3}(-59.0,-233.0)(-60.0,-231.0)}%
{\color{red}\dashline[200]{3}(-50.7,-222.9)(-41.3,-156.8)}%
{\color{red}\dashline[200]{3}(-41.3,-156.8)(-42.5,-158.7)}%
{\color{red}\dashline[200]{3}(-41.3,-156.8)(-40.6,-159.0)}%
{\color{red}\dashline[200]{3}(-52.1,-224.3)(-59.3,-217.1)}%
{\color{red}\dashline[200]{3}(-59.3,-217.1)(-58.6,-219.2)}%
{\color{red}\dashline[200]{3}(-59.3,-217.1)(-57.2,-217.8)}%
{\color{red}\dashline[200]{3}(-50.7,-227.1)(-57.9,-234.3)}%
{\color{red}\dashline[200]{3}(-57.9,-234.3)(-55.8,-233.6)}%
{\color{red}\dashline[200]{3}(-57.9,-234.3)(-57.2,-232.2)}%
{\color{red}\dashline[200]{3}(-61.5,-233.4)(-80.4,-167.2)}%
{\color{red}\dashline[200]{3}(-80.4,-167.2)(-80.8,-169.4)}%
{\color{red}\dashline[200]{3}(-80.4,-167.2)(-78.9,-168.8)}%
{\color{red}\dashline[200]{3}(-58.9,-233.3)(-21.1,-176.7)}%
{\color{red}\dashline[200]{3}(-21.1,-176.7)(-23.1,-177.8)}%
{\color{red}\dashline[200]{3}(-21.1,-176.7)(-21.4,-178.9)}%
{\color{red}\dashline[200]{3}(-61.0,-233.0)(-61.0,-217.0)}%
{\color{red}\dashline[200]{3}(-61.0,-217.0)(-62.0,-219.0)}%
{\color{red}\dashline[200]{3}(-61.0,-217.0)(-60.0,-219.0)}%
{\color{green}\dashline[200]{3}(-59.3,-232.9)(-52.1,-225.7)}%
{\color{green}\dashline[200]{3}(-52.1,-225.7)(-54.2,-226.4)}%
{\color{green}\dashline[200]{3}(-52.1,-225.7)(-52.8,-227.8)}%
{\color{green}\dashline[200]{3}(-60.6,-236.9)(-69.4,-263.1)}%
{\color{green}\dashline[200]{3}(-69.4,-263.1)(-67.8,-261.5)}%
{\color{green}\dashline[200]{3}(-69.4,-263.1)(-69.7,-260.9)}%
{\color{red}\dashline[200]{3}(-10.0,-317.0)(-10.0,-383.0)}%
{\color{red}\dashline[200]{3}(-10.0,-383.0)(-9.0,-381.0)}%
{\color{red}\dashline[200]{3}(-10.0,-383.0)(-11.0,-381.0)}%
{\color{red}\dashline[200]{3}(-28.6,-336.4)(-21.4,-343.6)}%
{\color{red}\dashline[200]{3}(-21.4,-343.6)(-22.1,-341.5)}%
{\color{red}\dashline[200]{3}(-21.4,-343.6)(-23.5,-342.9)}%
{\color{red}\dashline[200]{3}(-29.1,-336.8)(-20.9,-353.2)}%
{\color{red}\dashline[200]{3}(-20.9,-353.2)(-20.9,-351.0)}%
{\color{red}\dashline[200]{3}(-20.9,-353.2)(-22.7,-351.9)}%
{\color{red}\dashline[200]{3}(-30.0,-337.0)(-30.0,-363.0)}%
{\color{red}\dashline[200]{3}(-30.0,-363.0)(-29.0,-361.0)}%
{\color{red}\dashline[200]{3}(-30.0,-363.0)(-31.0,-361.0)}%
{\color{red}\dashline[200]{3}(-20.0,-347.0)(-20.0,-353.0)}%
{\color{red}\dashline[200]{3}(-20.0,-353.0)(-19.0,-351.0)}%
{\color{red}\dashline[200]{3}(-20.0,-353.0)(-21.0,-351.0)}%
{\color{red}\dashline[200]{3}(-20.9,-346.8)(-29.1,-363.2)}%
{\color{red}\dashline[200]{3}(-29.1,-363.2)(-27.3,-361.9)}%
{\color{red}\dashline[200]{3}(-29.1,-363.2)(-29.1,-361.0)}%
{\color{red}\dashline[200]{3}(-10.0,-437.0)(-10.0,-523.0)}%
{\color{red}\dashline[200]{3}(-10.0,-523.0)(-9.0,-521.0)}%
{\color{red}\dashline[200]{3}(-10.0,-523.0)(-11.0,-521.0)}%
{\color{red}\dashline[200]{3}(-30.0,-467.0)(-30.0,-483.0)}%
{\color{red}\dashline[200]{3}(-30.0,-483.0)(-29.0,-481.0)}%
{\color{red}\dashline[200]{3}(-30.0,-483.0)(-31.0,-481.0)}%
{\color{red}\dashline[200]{3}(-20.7,-477.1)(-27.9,-484.3)}%
{\color{red}\dashline[200]{3}(-27.9,-484.3)(-25.8,-483.6)}%
{\color{red}\dashline[200]{3}(-27.9,-484.3)(-27.2,-482.2)}%
{\color{green}\dashline[200]{3}(-29.3,-482.9)(-22.1,-475.7)}%
{\color{green}\dashline[200]{3}(-22.1,-475.7)(-24.2,-476.4)}%
{\color{green}\dashline[200]{3}(-22.1,-475.7)(-22.8,-477.8)}%
{\color{red}\dashline[200]{3}(-30.0,-487.0)(-30.0,-503.0)}%
{\color{red}\dashline[200]{3}(-30.0,-503.0)(-29.0,-501.0)}%
{\color{red}\dashline[200]{3}(-30.0,-503.0)(-31.0,-501.0)}%
\end{picture}
\]
\hrulefill

\end{figure}

