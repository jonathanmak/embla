% -*- eval: (local-set-key "\M-q" 'undefined) -*-
%
% The line above will probably make Emacs ask if it is ok to evaluate the
% expression.  Answer y.

\section{The Tool}

Embla helps the user find dependencies in a program. Technically, a
dependece is two references, not both reads, to overlapping memory
locations with no interveaning write. What makes Embla special is the
way dependencies are reported. Since the idea is to support manual,
function call level parallelization, we are not really interested in
knowing that a memory read on line 11754 in function {\tt foo}
sometimes reads a value last written in line 3411 in function {\tt
bar}. Rather, the user is concerned with a completely different
function {\tt baz} and would like to know whether the call to {\tt
foobar} on line 2355 can be made in parallel with the call to {\tt
barfoo} on line 2356.



\subsection{Types of dependencies}

Another dimension that Embla recognizes is where in memory the
location associated with the dependency is situated. In particular, we
want to catch the cases where anti and output dependencies are created
by reuse of memory for new stack frames. This happens when the stack 
shrinks and
then grows again; the new stack frames use the same memory locations,
but only for memory management reasons. With this in mind we have the
following categories:
\begin{description}
\item[s(tack):]
The location has been a part of the stack from the first reference to
the second (inclusive).
\item[f(alse):]
The location was part of the stack when the first reference was made,
then the stack shrunk enough that the location was not part of the
stack and then it became part of the stack again.
\item[o(ther):]
The location has never been part of the stack.
\end{description}

We also categorize dependencies according to properties of the two 
memory references making up the source and target of the dependence. 
This is both in order to give more precise information and to flag
additional dependencies as spurious. 

Dependency endpoints correspond to instruction execution events and
are reported as line numbers with one of the following codes:
\begin{description}
\item[$\epsilon$:] 
No code at all represents an instruction that is
generated from the source code at the indicated line. Thus the 
instruction is part of the function or procedure containing the 
line.
\item[c:]
The instruction was executed by a function (procedure) that was 
(transitively) invoked by a function (procedure) call at the indicated 
line.
\item[h:]
Like {\bf c}, but the function containing the instruction was 
{\em hidden}, that is, it was on a blacklist of functions which, 
like {\tt malloc}, would not give rise to dependencies in the 
parallelized program.
\end{description}


\begin{figure} 
\small
\verbdef\jkffxb$#include <stdlib.h> $
\verbdef\jkffxc$#include <stdio.h> $
\verbdef\jkffxd$ $
\verbdef\jkffxe$struct {int a; int b;} has_ab; $
\verbdef\jkffxf$ $
\verbdef\jkffxg$int main(int argc, char **argv) $
\verbdef\jkffxh${ $
\verbdef\jkffxi$   has_ab.a = 1; $
\verbdef\jkffxj$   has_ab.b = 1; $
\verbdef\jkffxba$   printf("%d ", has_ab.a); $
\verbdef\jkffxbb$   has_ab.a = 2; $
\verbdef\jkffxbc$   has_ab.b = 2; $
\verbdef\jkffxbd$   printf("%d %d\n", has_ab.a, has_ab.b); $
\verbdef\jkffxbe$ $
\verbdef\jkffxbf$} $
\hrulefill
\[
\begin{picture}(420,150)(-54,-150)

\put(0,0){\makebox(100,10)[l]{}}
\put(0,-10){\makebox(100,10)[l]{\jkffxb}}
\put(0,-20){\makebox(100,10)[l]{\jkffxc}}
\put(0,-30){\makebox(100,10)[l]{\jkffxd}}
\put(0,-40){\makebox(100,10)[l]{\jkffxe}}
\put(0,-50){\makebox(100,10)[l]{\jkffxf}}
\put(0,-60){\makebox(100,10)[l]{\jkffxg}}
\put(0,-70){\makebox(100,10)[l]{\jkffxh}}
{\color{black} \dottedline{3}(-51,-70)(-5,-70)}\put(0,-80){\makebox(100,10)[l]{\jkffxi}}
{\color{black} \dottedline{3}(-51,-80)(-5,-80)}\put(0,-90){\makebox(100,10)[l]{\jkffxj}}
{\color{black} \dottedline{3}(-51,-90)(-5,-90)}\put(0,-100){\makebox(100,10)[l]{\jkffxba}}
{\color{black} \dottedline{3}(-51,-100)(-5,-100)}\put(0,-110){\makebox(100,10)[l]{\jkffxbb}}
{\color{black} \dottedline{3}(-51,-110)(-5,-110)}\put(0,-120){\makebox(100,10)[l]{\jkffxbc}}
{\color{black} \dottedline{3}(-51,-120)(-5,-120)}\put(0,-130){\makebox(100,10)[l]{\jkffxbd}}
{\color{black} \dottedline{3}(-51,-130)(-5,-130)}\put(0,-140){\makebox(100,10)[l]{\jkffxbe}}
{\color{black} \dottedline{3}(-51,-140)(-5,-140)}\put(0,-150){\makebox(100,10)[l]{\jkffxbf}}

\color{red}
\put(-6,-65){\circle*{2}}
\put(-6,-67){\vector(0,-1){76}}
\put(-8,-67){\linethickness{0.7pt}\line(1,0){0}}
\put(-6,-145){\circle*{2}}

\color{red}
\put(-12,-75){\circle*{2}}
\put(-12,-77){\vector(0,-1){16}}
\put(-14,-77){\linethickness{0.7pt}\line(1,0){4}}
\put(-12,-95){\circle*{2}}

\color{blue}
\put(-18,-85){\circle*{2}}
\put(-18,-87){\vector(0,-1){26}}
\put(-20,-87){\linethickness{0.7pt}\line(1,0){4}}
\put(-18,-115){\circle*{2}}

\color{red}
\put(-24,-92){\circle*{2}}
\put(-26,-94){\linethickness{0.7pt}\line(1,0){0}}
\put(-24,-97){\circle*{2}\hspace{-2\unitlength}\circle{4}}

\color{green}
\put(-30,-95){\circle*{2}}
\put(-30,-97){\vector(0,-1){6}}
\put(-32,-97){\linethickness{0.7pt}\line(1,0){4}}
\put(-30,-105){\circle*{2}}

\color{red}
\put(-36,-95){\circle*{2}\hspace{-2\unitlength}\circle{4}}
\put(-36,-97){\vector(0,-1){26}}
\put(-38,-97){\linethickness{0.7pt}\line(1,0){4}}
\put(-36,-125){\circle*{2}\hspace{-2\unitlength}\circle{4}}

\color{green}
\put(-42,-95){\circle*{2}\hspace{-2\unitlength}\circle{4}}
\put(-42,-97){\vector(0,-1){26}}
\put(-44,-97){\linethickness{0.7pt}\line(1,0){4}}
\put(-42,-125){\circle*{2}\hspace{-2\unitlength}\circle{4}}

\color{blue}
\put(-48,-95){\circle*{2}\hspace{-2\unitlength}\circle{4}}
\put(-48,-97){\vector(0,-1){26}}
\put(-50,-97){\linethickness{0.7pt}\line(1,0){4}}
\put(-48,-125){\circle*{2}\hspace{-2\unitlength}\circle{4}}

\color{red}
\put(-12,-105){\circle*{2}}
\put(-12,-107){\vector(0,-1){16}}
\put(-14,-107){\linethickness{0.7pt}\line(1,0){4}}
\put(-12,-125){\circle*{2}}

\color{red}
\put(-24,-115){\circle*{2}}
\put(-24,-117){\vector(0,-1){6}}
\put(-26,-117){\linethickness{0.7pt}\line(1,0){4}}
\put(-24,-125){\circle*{2}}

\color{red}
\put(-18,-122){\circle*{2}}
\put(-20,-124){\linethickness{0.7pt}\line(1,0){0}}
\put(-18,-127){\circle*{2}\hspace{-2\unitlength}\circle{4}}

\end{picture}
\]
\hrulefill

\end{figure}

\begin{figure} 
\small
\verbdef\jkffxb$#include <stdio.h> $
\verbdef\jkffxc$#include <stdlib.h> $
\verbdef\jkffxd$ $
\verbdef\jkffxe$#define N 5 $
\verbdef\jkffxf$ $
\verbdef\jkffxg$typedef struct _ilist { $
\verbdef\jkffxh$   int val; $
\verbdef\jkffxi$   struct _ilist *next; $
\verbdef\jkffxj$} ilist; $
\verbdef\jkffxba$ $
\verbdef\jkffxbb$static ilist* mklist(int n) $
\verbdef\jkffxbc${ $
\verbdef\jkffxbd$  int i; $
\verbdef\jkffxbe$  ilist *p = NULL; $
\verbdef\jkffxbf$ $
\verbdef\jkffxbg$  for( i=0; i<n; i++ ) { $
\verbdef\jkffxbh$    ilist *t = (ilist *) malloc( sizeof(ilist) ); $
\verbdef\jkffxbi$    t->val = i; $
\verbdef\jkffxbj$    t->next = p; $
\verbdef\jkffxca$    p = t; $
\verbdef\jkffxcb$  } $
\verbdef\jkffxcc$  return p; $
\verbdef\jkffxcd$} $
\verbdef\jkffxce$ $
\verbdef\jkffxcf$static int sumlist(ilist* p) $
\verbdef\jkffxcg${ $
\verbdef\jkffxch$   ilist *q; $
\verbdef\jkffxci$   int  sum = 0; $
\verbdef\jkffxcj$ $
\verbdef\jkffxda$   for( q=p; q!=NULL; q=q->next ) { $
\verbdef\jkffxdb$      sum += q->val; $
\verbdef\jkffxdc$   } $
\verbdef\jkffxdd$   return sum; $
\verbdef\jkffxde$} $
\verbdef\jkffxdf$ $
\verbdef\jkffxdg$ $
\verbdef\jkffxdh$int main(int argc, char **argv)  $
\verbdef\jkffxdi${ $
\verbdef\jkffxdj$  int m,n; $
\verbdef\jkffxea$  ilist *p,*q; $
\verbdef\jkffxeb$ $
\verbdef\jkffxec$  p = mklist( N ); $
\verbdef\jkffxed$  q = mklist( N ); $
\verbdef\jkffxee$  m = sumlist( p ); $
\verbdef\jkffxef$  n = sumlist( q ); $
\verbdef\jkffxeg$ $
\verbdef\jkffxeh$  printf("%d\n", m+n); $
\verbdef\jkffxei$ $
\verbdef\jkffxej$  return 0; $
\verbdef\jkffxfa$ $
\verbdef\jkffxfb$} $
\hrulefill
\[
\begin{picture}(420,510)(-78,-510)

\put(0,0){\makebox(100,10)[l]{}}
\put(0,-10){\makebox(100,10)[l]{\jkffxb}}
\put(0,-20){\makebox(100,10)[l]{\jkffxc}}
\put(0,-30){\makebox(100,10)[l]{\jkffxd}}
\put(0,-40){\makebox(100,10)[l]{\jkffxe}}
\put(0,-50){\makebox(100,10)[l]{\jkffxf}}
\put(0,-60){\makebox(100,10)[l]{\jkffxg}}
\put(0,-70){\makebox(100,10)[l]{\jkffxh}}
\put(0,-80){\makebox(100,10)[l]{\jkffxi}}
\put(0,-90){\makebox(100,10)[l]{\jkffxj}}
\put(0,-100){\makebox(100,10)[l]{\jkffxba}}
\put(0,-110){\makebox(100,10)[l]{\jkffxbb}}
\put(0,-120){\makebox(100,10)[l]{\jkffxbc}}
{\color{black} \dottedline{3}(-75,-120)(-5,-120)}\put(0,-130){\makebox(100,10)[l]{\jkffxbd}}
{\color{black} \dottedline{3}(-75,-130)(-5,-130)}\put(0,-140){\makebox(100,10)[l]{\jkffxbe}}
{\color{black} \dottedline{3}(-75,-140)(-5,-140)}\put(0,-150){\makebox(100,10)[l]{\jkffxbf}}
{\color{black} \dottedline{3}(-75,-150)(-5,-150)}\put(0,-160){\makebox(100,10)[l]{\jkffxbg}}
{\color{black} \dottedline{3}(-75,-160)(-5,-160)}\put(0,-170){\makebox(100,10)[l]{\jkffxbh}}
{\color{black} \dottedline{3}(-75,-170)(-5,-170)}\put(0,-180){\makebox(100,10)[l]{\jkffxbi}}
{\color{black} \dottedline{3}(-75,-180)(-5,-180)}\put(0,-190){\makebox(100,10)[l]{\jkffxbj}}
{\color{black} \dottedline{3}(-75,-190)(-5,-190)}\put(0,-200){\makebox(100,10)[l]{\jkffxca}}
{\color{black} \dottedline{3}(-75,-200)(-5,-200)}\put(0,-210){\makebox(100,10)[l]{\jkffxcb}}
{\color{black} \dottedline{3}(-75,-210)(-5,-210)}\put(0,-220){\makebox(100,10)[l]{\jkffxcc}}
{\color{black} \dottedline{3}(-75,-220)(-5,-220)}\put(0,-230){\makebox(100,10)[l]{\jkffxcd}}
\put(0,-240){\makebox(100,10)[l]{\jkffxce}}
\put(0,-250){\makebox(100,10)[l]{\jkffxcf}}
\put(0,-260){\makebox(100,10)[l]{\jkffxcg}}
{\color{black} \dottedline{3}(-75,-260)(-5,-260)}\put(0,-270){\makebox(100,10)[l]{\jkffxch}}
{\color{black} \dottedline{3}(-75,-270)(-5,-270)}\put(0,-280){\makebox(100,10)[l]{\jkffxci}}
{\color{black} \dottedline{3}(-75,-280)(-5,-280)}\put(0,-290){\makebox(100,10)[l]{\jkffxcj}}
{\color{black} \dottedline{3}(-75,-290)(-5,-290)}\put(0,-300){\makebox(100,10)[l]{\jkffxda}}
{\color{black} \dottedline{3}(-75,-300)(-5,-300)}\put(0,-310){\makebox(100,10)[l]{\jkffxdb}}
{\color{black} \dottedline{3}(-75,-310)(-5,-310)}\put(0,-320){\makebox(100,10)[l]{\jkffxdc}}
{\color{black} \dottedline{3}(-75,-320)(-5,-320)}\put(0,-330){\makebox(100,10)[l]{\jkffxdd}}
{\color{black} \dottedline{3}(-75,-330)(-5,-330)}\put(0,-340){\makebox(100,10)[l]{\jkffxde}}
\put(0,-350){\makebox(100,10)[l]{\jkffxdf}}
\put(0,-360){\makebox(100,10)[l]{\jkffxdg}}
\put(0,-370){\makebox(100,10)[l]{\jkffxdh}}
\put(0,-380){\makebox(100,10)[l]{\jkffxdi}}
{\color{black} \dottedline{3}(-75,-380)(-5,-380)}\put(0,-390){\makebox(100,10)[l]{\jkffxdj}}
{\color{black} \dottedline{3}(-75,-390)(-5,-390)}\put(0,-400){\makebox(100,10)[l]{\jkffxea}}
{\color{black} \dottedline{3}(-75,-400)(-5,-400)}\put(0,-410){\makebox(100,10)[l]{\jkffxeb}}
{\color{black} \dottedline{3}(-75,-410)(-5,-410)}\put(0,-420){\makebox(100,10)[l]{\jkffxec}}
{\color{black} \dottedline{3}(-75,-420)(-5,-420)}\put(0,-430){\makebox(100,10)[l]{\jkffxed}}
{\color{black} \dottedline{3}(-75,-430)(-5,-430)}\put(0,-440){\makebox(100,10)[l]{\jkffxee}}
{\color{black} \dottedline{3}(-75,-440)(-5,-440)}\put(0,-450){\makebox(100,10)[l]{\jkffxef}}
{\color{black} \dottedline{3}(-75,-450)(-5,-450)}\put(0,-460){\makebox(100,10)[l]{\jkffxeg}}
{\color{black} \dottedline{3}(-75,-460)(-5,-460)}\put(0,-470){\makebox(100,10)[l]{\jkffxeh}}
{\color{black} \dottedline{3}(-75,-470)(-5,-470)}\put(0,-480){\makebox(100,10)[l]{\jkffxei}}
{\color{black} \dottedline{3}(-75,-480)(-5,-480)}\put(0,-490){\makebox(100,10)[l]{\jkffxej}}
{\color{black} \dottedline{3}(-75,-490)(-5,-490)}\put(0,-500){\makebox(100,10)[l]{\jkffxfa}}
{\color{black} \dottedline{3}(-75,-500)(-5,-500)}\put(0,-510){\makebox(100,10)[l]{\jkffxfb}}

\color{red}
\put(-6,-255){\circle*{2}}
\put(-6,-257){\vector(0,-1){76}}
\put(-8,-257){\linethickness{0.7pt}\line(1,0){0}}
\put(-6,-335){\circle*{2}}

\color{red}
\put(-12,-275){\circle*{2}}
\put(-12,-277){\vector(0,-1){26}}
\put(-14,-277){\linethickness{0.7pt}\line(1,0){0}}
\put(-12,-305){\circle*{2}}

\color{red}
\put(-18,-292){\circle*{2}}
\put(-20,-294){\linethickness{0.7pt}\line(1,0){0}}
\put(-18,-297){\circle*{2}}

\color{green}
\put(-24,-292){\circle*{2}}
\put(-26,-294){\linethickness{0.7pt}\line(1,0){0}}
\put(-24,-297){\circle*{2}}

\color{red}
\put(-30,-295){\circle*{2}}
\put(-30,-297){\vector(0,-1){6}}
\put(-32,-297){\linethickness{0.7pt}\line(1,0){0}}
\put(-30,-305){\circle*{2}}

\color{green}
\put(-36,-305){\circle*{2}}
\put(-36,-303){\vector(0,1){6}}
\put(-38,-297){\linethickness{0.7pt}\line(1,0){0}}
\put(-36,-295){\circle*{2}}

\color{red}
\put(-18,-302){\circle*{2}}
\put(-20,-304){\linethickness{0.7pt}\line(1,0){0}}
\put(-18,-307){\circle*{2}}

\color{green}
\put(-24,-302){\circle*{2}}
\put(-26,-304){\linethickness{0.7pt}\line(1,0){0}}
\put(-24,-307){\circle*{2}}

\color{red}
\put(-42,-305){\circle*{2}}
\put(-42,-307){\vector(0,-1){16}}
\put(-44,-307){\linethickness{0.7pt}\line(1,0){0}}
\put(-42,-325){\circle*{2}}

\color{red}
\put(-6,-115){\circle*{2}}
\put(-6,-117){\vector(0,-1){106}}
\put(-8,-117){\linethickness{0.7pt}\line(1,0){0}}
\put(-6,-225){\circle*{2}}

\color{red}
\put(-12,-135){\circle*{2}}
\put(-12,-137){\vector(0,-1){46}}
\put(-14,-137){\linethickness{0.7pt}\line(1,0){0}}
\put(-12,-185){\circle*{2}}

\color{red}
\put(-18,-152){\circle*{2}}
\put(-20,-154){\linethickness{0.7pt}\line(1,0){0}}
\put(-18,-157){\circle*{2}}

\color{green}
\put(-24,-152){\circle*{2}}
\put(-26,-154){\linethickness{0.7pt}\line(1,0){0}}
\put(-24,-157){\circle*{2}}

\color{red}
\put(-30,-155){\circle*{2}}
\put(-30,-157){\vector(0,-1){16}}
\put(-32,-157){\linethickness{0.7pt}\line(1,0){0}}
\put(-30,-175){\circle*{2}}

\color{red}
\put(-18,-162){\circle*{2}}
\put(-20,-164){\linethickness{0.7pt}\line(1,0){0}}
\put(-18,-167){\circle{4}}

\color{red}
\put(-24,-165){\circle*{2}}
\put(-24,-167){\vector(0,-1){6}}
\put(-26,-167){\linethickness{0.7pt}\line(1,0){0}}
\put(-24,-175){\circle*{2}}

\color{red}
\put(-36,-165){\circle*{2}}
\put(-36,-167){\vector(0,-1){16}}
\put(-38,-167){\linethickness{0.7pt}\line(1,0){0}}
\put(-36,-185){\circle*{2}}

\color{red}
\put(-42,-165){\circle*{2}}
\put(-42,-167){\vector(0,-1){26}}
\put(-44,-167){\linethickness{0.7pt}\line(1,0){0}}
\put(-42,-195){\circle*{2}}

\color{red}
\put(-48,-162){\circle{4}}
\put(-50,-164){\linethickness{0.7pt}\line(1,0){4}}
\put(-48,-167){\circle{4}}

\color{green}
\put(-54,-175){\circle*{2}}
\put(-54,-173){\vector(0,1){16}}
\put(-56,-157){\linethickness{0.7pt}\line(1,0){0}}
\put(-54,-155){\circle*{2}}

\color{green}
\put(-60,-175){\circle*{2}}
\put(-60,-173){\vector(0,1){6}}
\put(-62,-167){\linethickness{0.7pt}\line(1,0){0}}
\put(-60,-165){\circle*{2}}

\color{green}
\put(-66,-185){\circle*{2}}
\put(-66,-183){\vector(0,1){16}}
\put(-68,-167){\linethickness{0.7pt}\line(1,0){0}}
\put(-66,-165){\circle*{2}}

\color{green}
\put(-18,-185){\circle*{2}}
\put(-18,-187){\vector(0,-1){6}}
\put(-20,-187){\linethickness{0.7pt}\line(1,0){0}}
\put(-18,-195){\circle*{2}}

\color{green}
\put(-72,-195){\circle*{2}}
\put(-72,-193){\vector(0,1){26}}
\put(-74,-167){\linethickness{0.7pt}\line(1,0){0}}
\put(-72,-165){\circle*{2}}

\color{red}
\put(-24,-195){\circle*{2}}
\put(-24,-193){\vector(0,1){6}}
\put(-26,-187){\linethickness{0.7pt}\line(1,0){0}}
\put(-24,-185){\circle*{2}}

\color{red}
\put(-12,-195){\circle*{2}}
\put(-12,-197){\vector(0,-1){16}}
\put(-14,-197){\linethickness{0.7pt}\line(1,0){0}}
\put(-12,-215){\circle*{2}}

\color{red}
\put(-6,-375){\circle*{2}}
\put(-6,-377){\vector(0,-1){126}}
\put(-8,-377){\linethickness{0.7pt}\line(1,0){0}}
\put(-6,-505){\circle*{2}}

\color{red}
\put(-12,-412){\circle*{2}}
\put(-14,-414){\linethickness{0.7pt}\line(1,0){0}}
\put(-12,-417){\circle*{2}\hspace{-2\unitlength}\circle{4}}

\color{red}
\put(-18,-415){\circle*{2}}
\put(-18,-417){\vector(0,-1){16}}
\put(-20,-417){\linethickness{0.7pt}\line(1,0){0}}
\put(-18,-435){\circle*{2}}

\color{red}
\put(-24,-415){\circle*{2}\hspace{-2\unitlength}\circle{4}}
\put(-24,-417){\vector(0,-1){16}}
\put(-26,-417){\linethickness{0.7pt}\line(1,0){4}}
\put(-24,-435){\circle*{2}\hspace{-2\unitlength}\circle{4}}

\color{red}
\put(-30,-415){\circle{4}}
\put(-30,-417){\vector(0,-1){6}}
\put(-32,-417){\linethickness{0.7pt}\line(1,0){4}}
\put(-30,-425){\circle{4}}

\color{red}
\put(-12,-422){\circle*{2}}
\put(-14,-424){\linethickness{0.7pt}\line(1,0){0}}
\put(-12,-427){\circle*{2}\hspace{-2\unitlength}\circle{4}}

\color{red}
\put(-36,-425){\circle*{2}}
\put(-36,-427){\vector(0,-1){16}}
\put(-38,-427){\linethickness{0.7pt}\line(1,0){0}}
\put(-36,-445){\circle*{2}}

\color{red}
\put(-42,-425){\circle*{2}\hspace{-2\unitlength}\circle{4}}
\put(-42,-427){\vector(0,-1){16}}
\put(-44,-427){\linethickness{0.7pt}\line(1,0){4}}
\put(-42,-445){\circle*{2}\hspace{-2\unitlength}\circle{4}}

\color{red}
\put(-12,-432){\circle*{2}}
\put(-14,-434){\linethickness{0.7pt}\line(1,0){0}}
\put(-12,-437){\circle*{2}\hspace{-2\unitlength}\circle{4}}

\color{red}
\put(-30,-435){\circle*{2}}
\put(-30,-437){\vector(0,-1){26}}
\put(-32,-437){\linethickness{0.7pt}\line(1,0){0}}
\put(-30,-465){\circle*{2}}

\color{red}
\put(-12,-442){\circle*{2}}
\put(-14,-444){\linethickness{0.7pt}\line(1,0){0}}
\put(-12,-447){\circle*{2}\hspace{-2\unitlength}\circle{4}}

\color{red}
\put(-18,-445){\circle*{2}}
\put(-18,-447){\vector(0,-1){16}}
\put(-20,-447){\linethickness{0.7pt}\line(1,0){0}}
\put(-18,-465){\circle*{2}}

\color{red}
\put(-12,-462){\circle*{2}}
\put(-14,-464){\linethickness{0.7pt}\line(1,0){0}}
\put(-12,-467){\circle*{2}\hspace{-2\unitlength}\circle{4}}

\end{picture}
\]
\hrulefill

\end{figure}

\begin{figure} 
\small
\verbdef\jkffxb$#include <stdlib.h> $
\verbdef\jkffxc$#include <stdio.h> $
\verbdef\jkffxd$ $
\verbdef\jkffxe$struct {int a; int b;} has_ab; $
\verbdef\jkffxf$ $
\verbdef\jkffxg$int main(int argc, char **argv) $
\verbdef\jkffxh${ $
\verbdef\jkffxi$   has_ab.a = 1; $
\verbdef\jkffxj$   has_ab.b = 1; $
\verbdef\jkffxba$   printf("%d ", has_ab.a); $
\verbdef\jkffxbb$   has_ab.a = 2; $
\verbdef\jkffxbc$   has_ab.b = 2; $
\verbdef\jkffxbd$   printf("%d %d\n", has_ab.a, has_ab.b); $
\verbdef\jkffxbe$ $
\verbdef\jkffxbf$} $
\hrulefill
\[
\begin{picture}(420,150)(-40,-150)

\put(0,-10){\makebox(100,10)[l]{\jkffxb}}
\put(0,-20){\makebox(100,10)[l]{\jkffxc}}
\put(0,-30){\makebox(100,10)[l]{\jkffxd}}
\put(0,-40){\makebox(100,10)[l]{\jkffxe}}
\put(0,-50){\makebox(100,10)[l]{\jkffxf}}
\put(0,-60){\makebox(100,10)[l]{\jkffxg}}
\put(0,-70){\makebox(100,10)[l]{\jkffxh}}
{\color{black} \dottedline{3}(-10,-65)(-0,-65)}
\put(0,-80){\makebox(100,10)[l]{\jkffxi}}
{\color{black} \dottedline{3}(-10,-75)(-0,-75)}
\put(0,-90){\makebox(100,10)[l]{\jkffxj}}
{\color{black} \dottedline{3}(-10,-85)(-0,-85)}
\put(0,-100){\makebox(100,10)[l]{\jkffxba}}
{\color{black} \dottedline{3}(-10,-95)(-0,-95)}
\put(0,-110){\makebox(100,10)[l]{\jkffxbb}}
{\color{black} \dottedline{3}(-10,-105)(-0,-105)}
\put(0,-120){\makebox(100,10)[l]{\jkffxbc}}
{\color{black} \dottedline{3}(-10,-115)(-0,-115)}
\put(0,-130){\makebox(100,10)[l]{\jkffxbd}}
{\color{black} \dottedline{3}(-10,-125)(-0,-125)}
\put(0,-140){\makebox(100,10)[l]{\jkffxbe}}
\put(0,-150){\makebox(100,10)[l]{\jkffxbf}}
{\color{black} \dottedline{3}(-10,-145)(-0,-145)}

\put(-10,-65){\color{black}\circle*{2}}
\put(-40,-75){\color{black}\circle*{2}}
\put(-20,-85){\color{black}\circle*{2}}
\put(-40,-95){\color{black}\circle*{2}}
\put(-30,-105){\color{black}\circle*{2}}
\put(-30,-115){\color{black}\circle*{2}}
\put(-40,-125){\color{black}\circle*{2}}
\put(-10,-145){\color{black}\circle*{2}}
\put(-10.0,-67.0){\color{red}\vector(0,-1){76.0}}
{\color{red}\dashline[100]{2}(-10.0,-67.0)(-10.0,-143.0)}
\put(-40.0,-77.0){\color{red}\vector(0,-1){16.0}}
{\color{red}\dashline[100]{2}(-40.0,-77.0)(-40.0,-93.0)}
\put(-20.6,-86.9){\color{blue}\vector(-1,-3){8.7}}
{\color{blue}\dashline[100]{2}(-20.6,-86.9)(-29.4,-113.1)}
\put(-38.6,-96.4){\color{green}\vector(1,-1){7.2}}
{\color{green}\dashline[100]{2}(-38.6,-96.4)(-31.4,-103.6)}
\put(-40.0,-97.0){\color{red}\vector(0,-1){26.0}}
{\color{red}\dashline[100]{2}(-40.0,-97.0)(-40.0,-123.0)}
\put(-30.9,-106.8){\color{red}\vector(-1,-2){8.2}}
{\color{red}\dashline[100]{2}(-30.9,-106.8)(-39.1,-123.2)}
\put(-31.4,-116.4){\color{red}\vector(-1,-1){7.2}}
{\color{red}\dashline[100]{2}(-31.4,-116.4)(-38.6,-123.6)}
\end{picture}
\]
\hrulefill

\end{figure}

\begin{figure} 
\small
\verbdef\jkffxb$#include <stdio.h> $
\verbdef\jkffxc$#include <stdlib.h> $
\verbdef\jkffxd$ $
\verbdef\jkffxe$#define N 5 $
\verbdef\jkffxf$ $
\verbdef\jkffxg$typedef struct _ilist { $
\verbdef\jkffxh$   int val; $
\verbdef\jkffxi$   struct _ilist *next; $
\verbdef\jkffxj$} ilist; $
\verbdef\jkffxba$ $
\verbdef\jkffxbb$static ilist* mklist(int n) $
\verbdef\jkffxbc${ $
\verbdef\jkffxbd$  int i; $
\verbdef\jkffxbe$  ilist *p = NULL; $
\verbdef\jkffxbf$ $
\verbdef\jkffxbg$  for( i=0; i<n; i++ ) { $
\verbdef\jkffxbh$    ilist *t = (ilist *) malloc( sizeof(ilist) ); $
\verbdef\jkffxbi$    t->val = i; $
\verbdef\jkffxbj$    t->next = p; $
\verbdef\jkffxca$    p = t; $
\verbdef\jkffxcb$  } $
\verbdef\jkffxcc$  return p; $
\verbdef\jkffxcd$} $
\verbdef\jkffxce$ $
\verbdef\jkffxcf$static int sumlist(ilist* p) $
\verbdef\jkffxcg${ $
\verbdef\jkffxch$   ilist *q; $
\verbdef\jkffxci$   int  sum = 0; $
\verbdef\jkffxcj$ $
\verbdef\jkffxda$   for( q=p; q!=NULL; q=q->next ) { $
\verbdef\jkffxdb$      sum += q->val; $
\verbdef\jkffxdc$   } $
\verbdef\jkffxdd$   return sum; $
\verbdef\jkffxde$} $
\verbdef\jkffxdf$ $
\verbdef\jkffxdg$ $
\verbdef\jkffxdh$int main(int argc, char **argv)  $
\verbdef\jkffxdi${ $
\verbdef\jkffxdj$  int m,n; $
\verbdef\jkffxea$  ilist *p,*q; $
\verbdef\jkffxeb$ $
\verbdef\jkffxec$  p = mklist( N ); $
\verbdef\jkffxed$  q = mklist( N ); $
\verbdef\jkffxee$  m = sumlist( p ); $
\verbdef\jkffxef$  n = sumlist( q ); $
\verbdef\jkffxeg$ $
\verbdef\jkffxeh$  printf("%d\n", m+n); $
\verbdef\jkffxei$ $
\verbdef\jkffxej$  return 0; $
\verbdef\jkffxfa$ $
\verbdef\jkffxfb$} $
\hrulefill
\[
\begin{picture}(420,510)(-40,-510)

\put(0,-10){\makebox(100,10)[l]{\jkffxb}}
\put(0,-20){\makebox(100,10)[l]{\jkffxc}}
\put(0,-30){\makebox(100,10)[l]{\jkffxd}}
\put(0,-40){\makebox(100,10)[l]{\jkffxe}}
\put(0,-50){\makebox(100,10)[l]{\jkffxf}}
\put(0,-60){\makebox(100,10)[l]{\jkffxg}}
\put(0,-70){\makebox(100,10)[l]{\jkffxh}}
\put(0,-80){\makebox(100,10)[l]{\jkffxi}}
\put(0,-90){\makebox(100,10)[l]{\jkffxj}}
\put(0,-100){\makebox(100,10)[l]{\jkffxba}}
\put(0,-110){\makebox(100,10)[l]{\jkffxbb}}
\put(0,-120){\makebox(100,10)[l]{\jkffxbc}}
{\color{black} \dottedline{3}(-10,-115)(-0,-115)}
\put(0,-130){\makebox(100,10)[l]{\jkffxbd}}
\put(0,-140){\makebox(100,10)[l]{\jkffxbe}}
{\color{black} \dottedline{3}(-10,-135)(-0,-135)}
\put(0,-150){\makebox(100,10)[l]{\jkffxbf}}
\put(0,-160){\makebox(100,10)[l]{\jkffxbg}}
{\color{black} \dottedline{3}(-10,-155)(-0,-155)}
\put(0,-170){\makebox(100,10)[l]{\jkffxbh}}
{\color{black} \dottedline{3}(-10,-165)(-0,-165)}
\put(0,-180){\makebox(100,10)[l]{\jkffxbi}}
{\color{black} \dottedline{3}(-10,-175)(-0,-175)}
\put(0,-190){\makebox(100,10)[l]{\jkffxbj}}
{\color{black} \dottedline{3}(-10,-185)(-0,-185)}
\put(0,-200){\makebox(100,10)[l]{\jkffxca}}
{\color{black} \dottedline{3}(-10,-195)(-0,-195)}
\put(0,-210){\makebox(100,10)[l]{\jkffxcb}}
\put(0,-220){\makebox(100,10)[l]{\jkffxcc}}
{\color{black} \dottedline{3}(-10,-215)(-0,-215)}
\put(0,-230){\makebox(100,10)[l]{\jkffxcd}}
{\color{black} \dottedline{3}(-10,-225)(-0,-225)}
\put(0,-240){\makebox(100,10)[l]{\jkffxce}}
\put(0,-250){\makebox(100,10)[l]{\jkffxcf}}
\put(0,-260){\makebox(100,10)[l]{\jkffxcg}}
{\color{black} \dottedline{3}(-10,-255)(-0,-255)}
\put(0,-270){\makebox(100,10)[l]{\jkffxch}}
\put(0,-280){\makebox(100,10)[l]{\jkffxci}}
{\color{black} \dottedline{3}(-10,-275)(-0,-275)}
\put(0,-290){\makebox(100,10)[l]{\jkffxcj}}
\put(0,-300){\makebox(100,10)[l]{\jkffxda}}
{\color{black} \dottedline{3}(-10,-295)(-0,-295)}
\put(0,-310){\makebox(100,10)[l]{\jkffxdb}}
{\color{black} \dottedline{3}(-10,-305)(-0,-305)}
\put(0,-320){\makebox(100,10)[l]{\jkffxdc}}
\put(0,-330){\makebox(100,10)[l]{\jkffxdd}}
{\color{black} \dottedline{3}(-10,-325)(-0,-325)}
\put(0,-340){\makebox(100,10)[l]{\jkffxde}}
{\color{black} \dottedline{3}(-10,-335)(-0,-335)}
\put(0,-350){\makebox(100,10)[l]{\jkffxdf}}
\put(0,-360){\makebox(100,10)[l]{\jkffxdg}}
\put(0,-370){\makebox(100,10)[l]{\jkffxdh}}
\put(0,-380){\makebox(100,10)[l]{\jkffxdi}}
{\color{black} \dottedline{3}(-10,-375)(-0,-375)}
\put(0,-390){\makebox(100,10)[l]{\jkffxdj}}
\put(0,-400){\makebox(100,10)[l]{\jkffxea}}
\put(0,-410){\makebox(100,10)[l]{\jkffxeb}}
\put(0,-420){\makebox(100,10)[l]{\jkffxec}}
{\color{black} \dottedline{3}(-10,-415)(-0,-415)}
\put(0,-430){\makebox(100,10)[l]{\jkffxed}}
{\color{black} \dottedline{3}(-10,-425)(-0,-425)}
\put(0,-440){\makebox(100,10)[l]{\jkffxee}}
{\color{black} \dottedline{3}(-10,-435)(-0,-435)}
\put(0,-450){\makebox(100,10)[l]{\jkffxef}}
{\color{black} \dottedline{3}(-10,-445)(-0,-445)}
\put(0,-460){\makebox(100,10)[l]{\jkffxeg}}
\put(0,-470){\makebox(100,10)[l]{\jkffxeh}}
{\color{black} \dottedline{3}(-10,-465)(-0,-465)}
\put(0,-480){\makebox(100,10)[l]{\jkffxei}}
\put(0,-490){\makebox(100,10)[l]{\jkffxej}}
\put(0,-500){\makebox(100,10)[l]{\jkffxfa}}
\put(0,-510){\makebox(100,10)[l]{\jkffxfb}}
{\color{black} \dottedline{3}(-10,-505)(-0,-505)}

\put(-20,-435){\color{black}\circle*{2}}
\put(-40,-165){\color{black}\circle*{2}}
\put(-10,-255){\color{black}\circle*{2}}
\put(-30,-445){\color{black}\circle*{2}}
\put(-20,-175){\color{black}\circle*{2}}
\put(-30,-185){\color{black}\circle*{2}}
\put(-30,-275){\color{black}\circle*{2}}
\put(-20,-465){\color{black}\circle*{2}}
\put(-10,-375){\color{black}\circle*{2}}
\put(-40,-195){\color{black}\circle*{2}}
\put(-10,-115){\color{black}\circle*{2}}
\put(-20,-295){\color{black}\circle*{2}}
\put(-40,-215){\color{black}\circle*{2}}
\put(-30,-305){\color{black}\circle*{2}}
\put(-30,-135){\color{black}\circle*{2}}
\put(-10,-225){\color{black}\circle*{2}}
\put(-30,-415){\color{black}\circle*{2}}
\put(-10,-505){\color{black}\circle*{2}}
\put(-30,-325){\color{black}\circle*{2}}
\put(-30,-425){\color{black}\circle*{2}}
\put(-20,-155){\color{black}\circle*{2}}
\put(-10,-335){\color{black}\circle*{2}}
\put(-10,-117){\color{red}\vector(0,-1){106}}
\put(-30,-137){\color{red}\vector(0,-1){46}}
\put(-19,-157){\color{red}\vector(0,-1){16}}
\put(-37,-166){\color{red}\vector(2,-1){16}}
\put(-38,-167){\color{red}\vector(1,-2){8}}
\put(-39,-167){\color{red}\vector(0,-1){26}}
\put(-21,-173){\color{green}\vector(0,1){16}}
\put(-22,-173){\color{green}\vector(-2,1){16}}
\put(-31,-182){\color{green}\vector(-1,2){8}}
\put(-30,-185){\color{green}\vector(-1,-1){7}}
\put(-41,-193){\color{green}\vector(0,1){26}}
\put(-39,-194){\color{red}\vector(1,1){7}}
\put(-40,-197){\color{red}\vector(0,-1){16}}
\put(-10,-257){\color{red}\vector(0,-1){76}}
\put(-30,-277){\color{red}\vector(0,-1){26}}
\put(-20,-295){\color{red}\vector(-1,-1){7}}
\put(-29,-304){\color{green}\vector(1,1){7}}
\put(-30,-307){\color{red}\vector(0,-1){16}}
\put(-10,-377){\color{red}\vector(0,-1){126}}
\put(-30,-417){\color{cyan}\vector(0,-1){6}}
\put(-29,-416){\color{red}\vector(1,-2){8}}
\put(-30,-427){\color{red}\vector(0,-1){16}}
\put(-20,-437){\color{red}\vector(0,-1){26}}
\put(-29,-446){\color{red}\vector(1,-2){8}}
\end{picture}
\]
\hrulefill

\end{figure}

\begin{figure} 
\small
\verbdef\jkffxb$#include <stdlib.h> $
\verbdef\jkffxc$#include <stdio.h> $
\verbdef\jkffxd$ $
\verbdef\jkffxe$static int nfib(int n) $
\verbdef\jkffxf${ $
\verbdef\jkffxg$   int result; $
\verbdef\jkffxh$   if( n < 2 ) { $
\verbdef\jkffxi$     result = 1; $
\verbdef\jkffxj$   } else { $
\verbdef\jkffxba$     int a = nfib( n-1 ); $
\verbdef\jkffxbb$     int b = nfib( n-2 ); $
\verbdef\jkffxbc$     result = a+b; $
\verbdef\jkffxbd$   } $
\verbdef\jkffxbe$   return result; $
\verbdef\jkffxbf$} $
\verbdef\jkffxbg$      $
\verbdef\jkffxbh$  $
\verbdef\jkffxbi$ $
\verbdef\jkffxbj$int main(int argc, char **argv) $
\verbdef\jkffxca${ $
\verbdef\jkffxcb$   int m = nfib( 8 ); $
\verbdef\jkffxcc$    $
\verbdef\jkffxcd$   printf( "%d\n", m ); $
\verbdef\jkffxce$} $
\hrulefill
\[
\begin{picture}(420,240)(-36,-240)

\put(0,0){\makebox(100,10)[l]{}}
\put(0,-10){\makebox(100,10)[l]{\jkffxb}}
\put(0,-20){\makebox(100,10)[l]{\jkffxc}}
\put(0,-30){\makebox(100,10)[l]{\jkffxd}}
\put(0,-40){\makebox(100,10)[l]{\jkffxe}}
\put(0,-50){\makebox(100,10)[l]{\jkffxf}}
{\color{black} \dottedline{3}(-33,-50)(-5,-50)}\put(0,-60){\makebox(100,10)[l]{\jkffxg}}
{\color{black} \dottedline{3}(-33,-60)(-5,-60)}\put(0,-70){\makebox(100,10)[l]{\jkffxh}}
{\color{black} \dottedline{3}(-33,-70)(-5,-70)}\put(0,-80){\makebox(100,10)[l]{\jkffxi}}
{\color{black} \dottedline{3}(-33,-80)(-5,-80)}\put(0,-90){\makebox(100,10)[l]{\jkffxj}}
{\color{black} \dottedline{3}(-33,-90)(-5,-90)}\put(0,-100){\makebox(100,10)[l]{\jkffxba}}
{\color{black} \dottedline{3}(-33,-100)(-5,-100)}\put(0,-110){\makebox(100,10)[l]{\jkffxbb}}
{\color{black} \dottedline{3}(-33,-110)(-5,-110)}\put(0,-120){\makebox(100,10)[l]{\jkffxbc}}
{\color{black} \dottedline{3}(-33,-120)(-5,-120)}\put(0,-130){\makebox(100,10)[l]{\jkffxbd}}
{\color{black} \dottedline{3}(-33,-130)(-5,-130)}\put(0,-140){\makebox(100,10)[l]{\jkffxbe}}
{\color{black} \dottedline{3}(-33,-140)(-5,-140)}\put(0,-150){\makebox(100,10)[l]{\jkffxbf}}
\put(0,-160){\makebox(100,10)[l]{\jkffxbg}}
\put(0,-170){\makebox(100,10)[l]{\jkffxbh}}
\put(0,-180){\makebox(100,10)[l]{\jkffxbi}}
\put(0,-190){\makebox(100,10)[l]{\jkffxbj}}
\put(0,-200){\makebox(100,10)[l]{\jkffxca}}
{\color{black} \dottedline{3}(-33,-200)(-5,-200)}\put(0,-210){\makebox(100,10)[l]{\jkffxcb}}
{\color{black} \dottedline{3}(-33,-210)(-5,-210)}\put(0,-220){\makebox(100,10)[l]{\jkffxcc}}
{\color{black} \dottedline{3}(-33,-220)(-5,-220)}\put(0,-230){\makebox(100,10)[l]{\jkffxcd}}
{\color{black} \dottedline{3}(-33,-230)(-5,-230)}\put(0,-240){\makebox(100,10)[l]{\jkffxce}}

\color{red}
\put(-6,-45){\circle*{2}}
\put(-6,-47){\vector(0,-1){96}}
\put(-8,-47){\linethickness{0.7pt}\line(1,0){0}}
\put(-6,-145){\circle*{2}}

\color{red}
\put(-12,-75){\circle*{2}}
\put(-12,-77){\vector(0,-1){56}}
\put(-14,-77){\linethickness{0.7pt}\line(1,0){0}}
\put(-12,-135){\circle*{2}}

\color{red}
\put(-18,-92){\circle*{2}}
\put(-20,-94){\linethickness{0.7pt}\line(1,0){0}}
\put(-18,-97){\circle*{2}\hspace{-2\unitlength}\circle{4}}

\color{red}
\put(-24,-95){\circle*{2}}
\put(-24,-97){\vector(0,-1){16}}
\put(-26,-97){\linethickness{0.7pt}\line(1,0){0}}
\put(-24,-115){\circle*{2}}

\color{red}
\put(-18,-102){\circle*{2}}
\put(-20,-104){\linethickness{0.7pt}\line(1,0){0}}
\put(-18,-107){\circle*{2}\hspace{-2\unitlength}\circle{4}}

\color{red}
\put(-30,-105){\circle*{2}}
\put(-30,-107){\vector(0,-1){6}}
\put(-32,-107){\linethickness{0.7pt}\line(1,0){0}}
\put(-30,-115){\circle*{2}}

\color{red}
\put(-18,-115){\circle*{2}}
\put(-18,-117){\vector(0,-1){16}}
\put(-20,-117){\linethickness{0.7pt}\line(1,0){0}}
\put(-18,-135){\circle*{2}}

\color{red}
\put(-6,-195){\circle*{2}}
\put(-6,-197){\vector(0,-1){36}}
\put(-8,-197){\linethickness{0.7pt}\line(1,0){0}}
\put(-6,-235){\circle*{2}}

\color{red}
\put(-12,-202){\circle*{2}}
\put(-14,-204){\linethickness{0.7pt}\line(1,0){0}}
\put(-12,-207){\circle*{2}\hspace{-2\unitlength}\circle{4}}

\color{red}
\put(-18,-205){\circle*{2}}
\put(-18,-207){\vector(0,-1){16}}
\put(-20,-207){\linethickness{0.7pt}\line(1,0){0}}
\put(-18,-225){\circle*{2}}

\color{red}
\put(-12,-222){\circle*{2}}
\put(-14,-224){\linethickness{0.7pt}\line(1,0){0}}
\put(-12,-227){\circle*{2}\hspace{-2\unitlength}\circle{4}}

\end{picture}
\]
\hrulefill

\end{figure}

\begin{figure} 
\small
\verbdef\jkffxb$#include <stdlib.h> $
\verbdef\jkffxc$#include <stdio.h> $
\verbdef\jkffxd$ $
\verbdef\jkffxe$static int nfib(int n) $
\verbdef\jkffxf${ $
\verbdef\jkffxg$   int result; $
\verbdef\jkffxh$   if( n < 2 ) { $
\verbdef\jkffxi$     result = 1; $
\verbdef\jkffxj$   } else { $
\verbdef\jkffxba$     int a = nfib( n-1 ); $
\verbdef\jkffxbb$     int b = nfib( n-2 ); $
\verbdef\jkffxbc$     result = a+b; $
\verbdef\jkffxbd$   } $
\verbdef\jkffxbe$   return result; $
\verbdef\jkffxbf$} $
\verbdef\jkffxbg$      $
\verbdef\jkffxbh$  $
\verbdef\jkffxbi$ $
\verbdef\jkffxbj$int main(int argc, char **argv) $
\verbdef\jkffxca${ $
\verbdef\jkffxcb$   int m = nfib( 8 ); $
\verbdef\jkffxcc$    $
\verbdef\jkffxcd$   printf( "%d\n", m ); $
\verbdef\jkffxce$} $
\hrulefill
\[
\begin{picture}(420,240)(-40,-240)

\put(0,-10){\makebox(100,10)[l]{\jkffxb}}
\put(0,-20){\makebox(100,10)[l]{\jkffxc}}
\put(0,-30){\makebox(100,10)[l]{\jkffxd}}
\put(0,-40){\makebox(100,10)[l]{\jkffxe}}
\put(0,-50){\makebox(100,10)[l]{\jkffxf}}
{\color{black} \dottedline{3}(-10,-45)(-0,-45)}
\put(0,-60){\makebox(100,10)[l]{\jkffxg}}
\put(0,-70){\makebox(100,10)[l]{\jkffxh}}
\put(0,-80){\makebox(100,10)[l]{\jkffxi}}
{\color{black} \dottedline{3}(-10,-75)(-0,-75)}
\put(0,-90){\makebox(100,10)[l]{\jkffxj}}
\put(0,-100){\makebox(100,10)[l]{\jkffxba}}
{\color{black} \dottedline{3}(-10,-95)(-0,-95)}
\put(0,-110){\makebox(100,10)[l]{\jkffxbb}}
{\color{black} \dottedline{3}(-10,-105)(-0,-105)}
\put(0,-120){\makebox(100,10)[l]{\jkffxbc}}
{\color{black} \dottedline{3}(-10,-115)(-0,-115)}
\put(0,-130){\makebox(100,10)[l]{\jkffxbd}}
\put(0,-140){\makebox(100,10)[l]{\jkffxbe}}
{\color{black} \dottedline{3}(-10,-135)(-0,-135)}
\put(0,-150){\makebox(100,10)[l]{\jkffxbf}}
{\color{black} \dottedline{3}(-10,-145)(-0,-145)}
\put(0,-160){\makebox(100,10)[l]{\jkffxbg}}
\put(0,-170){\makebox(100,10)[l]{\jkffxbh}}
\put(0,-180){\makebox(100,10)[l]{\jkffxbi}}
\put(0,-190){\makebox(100,10)[l]{\jkffxbj}}
\put(0,-200){\makebox(100,10)[l]{\jkffxca}}
{\color{black} \dottedline{3}(-10,-195)(-0,-195)}
\put(0,-210){\makebox(100,10)[l]{\jkffxcb}}
{\color{black} \dottedline{3}(-10,-205)(-0,-205)}
\put(0,-220){\makebox(100,10)[l]{\jkffxcc}}
\put(0,-230){\makebox(100,10)[l]{\jkffxcd}}
{\color{black} \dottedline{3}(-10,-225)(-0,-225)}
\put(0,-240){\makebox(100,10)[l]{\jkffxce}}
{\color{black} \dottedline{3}(-10,-235)(-0,-235)}

\put(-10,-45){\color{black}\circle*{2}}
\put(-30,-75){\color{black}\circle*{2}}
\put(-20,-95){\color{black}\circle*{2}}
\put(-10,-195){\color{black}\circle*{2}}
\put(-40,-105){\color{black}\circle*{2}}
\put(-20,-205){\color{black}\circle*{2}}
\put(-20,-115){\color{black}\circle*{2}}
\put(-20,-225){\color{black}\circle*{2}}
\put(-30,-135){\color{black}\circle*{2}}
\put(-10,-235){\color{black}\circle*{2}}
\put(-10,-145){\color{black}\circle*{2}}
\put(-10,-47){\color{red}\vector(0,-1){96}}
\put(-30,-77){\color{red}\vector(0,-1){56}}
\put(-20,-97){\color{red}\vector(0,-1){16}}
\put(-38,-105){\color{red}\vector(2,-1){16}}
\put(-20,-116){\color{red}\vector(-1,-2){8}}
\put(-10,-197){\color{red}\vector(0,-1){36}}
\put(-20,-207){\color{red}\vector(0,-1){16}}
\end{picture}
\]
\hrulefill

\end{figure}


\begin{figure*} 
\small
\verbdef\jkffxb$#include <stdlib.h> $
\verbdef\jkffxc$#include <stdio.h> $
\verbdef\jkffxd$ $
\verbdef\jkffxe$static int a[]  $
\verbdef\jkffxf$  = {17, 3, 84, 89, 4, 5, 23, 43,  $
\verbdef\jkffxg$     21, 7, 2, 1, 55, 63, 21}; $
\verbdef\jkffxh$static int n = 15; $
\verbdef\jkffxi$ $
\verbdef\jkffxj$static int *part(int *a, int n) $
\verbdef\jkffxba${ $
\verbdef\jkffxbb$   int i = a[0]; $
\verbdef\jkffxbc$   int k = a[n-1]; $
\verbdef\jkffxbd$   int *lp = a; $
\verbdef\jkffxbe$   int *hp = a+n-1; $
\verbdef\jkffxbf$ $
\verbdef\jkffxbg$   while( lp<hp ) { $
\verbdef\jkffxbh$      if( k<i ) { $
\verbdef\jkffxbi$         *lp = k; $
\verbdef\jkffxbj$          lp++; $
\verbdef\jkffxca$          k = *lp; $
\verbdef\jkffxcb$      } else { $
\verbdef\jkffxcc$          *hp = k; $
\verbdef\jkffxcd$           hp--; $
\verbdef\jkffxce$           k = *hp; $
\verbdef\jkffxcf$      } $
\verbdef\jkffxcg$   } $
\verbdef\jkffxch$   *lp = i; $
\verbdef\jkffxci$   return lp; $
\verbdef\jkffxcj$} $
\verbdef\jkffxda$ $
\verbdef\jkffxdb$static void qs(int *a, int n) $
\verbdef\jkffxdc${ $
\verbdef\jkffxdd$   if( n>1 ) { $
\verbdef\jkffxde$      int *lp = part( a, n ); $
\verbdef\jkffxdf$      int m = lp-a; $
\verbdef\jkffxdg$      qs( a, m ); $
\verbdef\jkffxdh$      qs( lp+1, n-m-1 ); $
\verbdef\jkffxdi$   } $
\verbdef\jkffxdj$} $
\verbdef\jkffxea$ $
\verbdef\jkffxeb$  $
\verbdef\jkffxec$ $
\verbdef\jkffxed$int main(int argc, char **argv) $
\verbdef\jkffxee${ $
\verbdef\jkffxef$   int i; $
\verbdef\jkffxeg$    $
\verbdef\jkffxeh$   qs( a, n ); $
\verbdef\jkffxei$   for( i=0; i<n; i++ ) { $
\verbdef\jkffxej$      printf( "%d ", a[i] ); $
\verbdef\jkffxfa$   } $
\verbdef\jkffxfb$   printf( "\n" ); $
\verbdef\jkffxfc$ $
\verbdef\jkffxfd$} $
\hrulefill
\[
\begin{picture}(420,530)(-192,-530)

\put(0,0){\makebox(100,10)[l]{}}
\put(0,-10){\makebox(100,10)[l]{\jkffxb}}
\put(0,-20){\makebox(100,10)[l]{\jkffxc}}
\put(0,-30){\makebox(100,10)[l]{\jkffxd}}
\put(0,-40){\makebox(100,10)[l]{\jkffxe}}
\put(0,-50){\makebox(100,10)[l]{\jkffxf}}
\put(0,-60){\makebox(100,10)[l]{\jkffxg}}
\put(0,-70){\makebox(100,10)[l]{\jkffxh}}
\put(0,-80){\makebox(100,10)[l]{\jkffxi}}
{\color{black} \dottedline{3}(-189,-80)(-5,-80)}\put(0,-90){\makebox(100,10)[l]{\jkffxj}}
{\color{black} \dottedline{3}(-189,-90)(-5,-90)}\put(0,-100){\makebox(100,10)[l]{\jkffxba}}
{\color{black} \dottedline{3}(-189,-100)(-5,-100)}\put(0,-110){\makebox(100,10)[l]{\jkffxbb}}
{\color{black} \dottedline{3}(-189,-110)(-5,-110)}\put(0,-120){\makebox(100,10)[l]{\jkffxbc}}
{\color{black} \dottedline{3}(-189,-120)(-5,-120)}\put(0,-130){\makebox(100,10)[l]{\jkffxbd}}
{\color{black} \dottedline{3}(-189,-130)(-5,-130)}\put(0,-140){\makebox(100,10)[l]{\jkffxbe}}
{\color{black} \dottedline{3}(-189,-140)(-5,-140)}\put(0,-150){\makebox(100,10)[l]{\jkffxbf}}
{\color{black} \dottedline{3}(-189,-150)(-5,-150)}\put(0,-160){\makebox(100,10)[l]{\jkffxbg}}
{\color{black} \dottedline{3}(-189,-160)(-5,-160)}\put(0,-170){\makebox(100,10)[l]{\jkffxbh}}
{\color{black} \dottedline{3}(-189,-170)(-5,-170)}\put(0,-180){\makebox(100,10)[l]{\jkffxbi}}
{\color{black} \dottedline{3}(-189,-180)(-5,-180)}\put(0,-190){\makebox(100,10)[l]{\jkffxbj}}
{\color{black} \dottedline{3}(-189,-190)(-5,-190)}\put(0,-200){\makebox(100,10)[l]{\jkffxca}}
{\color{black} \dottedline{3}(-189,-200)(-5,-200)}\put(0,-210){\makebox(100,10)[l]{\jkffxcb}}
{\color{black} \dottedline{3}(-189,-210)(-5,-210)}\put(0,-220){\makebox(100,10)[l]{\jkffxcc}}
{\color{black} \dottedline{3}(-189,-220)(-5,-220)}\put(0,-230){\makebox(100,10)[l]{\jkffxcd}}
{\color{black} \dottedline{3}(-189,-230)(-5,-230)}\put(0,-240){\makebox(100,10)[l]{\jkffxce}}
{\color{black} \dottedline{3}(-189,-240)(-5,-240)}\put(0,-250){\makebox(100,10)[l]{\jkffxcf}}
{\color{black} \dottedline{3}(-189,-250)(-5,-250)}\put(0,-260){\makebox(100,10)[l]{\jkffxcg}}
{\color{black} \dottedline{3}(-189,-260)(-5,-260)}\put(0,-270){\makebox(100,10)[l]{\jkffxch}}
\put(0,-280){\makebox(100,10)[l]{\jkffxci}}
\put(0,-290){\makebox(100,10)[l]{\jkffxcj}}
\put(0,-300){\makebox(100,10)[l]{\jkffxda}}
{\color{black} \dottedline{3}(-189,-300)(-5,-300)}\put(0,-310){\makebox(100,10)[l]{\jkffxdb}}
{\color{black} \dottedline{3}(-189,-310)(-5,-310)}\put(0,-320){\makebox(100,10)[l]{\jkffxdc}}
{\color{black} \dottedline{3}(-189,-320)(-5,-320)}\put(0,-330){\makebox(100,10)[l]{\jkffxdd}}
{\color{black} \dottedline{3}(-189,-330)(-5,-330)}\put(0,-340){\makebox(100,10)[l]{\jkffxde}}
{\color{black} \dottedline{3}(-189,-340)(-5,-340)}\put(0,-350){\makebox(100,10)[l]{\jkffxdf}}
{\color{black} \dottedline{3}(-189,-350)(-5,-350)}\put(0,-360){\makebox(100,10)[l]{\jkffxdg}}
{\color{black} \dottedline{3}(-189,-360)(-5,-360)}\put(0,-370){\makebox(100,10)[l]{\jkffxdh}}
\put(0,-380){\makebox(100,10)[l]{\jkffxdi}}
\put(0,-390){\makebox(100,10)[l]{\jkffxdj}}
\put(0,-400){\makebox(100,10)[l]{\jkffxea}}
\put(0,-410){\makebox(100,10)[l]{\jkffxeb}}
\put(0,-420){\makebox(100,10)[l]{\jkffxec}}
{\color{black} \dottedline{3}(-189,-420)(-5,-420)}\put(0,-430){\makebox(100,10)[l]{\jkffxed}}
{\color{black} \dottedline{3}(-189,-430)(-5,-430)}\put(0,-440){\makebox(100,10)[l]{\jkffxee}}
{\color{black} \dottedline{3}(-189,-440)(-5,-440)}\put(0,-450){\makebox(100,10)[l]{\jkffxef}}
{\color{black} \dottedline{3}(-189,-450)(-5,-450)}\put(0,-460){\makebox(100,10)[l]{\jkffxeg}}
{\color{black} \dottedline{3}(-189,-460)(-5,-460)}\put(0,-470){\makebox(100,10)[l]{\jkffxeh}}
{\color{black} \dottedline{3}(-189,-470)(-5,-470)}\put(0,-480){\makebox(100,10)[l]{\jkffxei}}
{\color{black} \dottedline{3}(-189,-480)(-5,-480)}\put(0,-490){\makebox(100,10)[l]{\jkffxej}}
{\color{black} \dottedline{3}(-189,-490)(-5,-490)}\put(0,-500){\makebox(100,10)[l]{\jkffxfa}}
{\color{black} \dottedline{3}(-189,-500)(-5,-500)}\put(0,-510){\makebox(100,10)[l]{\jkffxfb}}
\put(0,-520){\makebox(100,10)[l]{\jkffxfc}}
\put(0,-530){\makebox(100,10)[l]{\jkffxfd}}

\color{red}
\put(-6,-295){\circle*{2}}
\put(-6,-297){\vector(0,-1){66}}
\put(-8,-297){\linethickness{0.7pt}\line(1,0){0}}
\put(-6,-365){\circle*{2}}

\color{red}
\put(-12,-312){\circle*{2}}
\put(-14,-314){\linethickness{0.7pt}\line(1,0){0}}
\put(-12,-317){\circle*{2}\hspace{-2\unitlength}\circle{4}}

\color{red}
\put(-18,-315){\circle*{2}}
\put(-18,-317){\vector(0,-1){6}}
\put(-20,-317){\linethickness{0.7pt}\line(1,0){0}}
\put(-18,-325){\circle*{2}}

\color{red}
\put(-24,-315){\circle*{2}}
\put(-24,-317){\vector(0,-1){26}}
\put(-26,-317){\linethickness{0.7pt}\line(1,0){0}}
\put(-24,-345){\circle*{2}}

\color{red}
\put(-30,-315){\circle*{2}\hspace{-2\unitlength}\circle{4}}
\put(-30,-317){\vector(0,-1){16}}
\put(-32,-317){\linethickness{0.7pt}\line(1,0){4}}
\put(-30,-335){\circle*{2}\hspace{-2\unitlength}\circle{4}}

\color{red}
\put(-36,-315){\circle*{2}\hspace{-2\unitlength}\circle{4}}
\put(-36,-317){\vector(0,-1){26}}
\put(-38,-317){\linethickness{0.7pt}\line(1,0){4}}
\put(-36,-345){\circle*{2}\hspace{-2\unitlength}\circle{4}}

\color{red}
\put(-12,-325){\circle*{2}}
\put(-12,-327){\vector(0,-1){6}}
\put(-14,-327){\linethickness{0.7pt}\line(1,0){0}}
\put(-12,-335){\circle*{2}}

\color{red}
\put(-42,-325){\circle*{2}}
\put(-42,-327){\vector(0,-1){16}}
\put(-44,-327){\linethickness{0.7pt}\line(1,0){0}}
\put(-42,-345){\circle*{2}}

\color{red}
\put(-18,-332){\circle*{2}}
\put(-20,-334){\linethickness{0.7pt}\line(1,0){0}}
\put(-18,-337){\circle*{2}\hspace{-2\unitlength}\circle{4}}

\color{red}
\put(-12,-342){\circle*{2}}
\put(-14,-344){\linethickness{0.7pt}\line(1,0){0}}
\put(-12,-347){\circle*{2}\hspace{-2\unitlength}\circle{4}}

\color{red}
\put(-6,-75){\circle*{2}}
\put(-6,-77){\vector(0,-1){186}}
\put(-8,-77){\linethickness{0.7pt}\line(1,0){0}}
\put(-6,-265){\circle*{2}}

\color{red}
\put(-12,-85){\circle*{2}}
\put(-12,-87){\vector(0,-1){56}}
\put(-14,-87){\linethickness{0.7pt}\line(1,0){0}}
\put(-12,-145){\circle*{2}}

\color{green}
\put(-18,-85){\circle*{2}}
\put(-18,-87){\vector(0,-1){66}}
\put(-20,-87){\linethickness{0.7pt}\line(1,0){4}}
\put(-18,-155){\circle*{2}}

\color{green}
\put(-24,-85){\circle*{2}}
\put(-24,-87){\vector(0,-1){156}}
\put(-26,-87){\linethickness{0.7pt}\line(1,0){4}}
\put(-24,-245){\circle*{2}}

\color{red}
\put(-30,-85){\circle*{2}}
\put(-30,-87){\vector(0,-1){156}}
\put(-32,-87){\linethickness{0.7pt}\line(1,0){0}}
\put(-30,-245){\circle*{2}}

\color{red}
\put(-36,-95){\circle*{2}}
\put(-36,-97){\vector(0,-1){46}}
\put(-38,-97){\linethickness{0.7pt}\line(1,0){0}}
\put(-36,-145){\circle*{2}}

\color{red}
\put(-42,-95){\circle*{2}}
\put(-42,-97){\vector(0,-1){56}}
\put(-44,-97){\linethickness{0.7pt}\line(1,0){0}}
\put(-42,-155){\circle*{2}}

\color{green}
\put(-48,-95){\circle*{2}}
\put(-48,-97){\vector(0,-1){96}}
\put(-50,-97){\linethickness{0.7pt}\line(1,0){4}}
\put(-48,-195){\circle*{2}}

\color{red}
\put(-54,-95){\circle*{2}}
\put(-54,-97){\vector(0,-1){96}}
\put(-56,-97){\linethickness{0.7pt}\line(1,0){0}}
\put(-54,-195){\circle*{2}}

\color{green}
\put(-60,-95){\circle*{2}}
\put(-60,-97){\vector(0,-1){146}}
\put(-62,-97){\linethickness{0.7pt}\line(1,0){4}}
\put(-60,-245){\circle*{2}}

\color{red}
\put(-66,-105){\circle*{2}}
\put(-66,-107){\vector(0,-1){26}}
\put(-68,-107){\linethickness{0.7pt}\line(1,0){0}}
\put(-66,-135){\circle*{2}}

\color{red}
\put(-72,-105){\circle*{2}}
\put(-72,-107){\vector(0,-1){46}}
\put(-74,-107){\linethickness{0.7pt}\line(1,0){0}}
\put(-72,-155){\circle*{2}}

\color{red}
\put(-78,-105){\circle*{2}}
\put(-78,-107){\vector(0,-1){56}}
\put(-80,-107){\linethickness{0.7pt}\line(1,0){0}}
\put(-78,-165){\circle*{2}}

\color{red}
\put(-84,-105){\circle*{2}}
\put(-84,-107){\vector(0,-1){136}}
\put(-86,-107){\linethickness{0.7pt}\line(1,0){0}}
\put(-84,-245){\circle*{2}}

\color{red}
\put(-90,-105){\circle*{2}}
\put(-90,-107){\vector(0,-1){146}}
\put(-92,-107){\linethickness{0.7pt}\line(1,0){0}}
\put(-90,-255){\circle*{2}}

\color{red}
\put(-96,-115){\circle*{2}}
\put(-96,-117){\vector(0,-1){16}}
\put(-98,-117){\linethickness{0.7pt}\line(1,0){0}}
\put(-96,-135){\circle*{2}}

\color{red}
\put(-102,-115){\circle*{2}}
\put(-102,-117){\vector(0,-1){76}}
\put(-104,-117){\linethickness{0.7pt}\line(1,0){0}}
\put(-102,-195){\circle*{2}}

\color{red}
\put(-108,-115){\circle*{2}}
\put(-108,-117){\vector(0,-1){86}}
\put(-110,-117){\linethickness{0.7pt}\line(1,0){0}}
\put(-108,-205){\circle*{2}}

\color{green}
\put(-114,-135){\circle*{2}}
\put(-114,-137){\vector(0,-1){26}}
\put(-116,-137){\linethickness{0.7pt}\line(1,0){0}}
\put(-114,-165){\circle*{2}}

\color{green}
\put(-120,-135){\circle*{2}}
\put(-120,-137){\vector(0,-1){66}}
\put(-122,-137){\linethickness{0.7pt}\line(1,0){0}}
\put(-120,-205){\circle*{2}}

\color{green}
\put(-66,-145){\circle*{2}}
\put(-66,-147){\vector(0,-1){26}}
\put(-68,-147){\linethickness{0.7pt}\line(1,0){0}}
\put(-66,-175){\circle*{2}}

\color{green}
\put(-96,-145){\circle*{2}}
\put(-96,-147){\vector(0,-1){66}}
\put(-98,-147){\linethickness{0.7pt}\line(1,0){0}}
\put(-96,-215){\circle*{2}}

\color{green}
\put(-12,-155){\circle*{2}}
\put(-12,-157){\vector(0,-1){6}}
\put(-14,-157){\linethickness{0.7pt}\line(1,0){0}}
\put(-12,-165){\circle*{2}}

\color{green}
\put(-36,-155){\circle*{2}}
\put(-36,-157){\vector(0,-1){16}}
\put(-38,-157){\linethickness{0.7pt}\line(1,0){0}}
\put(-36,-175){\circle*{2}}

\color{red}
\put(-126,-165){\circle*{2}}
\put(-126,-163){\vector(0,1){26}}
\put(-128,-137){\linethickness{0.7pt}\line(1,0){0}}
\put(-126,-135){\circle*{2}}

\color{red}
\put(-132,-165){\circle*{2}}
\put(-132,-163){\vector(0,1){6}}
\put(-134,-157){\linethickness{0.7pt}\line(1,0){0}}
\put(-132,-155){\circle*{2}}

\color{red}
\put(-18,-162){\circle*{2}}
\put(-20,-164){\linethickness{0.7pt}\line(1,0){0}}
\put(-18,-167){\circle*{2}}

\color{green}
\put(-42,-162){\circle*{2}}
\put(-44,-164){\linethickness{0.7pt}\line(1,0){0}}
\put(-42,-167){\circle*{2}}

\color{red}
\put(-72,-165){\circle*{2}}
\put(-72,-167){\vector(0,-1){6}}
\put(-74,-167){\linethickness{0.7pt}\line(1,0){0}}
\put(-72,-175){\circle*{2}}

\color{red}
\put(-138,-165){\circle*{2}}
\put(-138,-167){\vector(0,-1){76}}
\put(-140,-167){\linethickness{0.7pt}\line(1,0){0}}
\put(-138,-245){\circle*{2}}

\color{red}
\put(-144,-165){\circle*{2}}
\put(-144,-167){\vector(0,-1){86}}
\put(-146,-167){\linethickness{0.7pt}\line(1,0){0}}
\put(-144,-255){\circle*{2}}

\color{red}
\put(-150,-175){\circle*{2}}
\put(-150,-173){\vector(0,1){26}}
\put(-152,-147){\linethickness{0.7pt}\line(1,0){0}}
\put(-150,-145){\circle*{2}}

\color{green}
\put(-156,-175){\circle*{2}}
\put(-156,-173){\vector(0,1){16}}
\put(-158,-157){\linethickness{0.7pt}\line(1,0){4}}
\put(-156,-155){\circle*{2}}

\color{red}
\put(-162,-175){\circle*{2}}
\put(-162,-173){\vector(0,1){16}}
\put(-164,-157){\linethickness{0.7pt}\line(1,0){0}}
\put(-162,-155){\circle*{2}}

\color{green}
\put(-168,-175){\circle*{2}}
\put(-168,-173){\vector(0,1){6}}
\put(-170,-167){\linethickness{0.7pt}\line(1,0){0}}
\put(-168,-165){\circle*{2}}

\color{red}
\put(-12,-175){\circle*{2}}
\put(-12,-177){\vector(0,-1){16}}
\put(-14,-177){\linethickness{0.7pt}\line(1,0){0}}
\put(-12,-195){\circle*{2}}

\color{green}
\put(-18,-175){\circle*{2}}
\put(-18,-177){\vector(0,-1){66}}
\put(-20,-177){\linethickness{0.7pt}\line(1,0){4}}
\put(-18,-245){\circle*{2}}

\color{green}
\put(-36,-195){\circle*{2}}
\put(-36,-197){\vector(0,-1){6}}
\put(-38,-197){\linethickness{0.7pt}\line(1,0){0}}
\put(-36,-205){\circle*{2}}

\color{green}
\put(-42,-195){\circle*{2}}
\put(-42,-197){\vector(0,-1){16}}
\put(-44,-197){\linethickness{0.7pt}\line(1,0){0}}
\put(-42,-215){\circle*{2}}

\color{red}
\put(-174,-205){\circle*{2}}
\put(-174,-203){\vector(0,1){66}}
\put(-176,-137){\linethickness{0.7pt}\line(1,0){0}}
\put(-174,-135){\circle*{2}}

\color{red}
\put(-66,-205){\circle*{2}}
\put(-66,-203){\vector(0,1){6}}
\put(-68,-197){\linethickness{0.7pt}\line(1,0){0}}
\put(-66,-195){\circle*{2}}

\color{red}
\put(-12,-202){\circle*{2}}
\put(-14,-204){\linethickness{0.7pt}\line(1,0){0}}
\put(-12,-207){\circle*{2}}

\color{green}
\put(-48,-202){\circle*{2}}
\put(-50,-204){\linethickness{0.7pt}\line(1,0){0}}
\put(-48,-207){\circle*{2}}

\color{red}
\put(-54,-205){\circle*{2}}
\put(-54,-207){\vector(0,-1){6}}
\put(-56,-207){\linethickness{0.7pt}\line(1,0){0}}
\put(-54,-215){\circle*{2}}

\color{red}
\put(-180,-215){\circle*{2}}
\put(-180,-213){\vector(0,1){66}}
\put(-182,-147){\linethickness{0.7pt}\line(1,0){0}}
\put(-180,-145){\circle*{2}}

\color{red}
\put(-186,-215){\circle*{2}}
\put(-186,-213){\vector(0,1){56}}
\put(-188,-157){\linethickness{0.7pt}\line(1,0){0}}
\put(-186,-155){\circle*{2}}

\color{green}
\put(-72,-215){\circle*{2}}
\put(-72,-213){\vector(0,1){16}}
\put(-74,-197){\linethickness{0.7pt}\line(1,0){4}}
\put(-72,-195){\circle*{2}}

\color{red}
\put(-78,-215){\circle*{2}}
\put(-78,-213){\vector(0,1){16}}
\put(-80,-197){\linethickness{0.7pt}\line(1,0){0}}
\put(-78,-195){\circle*{2}}

\color{green}
\put(-102,-215){\circle*{2}}
\put(-102,-213){\vector(0,1){6}}
\put(-104,-207){\linethickness{0.7pt}\line(1,0){0}}
\put(-102,-205){\circle*{2}}

\color{green}
\put(-12,-215){\circle*{2}}
\put(-12,-217){\vector(0,-1){26}}
\put(-14,-217){\linethickness{0.7pt}\line(1,0){4}}
\put(-12,-245){\circle*{2}}

\color{red}
\put(-6,-415){\circle*{2}}
\put(-6,-417){\vector(0,-1){86}}
\put(-8,-417){\linethickness{0.7pt}\line(1,0){0}}
\put(-6,-505){\circle*{2}}

\color{red}
\put(-12,-442){\circle*{2}}
\put(-14,-444){\linethickness{0.7pt}\line(1,0){0}}
\put(-12,-447){\circle*{2}\hspace{-2\unitlength}\circle{4}}

\color{red}
\put(-18,-445){\circle*{2}\hspace{-2\unitlength}\circle{4}}
\put(-18,-447){\vector(0,-1){16}}
\put(-20,-447){\linethickness{0.7pt}\line(1,0){4}}
\put(-18,-465){\circle*{2}}

\color{red}
\put(-12,-452){\circle*{2}}
\put(-14,-454){\linethickness{0.7pt}\line(1,0){0}}
\put(-12,-457){\circle*{2}}

\color{green}
\put(-24,-452){\circle*{2}}
\put(-26,-454){\linethickness{0.7pt}\line(1,0){0}}
\put(-24,-457){\circle*{2}}

\color{red}
\put(-30,-455){\circle*{2}}
\put(-30,-457){\vector(0,-1){6}}
\put(-32,-457){\linethickness{0.7pt}\line(1,0){0}}
\put(-30,-465){\circle*{2}}

\color{green}
\put(-36,-465){\circle*{2}}
\put(-36,-463){\vector(0,1){6}}
\put(-38,-457){\linethickness{0.7pt}\line(1,0){0}}
\put(-36,-455){\circle*{2}}

\color{red}
\put(-12,-462){\circle*{2}}
\put(-14,-464){\linethickness{0.7pt}\line(1,0){0}}
\put(-12,-467){\circle*{2}\hspace{-2\unitlength}\circle{4}}

\color{red}
\put(-24,-462){\circle*{2}\hspace{-2\unitlength}\circle{4}}
\put(-26,-464){\linethickness{0.7pt}\line(1,0){4}}
\put(-24,-467){\circle*{2}\hspace{-2\unitlength}\circle{4}}

\color{red}
\put(-42,-465){\circle*{2}\hspace{-2\unitlength}\circle{4}}
\put(-42,-467){\vector(0,-1){16}}
\put(-44,-467){\linethickness{0.7pt}\line(1,0){4}}
\put(-42,-485){\circle*{2}\hspace{-2\unitlength}\circle{4}}

\color{green}
\put(-48,-465){\circle*{2}\hspace{-2\unitlength}\circle{4}}
\put(-48,-467){\vector(0,-1){16}}
\put(-50,-467){\linethickness{0.7pt}\line(1,0){4}}
\put(-48,-485){\circle*{2}\hspace{-2\unitlength}\circle{4}}

\color{blue}
\put(-54,-465){\circle*{2}\hspace{-2\unitlength}\circle{4}}
\put(-54,-467){\vector(0,-1){16}}
\put(-56,-467){\linethickness{0.7pt}\line(1,0){4}}
\put(-54,-485){\circle*{2}\hspace{-2\unitlength}\circle{4}}

\color{red}
\put(-12,-482){\circle*{2}}
\put(-14,-484){\linethickness{0.7pt}\line(1,0){0}}
\put(-12,-487){\circle*{2}\hspace{-2\unitlength}\circle{4}}

\end{picture}
\]
\hrulefill

\end{figure*}

\begin{figure} 
\small
\verbdef\jkffxb$#include <stdlib.h> $
\verbdef\jkffxc$#include <stdio.h> $
\verbdef\jkffxd$ $
\verbdef\jkffxe$static int a[]  $
\verbdef\jkffxf$  = {17, 3, 84, 89, 4, 5, 23, 43,  $
\verbdef\jkffxg$     21, 7, 2, 1, 55, 63, 21}; $
\verbdef\jkffxh$static int n = 15; $
\verbdef\jkffxi$ $
\verbdef\jkffxj$static int *part(int *a, int n) $
\verbdef\jkffxba${ $
\verbdef\jkffxbb$   int i = a[0]; $
\verbdef\jkffxbc$   int k = a[n-1]; $
\verbdef\jkffxbd$   int *lp = a; $
\verbdef\jkffxbe$   int *hp = a+n-1; $
\verbdef\jkffxbf$ $
\verbdef\jkffxbg$   while( lp<hp ) { $
\verbdef\jkffxbh$      if( k<i ) { $
\verbdef\jkffxbi$         *lp = k; $
\verbdef\jkffxbj$          lp++; $
\verbdef\jkffxca$          k = *lp; $
\verbdef\jkffxcb$      } else { $
\verbdef\jkffxcc$          *hp = k; $
\verbdef\jkffxcd$           hp--; $
\verbdef\jkffxce$           k = *hp; $
\verbdef\jkffxcf$      } $
\verbdef\jkffxcg$   } $
\verbdef\jkffxch$   *lp = i; $
\verbdef\jkffxci$   return lp; $
\verbdef\jkffxcj$} $
\verbdef\jkffxda$ $
\verbdef\jkffxdb$static void qs(int *a, int n) $
\verbdef\jkffxdc${ $
\verbdef\jkffxdd$   if( n>1 ) { $
\verbdef\jkffxde$      int *lp = part( a, n ); $
\verbdef\jkffxdf$      int m = lp-a; $
\verbdef\jkffxdg$      qs( a, m ); $
\verbdef\jkffxdh$      qs( lp+1, n-m-1 ); $
\verbdef\jkffxdi$   } $
\verbdef\jkffxdj$} $
\verbdef\jkffxea$ $
\verbdef\jkffxeb$  $
\verbdef\jkffxec$ $
\verbdef\jkffxed$int main(int argc, char **argv) $
\verbdef\jkffxee${ $
\verbdef\jkffxef$   int i; $
\verbdef\jkffxeg$    $
\verbdef\jkffxeh$   qs( a, n ); $
\verbdef\jkffxei$   for( i=0; i<n; i++ ) { $
\verbdef\jkffxej$      printf( "%d ", a[i] ); $
\verbdef\jkffxfa$   } $
\verbdef\jkffxfb$   printf( "\n" ); $
\verbdef\jkffxfc$ $
\verbdef\jkffxfd$} $
\hrulefill
\[
\begin{picture}(420,530)(-80,-530)

\put(0,-10){\makebox(100,10)[l]{\jkffxb}}
\put(0,-20){\makebox(100,10)[l]{\jkffxc}}
\put(0,-30){\makebox(100,10)[l]{\jkffxd}}
\put(0,-40){\makebox(100,10)[l]{\jkffxe}}
\put(0,-50){\makebox(100,10)[l]{\jkffxf}}
\put(0,-60){\makebox(100,10)[l]{\jkffxg}}
\put(0,-70){\makebox(100,10)[l]{\jkffxh}}
\put(0,-80){\makebox(100,10)[l]{\jkffxi}}
{\color{black} \dottedline{3}(-10,-75)(-0,-75)}
\put(0,-90){\makebox(100,10)[l]{\jkffxj}}
{\color{black} \dottedline{3}(-10,-85)(-0,-85)}
\put(0,-100){\makebox(100,10)[l]{\jkffxba}}
{\color{black} \dottedline{3}(-10,-95)(-0,-95)}
\put(0,-110){\makebox(100,10)[l]{\jkffxbb}}
{\color{black} \dottedline{3}(-10,-105)(-0,-105)}
\put(0,-120){\makebox(100,10)[l]{\jkffxbc}}
{\color{black} \dottedline{3}(-10,-115)(-0,-115)}
\put(0,-130){\makebox(100,10)[l]{\jkffxbd}}
\put(0,-140){\makebox(100,10)[l]{\jkffxbe}}
{\color{black} \dottedline{3}(-10,-135)(-0,-135)}
\put(0,-150){\makebox(100,10)[l]{\jkffxbf}}
{\color{black} \dottedline{3}(-10,-145)(-0,-145)}
\put(0,-160){\makebox(100,10)[l]{\jkffxbg}}
{\color{black} \dottedline{3}(-10,-155)(-0,-155)}
\put(0,-170){\makebox(100,10)[l]{\jkffxbh}}
{\color{black} \dottedline{3}(-10,-165)(-0,-165)}
\put(0,-180){\makebox(100,10)[l]{\jkffxbi}}
{\color{black} \dottedline{3}(-10,-175)(-0,-175)}
\put(0,-190){\makebox(100,10)[l]{\jkffxbj}}
\put(0,-200){\makebox(100,10)[l]{\jkffxca}}
{\color{black} \dottedline{3}(-10,-195)(-0,-195)}
\put(0,-210){\makebox(100,10)[l]{\jkffxcb}}
{\color{black} \dottedline{3}(-10,-205)(-0,-205)}
\put(0,-220){\makebox(100,10)[l]{\jkffxcc}}
{\color{black} \dottedline{3}(-10,-215)(-0,-215)}
\put(0,-230){\makebox(100,10)[l]{\jkffxcd}}
\put(0,-240){\makebox(100,10)[l]{\jkffxce}}
\put(0,-250){\makebox(100,10)[l]{\jkffxcf}}
{\color{black} \dottedline{3}(-10,-245)(-0,-245)}
\put(0,-260){\makebox(100,10)[l]{\jkffxcg}}
{\color{black} \dottedline{3}(-10,-255)(-0,-255)}
\put(0,-270){\makebox(100,10)[l]{\jkffxch}}
{\color{black} \dottedline{3}(-10,-265)(-0,-265)}
\put(0,-280){\makebox(100,10)[l]{\jkffxci}}
\put(0,-290){\makebox(100,10)[l]{\jkffxcj}}
\put(0,-300){\makebox(100,10)[l]{\jkffxda}}
{\color{black} \dottedline{3}(-10,-295)(-0,-295)}
\put(0,-310){\makebox(100,10)[l]{\jkffxdb}}
\put(0,-320){\makebox(100,10)[l]{\jkffxdc}}
{\color{black} \dottedline{3}(-10,-315)(-0,-315)}
\put(0,-330){\makebox(100,10)[l]{\jkffxdd}}
{\color{black} \dottedline{3}(-10,-325)(-0,-325)}
\put(0,-340){\makebox(100,10)[l]{\jkffxde}}
{\color{black} \dottedline{3}(-10,-335)(-0,-335)}
\put(0,-350){\makebox(100,10)[l]{\jkffxdf}}
{\color{black} \dottedline{3}(-10,-345)(-0,-345)}
\put(0,-360){\makebox(100,10)[l]{\jkffxdg}}
\put(0,-370){\makebox(100,10)[l]{\jkffxdh}}
{\color{black} \dottedline{3}(-10,-365)(-0,-365)}
\put(0,-380){\makebox(100,10)[l]{\jkffxdi}}
\put(0,-390){\makebox(100,10)[l]{\jkffxdj}}
\put(0,-400){\makebox(100,10)[l]{\jkffxea}}
\put(0,-410){\makebox(100,10)[l]{\jkffxeb}}
\put(0,-420){\makebox(100,10)[l]{\jkffxec}}
{\color{black} \dottedline{3}(-10,-415)(-0,-415)}
\put(0,-430){\makebox(100,10)[l]{\jkffxed}}
\put(0,-440){\makebox(100,10)[l]{\jkffxee}}
\put(0,-450){\makebox(100,10)[l]{\jkffxef}}
{\color{black} \dottedline{3}(-10,-445)(-0,-445)}
\put(0,-460){\makebox(100,10)[l]{\jkffxeg}}
{\color{black} \dottedline{3}(-10,-455)(-0,-455)}
\put(0,-470){\makebox(100,10)[l]{\jkffxeh}}
{\color{black} \dottedline{3}(-10,-465)(-0,-465)}
\put(0,-480){\makebox(100,10)[l]{\jkffxei}}
\put(0,-490){\makebox(100,10)[l]{\jkffxej}}
{\color{black} \dottedline{3}(-10,-485)(-0,-485)}
\put(0,-500){\makebox(100,10)[l]{\jkffxfa}}
\put(0,-510){\makebox(100,10)[l]{\jkffxfb}}
{\color{black} \dottedline{3}(-10,-505)(-0,-505)}
\put(0,-520){\makebox(100,10)[l]{\jkffxfc}}
\put(0,-530){\makebox(100,10)[l]{\jkffxfd}}

\put(-30,-255){\color{black}\circle*{2}}
\put(-20,-165){\color{black}\circle*{2}}
\put(-30,-345){\color{black}\circle*{2}}
\put(-30,-445){\color{black}\circle*{2}}
\put(-80,-175){\color{black}\circle*{2}}
\put(-10,-265){\color{black}\circle*{2}}
\put(-20,-455){\color{black}\circle*{2}}
\put(-10,-365){\color{black}\circle*{2}}
\put(-30,-465){\color{black}\circle*{2}}
\put(-10,-75){\color{black}\circle*{2}}
\put(-30,-485){\color{black}\circle*{2}}
\put(-70,-85){\color{black}\circle*{2}}
\put(-60,-95){\color{black}\circle*{2}}
\put(-40,-105){\color{black}\circle*{2}}
\put(-60,-195){\color{black}\circle*{2}}
\put(-50,-205){\color{black}\circle*{2}}
\put(-50,-115){\color{black}\circle*{2}}
\put(-10,-295){\color{black}\circle*{2}}
\put(-60,-215){\color{black}\circle*{2}}
\put(-40,-135){\color{black}\circle*{2}}
\put(-30,-315){\color{black}\circle*{2}}
\put(-10,-505){\color{black}\circle*{2}}
\put(-10,-415){\color{black}\circle*{2}}
\put(-80,-145){\color{black}\circle*{2}}
\put(-20,-325){\color{black}\circle*{2}}
\put(-70,-245){\color{black}\circle*{2}}
\put(-20,-155){\color{black}\circle*{2}}
\put(-20,-335){\color{black}\circle*{2}}
{\color{red}\dashline[200]{3}(-10.0,-77.0)(-10.0,-263.0)}%
{\color{red}\dashline[200]{3}(-10.0,-263.0)(-9.0,-261.0)}%
{\color{red}\dashline[200]{3}(-10.0,-263.0)(-11.0,-261.0)}%
{\color{red}\dashline[200]{3}(-70.3,-87.0)(-79.7,-143.0)}%
{\color{red}\dashline[200]{3}(-79.7,-143.0)(-78.4,-141.2)}%
{\color{red}\dashline[200]{3}(-79.7,-143.0)(-80.3,-140.9)}%
{\color{green}\dashline[200]{3}(-68.8,-86.6)(-21.2,-153.4)}%
{\color{green}\dashline[200]{3}(-21.2,-153.4)(-21.5,-151.2)}%
{\color{green}\dashline[200]{3}(-21.2,-153.4)(-23.1,-152.3)}%
{\color{red}\dashline[200]{3}(-70.0,-87.0)(-70.0,-243.0)}%
{\color{red}\dashline[200]{3}(-70.0,-243.0)(-69.0,-241.0)}%
{\color{red}\dashline[200]{3}(-70.0,-243.0)(-71.0,-241.0)}%
{\color{red}\dashline[200]{3}(-60.7,-96.9)(-79.3,-143.1)}%
{\color{red}\dashline[200]{3}(-79.3,-143.1)(-77.6,-141.7)}%
{\color{red}\dashline[200]{3}(-79.3,-143.1)(-79.4,-140.9)}%
{\color{red}\dashline[200]{3}(-58.9,-96.7)(-21.1,-153.3)}%
{\color{red}\dashline[200]{3}(-21.1,-153.3)(-21.4,-151.1)}%
{\color{red}\dashline[200]{3}(-21.1,-153.3)(-23.1,-152.2)}%
{\color{red}\dashline[200]{3}(-60.0,-97.0)(-60.0,-193.0)}%
{\color{red}\dashline[200]{3}(-60.0,-193.0)(-59.0,-191.0)}%
{\color{red}\dashline[200]{3}(-60.0,-193.0)(-61.0,-191.0)}%
{\color{green}\dashline[200]{3}(-60.1,-97.0)(-69.9,-243.0)}%
{\color{green}\dashline[200]{3}(-69.9,-243.0)(-68.7,-241.1)}%
{\color{green}\dashline[200]{3}(-69.9,-243.0)(-70.7,-240.9)}%
{\color{red}\dashline[200]{3}(-40.0,-107.0)(-40.0,-133.0)}%
{\color{red}\dashline[200]{3}(-40.0,-133.0)(-39.0,-131.0)}%
{\color{red}\dashline[200]{3}(-40.0,-133.0)(-41.0,-131.0)}%
{\color{red}\dashline[200]{3}(-39.3,-106.9)(-20.7,-153.1)}%
{\color{red}\dashline[200]{3}(-20.7,-153.1)(-20.6,-150.9)}%
{\color{red}\dashline[200]{3}(-20.7,-153.1)(-22.4,-151.7)}%
{\color{red}\dashline[200]{3}(-39.4,-106.9)(-20.6,-163.1)}%
{\color{red}\dashline[200]{3}(-20.6,-163.1)(-20.3,-160.9)}%
{\color{red}\dashline[200]{3}(-20.6,-163.1)(-22.2,-161.5)}%
{\color{red}\dashline[200]{3}(-40.4,-107.0)(-69.6,-243.0)}%
{\color{red}\dashline[200]{3}(-69.6,-243.0)(-68.2,-241.3)}%
{\color{red}\dashline[200]{3}(-69.6,-243.0)(-70.1,-240.9)}%
{\color{red}\dashline[200]{3}(-39.9,-107.0)(-30.1,-253.0)}%
{\color{red}\dashline[200]{3}(-30.1,-253.0)(-29.3,-250.9)}%
{\color{red}\dashline[200]{3}(-30.1,-253.0)(-31.3,-251.1)}%
{\color{red}\dashline[200]{3}(-49.1,-116.8)(-40.9,-133.2)}%
{\color{red}\dashline[200]{3}(-40.9,-133.2)(-40.9,-131.0)}%
{\color{red}\dashline[200]{3}(-40.9,-133.2)(-42.7,-131.9)}%
{\color{red}\dashline[200]{3}(-50.2,-117.0)(-59.8,-193.0)}%
{\color{red}\dashline[200]{3}(-59.8,-193.0)(-58.5,-191.2)}%
{\color{red}\dashline[200]{3}(-59.8,-193.0)(-60.5,-190.9)}%
{\color{red}\dashline[200]{3}(-50.0,-117.0)(-50.0,-203.0)}%
{\color{red}\dashline[200]{3}(-50.0,-203.0)(-49.0,-201.0)}%
{\color{red}\dashline[200]{3}(-50.0,-203.0)(-51.0,-201.0)}%
{\color{green}\dashline[200]{3}(-38.1,-136.1)(-20.3,-162.8)}%
{\color{green}\dashline[200]{3}(-20.3,-162.8)(-20.6,-160.6)}%
{\color{green}\dashline[200]{3}(-20.3,-162.8)(-22.2,-161.7)}%
{\color{green}\dashline[200]{3}(-39.3,-137.1)(-48.7,-203.2)}%
{\color{green}\dashline[200]{3}(-48.7,-203.2)(-47.5,-201.3)}%
{\color{green}\dashline[200]{3}(-48.7,-203.2)(-49.4,-201.0)}%
{\color{green}\dashline[200]{3}(-79.0,-147.0)(-79.0,-173.0)}%
{\color{green}\dashline[200]{3}(-79.0,-173.0)(-78.0,-171.0)}%
{\color{green}\dashline[200]{3}(-79.0,-173.0)(-80.0,-171.0)}%
{\color{green}\dashline[200]{3}(-78.5,-146.6)(-59.6,-212.8)}%
{\color{green}\dashline[200]{3}(-59.6,-212.8)(-59.2,-210.6)}%
{\color{green}\dashline[200]{3}(-59.6,-212.8)(-61.1,-211.2)}%
{\color{green}\dashline[200]{3}(-19.0,-157.0)(-19.0,-163.0)}%
{\color{green}\dashline[200]{3}(-19.0,-163.0)(-18.0,-161.0)}%
{\color{green}\dashline[200]{3}(-19.0,-163.0)(-20.0,-161.0)}%
{\color{green}\dashline[200]{3}(-21.6,-156.6)(-77.8,-175.3)}%
{\color{green}\dashline[200]{3}(-77.8,-175.3)(-75.6,-175.6)}%
{\color{green}\dashline[200]{3}(-77.8,-175.3)(-76.2,-173.7)}%
{\color{red}\dashline[200]{3}(-21.9,-163.9)(-39.7,-137.2)}%
{\color{red}\dashline[200]{3}(-39.7,-137.2)(-39.4,-139.4)}%
{\color{red}\dashline[200]{3}(-39.7,-137.2)(-37.8,-138.3)}%
{\color{red}\dashline[200]{3}(-21.0,-163.0)(-21.0,-157.0)}%
{\color{red}\dashline[200]{3}(-21.0,-157.0)(-22.0,-159.0)}%
{\color{red}\dashline[200]{3}(-21.0,-157.0)(-20.0,-159.0)}%
{\color{red}\dashline[200]{3}(-21.8,-166.3)(-77.9,-175.7)}%
{\color{red}\dashline[200]{3}(-77.9,-175.7)(-75.7,-176.3)}%
{\color{red}\dashline[200]{3}(-77.9,-175.7)(-76.1,-174.3)}%
{\color{red}\dashline[200]{3}(-21.1,-166.7)(-68.9,-243.3)}%
{\color{red}\dashline[200]{3}(-68.9,-243.3)(-67.0,-242.1)}%
{\color{red}\dashline[200]{3}(-68.9,-243.3)(-68.7,-241.1)}%
{\color{red}\dashline[200]{3}(-20.2,-167.0)(-29.8,-253.0)}%
{\color{red}\dashline[200]{3}(-29.8,-253.0)(-28.6,-251.1)}%
{\color{red}\dashline[200]{3}(-29.8,-253.0)(-30.6,-250.9)}%
{\color{red}\dashline[200]{3}(-81.0,-173.0)(-81.0,-147.0)}%
{\color{red}\dashline[200]{3}(-81.0,-147.0)(-82.0,-149.0)}%
{\color{red}\dashline[200]{3}(-81.0,-147.0)(-80.0,-149.0)}%
{\color{red}\dashline[200]{3}(-78.4,-173.4)(-22.2,-154.7)}%
{\color{red}\dashline[200]{3}(-22.2,-154.7)(-24.4,-154.4)}%
{\color{red}\dashline[200]{3}(-22.2,-154.7)(-23.8,-156.3)}%
{\color{green}\dashline[200]{3}(-78.2,-173.7)(-22.1,-164.3)}%
{\color{green}\dashline[200]{3}(-22.1,-164.3)(-24.3,-163.7)}%
{\color{green}\dashline[200]{3}(-22.1,-164.3)(-23.9,-165.7)}%
{\color{red}\dashline[200]{3}(-78.6,-176.4)(-61.4,-193.6)}%
{\color{red}\dashline[200]{3}(-61.4,-193.6)(-62.1,-191.5)}%
{\color{red}\dashline[200]{3}(-61.4,-193.6)(-63.5,-192.9)}%
{\color{green}\dashline[200]{3}(-79.7,-177.0)(-70.3,-243.0)}%
{\color{green}\dashline[200]{3}(-70.3,-243.0)(-69.6,-240.9)}%
{\color{green}\dashline[200]{3}(-70.3,-243.0)(-71.6,-241.2)}%
{\color{green}\dashline[200]{3}(-57.9,-195.7)(-50.7,-202.9)}%
{\color{green}\dashline[200]{3}(-50.7,-202.9)(-51.4,-200.8)}%
{\color{green}\dashline[200]{3}(-50.7,-202.9)(-52.8,-202.2)}%
{\color{green}\dashline[200]{3}(-59.0,-197.0)(-59.0,-213.0)}%
{\color{green}\dashline[200]{3}(-59.0,-213.0)(-58.0,-211.0)}%
{\color{green}\dashline[200]{3}(-59.0,-213.0)(-60.0,-211.0)}%
{\color{red}\dashline[200]{3}(-50.7,-202.9)(-41.3,-136.8)}%
{\color{red}\dashline[200]{3}(-41.3,-136.8)(-42.5,-138.7)}%
{\color{red}\dashline[200]{3}(-41.3,-136.8)(-40.6,-139.0)}%
{\color{red}\dashline[200]{3}(-52.1,-204.3)(-59.3,-197.1)}%
{\color{red}\dashline[200]{3}(-59.3,-197.1)(-58.6,-199.2)}%
{\color{red}\dashline[200]{3}(-59.3,-197.1)(-57.2,-197.8)}%
{\color{red}\dashline[200]{3}(-50.7,-207.1)(-57.9,-214.3)}%
{\color{red}\dashline[200]{3}(-57.9,-214.3)(-55.8,-213.6)}%
{\color{red}\dashline[200]{3}(-57.9,-214.3)(-57.2,-212.2)}%
{\color{red}\dashline[200]{3}(-61.5,-213.4)(-80.4,-147.2)}%
{\color{red}\dashline[200]{3}(-80.4,-147.2)(-80.8,-149.4)}%
{\color{red}\dashline[200]{3}(-80.4,-147.2)(-78.9,-148.8)}%
{\color{red}\dashline[200]{3}(-58.9,-213.3)(-21.1,-156.7)}%
{\color{red}\dashline[200]{3}(-21.1,-156.7)(-23.1,-157.8)}%
{\color{red}\dashline[200]{3}(-21.1,-156.7)(-21.4,-158.9)}%
{\color{red}\dashline[200]{3}(-61.0,-213.0)(-61.0,-197.0)}%
{\color{red}\dashline[200]{3}(-61.0,-197.0)(-62.0,-199.0)}%
{\color{red}\dashline[200]{3}(-61.0,-197.0)(-60.0,-199.0)}%
{\color{green}\dashline[200]{3}(-59.3,-212.9)(-52.1,-205.7)}%
{\color{green}\dashline[200]{3}(-52.1,-205.7)(-54.2,-206.4)}%
{\color{green}\dashline[200]{3}(-52.1,-205.7)(-52.8,-207.8)}%
{\color{green}\dashline[200]{3}(-60.6,-216.9)(-69.4,-243.1)}%
{\color{green}\dashline[200]{3}(-69.4,-243.1)(-67.8,-241.5)}%
{\color{green}\dashline[200]{3}(-69.4,-243.1)(-69.7,-240.9)}%
{\color{red}\dashline[200]{3}(-10.0,-297.0)(-10.0,-363.0)}%
{\color{red}\dashline[200]{3}(-10.0,-363.0)(-9.0,-361.0)}%
{\color{red}\dashline[200]{3}(-10.0,-363.0)(-11.0,-361.0)}%
{\color{red}\dashline[200]{3}(-28.6,-316.4)(-21.4,-323.6)}%
{\color{red}\dashline[200]{3}(-21.4,-323.6)(-22.1,-321.5)}%
{\color{red}\dashline[200]{3}(-21.4,-323.6)(-23.5,-322.9)}%
{\color{red}\dashline[200]{3}(-29.1,-316.8)(-20.9,-333.2)}%
{\color{red}\dashline[200]{3}(-20.9,-333.2)(-20.9,-331.0)}%
{\color{red}\dashline[200]{3}(-20.9,-333.2)(-22.7,-331.9)}%
{\color{red}\dashline[200]{3}(-30.0,-317.0)(-30.0,-343.0)}%
{\color{red}\dashline[200]{3}(-30.0,-343.0)(-29.0,-341.0)}%
{\color{red}\dashline[200]{3}(-30.0,-343.0)(-31.0,-341.0)}%
{\color{red}\dashline[200]{3}(-20.0,-327.0)(-20.0,-333.0)}%
{\color{red}\dashline[200]{3}(-20.0,-333.0)(-19.0,-331.0)}%
{\color{red}\dashline[200]{3}(-20.0,-333.0)(-21.0,-331.0)}%
{\color{red}\dashline[200]{3}(-20.9,-326.8)(-29.1,-343.2)}%
{\color{red}\dashline[200]{3}(-29.1,-343.2)(-27.3,-341.9)}%
{\color{red}\dashline[200]{3}(-29.1,-343.2)(-29.1,-341.0)}%
{\color{red}\dashline[200]{3}(-10.0,-417.0)(-10.0,-503.0)}%
{\color{red}\dashline[200]{3}(-10.0,-503.0)(-9.0,-501.0)}%
{\color{red}\dashline[200]{3}(-10.0,-503.0)(-11.0,-501.0)}%
{\color{red}\dashline[200]{3}(-30.0,-447.0)(-30.0,-463.0)}%
{\color{red}\dashline[200]{3}(-30.0,-463.0)(-29.0,-461.0)}%
{\color{red}\dashline[200]{3}(-30.0,-463.0)(-31.0,-461.0)}%
{\color{red}\dashline[200]{3}(-20.7,-457.1)(-27.9,-464.3)}%
{\color{red}\dashline[200]{3}(-27.9,-464.3)(-25.8,-463.6)}%
{\color{red}\dashline[200]{3}(-27.9,-464.3)(-27.2,-462.2)}%
{\color{green}\dashline[200]{3}(-29.3,-462.9)(-22.1,-455.7)}%
{\color{green}\dashline[200]{3}(-22.1,-455.7)(-24.2,-456.4)}%
{\color{green}\dashline[200]{3}(-22.1,-455.7)(-22.8,-457.8)}%
{\color{red}\dashline[200]{3}(-30.0,-467.0)(-30.0,-483.0)}%
{\color{red}\dashline[200]{3}(-30.0,-483.0)(-29.0,-481.0)}%
{\color{red}\dashline[200]{3}(-30.0,-483.0)(-31.0,-481.0)}%
\end{picture}
\]
\hrulefill

\end{figure}

