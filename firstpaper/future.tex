% -*- eval: (local-set-key "\M-q" 'undefined) -*-
%
% The line above will probably make Emacs ask if it is ok to evaluate the
% expression.  Answer y.

\section{Future Work}

Embla is a prototype demonstrator, and not yet a sophisticated tool.  We
believe that there are many interesting research issues worht
investigating, which could lead to improvements in the usability.  First,
Embla needs a scalable technique for visualisating the results.  Latex
illustrations work for illustrating the profiling data collected, but does
not scale to large programs.  

The current Embla implementation only presents all independent subgraphs it
finds, and does not consider run-time performance at all.  Many of the
independent subgraphs are short, and have little potential for speedup,
however.  A more mature implementation could measure run-time performance
of arcs in the data dependency graph.  evaluate the potential performance
gain, and present the longest mutually independent subgraphs first.  In
this case, it would be reasonable to use a table-based presentation,
similar to traditional performance profilers.

The current implementation only presents mutually independent subgraphs
that begin and end in the same procedure call.  There is an opportunity for
finding more parallelism if Embla could also find and present independent
subgraphs that begin and join in different procedures.  There are also
cases where subgraphs are mutually independent, except for a few data
interdependencies.  If these interdependencies can be removed or stem from
data accesses that could be reordered without harm, as could be the case
e.g. with log file accesses or transactions, the procedure calls can be
parallelised anyway.  Hence, they could be presented by Embla, along with
information on the dependencies.  It would also be beneficial to improve
Embla in order to allow developers to specify data access patterns that
should be ignored for dependency calculation.

Similarly to a traditional profiler, Embla collects information from a
single execution on input data that is assumed to be representative.  When
the developer judges whether it would be reasonable to parallelise
procedure calls, however, he will want to consider whether there are other
data dependencies when executing the program with other input data sets.
Hence, it is desirable to make Embla aggregate data dependencies from
multiple executions, and only present the subgraphs that have few or no
interdependencies on all the executions.
