\section{Using Embla}

\begin{figure} 
\small
\verbdef\jkffxb$#include <stdlib.h> $
\verbdef\jkffxc$#include <stdio.h> $
\verbdef\jkffxd$ $
\verbdef\jkffxe$static void inc(int *p) $
\verbdef\jkffxf${ $
\verbdef\jkffxg$   *p=*p+1; $
\verbdef\jkffxh$} $
\verbdef\jkffxi$ $
\verbdef\jkffxj$int main(int argc, char **argv) $
\verbdef\jkffxba${ $
\verbdef\jkffxbb$   int *q=NULL,n=0; $
\verbdef\jkffxbc$ $
\verbdef\jkffxbd$   q = (int*) malloc( sizeof(int) ); $
\verbdef\jkffxbe$   inc(q); $
\verbdef\jkffxbf$   inc(&n); $
\verbdef\jkffxbg$   inc(q); $
\verbdef\jkffxbh$   printf( "%d\n", *q+n ); $
\verbdef\jkffxbi$   q = (int*) malloc( sizeof(int) ); $
\verbdef\jkffxbj$   return q==NULL; $
\verbdef\jkffxca$} $
\hrulefill
\[
\begin{picture}(420,200)(-65,-200)

\put(-65,-10){\makebox(15,10)[r]{\it 1:}}%
\put(0,-10){\makebox(100,10)[l]{\jkffxb}}
\put(-65,-20){\makebox(15,10)[r]{\it 2:}}%
\put(0,-20){\makebox(100,10)[l]{\jkffxc}}
\put(-65,-30){\makebox(15,10)[r]{\it 3:}}%
\put(0,-30){\makebox(100,10)[l]{\jkffxd}}
\put(-65,-40){\makebox(15,10)[r]{\it 4:}}%
\put(0,-40){\makebox(100,10)[l]{\jkffxe}}
\put(-65,-50){\makebox(15,10)[r]{\it 5:}}%
\put(0,-50){\makebox(100,10)[l]{\jkffxf}}
{\color{black} \dottedline{3}(-10,-45)(0,-45)}
\put(-65,-60){\makebox(15,10)[r]{\it 6:}}%
\put(0,-60){\makebox(100,10)[l]{\jkffxg}}
\put(-65,-70){\makebox(15,10)[r]{\it 7:}}%
\put(0,-70){\makebox(100,10)[l]{\jkffxh}}
{\color{black} \dottedline{3}(-10,-65)(0,-65)}
\put(-65,-80){\makebox(15,10)[r]{\it 8:}}%
\put(0,-80){\makebox(100,10)[l]{\jkffxi}}
\put(-65,-90){\makebox(15,10)[r]{\it 9:}}%
\put(0,-90){\makebox(100,10)[l]{\jkffxj}}
\put(-65,-100){\makebox(15,10)[r]{\it 10:}}%
\put(0,-100){\makebox(100,10)[l]{\jkffxba}}
{\color{black} \dottedline{3}(-10,-95)(0,-95)}
\put(-65,-110){\makebox(15,10)[r]{\it 11:}}%
\put(0,-110){\makebox(100,10)[l]{\jkffxbb}}
{\color{black} \dottedline{3}(-10,-105)(0,-105)}
\put(-65,-120){\makebox(15,10)[r]{\it 12:}}%
\put(0,-120){\makebox(100,10)[l]{\jkffxbc}}
\put(-65,-130){\makebox(15,10)[r]{\it 13:}}%
\put(0,-130){\makebox(100,10)[l]{\jkffxbd}}
{\color{black} \dottedline{3}(-10,-125)(0,-125)}
\put(-65,-140){\makebox(15,10)[r]{\it 14:}}%
\put(0,-140){\makebox(100,10)[l]{\jkffxbe}}
{\color{black} \dottedline{3}(-10,-135)(0,-135)}
\put(-65,-150){\makebox(15,10)[r]{\it 15:}}%
\put(0,-150){\makebox(100,10)[l]{\jkffxbf}}
{\color{black} \dottedline{3}(-10,-145)(0,-145)}
\put(-65,-160){\makebox(15,10)[r]{\it 16:}}%
\put(0,-160){\makebox(100,10)[l]{\jkffxbg}}
{\color{black} \dottedline{3}(-10,-155)(0,-155)}
\put(-65,-170){\makebox(15,10)[r]{\it 17:}}%
\put(0,-170){\makebox(100,10)[l]{\jkffxbh}}
{\color{black} \dottedline{3}(-10,-165)(0,-165)}
\put(-65,-180){\makebox(15,10)[r]{\it 18:}}%
\put(0,-180){\makebox(100,10)[l]{\jkffxbi}}
{\color{black} \dottedline{3}(-10,-175)(0,-175)}
\put(-65,-190){\makebox(15,10)[r]{\it 19:}}%
\put(0,-190){\makebox(100,10)[l]{\jkffxbj}}
{\color{black} \dottedline{3}(-10,-185)(0,-185)}
\put(-65,-200){\makebox(15,10)[r]{\it 20:}}%
\put(0,-200){\makebox(100,10)[l]{\jkffxca}}
{\color{black} \dottedline{3}(-10,-195)(0,-195)}

\put(-40,-165){\color{black}\circle*{2}}
\put(-10,-45){\color{black}\circle*{2}}
\put(-30,-175){\color{black}\circle*{2}}
\put(-30,-185){\color{black}\circle*{2}}
\put(-10,-65){\color{black}\circle*{2}}
\put(-10,-95){\color{black}\circle*{2}}
\put(-40,-105){\color{black}\circle*{2}}
\put(-10,-195){\color{black}\circle*{2}}
\put(-30,-125){\color{black}\circle*{2}}
\put(-20,-135){\color{black}\circle*{2}}
\put(-40,-145){\color{black}\circle*{2}}
\put(-20,-155){\color{black}\circle*{2}}
{\color{red}\dashline[200]{3}(-10.0,-48.0)(-10.0,-62.0)}%
{\color{red}\dashline[200]{3}(-10.0,-62.0)(-9.0,-59.0)}%
{\color{red}\dashline[200]{3}(-10.0,-62.0)(-11.0,-59.0)}%
{\color{red}\dashline[200]{3}(-10.0,-98.0)(-10.0,-192.0)}%
{\color{red}\dashline[200]{3}(-10.0,-192.0)(-9.0,-189.0)}%
{\color{red}\dashline[200]{3}(-10.0,-192.0)(-11.0,-189.0)}%
{\color{blue}\dashline[200]{3}(-38.7,-107.7)(-31.3,-122.3)}%
{\color{blue}\dashline[200]{3}(-31.3,-122.3)(-31.8,-119.2)}%
{\color{blue}\dashline[200]{3}(-31.3,-122.3)(-33.6,-120.1)}%
{\color{red}\dashline[200]{3}(-40.0,-108.0)(-40.0,-142.0)}%
{\color{red}\dashline[200]{3}(-40.0,-142.0)(-39.0,-139.0)}%
{\color{red}\dashline[200]{3}(-40.0,-142.0)(-41.0,-139.0)}%
{\color{red}\dashline[200]{3}(-27.9,-127.1)(-22.1,-132.9)}%
{\color{red}\dashline[200]{3}(-22.1,-132.9)(-23.5,-130.1)}%
{\color{red}\dashline[200]{3}(-22.1,-132.9)(-24.9,-131.5)}%
{\color{red}\dashline[200]{3}(-29.1,-127.8)(-20.9,-152.2)}%
{\color{red}\dashline[200]{3}(-20.9,-152.2)(-20.9,-149.0)}%
{\color{red}\dashline[200]{3}(-20.9,-152.2)(-22.8,-149.6)}%
{\color{red}\dashline[200]{3}(-30.7,-127.9)(-39.3,-162.1)}%
{\color{red}\dashline[200]{3}(-39.3,-162.1)(-37.6,-159.4)}%
{\color{red}\dashline[200]{3}(-39.3,-162.1)(-39.5,-158.9)}%
{\color{cyan}\dashline[200]{3}(-30.0,-128.0)(-30.0,-172.0)}%
{\color{cyan}\dashline[200]{3}(-30.0,-172.0)(-29.0,-169.0)}%
{\color{cyan}\dashline[200]{3}(-30.0,-172.0)(-31.0,-169.0)}%
{\color{red}\dashline[200]{3}(-20.0,-138.0)(-20.0,-152.0)}%
{\color{red}\dashline[200]{3}(-20.0,-152.0)(-19.0,-149.0)}%
{\color{red}\dashline[200]{3}(-20.0,-152.0)(-21.0,-149.0)}%
{\color{green}\dashline[200]{3}(-20.7,-137.9)(-29.3,-172.1)}%
{\color{green}\dashline[200]{3}(-29.3,-172.1)(-27.6,-169.4)}%
{\color{green}\dashline[200]{3}(-29.3,-172.1)(-29.5,-168.9)}%
{\color{red}\dashline[200]{3}(-40.0,-148.0)(-40.0,-162.0)}%
{\color{red}\dashline[200]{3}(-40.0,-162.0)(-39.0,-159.0)}%
{\color{red}\dashline[200]{3}(-40.0,-162.0)(-41.0,-159.0)}%
{\color{red}\dashline[200]{3}(-22.7,-156.3)(-37.3,-163.7)}%
{\color{red}\dashline[200]{3}(-37.3,-163.7)(-34.2,-163.2)}%
{\color{red}\dashline[200]{3}(-37.3,-163.7)(-35.1,-161.4)}%
{\color{green}\dashline[200]{3}(-21.3,-157.7)(-28.7,-172.3)}%
{\color{green}\dashline[200]{3}(-28.7,-172.3)(-26.4,-170.1)}%
{\color{green}\dashline[200]{3}(-28.7,-172.3)(-28.2,-169.2)}%
{\color{green}\dashline[200]{3}(-37.9,-167.1)(-32.1,-172.9)}%
{\color{green}\dashline[200]{3}(-32.1,-172.9)(-33.5,-170.1)}%
{\color{green}\dashline[200]{3}(-32.1,-172.9)(-34.9,-171.5)}%
{\color{red}\dashline[200]{3}(-30.0,-178.0)(-30.0,-182.0)}%
{\color{red}\dashline[200]{3}(-30.0,-182.0)(-29.0,-179.0)}%
{\color{red}\dashline[200]{3}(-30.0,-182.0)(-31.0,-179.0)}%
\end{picture}
\]
\hrulefill

\caption{Example program with dependency graph} \label{ffirstex}
\end{figure}

To get a feeling for what dependency profiling is and what Embla can do, 
let us turn to the admittedly contrieved program in Figure~\ref{ffirstex}. 
Here we see, from 
left to right, line numbers, colored data dependency arrows and source 
lines. 

The presence of a data dependency arrow between a pair of lines indicates
that Embla has found one or more data dependencies between them
(Embla does not find control dependencies). A data dependece is a pair
of references, not both reads, to overlapping memory
locations with no interveaning write. For instance, in the figure, 
there is a (red) arrow from line 13 to line 14 corresponding to
the assignment to {\tt q} followed by its use as an argument in 
{\tt inc(q)}.

Depending on whether the two references corresponding to a dependence
are reads or writes, data dependencies are typically divided into 
three classes:
\begin{description}
\item[Flow:]
A true data dependency where the location is first written then
read. Also known as {\em read after write} (RAW). Shown in 
{\color{red} red} in the figure.
\item[Anti:]
A dependence caused by reuse of a location that is first read and then
written. Also known as {\em write after read} (WAR). Shown in 
{\color{green} green} in the figure.
\item[Output:]
Similar to an anti dependence, but the second reference is also a
write. Also known as {\em write after write} (WAW). Shown in 
{\color{blue} blue} in the figure.
\end{description}
If there are multiple dependencies of different class between the 
same pair of lines, the color selection is prioritized in the
order the dependencies are presented above. That is, if there is
some flow dependence, the arrow is red. Otherwise if there is 
some antidependence, the arrow is green, otherwise blue. We will 
come to the light blue arrows shortly.


We show dependencies in different colors since there are program
transformations that in some cases can be used to eliminate 
dependencies. Since we are discussing an approach to tool-supported
hand parallelisation, program transformation to remove dependencies
is potentially useful so Embla is designed to support it.

Flow dependencies are unavoidable since they represent the data flow 
in the program, but anti and output dependencies can sometimes be 
removed by {\em renaming}. Consider the (green) arrows from line 
14, 16 and 17 to line 18 in the figure. These correspond to three
reads of {\tt q} followed by an assignment to {\tt q}. Here we 
are dealing with two different values which happen to be stored 
in the same variable. We could easily invent a new variable, e.g. 
{\tt r}, and replace {\tt q} by {\tt r} on line 18 and 19. This 
transformation would remove the dependencies from line 14, 16 and
17 to line 18. 

Similarly, an output dependence like the one from line 11 to line 13
(blue) can sometimes, and indeed in this case, be eliminated by 
removing the first write (the initialization of {\tt q} on line 11).
Renaming is of course also applicable to output dependencies.

For each of the dependence arrows in the figure that 
we have dicussed up to now, the endpoints have been part of the 
code for {\tt main}
itself. Embla also tracks references made in function calls. For
instance, there is a (red) flow dependence from line 14 to line 16
representing the write in the first call to {\tt inc} to the 
{\tt malloc}'d area pointed to by {\tt q} and the subsequent read 
of the same location by {\tt inc}. Note that these dependencies 
are reported as pertaining to {\tt main} rather than {\tt inc},
although the referencing instructions are part of the latter function. 
But the importance of the dependence is that, in {\tt main}, the calls
on line 14 and 16 can not be made in parallel.

Dependencies involving {\tt malloc} are especially interesting.
Each call to {\tt malloc} manipulates (updates) 
administrative data structures like the free list. Embla will
report these as dependencies, effectively serializing all calls to 
{\tt malloc}, and thus in addition all calls to functions calling
{\tt malloc}. In reality, these calls need not be serialized since
program semantics typically does not depend on the exact addresses 
that data are allocated. It suffices that a thread safe 
implementation of {\tt malloc} is used.

Embla maintains a black list of functions that behave in this way.
In the exampples in this paper, the black list consists of the 
{\tt malloc} family, but of course other functions can be included 
as appropriate. 

An important thing to remember when interpreting the output of Embla 
is that although it takes the form of annotated source code, the 
profiling is performed on the executable machine code. Thus one must
keep the mappning of source code to machine code in mind. A case in 
point is the (red) flow dependence between the opening brace of a 
function and its closing brace (for instance from line 5 to line 7).
This dependence corresponds to pushing the frame pointer onto the 
stack as part of the function prologue and popping it in the 
epilogue.

\begin{figure} 
\small
\verbdef\jkffxb$#include <stdlib.h> $
\verbdef\jkffxc$#include <stdio.h> $
\verbdef\jkffxd$ $
\verbdef\jkffxe$static void inc(int *p) $
\verbdef\jkffxf${ $
\verbdef\jkffxg$   *p=*p+1; $
\verbdef\jkffxh$} $
\verbdef\jkffxi$ $
\verbdef\jkffxj$int main(int argc, char **argv) $
\verbdef\jkffxba${ $
\verbdef\jkffxbb$   int *q=NULL,n=0; $
\verbdef\jkffxbc$ $
\verbdef\jkffxbd$   q = (int*) malloc( sizeof(int) ); $
\verbdef\jkffxbe$   inc(q); $
\verbdef\jkffxbf$   inc(&n); $
\verbdef\jkffxbg$   inc(q); $
\verbdef\jkffxbh$   printf( "%d\n", *q+n ); $
\verbdef\jkffxbi$   q = (int*) malloc( sizeof(int) ); $
\verbdef\jkffxbj$   return q==NULL; $
\verbdef\jkffxca$} $
\hrulefill
\[
\begin{picture}(420,200)(-86,-200)

\put(-86,0){\makebox(15,10)[r]{\it 0:}}%
\put(0,0){\makebox(100,10)[l]{}}
\put(-86,-10){\makebox(15,10)[r]{\it 1:}}%
\put(0,-10){\makebox(100,10)[l]{\jkffxb}}
\put(-86,-20){\makebox(15,10)[r]{\it 2:}}%
\put(0,-20){\makebox(100,10)[l]{\jkffxc}}
\put(-86,-30){\makebox(15,10)[r]{\it 3:}}%
\put(0,-30){\makebox(100,10)[l]{\jkffxd}}
\put(-86,-40){\makebox(15,10)[r]{\it 4:}}%
\put(0,-40){\makebox(100,10)[l]{\jkffxe}}
\put(-86,-50){\makebox(15,10)[r]{\it 5:}}%
\put(0,-50){\makebox(100,10)[l]{\jkffxf}}
{\color{black} \dottedline{3}(-63,-50)(-5,-50)}\put(-86,-60){\makebox(15,10)[r]{\it 6:}}%
\put(0,-60){\makebox(100,10)[l]{\jkffxg}}
{\color{black} \dottedline{3}(-63,-60)(-5,-60)}\put(-86,-70){\makebox(15,10)[r]{\it 7:}}%
\put(0,-70){\makebox(100,10)[l]{\jkffxh}}
\put(-86,-80){\makebox(15,10)[r]{\it 8:}}%
\put(0,-80){\makebox(100,10)[l]{\jkffxi}}
\put(-86,-90){\makebox(15,10)[r]{\it 9:}}%
\put(0,-90){\makebox(100,10)[l]{\jkffxj}}
\put(-86,-100){\makebox(15,10)[r]{\it 10:}}%
\put(0,-100){\makebox(100,10)[l]{\jkffxba}}
{\color{black} \dottedline{3}(-63,-100)(-5,-100)}\put(-86,-110){\makebox(15,10)[r]{\it 11:}}%
\put(0,-110){\makebox(100,10)[l]{\jkffxbb}}
{\color{black} \dottedline{3}(-63,-110)(-5,-110)}\put(-86,-120){\makebox(15,10)[r]{\it 12:}}%
\put(0,-120){\makebox(100,10)[l]{\jkffxbc}}
{\color{black} \dottedline{3}(-63,-120)(-5,-120)}\put(-86,-130){\makebox(15,10)[r]{\it 13:}}%
\put(0,-130){\makebox(100,10)[l]{\jkffxbd}}
{\color{black} \dottedline{3}(-63,-130)(-5,-130)}\put(-86,-140){\makebox(15,10)[r]{\it 14:}}%
\put(0,-140){\makebox(100,10)[l]{\jkffxbe}}
{\color{black} \dottedline{3}(-63,-140)(-5,-140)}\put(-86,-150){\makebox(15,10)[r]{\it 15:}}%
\put(0,-150){\makebox(100,10)[l]{\jkffxbf}}
{\color{black} \dottedline{3}(-63,-150)(-5,-150)}\put(-86,-160){\makebox(15,10)[r]{\it 16:}}%
\put(0,-160){\makebox(100,10)[l]{\jkffxbg}}
{\color{black} \dottedline{3}(-63,-160)(-5,-160)}\put(-86,-170){\makebox(15,10)[r]{\it 17:}}%
\put(0,-170){\makebox(100,10)[l]{\jkffxbh}}
{\color{black} \dottedline{3}(-63,-170)(-5,-170)}\put(-86,-180){\makebox(15,10)[r]{\it 18:}}%
\put(0,-180){\makebox(100,10)[l]{\jkffxbi}}
{\color{black} \dottedline{3}(-63,-180)(-5,-180)}\put(-86,-190){\makebox(15,10)[r]{\it 19:}}%
\put(0,-190){\makebox(100,10)[l]{\jkffxbj}}
{\color{black} \dottedline{3}(-63,-190)(-5,-190)}\put(-86,-200){\makebox(15,10)[r]{\it 20:}}%
\put(0,-200){\makebox(100,10)[l]{\jkffxca}}

\color{red}
\put(-6,-95){\circle*{2}}
\put(-6,-97){\vector(0,-1){96}}
\put(-8,-97){\linethickness{0.7pt}\line(1,0){0}}
\put(-6,-195){\circle*{2}}

\color{blue}
\put(-12,-105){\circle*{2}}
\put(-12,-107){\vector(0,-1){16}}
\put(-14,-107){\linethickness{0.7pt}\line(1,0){0}}
\put(-12,-125){\circle*{2}}

\color{red}
\put(-18,-105){\circle*{2}}
\put(-18,-107){\vector(0,-1){36}}
\put(-20,-107){\linethickness{0.7pt}\line(1,0){0}}
\put(-18,-145){\circle*{2}\hspace{-2\unitlength}\circle{4}}

\color{red}
\put(-24,-122){\circle*{2}}
\put(-26,-124){\linethickness{0.7pt}\line(1,0){0}}
\put(-24,-127){\circle{4}}

\color{red}
\put(-30,-125){\circle*{2}}
\put(-30,-127){\vector(0,-1){6}}
\put(-32,-127){\linethickness{0.7pt}\line(1,0){0}}
\put(-30,-135){\circle*{2}}

\color{red}
\put(-36,-125){\circle*{2}}
\put(-36,-127){\vector(0,-1){26}}
\put(-38,-127){\linethickness{0.7pt}\line(1,0){0}}
\put(-36,-155){\circle*{2}}

\color{red}
\put(-42,-125){\circle*{2}}
\put(-42,-127){\vector(0,-1){36}}
\put(-44,-127){\linethickness{0.7pt}\line(1,0){0}}
\put(-42,-165){\circle*{2}}

\color{red}
\put(-48,-125){\circle{4}}
\put(-48,-127){\vector(0,-1){46}}
\put(-50,-127){\linethickness{0.7pt}\line(1,0){4}}
\put(-48,-175){\circle{4}}

\color{red}
\put(-12,-132){\circle*{2}}
\put(-14,-134){\linethickness{0.7pt}\line(1,0){0}}
\put(-12,-137){\circle*{2}\hspace{-2\unitlength}\circle{4}}

\color{green}
\put(-24,-135){\circle*{2}}
\put(-24,-137){\vector(0,-1){36}}
\put(-26,-137){\linethickness{0.7pt}\line(1,0){0}}
\put(-24,-175){\circle*{2}}

\color{red}
\put(-54,-135){\circle*{2}\hspace{-2\unitlength}\circle{4}}
\put(-54,-137){\vector(0,-1){16}}
\put(-56,-137){\linethickness{0.7pt}\line(1,0){4}}
\put(-54,-155){\circle*{2}\hspace{-2\unitlength}\circle{4}}

\color{red}
\put(-12,-142){\circle*{2}}
\put(-14,-144){\linethickness{0.7pt}\line(1,0){0}}
\put(-12,-147){\circle*{2}\hspace{-2\unitlength}\circle{4}}

\color{red}
\put(-30,-145){\circle*{2}\hspace{-2\unitlength}\circle{4}}
\put(-30,-147){\vector(0,-1){16}}
\put(-32,-147){\linethickness{0.7pt}\line(1,0){0}}
\put(-30,-165){\circle*{2}}

\color{red}
\put(-12,-152){\circle*{2}}
\put(-14,-154){\linethickness{0.7pt}\line(1,0){0}}
\put(-12,-157){\circle*{2}\hspace{-2\unitlength}\circle{4}}

\color{green}
\put(-18,-155){\circle*{2}}
\put(-18,-157){\vector(0,-1){16}}
\put(-20,-157){\linethickness{0.7pt}\line(1,0){0}}
\put(-18,-175){\circle*{2}}

\color{red}
\put(-60,-155){\circle*{2}\hspace{-2\unitlength}\circle{4}}
\put(-60,-157){\vector(0,-1){6}}
\put(-62,-157){\linethickness{0.7pt}\line(1,0){4}}
\put(-60,-165){\circle*{2}}

\color{red}
\put(-12,-162){\circle*{2}}
\put(-14,-164){\linethickness{0.7pt}\line(1,0){0}}
\put(-12,-167){\circle*{2}\hspace{-2\unitlength}\circle{4}}

\color{green}
\put(-36,-165){\circle*{2}}
\put(-36,-167){\vector(0,-1){6}}
\put(-38,-167){\linethickness{0.7pt}\line(1,0){0}}
\put(-36,-175){\circle*{2}}

\color{red}
\put(-12,-172){\circle*{2}}
\put(-14,-174){\linethickness{0.7pt}\line(1,0){0}}
\put(-12,-177){\circle{4}}

\color{red}
\put(-30,-175){\circle*{2}}
\put(-30,-177){\vector(0,-1){6}}
\put(-32,-177){\linethickness{0.7pt}\line(1,0){0}}
\put(-30,-185){\circle*{2}}

\color{red}
\put(-6,-45){\circle*{2}}
\put(-6,-47){\vector(0,-1){16}}
\put(-8,-47){\linethickness{0.7pt}\line(1,0){0}}
\put(-6,-65){\circle*{2}}

\color{green}
\put(-12,-52){\circle*{2}}
\put(-14,-54){\linethickness{0.7pt}\line(1,0){4}}
\put(-12,-57){\circle*{2}}

\color{green}
\put(-18,-52){\circle*{2}}
\put(-20,-54){\linethickness{0.7pt}\line(1,0){0}}
\put(-18,-57){\circle*{2}}

\end{picture}
\]
\hrulefill

\end{figure}

