\begin{figure} \small
\hrulefill
\verbdef\one^bastockmos^
\verbdef\two^       blandsaft^
\verbdef\thr^       och hoc^
\[
\begin{picture}(200,300)

\put(20,50){\makebox(100,10)[l]{\one}}
\put(20,40){\makebox(100,10)[l]{\two}}
\put(20,30){\makebox(100,10)[l]{\thr}}
\color{red}
\put(10,53){\vector(0,-1){6}}
\put(10,55){\circle{4}}
\put(10,45){\circle*{4}}
\color{blue}
\put(0,53){\vector(0,-1){16}}
\put(0,55){\circle{4}}
\put(0,35){\circle*{4}}

\end{picture}
\]
\hrulefill
\end{figure}



\begin{figure}
\small
\hrulefill
\begin{verbatim}
                                | #include <stdio.h>
                                | #include <stdlib.h>
                                | 
                                | #define N 5
                                | 
                                | typedef struct _ilist {
                                |    int val;
                                |    struct _ilist *next;
                                | } ilist;
                                | 
                                | static ilist* mklist(int n)
                                | {
                                |   int i;
                                |   ilist *p = NULL;
                                | 
                                |   for( i=0; i<n; i++ ) {
                                |     ilist *t = (ilist *) malloc( sizeof(ilist) );
                                |     t->val = i;
                                |     t->next = p;
                                |     p = t;
                                |   }
                                |   return p;
                                | }
                                | 
                                | static int sumlist(ilist* p)
                                | {
                                |    ilist *q;
                                |    int  sum = 0;
                                | 
                                |    for( q=p; q!=NULL; q=q->next ) {
                                |       sum += q->val;
                                |    }
                                |    return sum;
                              F | }
                              . | 
                              s | 
                              . | int main(int argc, char **argv) 
  A   F F F                   V | {
  c   h . c A   F   F         | |   int m,n;
  s   o s o c   . A c   F     | |   ilist *p,*q;
  .   h . c s   s c o   . A   | | 
= V   V V V c   . s c   s c   | |   p = mklist( N );
F > = > | | V   V . V   . s   | |   q = mklist( N );
.   F   > > > = | V |   V .   | |   m = sumlist( p );
s   .         F > > > = | V   | |   n = sumlist( q );
c   s         .       F | |   | | 
    c         s       . > > = | |   printf("%d\n", m);
              c       s     F | | 
                      c     . | |   return 0;
                            s | | 
                            c > | }
\end{verbatim}


\hrulefill
\caption{The list program, annotated with the dependencies in {\tt main}.}
\label{flist}
\end{figure}


---------------------------------------

\begin{figure}
\small
\hrulefill
\begin{alltt}
#include <stdlib.h>
#include <stdio.h>

static void inc(int *p)
\verb+{+
   *p=*p+1;
\verb+}+

int main(int argc, char **argv)
\verb+{+
   int *q=NULL,n=0;

   {\color{red}spawn} inc(&n);
   q = (int*) malloc( sizeof(int) );
   inc(q);
   inc(q);
   {\color{red}synch;}
   printf( "%d\verb+\+n", *q+n );
   q = (int*) malloc( sizeof(int) );
   return q==NULL;
\verb+}+
\end{alltt}
\hrulefill
\caption{The first example parallelized.}
\label{fparfirstex}
\end{figure}

What, then, does the graph in Figure~\ref{ffirstex} tell us about
the available parallelism in the program? It tells us that 
the call to {\tt inc} on line 15 is independent on lines 13, 14
and 16. Consequently, it can be parallelized, and we show the
result in Figure~\ref{fparfirstex} where we have again used the 
Cilk primitives {\tt spawn} and {\tt synch}.

\begin{figure} 
\small
\verbdef\jkffxb$#include <stdlib.h> $
\verbdef\jkffxc$#include <stdio.h> $
\verbdef\jkffxd$ $
\verbdef\jkffxe$static void inc(int *p) $
\verbdef\jkffxf${ $
\verbdef\jkffxg$   *p=*p+1; $
\verbdef\jkffxh$} $
\verbdef\jkffxi$ $
\verbdef\jkffxj$int main(int argc, char **argv) $
\verbdef\jkffxba${ $
\verbdef\jkffxbb$   int *q=NULL,n=0; $
\verbdef\jkffxbc$ $
\verbdef\jkffxbd$   q = (int*) malloc( sizeof(int) ); $
\verbdef\jkffxbe$   inc(q); $
\verbdef\jkffxbf$   inc(&n); $
\verbdef\jkffxbg$   inc(q); $
\verbdef\jkffxbh$   printf( "%d\n", *q+n ); $
\verbdef\jkffxbi$   q = (int*) malloc( sizeof(int) ); $
\verbdef\jkffxbj$   return q==NULL; $
\verbdef\jkffxca$} $
\hrulefill
\[
\begin{picture}(420,200)(-86,-200)

\put(-86,0){\makebox(15,10)[r]{\it 0:}}%
\put(0,0){\makebox(100,10)[l]{}}
\put(-86,-10){\makebox(15,10)[r]{\it 1:}}%
\put(0,-10){\makebox(100,10)[l]{\jkffxb}}
\put(-86,-20){\makebox(15,10)[r]{\it 2:}}%
\put(0,-20){\makebox(100,10)[l]{\jkffxc}}
\put(-86,-30){\makebox(15,10)[r]{\it 3:}}%
\put(0,-30){\makebox(100,10)[l]{\jkffxd}}
\put(-86,-40){\makebox(15,10)[r]{\it 4:}}%
\put(0,-40){\makebox(100,10)[l]{\jkffxe}}
\put(-86,-50){\makebox(15,10)[r]{\it 5:}}%
\put(0,-50){\makebox(100,10)[l]{\jkffxf}}
{\color{black} \dottedline{3}(-63,-50)(-5,-50)}\put(-86,-60){\makebox(15,10)[r]{\it 6:}}%
\put(0,-60){\makebox(100,10)[l]{\jkffxg}}
{\color{black} \dottedline{3}(-63,-60)(-5,-60)}\put(-86,-70){\makebox(15,10)[r]{\it 7:}}%
\put(0,-70){\makebox(100,10)[l]{\jkffxh}}
\put(-86,-80){\makebox(15,10)[r]{\it 8:}}%
\put(0,-80){\makebox(100,10)[l]{\jkffxi}}
\put(-86,-90){\makebox(15,10)[r]{\it 9:}}%
\put(0,-90){\makebox(100,10)[l]{\jkffxj}}
\put(-86,-100){\makebox(15,10)[r]{\it 10:}}%
\put(0,-100){\makebox(100,10)[l]{\jkffxba}}
{\color{black} \dottedline{3}(-63,-100)(-5,-100)}\put(-86,-110){\makebox(15,10)[r]{\it 11:}}%
\put(0,-110){\makebox(100,10)[l]{\jkffxbb}}
{\color{black} \dottedline{3}(-63,-110)(-5,-110)}\put(-86,-120){\makebox(15,10)[r]{\it 12:}}%
\put(0,-120){\makebox(100,10)[l]{\jkffxbc}}
{\color{black} \dottedline{3}(-63,-120)(-5,-120)}\put(-86,-130){\makebox(15,10)[r]{\it 13:}}%
\put(0,-130){\makebox(100,10)[l]{\jkffxbd}}
{\color{black} \dottedline{3}(-63,-130)(-5,-130)}\put(-86,-140){\makebox(15,10)[r]{\it 14:}}%
\put(0,-140){\makebox(100,10)[l]{\jkffxbe}}
{\color{black} \dottedline{3}(-63,-140)(-5,-140)}\put(-86,-150){\makebox(15,10)[r]{\it 15:}}%
\put(0,-150){\makebox(100,10)[l]{\jkffxbf}}
{\color{black} \dottedline{3}(-63,-150)(-5,-150)}\put(-86,-160){\makebox(15,10)[r]{\it 16:}}%
\put(0,-160){\makebox(100,10)[l]{\jkffxbg}}
{\color{black} \dottedline{3}(-63,-160)(-5,-160)}\put(-86,-170){\makebox(15,10)[r]{\it 17:}}%
\put(0,-170){\makebox(100,10)[l]{\jkffxbh}}
{\color{black} \dottedline{3}(-63,-170)(-5,-170)}\put(-86,-180){\makebox(15,10)[r]{\it 18:}}%
\put(0,-180){\makebox(100,10)[l]{\jkffxbi}}
{\color{black} \dottedline{3}(-63,-180)(-5,-180)}\put(-86,-190){\makebox(15,10)[r]{\it 19:}}%
\put(0,-190){\makebox(100,10)[l]{\jkffxbj}}
{\color{black} \dottedline{3}(-63,-190)(-5,-190)}\put(-86,-200){\makebox(15,10)[r]{\it 20:}}%
\put(0,-200){\makebox(100,10)[l]{\jkffxca}}

\color{red}
\put(-6,-95){\circle*{2}}
\put(-6,-97){\vector(0,-1){96}}
\put(-8,-97){\linethickness{0.7pt}\line(1,0){0}}
\put(-6,-195){\circle*{2}}

\color{blue}
\put(-12,-105){\circle*{2}}
\put(-12,-107){\vector(0,-1){16}}
\put(-14,-107){\linethickness{0.7pt}\line(1,0){0}}
\put(-12,-125){\circle*{2}}

\color{red}
\put(-18,-105){\circle*{2}}
\put(-18,-107){\vector(0,-1){36}}
\put(-20,-107){\linethickness{0.7pt}\line(1,0){0}}
\put(-18,-145){\circle*{2}\hspace{-2\unitlength}\circle{4}}

\color{red}
\put(-24,-122){\circle*{2}}
\put(-26,-124){\linethickness{0.7pt}\line(1,0){0}}
\put(-24,-127){\circle{4}}

\color{red}
\put(-30,-125){\circle*{2}}
\put(-30,-127){\vector(0,-1){6}}
\put(-32,-127){\linethickness{0.7pt}\line(1,0){0}}
\put(-30,-135){\circle*{2}}

\color{red}
\put(-36,-125){\circle*{2}}
\put(-36,-127){\vector(0,-1){26}}
\put(-38,-127){\linethickness{0.7pt}\line(1,0){0}}
\put(-36,-155){\circle*{2}}

\color{red}
\put(-42,-125){\circle*{2}}
\put(-42,-127){\vector(0,-1){36}}
\put(-44,-127){\linethickness{0.7pt}\line(1,0){0}}
\put(-42,-165){\circle*{2}}

\color{red}
\put(-48,-125){\circle{4}}
\put(-48,-127){\vector(0,-1){46}}
\put(-50,-127){\linethickness{0.7pt}\line(1,0){4}}
\put(-48,-175){\circle{4}}

\color{red}
\put(-12,-132){\circle*{2}}
\put(-14,-134){\linethickness{0.7pt}\line(1,0){0}}
\put(-12,-137){\circle*{2}\hspace{-2\unitlength}\circle{4}}

\color{green}
\put(-24,-135){\circle*{2}}
\put(-24,-137){\vector(0,-1){36}}
\put(-26,-137){\linethickness{0.7pt}\line(1,0){0}}
\put(-24,-175){\circle*{2}}

\color{red}
\put(-54,-135){\circle*{2}\hspace{-2\unitlength}\circle{4}}
\put(-54,-137){\vector(0,-1){16}}
\put(-56,-137){\linethickness{0.7pt}\line(1,0){4}}
\put(-54,-155){\circle*{2}\hspace{-2\unitlength}\circle{4}}

\color{red}
\put(-12,-142){\circle*{2}}
\put(-14,-144){\linethickness{0.7pt}\line(1,0){0}}
\put(-12,-147){\circle*{2}\hspace{-2\unitlength}\circle{4}}

\color{red}
\put(-30,-145){\circle*{2}\hspace{-2\unitlength}\circle{4}}
\put(-30,-147){\vector(0,-1){16}}
\put(-32,-147){\linethickness{0.7pt}\line(1,0){0}}
\put(-30,-165){\circle*{2}}

\color{red}
\put(-12,-152){\circle*{2}}
\put(-14,-154){\linethickness{0.7pt}\line(1,0){0}}
\put(-12,-157){\circle*{2}\hspace{-2\unitlength}\circle{4}}

\color{green}
\put(-18,-155){\circle*{2}}
\put(-18,-157){\vector(0,-1){16}}
\put(-20,-157){\linethickness{0.7pt}\line(1,0){0}}
\put(-18,-175){\circle*{2}}

\color{red}
\put(-60,-155){\circle*{2}\hspace{-2\unitlength}\circle{4}}
\put(-60,-157){\vector(0,-1){6}}
\put(-62,-157){\linethickness{0.7pt}\line(1,0){4}}
\put(-60,-165){\circle*{2}}

\color{red}
\put(-12,-162){\circle*{2}}
\put(-14,-164){\linethickness{0.7pt}\line(1,0){0}}
\put(-12,-167){\circle*{2}\hspace{-2\unitlength}\circle{4}}

\color{green}
\put(-36,-165){\circle*{2}}
\put(-36,-167){\vector(0,-1){6}}
\put(-38,-167){\linethickness{0.7pt}\line(1,0){0}}
\put(-36,-175){\circle*{2}}

\color{red}
\put(-12,-172){\circle*{2}}
\put(-14,-174){\linethickness{0.7pt}\line(1,0){0}}
\put(-12,-177){\circle{4}}

\color{red}
\put(-30,-175){\circle*{2}}
\put(-30,-177){\vector(0,-1){6}}
\put(-32,-177){\linethickness{0.7pt}\line(1,0){0}}
\put(-30,-185){\circle*{2}}

\color{red}
\put(-6,-45){\circle*{2}}
\put(-6,-47){\vector(0,-1){16}}
\put(-8,-47){\linethickness{0.7pt}\line(1,0){0}}
\put(-6,-65){\circle*{2}}

\color{green}
\put(-12,-52){\circle*{2}}
\put(-14,-54){\linethickness{0.7pt}\line(1,0){4}}
\put(-12,-57){\circle*{2}}

\color{green}
\put(-18,-52){\circle*{2}}
\put(-20,-54){\linethickness{0.7pt}\line(1,0){0}}
\put(-18,-57){\circle*{2}}

\end{picture}
\]
\hrulefill

\caption{First example program with dependency list.}
\label{ffirstexlist}
\end{figure}

\subsection{Digging deeper: the dependency list}

The dependency graph we have discussed above is an abstracted form of
the information provided by Embla. If the region of interest lacks
dependencies we are done; in order to motivate the transformation
shown in Figure~\ref{fparfirstex}, we only need the dependency
graph. But if Embla discovers more dependencies than we had hoped, we
might wish to see if we can transform the program to eliminate them.
For that task, it is useful to have a more detailed look.

Hence we now turn to Figure~\ref{ffirstexlist} where we see a
graphical representation of the individual dependencies found by
Embla. There are three major dimensions in which this plot gives more
details:

\paragraph*{All dependencies are shown} 

Rather than giving a single arrow if there is more than one kind of
dependence, every kind is shown individually. If there is both a flow
and an anti dependence between a pair of lines, there are two arrows
(one red and one green). There is however no example of this in the
figure.

Dependency lists also show {\em self dependencies}, where the source
and target of the dependence is on the same line. in that case, no
arrow is shown, only the two endpoint symbols. 

\paragraph*{Characterization of dependence endpoints}

% Each dependence corresponds to a pair of memory referencing 
% instruction executions, referred to as the {\em dependence 
% endpoints}. 
The dependence endpoints are not always part of the function
containing the dependence but may be part of a function called
(indirectly) from the line indicated by the dependency arrow. For instance, the
endpoints of the flow dependence from line 14 to line 16 in the
example are part of the function {\tt inc}. This observation gives
rise to the following three kinds of endpoints:
\begin{description}
\item[$\cdot\ $] 
A small dot represents a {\em direct dependency endpoint}, an 
instruction that is
generated from the source code at the indicated line. 
% Thus the 
% instruction is part of the function or procedure containing the 
% line.
\item[$\circ\hskip-0.35em\cdot\ $]
A dot in a circle represents an {\em indirect dependency endpoint},
an instruction that is part of a function (procedure) that was 
(transitively) invoked by a function (procedure) call at the indicated 
line.
\item[$\circ\ $]
An unfilled circle represents a {\em weak dependecy endpoint},
which is like an indirect endpoint but the function containing 
the instruction is on a blacklist of functions which, 
like {\tt malloc}, would not give rise to dependencies in the 
parallelized program.
\end{description}
A dependency can have endpoints of different kind. For instance,
function calls give rise to flow dependencies between the write of the
argument and its subsequent read in the called function (seen in the
second column from the right in Figure~\ref{ffirstexlist} for the
calls on lines 14 through 18.

\paragraph*{Data location}

Another dimension that the dependency list shows is where in memory
the location associated with the dependency is situated. Currently, we
distinguish references to the stack from other references but a more
fine-grained characterization is certainly possible.
\begin{description}
\item[$\downarrow\ $]
An unadorned arrow represents a
 {\em stack dependence} where the location has been a part of 
the stack from the first reference to the second (inclusive).
\item[$\bar{\downarrow}\ $]
An arrow with a bar across represents a
{\em heap dependence} where the location has never been part of the stack.
\end{description}
Dependencies where the data location has been part of the stack, then
not part of the stack, then part of the stack again are filtered away


------------------------------------------

\begin{figure} 
\small
\verbdef\jkffxb$#include <stdlib.h> $
\verbdef\jkffxc$#include <stdio.h> $
\verbdef\jkffxd$ $
\verbdef\jkffxe$static int a[]  $
\verbdef\jkffxf$  = {17, 3, 84, 89, 4, 5, 23, 43,  $
\verbdef\jkffxg$     21, 7, 2, 1, 55, 63, 21}; $
\verbdef\jkffxh$static int n = 15; $
\verbdef\jkffxi$ $
\verbdef\jkffxj$static int *part(int *a, int n) $
\verbdef\jkffxba${ $
\verbdef\jkffxbb$   int i = a[0]; $
\verbdef\jkffxbc$   int k = a[n-1]; $
\verbdef\jkffxbd$   int *lp = a; $
\verbdef\jkffxbe$   int *hp = a+n-1; $
\verbdef\jkffxbf$ $
\verbdef\jkffxbg$   while( lp<hp ) { $
\verbdef\jkffxbh$      if( k<i ) { $
\verbdef\jkffxbi$         *lp = k; $
\verbdef\jkffxbj$          lp++; $
\verbdef\jkffxca$          k = *lp; $
\verbdef\jkffxcb$      } else { $
\verbdef\jkffxcc$          *hp = k; $
\verbdef\jkffxcd$           hp--; $
\verbdef\jkffxce$           k = *hp; $
\verbdef\jkffxcf$      } $
\verbdef\jkffxcg$   } $
\verbdef\jkffxch$   *lp = i; $
\verbdef\jkffxci$   return lp; $
\verbdef\jkffxcj$} $
\verbdef\jkffxda$ $
\verbdef\jkffxdb$static void qs(int *a, int n) $
\verbdef\jkffxdc${ $
\verbdef\jkffxdd$   if( n>1 ) { $
\verbdef\jkffxde$      int *lp = part( a, n ); $
\verbdef\jkffxdf$      int m = lp-a; $
\verbdef\jkffxdg$      qs( a, m ); $
\verbdef\jkffxdh$      qs( lp+1, n-m-1 ); $
\verbdef\jkffxdi$   } $
\verbdef\jkffxdj$} $
\verbdef\jkffxea$ $
\verbdef\jkffxeb$  $
\verbdef\jkffxec$ $
\verbdef\jkffxed$int main(int argc, char **argv) $
\verbdef\jkffxee${ $
\verbdef\jkffxef$   int i; $
\verbdef\jkffxeg$    $
\verbdef\jkffxeh$   qs( a, n ); $
\verbdef\jkffxei$   for( i=0; i<n; i++ ) { $
\verbdef\jkffxej$      printf( "%d ", a[i] ); $
\verbdef\jkffxfa$   } $
\verbdef\jkffxfb$   printf( "\n" ); $
\verbdef\jkffxfc$ $
\verbdef\jkffxfd$} $
\hrulefill
\[
\begin{picture}(420,530)(-192,-530)

\put(0,0){\makebox(100,10)[l]{}}
\put(0,-10){\makebox(100,10)[l]{\jkffxb}}
\put(0,-20){\makebox(100,10)[l]{\jkffxc}}
\put(0,-30){\makebox(100,10)[l]{\jkffxd}}
\put(0,-40){\makebox(100,10)[l]{\jkffxe}}
\put(0,-50){\makebox(100,10)[l]{\jkffxf}}
\put(0,-60){\makebox(100,10)[l]{\jkffxg}}
\put(0,-70){\makebox(100,10)[l]{\jkffxh}}
\put(0,-80){\makebox(100,10)[l]{\jkffxi}}
{\color{black} \dottedline{3}(-189,-80)(-5,-80)}\put(0,-90){\makebox(100,10)[l]{\jkffxj}}
{\color{black} \dottedline{3}(-189,-90)(-5,-90)}\put(0,-100){\makebox(100,10)[l]{\jkffxba}}
{\color{black} \dottedline{3}(-189,-100)(-5,-100)}\put(0,-110){\makebox(100,10)[l]{\jkffxbb}}
{\color{black} \dottedline{3}(-189,-110)(-5,-110)}\put(0,-120){\makebox(100,10)[l]{\jkffxbc}}
{\color{black} \dottedline{3}(-189,-120)(-5,-120)}\put(0,-130){\makebox(100,10)[l]{\jkffxbd}}
{\color{black} \dottedline{3}(-189,-130)(-5,-130)}\put(0,-140){\makebox(100,10)[l]{\jkffxbe}}
{\color{black} \dottedline{3}(-189,-140)(-5,-140)}\put(0,-150){\makebox(100,10)[l]{\jkffxbf}}
{\color{black} \dottedline{3}(-189,-150)(-5,-150)}\put(0,-160){\makebox(100,10)[l]{\jkffxbg}}
{\color{black} \dottedline{3}(-189,-160)(-5,-160)}\put(0,-170){\makebox(100,10)[l]{\jkffxbh}}
{\color{black} \dottedline{3}(-189,-170)(-5,-170)}\put(0,-180){\makebox(100,10)[l]{\jkffxbi}}
{\color{black} \dottedline{3}(-189,-180)(-5,-180)}\put(0,-190){\makebox(100,10)[l]{\jkffxbj}}
{\color{black} \dottedline{3}(-189,-190)(-5,-190)}\put(0,-200){\makebox(100,10)[l]{\jkffxca}}
{\color{black} \dottedline{3}(-189,-200)(-5,-200)}\put(0,-210){\makebox(100,10)[l]{\jkffxcb}}
{\color{black} \dottedline{3}(-189,-210)(-5,-210)}\put(0,-220){\makebox(100,10)[l]{\jkffxcc}}
{\color{black} \dottedline{3}(-189,-220)(-5,-220)}\put(0,-230){\makebox(100,10)[l]{\jkffxcd}}
{\color{black} \dottedline{3}(-189,-230)(-5,-230)}\put(0,-240){\makebox(100,10)[l]{\jkffxce}}
{\color{black} \dottedline{3}(-189,-240)(-5,-240)}\put(0,-250){\makebox(100,10)[l]{\jkffxcf}}
{\color{black} \dottedline{3}(-189,-250)(-5,-250)}\put(0,-260){\makebox(100,10)[l]{\jkffxcg}}
{\color{black} \dottedline{3}(-189,-260)(-5,-260)}\put(0,-270){\makebox(100,10)[l]{\jkffxch}}
\put(0,-280){\makebox(100,10)[l]{\jkffxci}}
\put(0,-290){\makebox(100,10)[l]{\jkffxcj}}
\put(0,-300){\makebox(100,10)[l]{\jkffxda}}
{\color{black} \dottedline{3}(-189,-300)(-5,-300)}\put(0,-310){\makebox(100,10)[l]{\jkffxdb}}
{\color{black} \dottedline{3}(-189,-310)(-5,-310)}\put(0,-320){\makebox(100,10)[l]{\jkffxdc}}
{\color{black} \dottedline{3}(-189,-320)(-5,-320)}\put(0,-330){\makebox(100,10)[l]{\jkffxdd}}
{\color{black} \dottedline{3}(-189,-330)(-5,-330)}\put(0,-340){\makebox(100,10)[l]{\jkffxde}}
{\color{black} \dottedline{3}(-189,-340)(-5,-340)}\put(0,-350){\makebox(100,10)[l]{\jkffxdf}}
{\color{black} \dottedline{3}(-189,-350)(-5,-350)}\put(0,-360){\makebox(100,10)[l]{\jkffxdg}}
{\color{black} \dottedline{3}(-189,-360)(-5,-360)}\put(0,-370){\makebox(100,10)[l]{\jkffxdh}}
\put(0,-380){\makebox(100,10)[l]{\jkffxdi}}
\put(0,-390){\makebox(100,10)[l]{\jkffxdj}}
\put(0,-400){\makebox(100,10)[l]{\jkffxea}}
\put(0,-410){\makebox(100,10)[l]{\jkffxeb}}
\put(0,-420){\makebox(100,10)[l]{\jkffxec}}
{\color{black} \dottedline{3}(-189,-420)(-5,-420)}\put(0,-430){\makebox(100,10)[l]{\jkffxed}}
{\color{black} \dottedline{3}(-189,-430)(-5,-430)}\put(0,-440){\makebox(100,10)[l]{\jkffxee}}
{\color{black} \dottedline{3}(-189,-440)(-5,-440)}\put(0,-450){\makebox(100,10)[l]{\jkffxef}}
{\color{black} \dottedline{3}(-189,-450)(-5,-450)}\put(0,-460){\makebox(100,10)[l]{\jkffxeg}}
{\color{black} \dottedline{3}(-189,-460)(-5,-460)}\put(0,-470){\makebox(100,10)[l]{\jkffxeh}}
{\color{black} \dottedline{3}(-189,-470)(-5,-470)}\put(0,-480){\makebox(100,10)[l]{\jkffxei}}
{\color{black} \dottedline{3}(-189,-480)(-5,-480)}\put(0,-490){\makebox(100,10)[l]{\jkffxej}}
{\color{black} \dottedline{3}(-189,-490)(-5,-490)}\put(0,-500){\makebox(100,10)[l]{\jkffxfa}}
{\color{black} \dottedline{3}(-189,-500)(-5,-500)}\put(0,-510){\makebox(100,10)[l]{\jkffxfb}}
\put(0,-520){\makebox(100,10)[l]{\jkffxfc}}
\put(0,-530){\makebox(100,10)[l]{\jkffxfd}}

\color{red}
\put(-6,-295){\circle*{2}}
\put(-6,-297){\vector(0,-1){66}}
\put(-8,-297){\linethickness{0.7pt}\line(1,0){0}}
\put(-6,-365){\circle*{2}}

\color{red}
\put(-12,-312){\circle*{2}}
\put(-14,-314){\linethickness{0.7pt}\line(1,0){0}}
\put(-12,-317){\circle*{2}\hspace{-2\unitlength}\circle{4}}

\color{red}
\put(-18,-315){\circle*{2}}
\put(-18,-317){\vector(0,-1){6}}
\put(-20,-317){\linethickness{0.7pt}\line(1,0){0}}
\put(-18,-325){\circle*{2}}

\color{red}
\put(-24,-315){\circle*{2}}
\put(-24,-317){\vector(0,-1){26}}
\put(-26,-317){\linethickness{0.7pt}\line(1,0){0}}
\put(-24,-345){\circle*{2}}

\color{red}
\put(-30,-315){\circle*{2}\hspace{-2\unitlength}\circle{4}}
\put(-30,-317){\vector(0,-1){16}}
\put(-32,-317){\linethickness{0.7pt}\line(1,0){4}}
\put(-30,-335){\circle*{2}\hspace{-2\unitlength}\circle{4}}

\color{red}
\put(-36,-315){\circle*{2}\hspace{-2\unitlength}\circle{4}}
\put(-36,-317){\vector(0,-1){26}}
\put(-38,-317){\linethickness{0.7pt}\line(1,0){4}}
\put(-36,-345){\circle*{2}\hspace{-2\unitlength}\circle{4}}

\color{red}
\put(-12,-325){\circle*{2}}
\put(-12,-327){\vector(0,-1){6}}
\put(-14,-327){\linethickness{0.7pt}\line(1,0){0}}
\put(-12,-335){\circle*{2}}

\color{red}
\put(-42,-325){\circle*{2}}
\put(-42,-327){\vector(0,-1){16}}
\put(-44,-327){\linethickness{0.7pt}\line(1,0){0}}
\put(-42,-345){\circle*{2}}

\color{red}
\put(-18,-332){\circle*{2}}
\put(-20,-334){\linethickness{0.7pt}\line(1,0){0}}
\put(-18,-337){\circle*{2}\hspace{-2\unitlength}\circle{4}}

\color{red}
\put(-12,-342){\circle*{2}}
\put(-14,-344){\linethickness{0.7pt}\line(1,0){0}}
\put(-12,-347){\circle*{2}\hspace{-2\unitlength}\circle{4}}

\color{red}
\put(-6,-75){\circle*{2}}
\put(-6,-77){\vector(0,-1){186}}
\put(-8,-77){\linethickness{0.7pt}\line(1,0){0}}
\put(-6,-265){\circle*{2}}

\color{red}
\put(-12,-85){\circle*{2}}
\put(-12,-87){\vector(0,-1){56}}
\put(-14,-87){\linethickness{0.7pt}\line(1,0){0}}
\put(-12,-145){\circle*{2}}

\color{green}
\put(-18,-85){\circle*{2}}
\put(-18,-87){\vector(0,-1){66}}
\put(-20,-87){\linethickness{0.7pt}\line(1,0){4}}
\put(-18,-155){\circle*{2}}

\color{green}
\put(-24,-85){\circle*{2}}
\put(-24,-87){\vector(0,-1){156}}
\put(-26,-87){\linethickness{0.7pt}\line(1,0){4}}
\put(-24,-245){\circle*{2}}

\color{red}
\put(-30,-85){\circle*{2}}
\put(-30,-87){\vector(0,-1){156}}
\put(-32,-87){\linethickness{0.7pt}\line(1,0){0}}
\put(-30,-245){\circle*{2}}

\color{red}
\put(-36,-95){\circle*{2}}
\put(-36,-97){\vector(0,-1){46}}
\put(-38,-97){\linethickness{0.7pt}\line(1,0){0}}
\put(-36,-145){\circle*{2}}

\color{red}
\put(-42,-95){\circle*{2}}
\put(-42,-97){\vector(0,-1){56}}
\put(-44,-97){\linethickness{0.7pt}\line(1,0){0}}
\put(-42,-155){\circle*{2}}

\color{green}
\put(-48,-95){\circle*{2}}
\put(-48,-97){\vector(0,-1){96}}
\put(-50,-97){\linethickness{0.7pt}\line(1,0){4}}
\put(-48,-195){\circle*{2}}

\color{red}
\put(-54,-95){\circle*{2}}
\put(-54,-97){\vector(0,-1){96}}
\put(-56,-97){\linethickness{0.7pt}\line(1,0){0}}
\put(-54,-195){\circle*{2}}

\color{green}
\put(-60,-95){\circle*{2}}
\put(-60,-97){\vector(0,-1){146}}
\put(-62,-97){\linethickness{0.7pt}\line(1,0){4}}
\put(-60,-245){\circle*{2}}

\color{red}
\put(-66,-105){\circle*{2}}
\put(-66,-107){\vector(0,-1){26}}
\put(-68,-107){\linethickness{0.7pt}\line(1,0){0}}
\put(-66,-135){\circle*{2}}

\color{red}
\put(-72,-105){\circle*{2}}
\put(-72,-107){\vector(0,-1){46}}
\put(-74,-107){\linethickness{0.7pt}\line(1,0){0}}
\put(-72,-155){\circle*{2}}

\color{red}
\put(-78,-105){\circle*{2}}
\put(-78,-107){\vector(0,-1){56}}
\put(-80,-107){\linethickness{0.7pt}\line(1,0){0}}
\put(-78,-165){\circle*{2}}

\color{red}
\put(-84,-105){\circle*{2}}
\put(-84,-107){\vector(0,-1){136}}
\put(-86,-107){\linethickness{0.7pt}\line(1,0){0}}
\put(-84,-245){\circle*{2}}

\color{red}
\put(-90,-105){\circle*{2}}
\put(-90,-107){\vector(0,-1){146}}
\put(-92,-107){\linethickness{0.7pt}\line(1,0){0}}
\put(-90,-255){\circle*{2}}

\color{red}
\put(-96,-115){\circle*{2}}
\put(-96,-117){\vector(0,-1){16}}
\put(-98,-117){\linethickness{0.7pt}\line(1,0){0}}
\put(-96,-135){\circle*{2}}

\color{red}
\put(-102,-115){\circle*{2}}
\put(-102,-117){\vector(0,-1){76}}
\put(-104,-117){\linethickness{0.7pt}\line(1,0){0}}
\put(-102,-195){\circle*{2}}

\color{red}
\put(-108,-115){\circle*{2}}
\put(-108,-117){\vector(0,-1){86}}
\put(-110,-117){\linethickness{0.7pt}\line(1,0){0}}
\put(-108,-205){\circle*{2}}

\color{green}
\put(-114,-135){\circle*{2}}
\put(-114,-137){\vector(0,-1){26}}
\put(-116,-137){\linethickness{0.7pt}\line(1,0){0}}
\put(-114,-165){\circle*{2}}

\color{green}
\put(-120,-135){\circle*{2}}
\put(-120,-137){\vector(0,-1){66}}
\put(-122,-137){\linethickness{0.7pt}\line(1,0){0}}
\put(-120,-205){\circle*{2}}

\color{green}
\put(-66,-145){\circle*{2}}
\put(-66,-147){\vector(0,-1){26}}
\put(-68,-147){\linethickness{0.7pt}\line(1,0){0}}
\put(-66,-175){\circle*{2}}

\color{green}
\put(-96,-145){\circle*{2}}
\put(-96,-147){\vector(0,-1){66}}
\put(-98,-147){\linethickness{0.7pt}\line(1,0){0}}
\put(-96,-215){\circle*{2}}

\color{green}
\put(-12,-155){\circle*{2}}
\put(-12,-157){\vector(0,-1){6}}
\put(-14,-157){\linethickness{0.7pt}\line(1,0){0}}
\put(-12,-165){\circle*{2}}

\color{green}
\put(-36,-155){\circle*{2}}
\put(-36,-157){\vector(0,-1){16}}
\put(-38,-157){\linethickness{0.7pt}\line(1,0){0}}
\put(-36,-175){\circle*{2}}

\color{red}
\put(-126,-165){\circle*{2}}
\put(-126,-163){\vector(0,1){26}}
\put(-128,-137){\linethickness{0.7pt}\line(1,0){0}}
\put(-126,-135){\circle*{2}}

\color{red}
\put(-132,-165){\circle*{2}}
\put(-132,-163){\vector(0,1){6}}
\put(-134,-157){\linethickness{0.7pt}\line(1,0){0}}
\put(-132,-155){\circle*{2}}

\color{red}
\put(-18,-162){\circle*{2}}
\put(-20,-164){\linethickness{0.7pt}\line(1,0){0}}
\put(-18,-167){\circle*{2}}

\color{green}
\put(-42,-162){\circle*{2}}
\put(-44,-164){\linethickness{0.7pt}\line(1,0){0}}
\put(-42,-167){\circle*{2}}

\color{red}
\put(-72,-165){\circle*{2}}
\put(-72,-167){\vector(0,-1){6}}
\put(-74,-167){\linethickness{0.7pt}\line(1,0){0}}
\put(-72,-175){\circle*{2}}

\color{red}
\put(-138,-165){\circle*{2}}
\put(-138,-167){\vector(0,-1){76}}
\put(-140,-167){\linethickness{0.7pt}\line(1,0){0}}
\put(-138,-245){\circle*{2}}

\color{red}
\put(-144,-165){\circle*{2}}
\put(-144,-167){\vector(0,-1){86}}
\put(-146,-167){\linethickness{0.7pt}\line(1,0){0}}
\put(-144,-255){\circle*{2}}

\color{red}
\put(-150,-175){\circle*{2}}
\put(-150,-173){\vector(0,1){26}}
\put(-152,-147){\linethickness{0.7pt}\line(1,0){0}}
\put(-150,-145){\circle*{2}}

\color{green}
\put(-156,-175){\circle*{2}}
\put(-156,-173){\vector(0,1){16}}
\put(-158,-157){\linethickness{0.7pt}\line(1,0){4}}
\put(-156,-155){\circle*{2}}

\color{red}
\put(-162,-175){\circle*{2}}
\put(-162,-173){\vector(0,1){16}}
\put(-164,-157){\linethickness{0.7pt}\line(1,0){0}}
\put(-162,-155){\circle*{2}}

\color{green}
\put(-168,-175){\circle*{2}}
\put(-168,-173){\vector(0,1){6}}
\put(-170,-167){\linethickness{0.7pt}\line(1,0){0}}
\put(-168,-165){\circle*{2}}

\color{red}
\put(-12,-175){\circle*{2}}
\put(-12,-177){\vector(0,-1){16}}
\put(-14,-177){\linethickness{0.7pt}\line(1,0){0}}
\put(-12,-195){\circle*{2}}

\color{green}
\put(-18,-175){\circle*{2}}
\put(-18,-177){\vector(0,-1){66}}
\put(-20,-177){\linethickness{0.7pt}\line(1,0){4}}
\put(-18,-245){\circle*{2}}

\color{green}
\put(-36,-195){\circle*{2}}
\put(-36,-197){\vector(0,-1){6}}
\put(-38,-197){\linethickness{0.7pt}\line(1,0){0}}
\put(-36,-205){\circle*{2}}

\color{green}
\put(-42,-195){\circle*{2}}
\put(-42,-197){\vector(0,-1){16}}
\put(-44,-197){\linethickness{0.7pt}\line(1,0){0}}
\put(-42,-215){\circle*{2}}

\color{red}
\put(-174,-205){\circle*{2}}
\put(-174,-203){\vector(0,1){66}}
\put(-176,-137){\linethickness{0.7pt}\line(1,0){0}}
\put(-174,-135){\circle*{2}}

\color{red}
\put(-66,-205){\circle*{2}}
\put(-66,-203){\vector(0,1){6}}
\put(-68,-197){\linethickness{0.7pt}\line(1,0){0}}
\put(-66,-195){\circle*{2}}

\color{red}
\put(-12,-202){\circle*{2}}
\put(-14,-204){\linethickness{0.7pt}\line(1,0){0}}
\put(-12,-207){\circle*{2}}

\color{green}
\put(-48,-202){\circle*{2}}
\put(-50,-204){\linethickness{0.7pt}\line(1,0){0}}
\put(-48,-207){\circle*{2}}

\color{red}
\put(-54,-205){\circle*{2}}
\put(-54,-207){\vector(0,-1){6}}
\put(-56,-207){\linethickness{0.7pt}\line(1,0){0}}
\put(-54,-215){\circle*{2}}

\color{red}
\put(-180,-215){\circle*{2}}
\put(-180,-213){\vector(0,1){66}}
\put(-182,-147){\linethickness{0.7pt}\line(1,0){0}}
\put(-180,-145){\circle*{2}}

\color{red}
\put(-186,-215){\circle*{2}}
\put(-186,-213){\vector(0,1){56}}
\put(-188,-157){\linethickness{0.7pt}\line(1,0){0}}
\put(-186,-155){\circle*{2}}

\color{green}
\put(-72,-215){\circle*{2}}
\put(-72,-213){\vector(0,1){16}}
\put(-74,-197){\linethickness{0.7pt}\line(1,0){4}}
\put(-72,-195){\circle*{2}}

\color{red}
\put(-78,-215){\circle*{2}}
\put(-78,-213){\vector(0,1){16}}
\put(-80,-197){\linethickness{0.7pt}\line(1,0){0}}
\put(-78,-195){\circle*{2}}

\color{green}
\put(-102,-215){\circle*{2}}
\put(-102,-213){\vector(0,1){6}}
\put(-104,-207){\linethickness{0.7pt}\line(1,0){0}}
\put(-102,-205){\circle*{2}}

\color{green}
\put(-12,-215){\circle*{2}}
\put(-12,-217){\vector(0,-1){26}}
\put(-14,-217){\linethickness{0.7pt}\line(1,0){4}}
\put(-12,-245){\circle*{2}}

\color{red}
\put(-6,-415){\circle*{2}}
\put(-6,-417){\vector(0,-1){86}}
\put(-8,-417){\linethickness{0.7pt}\line(1,0){0}}
\put(-6,-505){\circle*{2}}

\color{red}
\put(-12,-442){\circle*{2}}
\put(-14,-444){\linethickness{0.7pt}\line(1,0){0}}
\put(-12,-447){\circle*{2}\hspace{-2\unitlength}\circle{4}}

\color{red}
\put(-18,-445){\circle*{2}\hspace{-2\unitlength}\circle{4}}
\put(-18,-447){\vector(0,-1){16}}
\put(-20,-447){\linethickness{0.7pt}\line(1,0){4}}
\put(-18,-465){\circle*{2}}

\color{red}
\put(-12,-452){\circle*{2}}
\put(-14,-454){\linethickness{0.7pt}\line(1,0){0}}
\put(-12,-457){\circle*{2}}

\color{green}
\put(-24,-452){\circle*{2}}
\put(-26,-454){\linethickness{0.7pt}\line(1,0){0}}
\put(-24,-457){\circle*{2}}

\color{red}
\put(-30,-455){\circle*{2}}
\put(-30,-457){\vector(0,-1){6}}
\put(-32,-457){\linethickness{0.7pt}\line(1,0){0}}
\put(-30,-465){\circle*{2}}

\color{green}
\put(-36,-465){\circle*{2}}
\put(-36,-463){\vector(0,1){6}}
\put(-38,-457){\linethickness{0.7pt}\line(1,0){0}}
\put(-36,-455){\circle*{2}}

\color{red}
\put(-12,-462){\circle*{2}}
\put(-14,-464){\linethickness{0.7pt}\line(1,0){0}}
\put(-12,-467){\circle*{2}\hspace{-2\unitlength}\circle{4}}

\color{red}
\put(-24,-462){\circle*{2}\hspace{-2\unitlength}\circle{4}}
\put(-26,-464){\linethickness{0.7pt}\line(1,0){4}}
\put(-24,-467){\circle*{2}\hspace{-2\unitlength}\circle{4}}

\color{red}
\put(-42,-465){\circle*{2}\hspace{-2\unitlength}\circle{4}}
\put(-42,-467){\vector(0,-1){16}}
\put(-44,-467){\linethickness{0.7pt}\line(1,0){4}}
\put(-42,-485){\circle*{2}\hspace{-2\unitlength}\circle{4}}

\color{green}
\put(-48,-465){\circle*{2}\hspace{-2\unitlength}\circle{4}}
\put(-48,-467){\vector(0,-1){16}}
\put(-50,-467){\linethickness{0.7pt}\line(1,0){4}}
\put(-48,-485){\circle*{2}\hspace{-2\unitlength}\circle{4}}

\color{blue}
\put(-54,-465){\circle*{2}\hspace{-2\unitlength}\circle{4}}
\put(-54,-467){\vector(0,-1){16}}
\put(-56,-467){\linethickness{0.7pt}\line(1,0){4}}
\put(-54,-485){\circle*{2}\hspace{-2\unitlength}\circle{4}}

\color{red}
\put(-12,-482){\circle*{2}}
\put(-14,-484){\linethickness{0.7pt}\line(1,0){0}}
\put(-12,-487){\circle*{2}\hspace{-2\unitlength}\circle{4}}

\end{picture}
\]
\hrulefill

\caption{The dependency list of quicksort}
\label{fquickl}
\end{figure}

---------------------------------------------

\begin{figure*}
\small
\hrulefill
\[
\begin{picture}(500,210)(-30,0)
% The memory table
\put(-30,30){\framebox(40,180)[t]{\ }}
\put(-30,100){\line(1,0){40}}
\put(-30,110){\line(1,0){40}}
\put(-30,10){\makebox(60,10){The memory table}}
% A block
\put(10,110){\line(4,3){20}}
\put(10,100){\line(4,-3){20}}
\put(30,85){\framebox(40,40){\ }}
\put(30,115){\line(1,0){40}}
\put(30,115){\makebox(40,10){Last write}}
\put(30,105){\makebox(40,10){Last reads}}
\put(30,90){\makebox(40,10){\vdots}}
% An event
\put(70,125){\line(2,1){20}}
\put(70,115){\line(2,-1){20}}
\put(90,105){\framebox(40,30){\ }}
\put(90,125){\line(1,0){40}}
\put(90,115){\line(1,0){40}}
\put(90,105){\makebox(40,10){Time}}
\put(90,115){\makebox(40,10){Instr}}
\put(90,125){\makebox(40,10){Context}}
\put(130,130){\vector(1,-2){40}}
% The activation tree
% Level 0 (root)
\put(270,180){\framebox(30,30)[tl]{\it 1}}
% Level 1 (1 node)
\put(235,130){\framebox(30,30)[tl]{\it 2}} \put(250,160){\vector(3,2){30}}
\put(270,150){\makebox(50,10)[l]{\it NCA(4,7)}}
% Level 2 (2 nodes)
\put(190,80){\framebox(30,30)[tl]{\it 3}} \put(205,110){\vector(2,1){40}}
\put(280,80){\framebox(30,30)[tl]{\it 6}} \put(295,110){\vector(-2,1){40}}
% Level 3 (3 nodes)
\put(170,30){\framebox(30,30)[tl]{\it 4}} \put(185,60){\vector(3,4){15}}
\put(210,30){\framebox(30,30)[tl]{\it 5}} \put(225,60){\vector(-3,4){15}}
\put(260,30){\framebox(30,30)[tl]{\it 7}} \put(275,60){\vector(3,4){15}}
\put(295,50){\makebox(50,10)[l]{\it Current frame}}
% caption
\put(170,10){\makebox(80,10){The activation tree}}
% The result table
\put(360,30){\framebox(80,180){\ }}
\put(360,10){\makebox(80,10){The result table}}

\end{picture}
\]
\hrulefill
\caption{The main Embla data structures}
\label{fembladata}
\end{figure*}


-------------------------------------------------

The figure also shows how the data structures are used when the
profiled program does a memory read. From the memory table we find the
activation frame that was active when the last write was made and we
compute the {\em nearest common ancestor} (NCA) of that an the
current frame. The flow dependence we find will be attributed to the
function that the NCA corresponds to. 

If a dependency endpoint is part of the NCA (if the NCA is the current
frame or the frame pointed to from the memory table) it is labelled as
direct, otherwise it is indirect. If any of the activation frames on
the path from an endpoint to the NCA is on the black list, we have a
weak endpoint.

--------------------------------------------------



The {\em instruction trace} $T$ is a sequence of {\em instruction execution 
events} $e_1, \ldots, e_n$ which give rise to the {\em call tree} where a 
node is a subsequence $N \subseteq T$ bracketed by a pair of matched call 
and return events. The {\em parent} of a node $N$ is the smallest $M$ such
that $N \subseteq M$ and $N \not= M$, and we say that $N$ is a {\em child} 
of $M$. Two nodes with the same parent are {\em siblings}.
Thus the children of a node are all call events in it together 
with all instructions executed during these calls.

The program transformations we target, parallel execution of source statements,
essentially rearrange parts of the dynamic call tree by executing some subtrees 
in parallel. Two subtrees can be executed in parallel
if there are no dependencies between them. 
% Thus we want to know, for each
% call tree node $N$, the ordering constraints on the children of $N$ (that
% is, permissible rearrangements wrt the order in the original trace). 
Dependencies between instruction execution events are written as $e_i \prec e_j$
and induce dependencies between subtrees as follows: If $M \not= N$ are 
siblings, $e_i \in M$, $e_j \in N$ and $e_i \prec e_j$ then $M \prec N$.
We claim that each instruction level dependency $e_i \prec e_j$ gives rise to 
exactly one subtree dependency $M \prec N$ (different instruction level 
dependencies may give rise to the same tree dependency).

\newcommand{\li}{{\cal L}}
Given that we have computed a tree dependency $M \prec N$ we map it to
the source dependency $\li(M) \prec \li(N)$ where $\li$ maps a node to the
line number of the call site creating the node (the call instruction associated 
with the call that is the first event of the node). This is of course an
approximation; $\li(M) \prec \li(N)$ corresponds to the set of all node
dependencies $M' \prec N'$ such that $M' \not= N'$ are siblings, $\li(M)=\li(M')$
and $\li(N)=\li(N')$.


-----------------------------

Note that the same instructions in other calling contexts give rise to 
different dependencies, eg the flow dependence from line 15 (call to {\tt inc})
to line 17 (call to {\tt printf}).
