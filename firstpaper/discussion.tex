\section{Discussion}

In this section we would like to discuss our approach to parallel
programming and contrast it with two alternatives: Automatic 
(compiler) parallelisation and explicit thread parallelism.

While the information provided by Embla can be used for any hand 
parallelization strategy, it appears to be especially suitable to
combine with a fork-join framework. In this case, the programmer 
simply annotates function calls that should be made asynchronously 
and in addition determines the join points. For such programs we
know that in the absence of dependencies petween parallel parts,
the programs are free of both deadlocks and race conditions and are 
furthermore deterministic (if the sequential version of the program 
is deterministic).

Of course, since Embla is a testing tool, it may be the case that not
all relevant imputs have been used, allowing a dependency to slip
though the net. This will manifest itself as a bug in the parallel
program, possibly in a nondeterministic manner. However, the root
cause of the bug is the missed dependency, which is deterministic and
which can be found by running Embla with the offending input. Thus,
while Embla does not give proof of absence of dependencies (as a
static checker would), it reduces the problem of debugging a
multithreaded program to the problem of debugging a sequential
program. Here all of the standard testing machinery (regression 
testing, test coverage tools, \ldots) can be used.

