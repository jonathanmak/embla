\section{Introduction}

Parallel programming is no longer optional.  In order to enjoy continued
performance gains with future generation multi-core processors,
application developers must parallelize all software, old and
new.
While automatic parallelization based on static analysis 
is sometimes feasible, currently most software requires manual
parallelization.
Since this is a difficult task, there is urgent need for efficient tool support---in
particular, tools that assist the programmer in understanding the potential 
for parallelization in the code, parallelizing code with high potential, 
and validating that the resulting parallel code is correct.

In this paper we address the first two issues, by presenting a tool
which, when given a sequential program and an input, can estimate the 
amount of parallelism that would be available if the program was parallelized
using language constructs such as those in OpenMP~\cite{dagum98openmp}
or Cilk~\cite{blumofe96cilk}.
This parallelism measurement tool has been built by extending
Embla \cite{embla:08}, a Valgrind-based~\cite{valgrind:07} profiler which captures dependence information,
with functionality for simulating the effects of various parallel 
programming models based on independent fork-join task parallelism,
a framework used in many parallel programming environments.
We validate our results against timing 
measurements of explicitly parallel Cilk programs.

Unlike traditional studies of parallelism limits \cite{wall91limits,warg01limits}
which focus on hardware parameters, we focus on the 
parallelization of program source code, and so use a
different set of parameters to vary and different constraints for
parallelization. The aim is to address questions such as,
``Will thread-level speculation or parallel for-loops make much
difference to the possible speed-up when parallelizing
a program?'' Section~\ref{smethod} 
discusses our parallelization models in depth.

We have used our prototype tool to investigate the potential for 
parallelization of three collections of programs: the SPEC CPU 2000 
programs~\cite{henning00spec}, MiBench~\cite{guthaus01mibench} and the example programs in
the Cilk 5.4.6 distribution.
We contrast programs from these different sources and show the behaviour of 
the different parallelization models.
Section~\ref{sresults} reports the results of these experiments.

The contributions can be summarised as follows:
\begin{itemize}
\item
Dependences output by Embla assist the programmer in 
parallelization by identifying
appropriate source-level synchronization points. We show how synchronization
points identified in serial elisions of example Cilk programs
match those in the original hand-parallelized programs (Section~\ref{sresults:cilk-spawns}).
\item
By using Embla to map dynamic dependences back to program source,
our tool gives a realistic estimate of potential speed-up for
parallelization using frameworks such as Cilk and OpenMP.
Previous studies that we know of~\cite{Kreaseck00limitsof,warg01limits,oplinger99insearch}
have only considered \emph{speculative}
task-level parallelism---we give potential speed-ups for programs
both with and without the use of thread-level speculation.
Using this tool we present estimates of inherent parallelism in
various example programs (Sections~\ref{sresults:cilk-limits} and~\ref{sresults:benchmarks}).
\item
Critical paths output by our tool can
be used to identify at source-level bottlenecks that restrict the level of
inherent parallelism. We illustrate this using examples
from the MiBench benchmark suite, and suggest refactorings
or algorithmic changes to increase potential parallelism (Section~\ref{sresults:increasing}).
\end{itemize}
